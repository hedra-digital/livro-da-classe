% Bibliografia do texto de Juvenal e Gisele
% Texto do Zamorra

%Jorge: Conf. Aqueloo, mith
%	Diotima

\hyphenation{neo-pla-to-nis-mo}


\newcommand{\capitulo}[4][]{%
  \ifthenelse{\equal{#1}{}}{\def\sumario{#2}}{\def\sumario{#1}}
  \chapter[\sumario, \emph{por~#3}]%
		{#2
			\break{\large\bigskip \emph{#3} 
			({\small\textsc{#4})}%
			}%
		}%
	\refstepcounter{chapter}}

\makeatletter
\newcommand{\resume}[1]{%
  \begingroup%
    \list{}{%
      \listparindent=\parindent%
      \parsep=\parskip%
      \topsep=\z@%
      \rightmargin=\leftmargin%
     } %
    \item%
    \let\item\@undefined%
    {\vspace{1em}#1\vspace{1ex}}
    \endlist%
  \endgroup%
 } 
\makeatother


%\frontmatter

\capitulo{Apresentação}{Jean Marc Narbonne}{Université Laval/\,Quebec}
\markboth{Apresentação}{\textsc{j.m.} Narbonne}

\section{Neoplatonismo: influências e contemporaneidade}

Nos últimos quarenta anos, se multiplicaram, com uma intensidade
pouco comum e de diversos modos, edições e traduções, em diversas
línguas, dos textos mais importantes da tradição neoplatônica. É
importante ressaltar que, exceto Plotino, cuja obra tornou"-se cedo
integralmente acessível graças à influente tradução realizada por
Marsilo Ficino (1492), a maior parte dos autores da tradição
neoplatônica mais tardia, pouco editados, traduzidos e comentados,
não era conhecida senão por um pequeno círculo de especialistas, com
exceção talvez de Proclo e que Hegel contribuiu ao divulgar entre um
público mais amplo a partir do século \textsc{xix}. Figuras chaves do
neoplatonismo grego, tais como Porfírio, Jâmblico, Siriano, Damásio e
Simplício, costumavam aparecer como simples nomes ou eram estudados
somente de modo secundário, mas não por eles mesmos. No
entanto, esse florescimento aparentemente tardio do neoplatonismo
esconde a realidade das coisas.


De fato, o neoplatonismo formou muito cedo um verdadeiro delta com
ramificações extremamente numerosas. Logo depois de Plotino, todo o
mundo pagão até o século \textsc{vii} foi dominado pelo platonismo de tipo
neoplatônico, no qual se pode identificar várias diferentes escolas
que são ao mesmo tempo lugares de pensamento e de difusão: as escolas
de Roma, Apameia, Pérgamo, Atenas e Alexandria.

A estas se misturou muito cedo um neoplatonismo cristão, cujos
dois principais ramos são o agostiniano (ligado sobretudo ao
neoplatonismo de Plotino e Porfírio, notadamente através de Mario
Vitorino), e pseudo"-dionisiano (ligado a Proclo e ao neoplatonismo
jambliciano). Dois textos centrais, como se sabe, vão também
contribuir a espalhar durante a idade média esses dois ramos
neoplatônicos: a \emph{Teologia de Aristóteles} (escrito formado na
verdade a partir de vestígios árabes das três últimas
\emph{Enéadas} de Plotino, coleção possivelmente atribuível a
Porfírio); e o \emph{Liber de Causis} (tradução latina de um texto
árabe formado essencialmente de extratos dos \emph{Elementos de
Teologia} de Proclo).

Além do mais, ao lado do neoplatonismo cristão, pode"-se distinguir
um neoplatonismo bizantino, islâmico, judeu, e também, evidentemente,
renascentista e moderno (podemos pensar nos platônicos de Cambridge,
em Hegel ou Schelling, em Victor Cousin ou Bergson), com
prolongamentos importantes na filosofia contemporânea, e até em autores
como Heidegger, Levinas, Derrida ou
Marion.\footnote{ ``Levinas and the Greek Heritage'', Jean"-Marc Narbonne,
\& ``One Hundred Years of Neoplatonism in France'', Wayne
Hankey, Paris"-Leuven"-Dudley, Ma, Peeters, 2006; Stephen Gersh,
\emph{Neoplatonism after Derrida}, Brill, 2006}

O interesse por este tipo de filosofia é hoje tão forte que se
pode falar de um novo \emph{paradigma de leitura}, comparável ao
que foi, por exemplo, o \emph{tomismo} no século anterior. Quais
são os motivos deste recente êxito? São, sem dúvida, numerosos, mais
se pode, por hipótese, propor três principais:

\begin{enumerate}
\item O interesse pelo neoplatonismo se explica por um lado pela
necessidade de restabelecer o equilíbrio na apreciação da
contribuição da filosofia grega tardia à história das ideias. O
neoplatonismo foi visto durante muito tempo seja como uma filosofia
“neutra”, por ela servir essencialmente para transmitir com toda
imparcialidade as aquisições da reflexão platônica ou aristotélica,
seja como uma filosofia declinante (desprovida de originalidade),
trocada (desprovida de dimensão política), exaltada (desprovida de
racionalidade), ou, por fim, eclética (desprovida de
sistematicidade). Esta visão da história da filosofia se manteve até
pelo menos Zeller,\footnote{ W. Beierwaltes, «~Der Neuplatonismus
in Eduard Zellers \emph{Philosophie der Griechen~», Annali della
Scuola Normale Superiore de Pisa}, serie \textsc{iii}, Vol. xix, 3, 1989, p.
1179--1191.} e só será definitivamente derrubada com o grande
artigo de Praechter.\footnote{ K. Praechter, ``Richtungen und
Schulen im Neuplatonismus'', in \emph{Genethliakon für Carl
Robert}, Berlin, 1910 (K. Praechter, \emph{Kleine
Schriften}, hrsg. Von Dörrie, Hildesheim, 1973, p. 165--216. 
J.-M. Narbonne, ``Jamblique, le précurseur méconnu'',
\emph{Chôra} (Romênia) 5, 2007, p. 45--55.} Sabe"-se hoje que
ela é profundamente diversificada (malgrado a suposta disciplina de
escola); que ela oferece vozes muito originais com autores de
sensibilidades bem diferentes (Plotino o meditativo"-místico, por
exemplo, Jâmblico o teúrgico, e Damácio o aporético); que ela
apresenta um nível de sistematicidade em realidade quase sem
comparação em toda a história da filosofia, e que sua influência é
profunda tanto quanto variada.

\item O estudo do neoplatonismo permite renovar o olhar sobre as figuras
entre as mais importantes da tradição metafísica, sejam elas mais
especificamente decorrentes da tradição cristã, árabo"-muçulmana ou
judaica. Foi possível, por exemplo, constatar que o motivo central da
inefabilidade do princípio, os procedimentos da \emph{via negativa}
e a elaboração de múltiplos níveis de transcendência, fundamentados 
na estrutura das diferentes hipóteses do \emph{Parmênides} de
Platão, tais como reinterpretadas pelos neoplatônicos, haviam
exercido um papel determinante na reflexão de pensadores de crenças
bastante  diferentes. Aquilo que, por falta de melhor definição, é
frequentemente chamado de\emph{ tradições místicas} encontra
geralmente nessa fonte o seu denominador comum.

\item No espaço contemporâneo, o neoplatonismo responde a uma sede de
transcendência, satisfaz um desejo de absoluto e de sagrado de uma
maneira mais livre e pessoal, segundo um modo mais “filosófico” e à
distancia, poderia se dizer, dos sistemas de revelação ou de crenças
tradicionais (para os quais ele pode ao mesmo tempo servir de preparação, complemento
ou aprofundamento), enfim, ele é uma forma de teísmo em linha, com
muitas correntes da sociedade contemporânea.
\end{enumerate}

Por esses diferentes aspectos, vê-se que se pode interessar"-se
pelo neoplatonismo de uma maneira estritamente histórica, mas que se
pode também vislumbrar nele uma das faces do questionamento
contemporâneo sobre o lugar do homem no mundo, sua relação com
outrem, ou sua relação com o Absoluto. O conjunto dos estudos
apresentados na presente coletânea faz plenamente jus a esse vasto
leque. Portanto, o leitor poderá encontrar aqui várias ocasiões para não
somente reinterpretar o passado, mas também para meditar sobre o
presente.\\\bigskip

\hfill (Trad.~Cicero C.~Bezerra)
%\mainmatter


\capitulo{Plotino entre Narciso e Ulisses: Jogos de espelhos e a Nostalgia da casa}%
	{Marcus Reis Pinheiro}{uff}
\markboth{Plotino entre Narciso e Ulisses}{Marcus R.~Pinheiro}

\epigraph{\hfill{}\emph{A luz no fim do túnel é um espelho}}{}

Narciso é um mau amante: mancebo que despreza \emph{Eros},
recebe de modo justo a vingança de Nêmesis/ Afrodite por travar
guerra contra a força do amor. Supondo ser autossuficiente e não ter
necessidade dos influxos eróticos capazes de produzir o êxtase
divino, Narciso se mantém fechado e solitário e, como
Penteu,\footnote{ Penteu é um herói mitológico cujo desprezo por
Dionísio e seus rituais, termina por tê-lo dilacerado nas mãos das
bacantes. Ver Ovídio \textsc{iii}, 511f.} termina morto
pelas forças da loucura vingativa dos deuses.

Plotino é um dos autores mais antigos a interpretar alegoricamente
tal mito e, para ele, Narciso é a imagem do homem apegado aos
reflexos corpóreos do belo, homem que não compreende as forças
sublimatórias do amor, confundindo imagem com origem real da imagem,
toma o que é reflexo por substância, e termina por se afogar no seu
desprezo pelo \emph{Eros} alado que o guiaria, caso ele permitisse,
para fora desta ilusão narcótica.

Ulisses, por outro lado, é o antípoda de Narciso, herói filosófico
por excelência que vence as forças mágicas de Circe e Calipso, e
zarpa com sua frágil balsa no incerto das ondas do oceano, rumo à
pátria amada: \emph{philen es patrida}. O Lar, Ítaca, onde lhe
espera Penélope, a virtuosa, ambas símbolos do \emph{Noūs}, anseiam
por Ulisses para recebê-lo com delícias e prazeres libertadores.
Ulisses, o \emph{polútropon}, muito ardiloso, \emph{sabe
refletir} e assim sabe que o reflexo é reflexo e não se narcotiza nem
se obseda com a imagem muito bela que seus olhos lhe trazem das
formas de Calipso, e nem das guloseimas de Circe seu paladar se torna
enfeitiçado: sabe, como o filósofo da ascese platônica no
\emph{Banquete}, que o mundo sensível revela apenas imagens,
sombras, reflexos das reais belezas, e zarpa na aventura de
conquistá-las, pelos mares criativos da escada erótica sublimatória
de \emph{Eros}. O herói que sofre da dor do \emph{nostos}
(retorno), isto é, o herói da\emph{ nostalgia} do lar, tem como
companheiro o deus do \emph{logos}, Hermes, que lhe ajuda tanto no
episódio de Circe, dando"-lhe a erva \emph{molu} que inibe a magia
da bebida da feiticeira, quanto no de Calipso, trazendo a ordem de
Zeus para que a ninfa da ilha de Ogígia liberte o saudoso Ulisses.
Volta, então, para Casa, volta para o Lar, volta ao Inteligível que o
constitui e é fonte da beleza vista refletida em tantos corpos belos
que não o conseguem narcotizar. Ulisses é o Narciso que acorda de seu
sonho obsedante.

No presente artigo, iremos apresentar as quatro versões mais
antigas que temos do mito de Narciso assim como fazer alguns
apontamentos sobre a interpretação de Plotino. Ao longo do texto,
indicarei elementos importantes sobre o mito de Ulisses, mas não irei
analisá-lo tão detidamente. 

Pierre Hadot, em seu artigo \emph{Le mythe de Narcisse et son
interprétation par Plotin}, cita quatro versões mais extensas sobre o
mito de narciso, todas elas já do período romano: Ovídio, nas
\emph{Metamorfoses} (43 a.C. a 17 d.C.), Conon (\textsc{i} a.C.), em suas
\emph{Narrações}, Pausânias (\textsc{ii} d.C.) na \emph{Descrição da
Grécia} e Filóstrato, o Velho (\textsc{iii} d.C.), em sua descrições de
quadros, \emph{Imagines}.\footnote{ Para uma lista mais completa,
Hadot, 1999 indica a tese de doutorado de Vinge, 1967.}

Vamos primeiro trabalhar com a que Ovídio  
apresenta nas \emph{Metamorfoses}, que é a mais longa e
detalhada das versões. Vale já salientar que em algumas versões, o
contexto em que se insere o mito de Narciso tem uma forte ligação com
Dionísio. Estamos no livro \textsc{iii} das \emph{Metamorfoses}, em que
descrevemos a casa de Cadmo, irmão de Europa e fundador de Tebas, que
sofre as investidas de Hera, ciumenta da relação de Zeus com Europa.
O nascimento de Dionísio e o desprezo de Penteu por esse novo deus,
assim como seu consequente desmantelamento pelas Bacantes, vêm logo
em seguida ao mito de Narciso.

Ao narrar a história de Tirésias,\footnote{ Homem que havia sido
mulher que é chamado a decidir a disputa entre Hera e Zeus por quem
sentia mais prazer sexual. Zeus fundamentava suas traições pelo fato
de as mulheres sentirem mais prazer sexual em uma única relação, e
Tirésias oferece um veredicto a favor de Zeus.} Ovídio nos conta que
Liriope, ninfa azul"-onda,\footnote{ A edição principal que utilizo é
\textsc{ovídio}. \emph{As Metamorfoses}. Tr. Antônio Feliciano de Castilho.
Simões Editora: Rio de Janeiro, 1959. A edição em latim \textsc{ovid}.\emph{
Les} \emph{métamorphoses}. Paris: Les Belles Lettres, 1925--3 vv.
(Collection des Universités de France ). Ainda consultei três
traduções das \emph{Metamorfoses}, de Mary Innes, London: Penguin,
1955; de Charles Martin, New York: Norton \& Company, 2004; e a de A.
D. Melville, Oxford: Oxford University Press, 1996. O termo para
“azul"-onda” é \emph{caerulus}: azul, azul"-esverdeado, cor escura.}
teria sido a primeira a se consultar com o adivinho depois que este
recebe seus dons proféticos. Tendo tido um filho, chamado Narciso,
com o deus do rio Cefiso, Liriope pergunta a Tirésias se este terá
vida longa, ao que o adivinho responde que apenas se não se conhecer
a si mesmo (linha 348, \emph{si se non noverit}). Narciso, com sua
altivez e orgulho, desprezava todos os que se enamoravam por ele. Um
dos primeiros encontros de Narciso é com Eco, ninfa amaldiçoada por
Hera a repetir as últimas palavras de quem quer que se dirija a ela
(na tradução de Castilho: “Eco, a ninfa loquaz, a que não pode Falar
primeiro, nem calar"-se ouvindo”, linhas 356--357). Depois de um
interessante jogo de chamados, repetições e contra"-chamados, Narciso
a despreza e de tão envergonhada ela se esconde em uma caverna e
definha até virar só voz.\footnote{ Interessante a tradução
para o inglês de Martin (Ovid 2004), 396--403. “\emph{Unsleeping grief
wasted her sad body, reducing her to dried out skin and bones, then
voice and bones only; her skeleton turned, they say, into stone. Now
only voice is left of her, on wooded mountainsides, unseen by any,
although heard by all; for only the sound that lived in her lives
on.}”}

Um de seus inúmeros outros admiradores desprezados lhe dirige uma
maldição, para que ocorra com Narciso o mesmo que ocorre aos seus
admiradores: que ele ame e não possua quem ama,\footnote{ “\emph{Sic amet
ipse licet, sic non potiatur amato.}” “Que ele ame da mesma forma, sem
possuir o objeto amado”, 405. Exatamente por Narciso \emph{ser} o
que ama, ele não pode possuí-lo.} em uma espécie de lei de Talião, e
é Nêmesis, a deusa da vingança, quem ouve a oração. Em um dia de caça
(a indicação de Ártemis aqui é clara, a deusa da caça), tomado por
repentina sede, Narciso encontra uma superfície líquida que nada a
modifica,\footnote{ Muito interessante também a descrição negativa do
lago de Narciso: linhas 407--411, “\emph{There was a clear pool of
reflecting water unfrequented by shepherds with their flocks or
grazing moutain goats; no bird or beast, not even a fallen twig
stirred its surface}” na tradução de Martin. Em português de Castilho:
“Sem limos, toda esplêndida, manava Fonte argêntea, onde nunca os
pegureiros, nunca do monte as cabras repastadas, nem outra qualquer
grei, jamais desceram Ave alguma o cristal lhe não turbara, nem fera,
nem caduca arbórea rama. Com seu frescor em torno se lhe alastra mole
tapete ervoso, e a cingem bosques, do lago contra os sóis perene
escudo”} e se apaixona por seu próprio reflexo.\footnote{ “Deitou"-se,
e, onde cuidou matar a sede, outra mais forte achou”, “Dumque
sitim sedare cupit, sitis altera creuit”. Castilho, linhas 415--416.
Vejo um paralelo com o dizer de Jesus com a samaritana, João 4,
13--14, “Quem bebe dessa água tornará a ter sede, mas quem beber da
água que eu lhe der jamais terá sede. A água que eu lhe der será nele
uma fonte que jorra para a vida eterna.”} 

\begin{verse}
Levantando"-se um pouco, e alçando os braços\\
Aos bosques do arredor: “Ai, disse, ó Bosques\\
Houve jamais tão bárbaros amores?\\
Vós sabeis de bastantes, vós lhes destes\\
Nesta tácita sombra amigo amparo\\
Vós contais longos séculos; ah, Bosques!\\
Houve nunca infeliz, que assim morresse?\\
Vejo, amo; e não encontro, o que amo, e vejo:\\
Tanto, onde entrou paixão, reinais delírios!\\
Por cúmulo de dor, quem nos aparta,\\
Não é profuso mar, caminhos longos,\\
Ou fechada muralha, ou crespas serras;\\
Mas pobre fonte apenas! Ele mesmo,\\
Quer vir, quer dar"-se a mim; surge a beijar"-me\\
Todas as vezes, que a beijá-lo eu desço;\\
Quase, quase que os lábios se nos tocam;\\
Um nada a amor estorva. Oh! Sai da fonte,\\
Quem quer que sejas, singular menino;\\
Não zombes deste ardor mais longo tempo\\
\mbox{}[\ldots{}]\\
Não sei, que esp'rança meiga me está dando,\\
Esse aspecto benigno! Quando os braços\\
Te lanço, tu mos lanças; ris, se eu rio;\\
Choro, vejo"-te em lágrimas; teus olhos \\
Sempre à frase dos meus fiéis respondem.\\
E a crer da linda boca os movimentos,\\
Diriges"-me expressões, que ouvir não posso.\footnote{ Linhas
440--462.}
\end{verse}

Na versão de Ovídio, Narciso chega a tomar consciência do que ocorre,
mas não consegue se desvencilhar de sua obsessão pela linda figura
refletida, e míngua até sumir.\footnote{ Ele vai minguando e
desaparece igual a Eco, na tradução de Castilho: “Não pode mais; bem
como ao leve fogo Loura cera se funde; e ao sol temp'rado, De geosa
manhã se desgasta, se atenua: A mista cor da púrpura e da neve 490 Já
se esvaiu; sumiram"-se com ela, Forças, vigor, encanto, o próprio
corpo.” 487--491.} Eco fica ao seu lado até a morte e repete todos
seus lamentos por não poder consumar o seu amor. E até ao ser levado
pelo rio Estige para a casa dos mortos, ele ainda contemplava sua
própria imagem no rio. Por fim, as ninfas dos rios, suas irmãs,
fizeram lhe um funeral, utilizando especialmente a flor do narciso
que nasceu no lugar onde ele morreu.

\section{Conon}

Em Conon (Trazaskoma \emph{et all}, 2004),\footnote{ Da
obra de Canon, temos apenas resumos apresentados por um monge
bizantino do século \textsc{ix} chamado Fócio. Hadot dá uma referência:
\textsc{photius}, \emph{Bibliotheque.} Cod. 186, \textsc{iii}, p. 19.}temos uma
descrição mais forte do desprezo de Narciso pelos seus amantes, tanto
que ele chega a enviar uma espada para Amenias, um que não se
afastava nunca, com a qual este se mata na frente de Narciso. Nessa
versão, que parece ressaltar o intenso culto a \emph{Eros} em
Tespis, Narciso zomba explicitamente do deus do amor, e é esse deus
que se vinga e não Nêmesis. A Beócia, região onde fica Tespis é a
mesma região onde se encontra Tebas, cidade principal do canto
\textsc{iii}
das \emph{Metamorfoses} de Ovidio. Portanto, vemos que há uma
identificação geográfica entre essas versões.

\section{Pausânias}

De acordo com W. H. S. Jones (Pausanias, 1919), Pausânias, um
geógrafo que se preocupa com descrições mais factuais, tem em vista o
\emph{viajante} que toma prazer em passear pela Grécia. No século
\textsc{ii}
d.C. não eram poucos os que passeavam só para ver a região. Pausânias
cita umas 19 vezes o termo \emph{exegetai}, isto é, uma espécie de
cicerone que mostrava os lugares aos visitantes relatando as
histórias e lendas de cada lugar. 

A passagem em que ele remete a Narciso trata novamente da cidade de
Tespis e, colocando o mito claramente vinculado à região da Beócia.
Fraser (Pausanias, 1898, p 159, v 5) vai afirmar que provavelmente
Pausânias está também buscando explicar o culto ao deus \emph{Eros}
pelos habitantes de Tespis. Encontramos na \emph{Descrição da
Grécia} um repúdio à história mais comum de Narciso pela pouca
plausibilidade de uma pessoa se enganar com a própria imagem. Ele
apresenta uma versão menos conhecida (e também muito interessante) em
que Narciso teria uma irmã gêmea, e após a sua morte, nosso herói
alivia sua dor ao olhar para a fonte. Pausânias ainda cita o episódio
de Perséfone raptada por Hades ao se distrair com um belo narciso em
flor, vinculando assim os dois mitos.

\section{Filóstrato}

Filóstrato é um representante da segunda sofística, no século
\textsc{iii}
d.C. Além de biografias, (como a de Apolônio de Tiana) nos chegou
dele um livro de descrições curtas de quadros com elementos
mitológicos, feitas provavelmente como exercícios de retórica.

Encontramos ali a descrição de um quadro em que Narciso se
encontra em uma gruta recheada de elementos claramente dionisíacos.
Filóstrato descreve com muita vivacidade detalhes do olhar de Narciso
e do modo como ele estaria apaixonado pela figura refletida no lago.
A caverna é dedicada à Aqueloo, um deus dos rios, e às ninfas. O lago
tem cachos de uvas suspensos sobre, um vinhedo e também outras
plantas relacionadas aos \emph{thyrsi}. Pequenas flores brancas,
narcisos, começam a florescer ao lado do lago em honra ao jovem. Um
dos elementos mais interessantes da descrição de Filóstrato é a
relação entre a pintura e o reflexo do lago que engana nosso herói.
Vamos aqui salientar a passagem em que o autor faz tal relação e
conversa com Narciso.

\begin{quote}
Quanto a ti, jovém, não é alguma pintura que te ilude, nem tu estás
inscrito em cores ou cera, mas tu não percebes que a água “esculpiu”
tua imagem (\emph{eidos}); tu também não consegues refutar o
sofisma da fonte; seria necessário apenas mexer com sua cabeça, ou
mover sua mão e não ficar parado na mesma atitude; mas tu esperas uma
iniciativa dele, como se tivesse encontrado um outro. Será que
esperas que a fonte entre em conversas contigo? Este jovem não
entende nada do que lhe dizemos, mas está imerso, olhos e ouvidos, na
água, e nós mesmos devemos falar sobre ele como se fosse uma
pintura”.\footnote{ \textsc{philostratus}, 1931, p.~89.}
\end{quote}

Há aqui claramente uma crítica à passividade de Narciso. Parece
que Narciso permanece na mesma posição esperando, como sempre foi o
seu caso, que o outro tome a iniciativa. Assim como em Ovídio,
Narciso fica pasmado, estatelado olhando sua própria imagem e não
ousa mexer"-se e essa própria imobilidade o engana. Interessante notar
o modo como Filóstrato, um retórico, descreve a impotência de
Narciso, que não consegue refutar o sofisma (\emph{elencheis
sóphisma}) da fonte.

\section{Interpretação de Plotino}

A interpretação de Plotino afirma que Narciso, obcecado pela imagem e
ignorante da imagem como imagem, é um símbolo do homem que não
consegue fugir das belezas sensíveis e retornar para as belezas do
mundo inteligível. As duas passagens nas \emph{Enéadas} em que
Narciso parece ser mencionado estão vinculadas à ascese do Belo,
intimamente relacionada ao discurso da Diotima no \emph{Banquete}.
Tanto em Plotino como em Platão, o homem enamorado do belo deve ser
educado e deve aprender que há belezas em níveis de intensidades
diferentes, e que a beleza corpórea é apenas o primeiro nível de
beleza, sendo apenas um reflexo de uma beleza muito mais intensa no
nível inteligível. Sendo o mundo noético a origem do reflexo corpóreo
do belo, é para lá que nos leva nosso desejo, e tal jornada é em
verdade um retorno para o local de origem de nossas almas. Assim, o
mito de Ulisses é um contraponto ao mito de Narciso, pois Ulisses
consegue voltar para o Lar, e seu mito se torna a imagem de um
imperativo: devemos abrir as velas ao mar rumo ao inteligível que é
nossa origem. 

Pode"-se fazer a seguinte analogia simbólica como resumo da
interpretação de Plotino sobre o mito analisado: Narciso, ele mesmo,
está para a sua imagem refletida na água assim como a alma, no que
ela tem de identificação com o nível noético, está para as belezas
corpóreas.

\begin{center}
Narciso --- reflexo na água

Alma (noético) --- belezas corpóreas
\end{center}

\begin{quote}
Pois, vendo as belezas nos corpos, ele não deve se voltar para elas,
mas sabendo que são figuras, marcas\footnote{ O termo \emph{ichnos}
é bastante interessante neste contexto, pois remete à ideia de
pegadas, sinais deixados pelas realidades superiores, como pistas que
podem nos orientar no caminho de volta. Temos aqui ideia 
de que a beleza do mundo sensível são indicações e pistas
para que possamos resolver um mistério, um quebra cabeça. Pode"-se
fazer uma comparação com o movimento dos astros que deve ser
interpretado para que se possa perceber sua natureza heliocêntrica. A
natureza como enigma a ser desvendado, corroborando o fragmento de
Heráclito, ama esconder"-se. Ver Hadot, 2007.} e sombras,
ele deve fugir para aquela realidade da qual essas belezas são
figuras. Pois, se alguém corresse para pegar a imagem desejando pegar
o verdadeiro --- assim como aquele que ao desejar pegar a bela imagem
portada pela água (como me parece que algum mito indicou
(\emph{ainíttetai}), imergiu no rio e desapareceu --- desse mesmo
modo, aquele que se agarra aos belos corpos e não os descarta, irá
afundar não com o corpo, mas com a alma para a escuridão e para o
fundo, onde não há prazer para o intelecto, e então permanece cego no
Hades, convivendo lá com as sombras. “Fujamos para a pátria querida”,
alguém recomendaria de modo melhor. O que é, então, esta fuga e como
ela se processa? Devemos lançar velas ao mar como dizem que Ulisses
fez, fugindo da maga Circe e de Calipso\footnote{ Calipso vem do
verbo \emph{kalypto}, que quer dizer esconder.} (falando
enigmaticamente, penso eu), pois ele não aceitou permanecer, mesmo
tendo prazeres pelos olhos e convivendo com muita beleza sensível.
Nossa pátria, de onde viemos e onde está nosso pai, está lá.\footnote{
	\textsc{i}, 6 [1], 8, 6--23, \emph{Sobre o Belo}}
\end{quote}

E ainda:


\begin{quote}
Mas a natureza que produz coisas tão belas é bela de modo bem
anterior a elas. Nós não estamos acostumados a olhar o interior das
coisas, nem o conhecemos, mas procuramos o exterior ignorando que é o
interior que nos move. Assim como se alguém desejando sua própria
imagem, caçasse ela, ignorando de onde ela vem.\footnote{ \textsc{v}, 8 [31], 2,
31--35 \emph{Sobre a Beleza Inteligível}.  Vale apenas indicar que na segunda
passagem o termo ignorar (\emph{agnoo}) aparece como central.}
\end{quote}

Vale ainda ressaltar algumas características da interpretação
Plotino que são diferentes da teoria de Platão. Platão utiliza
largamente o linguajar da cópia (\emph{mímesis, eikon},
\emph{eidolon} etc.) para tratar do mundo sensível: de alguma
maneira, o mundo sensível é uma cópia do mundo inteligível. O que não
se pode pressupor em Platão\footnote{ Vale citar o \emph{Parmênides}
de Platão, em que o personagem homônimo levanta a possibilidade de
as Formas serem realidades mentais, presentes na alma: “A menos,
Parmênides, teria dito Sócrates, que cada uma dessas ideias não passe
de pensamento (\emph{nóema}), se não for, tão"-só, nas almas
(\emph{psykhaîs}). Assim, cada ideia seria uma e não ficaria
sujeita ao inconveniente de que falamos há pouco”.  Parmenides, em
seguida, refutará tal possibilidade. 132b. Tradução Carlos Alberto
Nunes modificada.} é que o nível inteligível seja interno ao sujeito
que pensa. Nesse sentido, não se pode dizer que em Platão o amor
pelos objetos sensíveis externos ao sujeito possa ter alguma
característica narcísica, já que eles não podem ser qualificados como
reflexos do que há internamente ao homem.\footnote{ Uma passagem que
talvez seja interessante ressaltar como aspecto narcísico em Platão é
a seguinte do \emph{Fedro}, 255c: “Esse desejo se insinua no amante,
e quando este se encontra cheio dele, transborda. Do mesmo modo que
um zéfiro ou que um som refletido por um corpo sólido (eco) e polido,
também as emanações da beleza, entrando pelos olhos através dos quais
--- como lhe é natural --- atingem a alma, volta esta ao belo, estende as
assas e, molhando"-as as torna capazes de gerar novas asas, inundando
também de amor a alma do amado”.}

Sobre Ulisses, vamos apenas indicar um breve estudo que faz
Lamberton.
Este nos diz que a passagem do \emph{Sobre o belo} é a única em que
Plotino diretamente cita um episódio dos textos homéricos (Lamberton,
1989). A expressão \emph{philen es patrida} aparece duas vezes na
\emph{Ilíada} (\textsc{ii}, 140; \textsc{ix}, 27) mas seu contexto não parece ter
relação com o que diz Plotino. Já na \emph{Odisseia}, além de
concentrar talvez a ideia geral de todo o épico, aparece 19 vezes.
Lamberton sugere que talvez Ulisses tenha sido escolhido como o herói
mítico melhor adaptado para interpretações alegóricas por ser ele
alguém que tem intenções secretas em seus dizeres, isto é, diz uma
coisa e intenciona outra. No entanto, Lamberton salienta que a
interpretação da astúcia de Ulisses como fundamento dos supostos
sentidos secretos em seus dizeres não aprece explicitamente na
Antiguidade.

Há ainda outra passagem em que, apesar de não ser explícita,
Plotino parece apontar para Ulisses. Trata"-se do primeiro capítulo de
um dos primeiros tratados das \emph{Enéadas}, em que Plotino retoma
um tema clássico da filosofia sobre os \emph{tipos de
vida}.\footnote{ Ver, por exemplo, o tratamento que Aristóteles dá a
esse tema na \emph{Ética a Nicômaco}.} Ele descreve três tipos de
vida: (1) a que tem no prazer o sumo bem, aquela dos homens que
permanecem presos à matéria sensível; depois, (2) aquela que tem o
sumo bem na honra que, mesmo sendo a vida de homens superiores, ainda
não contemplam a beleza do mundo inteligível; e por fim a vida
daqueles que se lançam para o mundo inteligível e retornam para casa
e para os deleites de lá.\footnote{ Brehier na introdução para a
edição da Belles Lettres apresenta uma interpretação que Plotino aqui
trata dos três tipos de filósofos, os Epicuristas, os Estoicos e os
platônicos.} 

\begin{quote}
A terceira raça de homens são divinos pela superioridade de seu poder
e da penetração de seu olhar. Eles veem com um olhar penetrante o
esplendor do alto, eles se elevam para além, como se estivessem sobre
as nuvens e a bruma daqui de baixo; e permanecem lá, percebendo do
alto todas as coisas daqui, se comprazendo com o lugar verdadeiro e
da real casa, assim como um homem que retorna para a pátria bem
governada desde longa errância.\footnote{ \textsc{v}, 9 [5], 1, 16--22,
\emph{Sobre o intelecto, as ideias e aquilo que é.}} 
\end{quote}

Acredito que a passagem em questão apenas reforça os mesmos elementos
da outra passagem presente no tratado 1 (cronologicamente bastante
próximo deste) que descreve Ulisses como o herói filosófico por
excelência, que abandona o mundo material e se lança rumo ao Lar. 

Assim, realçando as antíteses dos dois heróis, poderíamos dizer que a
beleza de Ogígia e de sua mestra, Calipso, assim como as magias de
Circe são correlatas à beleza narcótica da imagem de Narciso
refletida na água. Se Narciso conseguisse se libertar de sua
maldição, seria como Ulisses que consegue fugir da ninfa e da
feiticeira, assim como se Ulisses ficasse preso, estaria reproduzindo
o erro obsedante de Narciso.

De posse destes dados, vamos analisar alguns pontos relevantes do
mito de Narciso para o estudo de Plotino e do neoplatonismo.

\subparagraph{Os mitos correlatos a Narciso}
Um dos principais mitos com o qual Hadot traça paralelos com o de
Narciso é o mito de Hipólito, presente na tragédia homônima de
Eurípides. O desprezo por Afrodite é um traço fundamental no mito de
Narciso,\footnote{ Que aparece como desprezo por \emph{Eros} em
Conon e na vingança de \emph{Nemesis} (deusa muitas vezes
relacionada a Afrodite) em Ovídio.} e tal traço o aproxima do
caçador Hipólito, adorador de Ártemis, e amante da natureza. Seu
desprezo por Afrodite é o tema de toda a tragédia, que têm na
vingança da deusa do amor o centro de sua trama. O enredo básico da
tragédia é o seguinte: Por ser desprezada, Afrodite faz com que
Fedra, madrasta de Hipólito, se apaixone pelo enteado. Hipólito
termina por saber da paixão e Fedra se mata acusando, por carta,
Hipólito de ter tentado violentá-la. O pai, Teseu, expulsa o filho e
pede para Posseidon eliminá-lo, o que termina por acontecer justo no
momento em que o pai descobre todo o plano de Afrodite. 

Narciso também é um caçador e também despreza o amor e aqueles que o
amam. Em verdade, o desprezo por todos os seus amantes é o correlato
do amor fixo que Narciso sente sobre si mesmo: o amor obcecado da
única figura, cópia de sua imagem, faz de Narciso imediatamente um
total desprezador de todas as outras belezas (pensando aqui também
nas inteligíveis). Muito mais do que um amante, Narciso é um
desprezador, pois amando intensamente a sua única imagem, não acolhe
nenhuma outra beleza. 

Tal combate entre Narciso e Afrodite nos revela a intricada relação
entre o belo herói e a deusa do amor. Por um lado, temos um mau
amante, um herói que não sabe amar, que em suas desventuras irá
indicar um erro essencial do humano: apegar"-se a um reflexo e não
compreender a fonte deste reflexo. O desprezo pelo amor e o culto a
Ártemis, deusa virgem e impenetrável, podem também indicar um repúdio
radical a qualquer elemento de dependência, afirmando uma radical
liberdade do caçador na floresta. A vingança de \emph{Eros} e
\emph{Afrodite} vem pela total prisão em que colocam Narciso, e
esta mesma prisão pode ser pensada como uma impossibilidade de se
“fugir daqui”, como tanto nos recomenda Plotino citando Platão. Pela
incapacidade de um amor mais radical, que alcance níveis intensos de
beleza, o jovem Narciso prefere ficar preso a um amor estático e que
não lhe corresponde.

Outro mito relacionado com o de Narciso é o de Perséfone,\footnote{
Sobre seu mito, ver principalmente o hino homérico \emph{Hino a
Demeter}. Homer, 1995.} jovem que se enamora de um lindo narciso que
a Terra havia posto no seu caminho a pedido de Zeus que trama o rapto
junto com Hades. A imagem da jovem menina em flor assinala a ingênua
paixão pelo próprio poder de sedução, indicando o traço de
enamoramento da autoimagem dos adolescentes que descobrem o próprio
corpo em transformação. Pode"-se perceber no mito que o rapto se dá no
momento de certo hipnotismo, certo traço narcótico no amor próprio da
adolescente, que pela sua obsessão, se torna cativa da morte, Hades.
Hadot desenvolve essa ideia de “raptos da juventude”, tão frequente
na mitologia grega, como a fascinação narcótica da autoimagem em que
se perde a autonomia da individualidade, tornando"-se escravo do poder
dos deuses. Há, assim, uma relação estreita entre um fascínio
narcótico e a morte. Pode"-se, talvez, ver aqui que para Plotino o
homem narcísico, que ama as belezas corpóreas e não se lembra de sua
origem está vinculado a uma ideia de morte e monotonia.

Ainda podemos relacionar o mito de Dionísio, em sua versão
órfica,
com o espelho ofertado pelos titãs para distraí-lo e ser
estraçalhado. O próprio Plotino parece se referir a tal mito em
\textsc{iv}, 3
[27], 12,1, e Olimpiodoro, comentador neoplatônico tardio, se refere
ao mito de modo bastante semelhante à interpretação de
Plotino.\footnote{ \textsc{olimpiodore}, \emph{In Phaedonem,} p. 111, apud
\textsc{hadot}, 1999.} Hadot sugere que Plotino aqui talvez se refira a
práticas de se aceder as almas de mortos através de catoptromancias ou
lecanomancias, práticas mágicas de se comunicar com almas através de
espelhos ou reflexos na água. Aqui poderíamos ir ainda mais fundo e
descrever a própria identificação da alma com o corpo humano com a
magia catoptromântica, uma alma a ser capturada em um espelho. 

Assim, a própria existência humana encarnada é uma ilusão
narcísica, pois apaixonados pelo próprio corpo e suas realizações, a
alma acredita ser ela mesma esse corpo, e permanece presa ao reflexo.
A interpretação de Hadot do mito de Narciso em Plotino trabalha
bastante com essa perspectiva, apesar de as passagens de nosso
filósofo estarem explicitamente vinculadas à ascese do Belo e não ao
movimento de encarnação.

\subparagraph{Narciso, o imperativo délfico e a ambiguidade do autoconhecimento}
Narciso, estranhamente, recebe ao nascimento uma previsão de um
sacerdote de Apolo alertando"-o para o perigo de seguir o imperativo
délfico: Apolo, enigmaticamente, como lhe é apropriado, alerta contra
si mesmo. Vale aqui procurar ressaltar as ambiguidades deste
imperativo e o modo como Plotino interpreta o mito. Todo o ponto me
parece ser que o “si mesmo” pode ser compreendido em níveis
diferentes, ensejando um jogo de linguagem com o imperativo délfico. 

Em um primeiro nível, podemos chamar de si mesmo a imagem de Narciso,
o seu rosto que é projetado na água. Assim, Narciso ama e conhece de
alguma forma a si mesmo quando vê sua imagem na água. Se
permanecermos no nível sensível e não dispormos da teoria de Plotino
que vai defender níveis do “eu”, Tirésias tem razão ao precaver a
Narciso contra o imperativo délfico, pois sua imagem é, de alguma
forma, ele mesmo. Na versão de Ovídio, Narciso chega até a
compreender que a imagem do lago é ele mesmo, mas seu amor não cessa,
provavelmente revelando a força de Nemesis/\,Afrodite que consegue
superar até a consciência do reflexo como reflexo.  Narciso é
relativamente feliz até que não conheça a si mesmo: o respeito ao
imperativo délfico o mata. Apolo, semelhante à Noite, e suas flechas
terríveis.\footnote{ “A cada passo que dá, cheio de ira, ressoam"-lhe
as flechas nos ombros largos; à noite semelha, que baixa
terrível”\emph{ Ilíada} \textsc{i}, 46--47, trad. Carlos Alberto Nunes. Vale
indicar essa passagem como um exemplo de que mesmo Apolo sendo o deus
do sol e da luz, ele é também semelho à noite.}

No entanto, em um segundo nível, e especialmente se levarmos em conta
a filosofia de Plotino, o verdadeiro “si mesmo” de Narciso não é essa
autoimagem corporal, o que exclui um verdadeiro amor próprio ou
autoconhecimento no ato de obsedar"-se no lago, pois seu amor e
conhecimento são relativos a uma imagem criada exteriormente que o
narcotiza, o torna obcecado e ignorante de si, e assim, sempre com
mais sede. Para Plotino, o amor de Narciso pela sua imagem refletida
no lago não é fruto de um respeito ao imperativo délfico, pois a
verdade de Narciso não é a sombra na água, mas sim o seu “eu” noético
superior. Nosso filósofo talvez respondesse a Tirésias que somente no
momento em que Narciso realmente respeitar tal imperativo e, como
Ulisses, se lançar no mar do inteligível, irá ele inutilizar a força da
lecanomancia\footnote{ Lecanomancia ou a “arte divinatória que consiste em
observar o som produzido por metais ao tocarem o fundo de recipientes
com água, quando neles depositados” (\textsc{Houaiss}).} ou 
catoptromancia, que aprisiona e prende a alma.  

O que Narciso contempla no lago é um reflexo de uma beleza que tem sua
origem em outra ordem. Ao se fazer a transposição do mito par a
metafísica de Plotino, podemos ainda incluir as belezas dos astros e
do céu estrelado como recipientes do reflexo inteligível da alma.
Assim, surge a figura do astrônomo estupefato pela beleza dos astros
como um Narciso que precisa descobrir a origem desta beleza, para não
naufragar no mar brilhante do céu. As belezas que vemos no cosmo são
apenas um reflexo da beleza inteligível, e se por um lado elas são um
caminho na obediência ao imperativo délfico, pois são traços do amor
inteligível, por outro, é apenas na superação do amor a estas belezas
que podemos realmente respeitar o imperativo de Apolo. Assim, o
físico"-filósofo, enamorado pelas belezas da \emph{physis}, deve
procurar a ajuda de Hermes --- o deus do \emph{logos} que intervém
tanto no caso de Circe quanto no de Calipso --- para que possa inibir a
sua força mágica obsedante, e zarpar para os mares do Lar. Em uma
linda imagem, poderíamos pensar todo o céu estrelado como um grande
lago cósmico no qual olhamos e vemos refletidas as belezas
matematicamente dinâmicas, na grande dança cósmica embalada pela
harmonia das esferas, de que nos fala Plotino. O céu como um espelho,
com cujo reflexo, nós, narcisos astrônomos nos extasiamos.  

Por fim, ainda indicamos a analogia com o mito de Édipo, que na
tragédia de Sófocles, recebe sempre um alerta, daqueles que sabem de
seu destino, para que também não respeite o imperativo délfico.
(Episódio \textsc{i}, Tirésias; Episódio \textsc{iii}, Jocasta; Episódio
\textsc{iv}, o pastor
que o havia deixado dependurado). É como se Édipo e Narciso fossem
filhos da vingança de Apolo e do autoconhecimento do oráculo de
Delfos. A busca do autoconhecimento e o seu consequente perigo estão
no fundo dos dois mitos.

\subparagraph{Identidade e diferença no amor de si}

Seguindo essa mesma linha, e salientando a problemática do
autoconhecimento agora no nível erótico, podemos trabalhar a noção
de identidade e diferença no pseudo"-autoamor de Narciso. Toma"-se
normalmente Narciso por um apaixonado por si mesmo e esse seria o seu
erro: julga"-se que a moral da história nos ensina que o amor não deve
ser apenas por uma imagem de nós mesmos projetada sobre as outras
pessoas, como se o mito nos alertasse para não sermos tão
autocentrados, e nos abríssemos para encontrar o outro em sua
singularidade. Esse mito é usualmente interpretado como uma crítica
ao egoísmo e o próprio adjetivo, \emph{narcisista}, condena o homem
que não consegue se abrir para a diferença. Trata"-se, talvez, de uma
visão imbuída de cristianismo (amor ao próximo), que pode ser ainda
ornada com elementos talvez freudianos, em que repreendemos o
narcisista por não conseguir sair do amor próprio para usufruir um
amor mais dinâmico, saudável e criativo que seria o amor ao outro.
Concordando, de alguma forma, com a interpretação usual do mito,
Fraser (Pausanias, 1898. p. 159, volume \textsc{v}), em seu estudo sobre
Pausânias, comenta que talvez ele tenha surgido entre os círculos
gregos como forma de se criticar os jovens que não se entregam aos
amores de seus pretendentes.

No entanto, não é este problema “narcísico” que Plotino rechaça no
mito, pois nosso filósofo defende que o caminho ético autêntico é um
profundo mergulho em si mesmo. Narciso é, em verdade, pouco
\emph{narcísico}, pois ele não ama a si mesmo realmente, pois uma
crescente paixão por nossa profunda identidade deveria ser o caminho
de aprendizado sobre \emph{Eros}: Narciso é um mau amante de si. Na
educação erótica de Plotino, não se trata, de forma alguma, de
aprender a amar ao outro, mas sim de buscar a verdade do seu amor no
mais profundo do seu “eu”: o noético, em um primeiro momento, em um
segundo, o próprio Uno. 

Em um primeiro sentido, o noético pode ser considerado um “outro”,
pois enquanto estamos equivocados em relação a nós mesmos,
identificados ao âmbito sensível e obcecados pelo mundo material,
julgamos o noético como um outro --- como um possível Ulisses covarde e
esquecido do \emph{nostos} que se apegasse à segurança monótona da
Ilha de Calipso. No entanto, na medida em que se percebe que o nível
noético define a alma em sua identificação contemplativa, diríamos
que este é o verdadeiro “eu” de Narciso. Assim, amar verdadeiramente
a si mesmo é acordar das torpezas obsedantes das belezas sensíveis e
se voltar para o mundo noético, sua real pátria e identidade.

\subparagraph{Ambiguidade entre ser e possuir}

O momento em que Ovídio apresenta a autopercepção\footnote{ Pode"-se
ver em Narciso um traço dos deuses que se duplicam para criar o
mundo: Deus e Jesus (\emph{logos}) são um aspecto desse mito, e
também a autorreflexividade do primeiro motor imóvel de Aristóteles;
talvez até mesmo o \emph{Noūs} em Plotino, que é o pensamento e o
pensado ao mesmo tempo. Ver também a mística do espelho na época das
\emph{Beguines}, como por exemplo no título do livro de
Porrete,\emph{ The mirror of simple souls}, 1999.} do dolo em que
Narciso caiu é bem interessante para pensarmos alguns elementos do
neoplatonismo.

\begin{quote}
463 Eu sou esse. Percebo, e a minha imagem não me engana 464
Queimo de amor por mim, e eu inflamo a flama que carrego 465 O que
fazer? Suplicar ou esperar para ser suplicado? Que favor pedirei? 467
Aquilo que desejo está comigo; A posse me faz sem posse, 468 Oh, como
desejaria me afastar de meu corpo! 469 Um estranho pedido ao amante
esse, desejar longe o objeto amado.\footnote{ Tradução minha a partir de Castilho. 463 Iste ego sum;
sensi nec me mea fallit imago; 464 Vror amore mei, flammas moueoque
feroque. 465 Quid faciam? Roger anne rogem? Quid deinde rogabo? 467
Quod cupio mecum est; inopem me copia fecit. 468 O utinam a nostro
secedere corpore possem! 469 Votum in amante nouum, uellem quod
amamus abesset. A frase que aparece em \emph{Fastes} \textsc{v}, 226 também
de Ovídio é linda: “Narciso, triste por não ser diferente de si
mesmo”, “\emph{Infeliz quod non alter et alter eras.}”}
\end{quote}

Como já dissemos, essa é a única passagem, das versões antigas, em
que Narciso toma consciência de seu erro. Realmente, há elementos
muito interessantes para se pensar traços do neoplatonismo de Plotino
nesses versos de Ovídio. Há aqui um importante jogo entre possuir e
ser, muito caro a Plotino. Por um lado, (1) se utilizarmos o termo
\emph{possuir} de modo amplo, poderíamos dizer que Narciso possui tão
perfeitamente o amado que ele mesmo já é esse objeto amado. É como se
quiséssemos tornar sujeito e objeto implícitos na noção de posse tão
próximos que, no \emph{infinito} de sua proximidade, eles se
tornariam a mesma realidade. No limite, a posse de um sujeito sobre
um objeto se consuma na instanciação dos dois elementos em uma única
realidade. Aqui, a noção de possuir se alarga até a ideia de
identificar"-se. 

Por outro lado, (2) poderíamos utilizar o linguajar da
impossibilidade
de se possuir aquilo que já se é: a noção de posse remete a uma
separação entre sujeito e objeto que a torna contraditória à noção de
ser. Assim, enquanto a procura pela posse do objeto amado for
interpretada como um \emph{desejar algo outro}, \emph{buscar
apreender algo exterior}, será esse próprio procurar que impedirá o
\emph{encontro}. Nesse sentido, enquanto houver procura ainda não
haverá encontro, o próprio imperativo de procurar (externamente),
impede o verdadeiro encontro que é o reconhecimento de que já se é o
que se procura. Ao utilizar os dois sentidos do termo \emph{possuir}, o
linguajar do poema se torna ambíguo: por ser muito rico
(\emph{copia}), ele é extremamente pobre
(\emph{inópia}),\footnote{ Os dois termos latinos parecem derivar de
\emph{ops}: poder, força, habilidade, dinheiro, recursos.} isto
é, por já \emph{ser} tudo o que deseja, não pode \emph{possuir} o
que deseja.

Há uma ambiguidade muito interessante aqui que poderia ser mais
desenvolvida por Plotino, pois apesar de ser tão breve, acredito que
as seguintes ideias estão implícitas na sua interpretação do mito.
Para facilitar a compreensão do que se segue, temos que ter em mente
o seguinte esquema da alma em Plotino. Trata"-se de ressaltar que a
alma é e não é o nível inteligível: por um lado, na medida em que ela
se lembre de sua origem e se identifique com o noético, a alma
realizaria a compreensão de que um nível dela mesma nunca deixa o
inteligível, perdura eternamente em uma contemplação intendificatória
com o \emph{Noūs}. Por outro lado, na medida em que a alma esquece
sua origem e se volta para o mundo sensível, ela não é o
Intelecto.

O que torna Narciso pobre é ficar preso a uma forma de possuir o
objeto amado. Enquanto quiser possuir como quem possui um objeto
exterior, sua sede nunca será aplacada. O próprio imperativo de
tentar possuir aprofunda a ignorância do reflexo como reflexo, e
afasta a realização do desejo que perdura sempre desejante. O seu
querer é impossível, pois ele quer possuir como um objeto uma
realidade que ele mesmo já é. As belezas que ele vê como objeto são
em verdade reflexos de uma realidade interior e anterior, a noética.
Na medida em que ainda procura algo exterior para aplacar sua sede,
ele é completamente pobre, mas se Narciso aceitar o fato de que não
pode haver posse daquilo que já se é, e se ele conseguisse aceitar
esse outro modo da “posse”, uma posse que é identificação, uma
“posse” que é “ser”, e não ter, ele perceberia sua riqueza, seria
plenamente rico, e possuiria tudo que deseja. 

Em várias passagens das \emph{Enéadas}, Plotino procura fundamentar
a ideia de um tipo de posse que é ser: na discussão sobre a
felicidade,\footnote{ \textsc{i}, 4 [46], 4, 18--19. “O que é o bem para ele?
Ele mesmo já é o que ele tem”. Ver sobre este tratado de Plotino
MacGroarty, 2006.} por exemplo, ele trata a noção de sabedoria em
\emph{ato} com algo que não se pode possuir, mas que já se é.
Estamos falando da identificação da alma, em sua ascese de retorno,
com o nível noético. Sendo a sabedoria uma hipóstase identificada com
um nível da ascese da alma, esta só é sábia não se acaso
\emph{possuir} a sabedoria, mas na medida em que se identificar com
ela.  Assim, se Narciso desistisse de buscar essa beleza que ele
contempla no exterior, mas entrasse na verdadeira fonte de sua beleza
--- isto é, se ele se permitisse descobrir que ele mesmo sempre já foi
a fonte de seu objeto de desejo em um nível talvez inconsciente para
ele mesmo --- talvez ele pudesse, como Ulisses, reencontrar a pátria
amada, \emph{philen es patrida}. 
\bigskip

Assim, termino este artigo explicando a epígrafe “a luz no fim do
túnel é um espelho”. Que busca mostrar a ambiguidade da procura e do
encontro do objeto amado no âmbito da interpretação do mito de
Narciso por Plotino. Usualmente, o que se espera da luz no fim do
túnel é que ele seja uma saída da obscuridade claustrofóbica do
túnel, como em uma imagem do trágico e obscuro da vida. Já os
pessimistas, numa espécie de piada, poderiam afirmar que tal luz é um
trem vindo em nossa direção a nos atropelar. A ideia de colocar a luz
no fim do túnel como um espelho remete a ambiguidade que aqui
sublinhamos sobre a busca erótico"-filosófica em Plotino: buscamos
algo somente na medida em que ainda estamos ignorantes da fonte do
objeto que buscamos. Quando “encontrarmos” aquilo que buscamos, surge
a constatação de que nunca havíamos nos apartado do nosso objeto
amado, mas se encontrava tão próximo de nós mesmos que já o éramos: o
nível inteligível.

\section{Bibliografia}

\begin{description}\labelsep0ex\parsep0ex
\newcommand{\tit}[1]{\item[\textnormal{\textsc{\MakeTextLowercase{#1}}}]}
\newcommand{\titidem}{\item[\line(1,0){25}]}
\tit{HADOT}, Pierre. Le mythe de Narcisse et son interpretation par Plotin.
In \emph{Plotin, Porphyre. Études Néoplatoniciennes.} Paris: Les
Belles Lettres, 1999.

\titidem. \emph{O Véu de Ísis}. Loyola: Rio de Janeiro,
2007.

\tit{HAMILTON}, Victoria. \emph{Narcissus and Oedipus: the children of
psychoanalysis}. London: Routledge \& Kegan Paul, 1982.

\tit{HARDIE}, Philip (Ed.). \emph{The Cambridge companion to Ovid.}
Cambridge: Cambridge University Press, 2006. 

\tit{HOMER}. \emph{The Homeric Hymns}. Harvard: Harvard University Press,
1995. Loeb Classical Library.

\titidem. \emph{Ilíada.} Tradução Aroldo de Campos. São
Paulo: Arx, 2002.

\titidem. \emph{Ilíada}. Tradução Carlos Alberto Nunes.
Rio de Janeiro: Ediouro, 2000.

\titidem. \emph{Odisseia}\emph{.} Tradução Carlos
Alberto Nunes. Rio de Janeiro: Ediouro, 2000.

\tit{LAMBERTON}, Robert. \emph{Homer Theologian, Neoplatonist Allegorical
Reading and the Growth of the Epic Tradition.} Berkley: University of
California Press, 1989.

\tit{KNOESPEL}, K.J. \emph{Narcissus and the invention of personal
history.} New York: Garland, 1985.

\tit{KRISTEVA}, Julia. \emph{Histórias de Amor}. Madrid: Siglo Vientiuno
Edtores, 1987.

\tit{MACGROARTY}, Kieran. \emph{Plotinus, On Eudaimonia. A commentary on
the Ennead \textsc{i}.4.} Oxford: Oxford University Press, 2006

\tit{OVID}. \emph{Les métamorphoses}. Paris: Les Belles Lettres,
1925.

\titidem. \emph{As Metamorfoses}. Tr. Antônio Feliciano
de Castilho. Simões Editora: Rio de Janeiro, 1959. 

\titidem. \emph{The metamorphosis.} Mary
Innes,
London: Penguin, 1955.

\titidem. \emph{The metamorphosis.} Charles Martin, New
York: Norton \& Company, 2004.

\titidem. \emph{The metamorphosis.} A. D. Melville,
Oxford: Oxford University Press, 1996

\tit{PAUSANIAS}. \emph{Descrition of Greece}\emph{.}Trad. W. H. S.
Jones. Cambridge: Harvard University Press, 1918 (Loeb Classical
Library)

\titidem. \emph{\emph{Description of Greece.}} Trad.
James Fraser. \textsc{vi} volumes, London: Macmillan, 1898.

\tit{PHILOSTRATUS}. \emph{Imagines}; \textsc{callistus}. \emph{Descriptions}.
London: William Heineman, 1931. (Loeb Classical Library).

\tit{PLOTIN}. \emph{Ennéades.} Traduzido e estabelecido Émile Bréhier.
Paris: Les Belles Lettres, 1924.

\titidem. \emph{Traités}. 7 vols. Tradução sob a
direção
de Luc Brisson e Jean François Pradeaux. Paris: Flamarion, 2002.

\tit{PORRETE}, Marguerite.\emph{ The mirror of simple souls}.
Notre Dame: University of Notre Dame Press, 1999.

\tit{SPAAS}, L. (Ed.) \emph{Echoes of Narcissus}. Oxford: Berghahn Books,
2000.

\tit{TRAZASKOMA}. \emph{ et all} (Eds.) \emph{Anthology of Classical
Myth.} Cambridge: Hacket Publishing, 2004.

\tit{VINGE}, Louise, \emph{The Narcissus theme in Western European
Literature up to the Early 19th century}, Gleerups, 1967\emph{.}

\tit{WIESELER}, F. \emph{Narkissos}. Gottingen, 1856.
\end{description}

%%%%%%%%%%%%%%%%%%%%%%%%%%%%%%%%%%%%%%%%%%%%%%%%%%%%%%%%%%%%%%%%%%%%%%%%


\capitulo{O tratado de Porfírio \emph{Contra os cristãos}: \mbox{nova questão}}{José Maria Zamora}%
	{Univesidad Autónoma de Madrid}

\markboth{O tratado de Porfírio \emph{Contra os cristãos}}{José Maria Zamora}



%\footnote{ Este artigo é parte integrante do Projeto de Pesquisa
%“Éticas griegas y filosofia contemporânea” desenvolvido na Universidad
%Autónoma de Madrid. Tradução para o português de Josilene Simões
%Carvalho Bezerra e Maria da Conceição Rodrigues Palanca.}


Porfírio atacou os cristãos em seus escritos reconhecendo a ameaça
que o caráter intolerante da Igreja organizada representava para a
cultura politeísta do mundo Greco"-romano.  Os poucos fragmentos
conservados do tratado \textit{Contra cristãos}\footnote{ As citações
utilizadas do tratado de Porfírio correspondem a edição
de A. von Harnack,
\textit{Porphyrius. Gegen die Christen}{,
Berlin, 1916}{, cuja numeração dos fragmentos
seguimos.} Queremos agradecer aos colegas gaditanos, membros da equipe
de investigação com quem, sob a direção de Enrique, trabalhamos no
projeto: “A religión de Porfírio. Hacia una edición crítica con
traducción y comentario del \textit{Contra Christianos} de Porfirio”,
que resultou na edição espanhola{: E. Ramos Jurado}
{\textit{(et alii)}}{,}
{\textit{Porfirio de Tiro. Contra los
cristianos}}{. Recopilação de fragmentos, tradução,
introdução e notas, Cádiz, 2006.}  Porfirio foi considerado como o
precursor da crítica bíblica no século XIX. Foi o primeiro a mostrar
uma evidência histórica e filológica de que \textit{O livro de Daniel}
era de origem macabeia e também uma profecia póstuma. (Cf. Wilken,
\textit{The Christians as the Romans saw them}, New York, 1984, p.
1278--163; e Chr. Evangeliou, “Porphyry’s criticism of Christianity ant
the problem of Augustine’s platonism”, \textit{Dionysius}, 13, 1989,  p.
51--70).\par }  mostram a evidência da primeira crítica sistemática à
Igreja. O filósofo emprega sua formação filológica e filosófica em
defesa do helenismo revelando certa debilidade na estrutura lógica da
doutrina cristã, pelo que indiretamente e sem desejá-lo, ajudou aos
Padres da Igreja a articular o dogma ortodoxo de modo mais coerente
durante os séculos seguintes. Do mesmo modo, ajudou aos pagãos a verem
os perigos do dogmatismo e do irracionalismo da fé cristã. A Igreja
supunha uma clara ameaça aos olhos de Porfírio, não só desde um ponto
de vista filosófico e racional, mas também com respeito às antigas
tradições da simples piedade politeísta e tolerância religiosa.

\section{\textit{Contra Cristãos}: argumento e léxico}

Em um trabalho anterior\footnote{\textit{anthrōpos genomenos} : \textit{la divinité du Christ dans le} \textit{Contra
Christianos} de Porphyre{\textquotedbl}, París, setembro de 2009 (Actes
du Colloque sur le \textit{Contra Christianos} de Porphyre, S. Morlet
(ed.), [en prensa].} abordamos os critérios temáticos que levaram
Harnack a agrupar os fragmentos em cinco grandes blocos e consideramos
as razoáveis criticas que se formularam após essa classificação.
Benoit, por exemplo, apontou que, em algumas ocasiões, Harnack se
contradiz pelas escassas referencias aos livros concretos que se
conservam (sete ao todo). Deste modo, considerando independentes os
critérios seguidos para a edição dos fragmentos dos efeitos de um
estudo das argumentações empregadas por Porfírio, torna"-se
imprescindível uma organização temática do conteúdo, apoiada por
considerações sobre o léxico \textit{porfiriano }(com todas as
precauções e incertezas próprias de uma reconstrução que não se baseia
em citações literais e que está condicionada pelo intermediário,
geralmente cristão, que nos transmite a informação\footnote{ Deve"-se
utilizar, então, com extrema cautela trabalhos como os de J.M.
Demarolle (1972). }). Uma reflexão deste gênero nos remete a uma
organização dos argumentos anticristãos, mais lógica e coerente, que
obedece, estrita e exclusivamente, a razões conceituais, sem conexão, a
principio, com a localização dos \textit{fragmentos} na estrutura
original da obra. Propomos, por esta razão, para efeitos de nosso
estudo sobre o conteúdo, a divisão conceitual do tratado em três campos
temáticos:

\begin{enumerate}
\item Os textos sagrados como fonte de transmissão do cristianismo; 

\item A cristologia e o dogma cristão;

\item A comunidade cristã: práticas e modo de vida;
\end{enumerate}

A cada um destes campos correspondem diferentes argumentos, que
distribuímos tematicamente como segue:


\begin{enumerate}
\item Argumentos do primeiro campo;

\begin{enumerate}
\item Antigo e Novo Testamento: divergências e falsificações;

\item Fontes equivocadas, inexatidões históricas e contradições dos
autores do \textit{Novo testamento};

\item O uso impróprio da alegoria. 

\end{enumerate}

\item Argumentos do segundo campo: 

\begin{enumerate}
\item A divindade de Cristo. Politeísmo e monoteísmo;

\item O mistério cristão: Batismo, Eucaristia, Ressureição.
\end{enumerate}

\item Argumentos do terceiro campo:

  \begin{enumerate}
\item Ruptura com o tradicional e falta de integração cívica dos
cristãos;

\item A Igreja: hierarquia dos fiéis e das castas
sacerdotais. Lugares de culto.

\end{enumerate}
\end{enumerate}


A pesar do estado precário e suspeito em que as argumentações nos
chegaram, o conjunto das citações cristãs de Porfírio constitui uma
fonte de excepcional importância para formar um juízo sobre os temas do
polêmico pagão. Este \textit{corpus} é sua verdadeira Bíblia e, dado
nosso conhecimento fragmentário e secundário da obra, uma valiosa fonte
de informação. No que se refere ao léxico, se excluirmos estas citações
literais, que procedem do texto bíblico, o que resta é um raquítico
conjunto de termos gregos com acepção cristã, entre os quais se
sobressai o uso de substantivos\footnote{ \textsc{Demarolle} [1970; 199--200] 
faz notar que a utilização de substantivos, acima do uso de
verbos e adjetivos, revela uma orientação mais abstrata e estática da
polemica anticristã em Porfírio com relação aos tratados de Celso e
Juliano. }. Contudo, é impossível estabelecer, com mínima segurança, a
linha que separa o vocabulário porfiriano do léxico cristão daqueles
que nos transmitem seus argumentos. Com estas precauções, iniciamos
nosso percurso. 

\section{Antigo e Novo Testamento: divergências e falsificações}

Os evangelistas são “inventores, não historiadores” (fr. 60) da
doutrina de Cristo (\textit{epheretas oukh historas}).
Tal é a
proposição central da que parte Porfírio\footnote{ Daqui por diante nos
referiremos a Porfírio como o autor de todas as argumentações contidas
nos fragmentos. Admitimos tal procedimento por motivos de economia
expressiva, embora sejamos conscientes da insegurança da atribuição
porfiriana na maior parte dos casos.} para justificar sua crítica aos
textos bíblicos. Para o filósofo de Tiro, as sagradas escrituras são um
conjunto de falsidades, contradições e inexatidões, expressadas,
exageradamente, numa linguagem carregada de fraudes e exageros
tendenciosos. 

A literalidade, um tanto arbitrária e desproporcional de muitas
passagens, serve a Porfírio, de maneira bem cômoda, para sustentar sua
crítica. Por exemplo, com respeito à classificação de “mar” dado ao
lago Tiberíades por Mateus (fr. 70), declara abertamente que o
evangelista ultrapassou os limites da realidade, contando uma história
totalmente ridícula.
Nesta mesma
linha encontra"-se o relato de Marcos (fr. 70), quando se refere à
tempestade que Cristo conseguiu parar, fazendo acreditar que salvou
milagrosamente seus discípulos do abismo. Textos como estes nos
permitem qualificar o Evangelho de “mero artifício cênico” (\textit{skēnēn
sesophismenēn}).

Sobre o relato da Paixão classifica"-o de ficção incoerente (fr. 60),
pois os evangelistas não foram capazes de narrar “conforme a verdade”
(\textit{kata alētheian})
de que forma Jesus morreu e, portanto, não fizeram senão construir uma
farsa literária, daí se deduz que, nos demais casos não contaram nada que mereça
confiança. 


Os juízos de valor do polêmico Porfírio sobre o estilo e o conteúdo das
Sagradas Escrituras falam por si mesmos num léxico que não admite
contraposições a respeito. Sobre os apóstolos diz (fr. 26) que não são
mais que uns “pobres e rústicos pastores” (\textit{homines rusticani et
pauperes}), gente que se deixaria seduzir (fr. 34) pelo primeiro que
soubesse sacar partido de sua incompetência intelectual
(\textit{irrationabiliter quemlibet vocantem hominem sint secuti}). Das
posturas contrapostas por Pedro e Paulo, perante o difícil acesso dos
gentis ao cristianismo (frs. 28, 39, 42, 43 e 46), o filósofo conclui
que os dogmas propostos por ambos não são mais que autênticas ficções
(\textit{uolens et illi maculam erroris inuere}), produto da
falta de caráter de um e dos erros do outro. 

Além disso, há um uso não dissimulado de sarcasmo em expressões como as
referidas à “dialética” dos evangelhos:
``Que estupidez! Que cômica perda!” (fr. 68);
``Que história! Que bobagem! Que grande ridículo!” (fr.68);
``Que necessidade dessas palabras!” (fr. 80).

A opinião sobre o Evangelho apresenta"-se desmembrada em expressões como as
seguintes: “Semelhante prática – recursos para provocar o riso -- é própria das
representações teatrais”(fr.80); vai rir a gargalhadas como se estivesse no
teatro? Não vais lançar"-lhe improperios? Não vais assoviar com todas as suas
forças?” (fr. 76); ``partindo destas histórias infantis…” (fr. 70); ``deixemos estes asuntos
para que as crianças os julguem” (fr. 68); ``estas são histórias impróprias aos
homens, mas para mulheres decifradoras de sonhos…” (fr. 94); ``deixemos de lado
este assunto com o riso merecido” (fr. 87), entende"-se que este comentário foi
feito em revanche ao lado irracional do relato.

\section{Fontes equivocadas, inexatidões históricas e contradições dos autores
do \textit{Novo testamento}}

Cada um dos evangelistas recebe acusações que vão desde a mentira, até a
ignorância. Às acusações, digamos pessoais, se somam as contradições
das diferentes versões que apresentam os textos com respeito a um mesmo
relato e, finalmente as incongruências do texto por si mesmo.

Por exemplo, Mateus é acusado de “falsidade” quando se refere às duas
genealogias de Cristo (fr. 29). A ignorância e incultura dos
evangelistas se convertem em argumento quando repreende Marcos por
haver citado Malaquias em lugar de Isaías (fr. 52), fato que lhe serve
para dizer que “os evangelistas foram \textit{homens tão ignorantes},
não só de coisas profanas, mas também das escrituras divinas, que
atribuíram a um profeta diferente o testemunho que se acha escrito em
outra parte”.

As contradições no relato a respeito de um mesmo fato são inumeráveis e
repetidas: no fr. 53 se fala das diferentes versões sobre a morte de
Judas\footnote{ At 1.18 e Mt 18.5.}; no
fr. 61, dos desacordos sobre o relato da Paixão de Cristo\footnote{
Jo 19.33--34, quem foi o único que disse que um soldado
transpassou as costas de Jesus com uma lança e que dele saiu sangue e
água.}; no fr. 68 se faz referência ao diferente número de porcos que
se afogam no lago Tiberíades, segundo os testemunhos de
Marcos\footnote{ Mc 5.1--20.} e Mateus\footnote{
Mt 8.28--34.}, para sinalizar a própria inverossimilhança
dos fatos expostos, dado que uma manada de dois mil porcos na Judeia ia
contra dois argumentos óbvios: primeiro, que na Judeia o porco era
considerado um animal impuro e indesejado; e segundo, que as dimensões
reais do citado lago não admitiriam tal número de animais.

Numerosos são também os casos de contradições internas: no fr. 50, a
propósito da assistência de Jesus à festa dos Tabernáculos; no fr. 62,
manifesta"-se como Jesus rechaça a Paixão, apesar de ter declarado “não
tenhais medo dos que matam”\footnote{
Mt 10.28.}; no fr. 61, o filósofo reúne, a propósito,
estas duas declarações de Cristo: “Os pobres sempre estarão a nosso
lado, mas a mim não me tereis sempre”\footnote{ 
Mt 24.36.} e em outra parte, “estarei com
vós até o fim dos tempos”\footnote{
Mt 28.20.}. Os fragmentos 79, 89, 90 e 91, estão
carregados de incoerências argumentativas, tais como: sobrevivência do
Universo apesar do dito em Mateus 24.14 (fr. 89); morte vergonhosa de
Pedro, apesar dos privilégios supostamente concedidos a sua pessoa (fr.
79); decapitação de Paulo, apesar do assegurado em \textit{Feitos}
18.19--20 (fr. 90); e ausência de Cristo, cuja segunda vinda se
anunciava “desde trezentos anos e mais” (fr. 91).

No conjunto, a crítica dirigida por Porfírio aos aspectos doutrinais do
cristianismo estava baseada na análise rigorosa dos textos sagrados,
com toda a inclinação que quisesse\footnote{ Às vezes muito óbvia, o
que leva o intérprete a uma leitura forçada do texto evangélico.
Contudo, a rudez de algumas acusações atribuídas ao autor pagão lança
sérias dúvidas sobre a exatidão do autor que transmite o suposto
argumento “porfiriano”. Nota"-se, por exemplo, a chamada de
“incompetência” dirigida contra Jesus no fr. 35.}. 
Tratava"-se, acima de tudo, de evidenciar fatos
como os seguintes: a fragilidade do cristianismo como doutrina recente,
e, portanto, desprovida da autoridade nascida de uma longa tradição
religiosa (frs. 4, 15, 48)\footnote{ Cf. G. Rinaldi (1998) 392--396.}; a
base irracional e carente de lógica da doutrina cristã, infestada de
profecias e argumentos insustentáveis como tais (frs. 15, 19, 34), a
ignorancia e vulgaridade das pessoas a quem se destinavam as doutrinas
cristãs (frs. 20, 25, 26, 27, 45, 51) e, no conjunto, a falta de
credibilidade, baseada nas contradições, conflitos, e discordâncias,
que se deixavam mostrar, entre os próprios autores dos textos sagrados,
assim como entre os membros da comunidade cristã (frs. 28, 39, 42, 43,
44, 46)\footnote{ Cf. G. Rinaldi (1998) 489--493.}.

O maior número de fontes errôneas e de inexatidões históricas das Sagradas
Escrituras corresponde aos fragmentos que Harnack incluía em seu
segundo bloco temático: as críticas ao Antigo Testamento (frs. 16, 22,
24, 30, 37, 110, 111). Especialmente destacáveis pelo peso de suas
argumentações são o 22 e 24, provenientes de Eusébio de Cesareia (o fr.
22, através da tradução de Jerônimo) e o 30, de Jerônimo.

No fr. 24, o filósofo sustenta que Orígenes teria sido capaz de ocultar
que as palavras de Moisés e de outros escritores do Antigo Testamento
eram na realidade mistérios à espera da chegada do cristianismo, para
daí terem uma completa explicação. No fr. 22 se afirma que Semíramis, a
famosa rainha de Assíria, viveu setecentos anos depois de Moisés e 150
antes de Ínaco, ou seja, que Moisés viveu uns setecentos e cinquenta
anos antes da Guerra de Troia. E no fr. 30, procedente da extensa
refutação de Porfírio que é o \textit{Comentário a Daniel} de Jerônimo,
nosso autor afirma brilhantemente que Daniel, longe de ser um profeta
dos tempos do cativeiro judeu na Babilônia, na realidade encobre uma
obra composta por volta da época do reino do monarca selêucida Antíoco
Epífanes. Todas as argumentações porfirianas do fragmento pretendem
demonstrar que tudo isso é, na realidade, uma intencionada deturpação
cristã.

Para Porfírio a importância dessas críticas consistia em evidenciar a
coincidência entre as religiões antigas, gregas ou bárbaras, diante das
novidades introduzidas pelos cristãos. O juízo favorável relativo à
religião hebreia, que aparece em \textit{Contra Cristãos}, parece
destinado a evidenciar o descrédito que Porfírio tinha pela leitura que
a formação de cristãos através dos textos sagrados. Mesmo assim, apesar
de reconhecer a antiguidade da doutrina de Moisés, no que se refere à
Bíblia, Porfírio mantém uma postura cética. No fr. 67\footnote{ Cf. G.
Rinaldi (1998) 64; 375.} se afirma que as escrituras hebraicas são, na
realidade, obras pseudo"-epigráficas, já que os livros de Moisés foram
destruídos no incêndio do templo de Jerusalém e reconstruídos por
Esdras, e seus colaboradores, mil cento e cinquenta anos depois da
morte de Moisés. E o \textit{Comentário a Daniel} de Jerônimo permite
testemunhar uma importante atividade crítica de Porfírio a respeito do
\textit{Livro de Daniel}, do qual o filósofo pagão logrou desvendar o
caráter profético \textit{post eventum }(escrito, além dos outros, em
grego), rechaçando a interpretação do texto baseada na cristologia e
defendendo a leitura de uma epopeia da resistência judia contra Antíoco
\textsc{iv} Epífanes (fr. 30).

\section{A divindade de Cristo: Politeísmo e monoteísmo}

Neste ponto, a “estratégia” de Porfírio consiste em revalorizar tudo o
que se possa encontrar de positivo na tradição judaica para destacar a
torpeza e irracionalidade de certas pretensões cristãs. A tese
fundamental dos cristãos, segundo a qual Jesus é o \textit{Logos}
divino, o filho de Deus, é uma incoerência inadmissível. O irracional
da doutrina cristã consiste em ter identificado um indivíduo único,
pessoal e corporal com o princípio divino. Uma afirmação assim leva a
identificar Deus como algo passivo e irracional, coisa inadmissível
para um neoplatônico como Porfírio. A doutrina da encarnação do Verbo
(\textit{anthōpos genomenos} fr. 105) implica que o divino – por si mesmo puro e
santo -- veja"-se
submetido à transformação e, posto que a condição de Deus está acima de
qualquer outra realidade, essa transformação só pode ser entendida como
uma diminuição, a que é contraditória e ilógica. No fragmento 112,
Porfírio expõe a questão sob a perspectiva do seguinte argumento com
raiz estóica: se o filho de Deus é um \textit{logos}, “proferido”
(\textit{prophorikos})
ou é “interior” (\textit{endiathetos}); 
e se não é nenhuma das duas coisas, também não é um \textit{logos}.
Para começar, então, Porfírio nega a natureza divina de Jesus.

Por outro lado, sua opinião sobre Jesus como pessoa não falha em
severidade, e inclusive dureza, em certas passagens: sua presença na
festa dos Tabernáculos, na que havia anunciado previamente que não
participaria, valeu"-lhe a acusação de homem “inconstante” e “mutável”
(\textit{inconstantiae ac mutationis})\footnote{ Fr. 50. Cf.
\textit{etiam} fr. 35.}, ainda que, outras vezes, está disposto a
reconhecer nele virtudes de um homem superior (fr. 72). Entretanto,
sobretudo, o filósofo se mostra contra a representação de Jesus segundo
seus discípulos (fr. 72). Com efeito, a distinção entre o Jesus
histórico e o Cristo, conforme a fé dos cristãos desempenhará um papel
importante na hora de argumentar sobre temas relevantes da apologia
cristã, como é o da Ressurreição, conforme veremos no parágrafo
seguinte.

No que se refere ao caráter heróico da figura de Jesus, o tema da Paixão
é o melhor exemplo da contradição de sua identidade como Deus. No fr.
65, Porfírio argumenta que um “varão sábio e divino”
 não se pode
deixar ultrajar como se fosse um homem vulgar,
e que para um homem singular como Jesus, numa situação tão humilhante,
se espera uma atuação, no mínimo, comparável à do célebre Apolônio de
Tiana. A imagem dolorosa e humilhada de Jesus lhe parece medíocre e
desproporcionada, considerando o que se espera de um “homem divino”,
um \textit{theios anēr}\footnote{
Cf. L. Bieler (1967).}: vítima da fatalidade, mas com prestígio, e
sublime inclusive na hora da calamidade. Daí que critique a aparição
\textit{post mortem }de Jesus perante uma pobre mulher como Maria
Madalena, em lugar de aparecer para Pilatos, Herodes ou o próprio rei
dos judeus; ou que critique uma ascensão que poderia ter realizado
perante todos, como corresponde a um homem divino (fr. 1).

Relacionado com a identidade deste Deus, se encontra a questão da
pretendida “monarquia” que, seguindo a argumentação de Porfírio (fr.
98), não é mais que um politeísmo disfarçado: realmente não há um só
Deus no cristianismo. Há mais deuses, e de uma mesma natureza. Deus,
disse Porfírio, não pode ser monarca se não reina sobre outros seres de
sua mesma natureza e de sua espécie, ou seja, sobre outros deuses, isto
sim, é o adequado à grandeza divina e à dignidade celestial. Além
disso, o modo que inclusive os cristãos definem os anjos, “impassíveis,
imortais e de natureza incorruptível”
levaria também  à conclusão de que são deuses; sua essência é igualmente
divina: a resposta de Cristo aos caduceus permite perceber isto
claramente (fr. 99). 

Na passagem que justifica o culto aos ídolos, as imagens do paganismo
(fr. 99), alegando que ninguém pode confundir a madeira, da qual são
feitos os ídolos, com a própria natureza divina representada na imagem,
o filósofo de Tiro expõe um novo perfil sobre este ser divino, um tanto
mais incoerente, quando diz que se algum grego é tão simples a ponto de
acreditar que os deuses habitam no interior das estátuas, “muito mais
saudável tem, entretanto, seu juízo quando acreditam que a divindade
penetrou no ventre da Virgem Maria, se converteu em feto e, uma vez
parida foi envolta em panos”,
“cheia do sangue da placenta, de bílis e de outras coisas ainda mais
insólitas que estas” (fr. 100).  

Da mesma maneira lhe parece inaceitável a questão da “encarnação”. Como
é que Cristo chegou a revelar"-se tão tarde para os homens, privando a
humanidade por tantos séculos de sua bondade? Por que permitiu que se
perdessem tantas almas? (fr. 105) E, certamente, como conjugar com uma
natureza divina, “impassível”, a desse “Filho de Deus” que morre na cruz?

Independente dos problemas dogmáticos que a figura de Jesus (“Cristo” em
outras ocasiões) promove no pensador pagão, a terminologia com que se
refere a ele não suscita nenhum equívoco e se mostra de acordo com a
usada pelos cristãos. Os apelativos \textit{ho Iēsous} (fr.
77) e \textit{ho Khristos}
(fr.
70) fazem referência indistintamente à figura de Cristo. Seus
seguidores são definidos mediante os términos \textit{hoi mathētai}
(fr. 70) e \textit{hoi gnōrimoi} (fr.
66), se bem que o primeiro termo é mais frequente que o segundo, assim
como nos mesmos escritores cristãos. Com relação ao complicado tema da
virgindade, e a propósito da presunção com que alguns alardeiam sobre
ele, o autor se refere – uma só vez -- à mãe de Cristo com este apelativo
\textit{tēi teksamenēi ton Iēsoun}, 
“a que pariu Jesus” (fr. 86). 

Jesus é frequentemente apresentado como \textit{ho didaskalos}
(fr. 72) e \textit{ho Kyrios}  (fr. 99). Para situar a figura de Cristo na Trindade se usam duas
expressões, a primeira muito mais frequente que a segunda:
\textit{pais Theou}
(frs. 66 e 75) e \textit{ho Hyios} (fr. 93).

Entretanto, notamos que alguns termos referentes a certos aspectos da
vida de Jesus, apesar de traduzir literalmente o vocabulário cristão,
são utilizados num contexto que deixa transparecer certa distância ou
estranhamento, e inclusive às vezes se nota uma clara ironia com
respeito às próprias realidades argumentadas. É o caso de \textit{agōnia}
(fr. 60), \textit{pathos} (fr. 60), \textit{to pathein} (fr.62). 
Quando se fala da mensagem de Jesus, a ironia é igualmente
manifestada: assim nos disse (fr. 69) que Jesus não podia estar
ensinando o “cânone” da verdade, uma vez que permitia que seus fiéis chegassem a
identificar a pobreza com a salvação e a riqueza com o pecado. Este
\textit{kanōn} faz, portanto,
referencia a uma duvidosa “regra” de conduta. Em outros casos, o
sentido em que se emprega um vocábulo determinado é claramente
pejorativo: assim o verbo
\textit{thaumatopoiein}
(fr. 80) aplicado à capacidade que Jesus tem em “fazer
milagres”, remete ao significado literal helênico de “fazer
\textit{artes} com as mãos”, “fazer magia”.

\section{O mistério cristão: Batismo, Eucaristia, Ressurreição} 

No fr. 97 o filósofo julga a prática cristã do batismo sob esta
consideração: “A partir de agora, quem vai deixar de se atrever a todos
os males -- pronunciáveis ou impronunciáveis -- e de fazer o que não é
tolerável -- que se diga nem se faça --, sabendo que obterá
\textit{absolvição}
(\textit{apolysin})
por tantas ações infames, apenas por crer, \textit{batizar"-se}
(\textit{baptisamenos})
e esperar depois conseguir o \textit{perdão}
(\textit{sungnōmēs})
daquele que julgará vivos e mortos? Estas palavras convidam aqueles que
a escutam ao \textit{pecado}
(\textit{hamartanein});
estas palavras ensinam a cometer atos ilícitos; estas palavras são
capazes de banir os ensinamentos da Lei e de fazer com que a justiça
careça de todo vigor contra a injustiça; estas palavras introduzem no
mundo um estado de anarquia e ensinam a não temer em absoluto a
\textit{impiedade}
(\textit{asebeian})
\textit{desde o momento em que o homem, somente por se batizar, se
desfaz do acúmulo de incontáveis crimes”}.

A valorização é suficientemente explícita por si mesma. Porém, cabe
fazer certas apreciações relacionadas ao uso do léxico empregado nesta
passagem. Não se empregam o termo
\textit{baptisma}, mas
dois verbos (fr. 97):
“lavar"-se” (\textit{apolouesthai})
e
“batizar"-se” (\textit{baptizesthai}).
Efetivamente, o uso de “lavar"-se” como sinônimo de “batizar"-se” implica
um juízo moral explicitamente negativo, dado que em ambos os casos a
citação remete a um homem que, carregado de falhas morais, com um
simples ritual higiênico, livra"-se \textit{ipso facto} de suas
injustiças. Com tais afirmações, evidentemente, o batismo não é mais
que “uma astuta invenção” da doutrina cristã
(\textit{kompson, plasma}). Isso sim, nesta
crítica da eficácia do batismo como rito purificador, o autor das
argumentações, refere"-se, com certa fidelidade ao original, às mostras
de um preciso conhecimento dos termos “técnicos” relativos à penitencia
cristã:
\textit{sungnōmē, hamartia, hamartanein} e \textit{apolysis} (frs.
97, 78, 97 e 97, respectivamente).

O rito cristão da eucaristia produz em nosso autor uma especial
repugnância\footnote{ Esta linha de pensamento já é antiga no
paganismo: cf. Iuv. XV}. Para ele a comunhão é um ato de canibalismo,
evocando"-lhe, como exemplo mitológico, o banquete de Tiestes (fr. 72).
A passagem é interessante, ademais, por vezes, o filósofo aproveitaria
para unificar as diferentes cosmologias, teologias, etc. de qualquer
povo, incluindo o bárbaro, contrapondo"-os aos cristãos – que o filósofo
entendia como irracional e provocador em sua relação com a tradição da
cultura herdada, segundo se diz no próprio texto: “Muitos outros
autores inventaram tramas peregrinos, mas nenhum deles idealizou um
argumento trágico mais peregrino que este, nem historiador, nem
filósofo, nem bárbaro, nem nenhum dos antigos helênicos.” Porfírio
conclui alegando que Mateus, Marcos e Lucas omitem expressamente este
argumento, porque pensam que “não é apropriado, senão peregrino,
deselegante e tremendamente distanciado da vida civilizada”.
O emprego, efetivamente, da expressão
“carne e sangue” (\textit{sarks kai aima}), não deixa espaço para dúvidas sobre o pensamento de
Porfírio sobre o ritual cristão. “Isto, realmente, não é que seja, em
verdade, bestial e extravagante, é muito mais extravagante que qualquer
extravagância e mais bestial que qualquer comportamento bestial”.

Em terceiro lugar, para um neoplatônico como Porfírio a doutrina da
ressurreição é um mau plágio da tese pagã da reencarnação. A
ressurreição é sintoma da vulgaridade e da impureza dos cristãos,
totalmente ligados à matéria e incapazes de ascender a uma autentica
compreensão da realidade inteligível. A ordem que está na origem desta
comunidade, se expressa também neste dogma, que transgride toda a
hierarquia relativa a um valor natural e razoável, situada entre o
sensível e o inteligível (frs. 97, 78, 97 e 97, respectivamente).

Entre os fragmentos do \textit{Contra Cristãos}, Harnack incluiu dois
testemunhos relativos à ressurreição dos mortos, extraídos das cartas
de Agostinho e de Macário Magnes: são os fragmentos 3 e 102. Em todo
caso, nestes fragmentos seu autor argumenta utilizando os mesmos
princípios que Porfírio expõe em seu \textit{De regresu animae},
comentados extensamente por Agustin\footnote{ Agustin, referindo"-se
precisamente à ressurreição, cita uma afirmação que aparece uma e outra
vez nesta obra: \textit{omne corpus esse fugiendum, ut anima possit
beata permanere cum }\textit{deo} (\textit{Porph. }De regressu
animae\textit{ }297 F Sm. [=Aug. \textit{civ. }X 29.55--75]). Cf.
\textit{etiam}. Porph. \textit{Marc}. 8; 34.}. Nota"-se, claramente, que
o autor de tais argumentações seja Porfírio, era evidente que para o
filósofo de Tiro, a doutrina da ressurreição contradizia com a natureza
e o destino da alma. Seguramente, em \textit{De regresu animae,}
Porfírio interpreta a entrada da alma no corpo como um ato negativo à
matéria, que deve terminar por devolver à alma sua capacidade para
fugir do mal e voltar ao Pai\footnote{ Porph. \textit{De regressu
animae} 298 F Sm (=Aug. \textit{civ.} X 30).}.

Conhecendo tão bem como conhecia a força desta doutrina, base da fé
cristã\footnote{ Cf. Tert. De resurrectione carnis 1: fiducia
Christianorum resurrectio mortuorum.}, o filósofo pagão se preocupa em
resumir no fr. 102 todas as objeções tradicionalmente existentes. Do
conjunto das argumentações expostas, destacam"-se duas ideias
principais: uma, a contradição com a ideia de um Deus racional como
princípio do universo, princípio que para a mentalidade de um
neoplatônico não pode ser interrompido pela vontade inconsequente desse
mesmo demiurgo; outra é, consequentemente, a valorização moral negativa
que implica a imagem desse criador, quando permite a destruição de sua
própria obra e de suas criaturas.

Porfírio diz com respeito à ressurreição dos mortos
(\textit{tēs anastaseōs tōn nekrōn}): “O que Deus decidiu
uma só vez e vem observando"-se desde sempre, convém que assim seja para
sempre e que não receba reproches do demiurgo, nem que seja destruído
como se procedesse do homem e fosse obra mortal erguida por um mortal.
Deste modo, seria ilógico que a ressurreição seguisse a destruição do
todo.” E mais adiante: “E tampouco Deus poderia fazer o mal, jamais,
ainda quisesse, e, por ser bom, não poderia faltar a sua natureza…
Considerem, também, quão absurdo seria que o demiurgo permitisse que
desaparecesse o céu, o mais divino que alguém pode conceber no tocante
à beleza, que caíssem os astros e que se destruísse a Terra, e que, por
outro lado, ressuscitassem os corpos putrefatos e corrompidos dos
homens” (fr. 102).

Estes são os argumentos. A respeito do vocabulário, é importante
observar os seguintes fatos: com o termo \textit{epidēmia}
(fr. 96) o filósofo faz alusão à vinda de Cristo, nada mais. Na
terminologia pagã, os gregos usavam este vocábulo para as Epifanias dos
deuses, com um significado, portanto, similar ao que encontramos no
polêmico filósofo\footnote{ É o caso, por exemplo, de Orígenes (Cf. Or.
\textit{Fr. in Lam.} 28).}. Ao mistério da encarnação – ao
menos segundo o fr. 105 – dedicava"-lhe a conhecida expressão
\textit{anthrōpos genomenos}.
A segunda e esperada chegada de Cristo para o Juízo Final está,
entretanto, traduzida num termo estóico para o
restabelecimento universal,
\textit{apokatastasis},
(fr. 108). ``E
por esta \textit{apokatastasis} – dizem alguns – os cristãos imaginam a
ressurreição”, junto ao menos comprometido
\textit{eleusis} (fr. 90) usado,
referindo"-se à incerteza da “vinda que se espera”.

Em geral, o vocabulário referente à dogmática cristã não aporta grande
originalidade e implica poucas novidades no uso de termos com sentido
cristão. O filósofo parece conhecer com exatidão- como nos revela seu
uso em o \textit{Contra Cristãos} – o significado cristão de palavras
como
\textit{angelos} (fr.
99), \textit{dogma} (fr. 97),
\textit{ho dikaios} (fr.
96),
\textit{eirēnē} (fr.
55),
\textit{skandalon} (fr.
76), \textit{telos} (fr. 89),
\textit{diabolos} (frs.
63 e 64)\footnote{ Cf. fr. 76, onde aparece com o mesmo significado o
termo \textit{Satan}.} e \textit{ktisis} (fr.
68)\footnote{ Com o significado de “universo”, “queriam trastornar os
elementos confundindo"-os e destruir com sua ação perniciosa toda a
criação”, disse"-nos com referência aos demônios.}.
\textit{Prosdokō} (fr.
90)\footnote{…“a ressurreição e a vinda que esperam”, em referência à incerta
esperança dos seguidores de Cristo.} e \textit{kalō}
(fr. 96)\footnote{ …“o
que não foi chamado e não necessita da cura que oferecem os cristãos…”,
em referência à vocação cristã.} traduzem perfeitamente os
matizes da fé cristã.

Além de
\textit{ho ktisas} (fr.
82), \textit{ho kritēs} (fr.
92) e de \textit{ouranou kai gēs patēr} (fr.
93) para referir"-se a Deus, o \textit{corpus} utiliza também
qualificativos tomados da tradição cristã, tais como
\textit{apathēs} (frs.
64 e
99), \textit{aphthartos} (fr.
99)\footnote{ Embora Porfirio o aplica a divindades ou seres
intermediários, enquanto que na tradição cristã se aplica, como é
obvio, a Deus e a seu Filho, Jesus Cristo.} e
\textit{lypoumenos} (frs.
71 e 87. Finalmente transcrevem literalmente o significado
cristão de expressões como as seguintes:
\textit{genesis tou kosmou}
(fr. 102)
\textit{kharismata ke tou ouranou}
(fr. 93), \textit{plousioi kai penētes} (fr.
69), \textit{anastasis tōn nekrōn} (fr.
88), \textit{krinai tous zōntas te kai nekrous} (fr.
97), \textit{zōē aiōnos} (fr.
72), \textit{pneuma hagion} (fr.
86), \textit{kharis kai pistis} (fr.
90), \textit{basileia tōn ouranōn} (fr.
69).

\section{Ruptura com a tradição e a falta de integração cívica dos
cristãos}

Dos fragmentos obtêm"-se respostas interessantes a respeito da comunidade
cristã pertencente ao último terço do século \textsc{iii}. No fr.~62 encontramos
uma excelente pista para termos uma ideia sobre o princípio das
perseguições cristãs, quando se afirma que Cristo não fez nada para
evitar os piores castigos àqueles que o seguiam. Já que, se houvesse
sabido impor"-se perante todos – o Senado, o povo romano e inclusive as
autoridades judias – para que acreditassem nele, haveria impedido que
“condenassem por sacrilégio seus fiéis à morte mediante decreto
público. 

No fr. 90 se faz alusão aos sofrimentos e à morte de vastíssimo número
de seguidores de Pedro e Paulo. O número dos seguidores de Cristo era,
efetivamente, muito elevado no momento das perseguições de Décio, por
volta do ano 250. Porfírio pôde contemplar o desastroso e cruel
desenlace na comunidade cristã, só que, como bem fez notar
Labriolle\footnote{ \textsc{Labriolle} (1948) 285--286.}, o filósofo
serviu"-se deste tremendo drama, não para chamar atenção à piedade, mas
para abater mais uma vez seus adversários: no fr. 15 aparece um autor
dentro de todos os conformes com os rigores oficiais aplicados aos
seguidores de Cristo: “A que castigo não seria justo entregar
a\textit{os que se desterraram das tradições dos antepassados…?”}.

A desconfiança do filósofo ante esta nova religião tem a ver diretamente
com a dimensão pública do culto exercido pela comunidade cristã e
consequentemente, com a importância política desta prática
religiosa\footnote{ Nisso Porfirio compartilhava totalmente seu
critério com Celso.}: nisso Porfírio compartia totalmente com Celso. O
abandono dos cristãos aos costumes dos seus antepassados se
manifestava, por exemplo, no rechaço ao culto das imagens, rechaço que
por sua vez deixava assomar o caráter um tanto grosseiro do povo
cristão no que se refere ao verdadeiro conhecimento sobre o politeísmo.
Para Porfírio ninguém em são juízo poderia pensar que as múltiples
imagens dos deuses não são outra coisa que oferendas votivas, simples
representações de um único Deus; mas o monoteísmo hebreu e cristão não
soube ver a relação de todas as realidades com esta realidade superior,
que era Deus, inclusive chegou a identificar este transcendente com um
ser mortal e determinado. 

Estas críticas à perigosa falta de continuidade da tradição na nova
comunidade cristã encontram"-se em \textit{Contra Cristãos}, ali onde
Eusébio pretende explicar quem são os cristãos, e qual é seu modo de
vida, considerando este como uma novidade,
já que nem observava os costumes pagãos, nem o dos povos bárbaros. Os
cristãos, já o dissemos, foram acusados de abandonar os costumes de
seus antepassados
para acolherem a dos judeus, ainda que nem mesmo mantiveram"-se fiéis a
esta, senão que fundaram um novo culto (fr. 15).

No fr. 24, procedente de Eusébio, o tema se projeta sob outro
prisma. Critica"-se Orígenes, argumentando que enquanto Amônio, cristão
de nascimento, havia adaptado"-se, graças à filosofia, a “uma conduta
conforme a lei”,
este abandonou sua formação de homem grego para “arruinar"-se na audácia
bárbara do cristianismo'',
“vivendo à maneira cristã e contra a lei”.

O cristianismo é uma religião
recente e, portanto, desprovista da autoridade que possui as religiões
antigas; assim mesmo, se funda numa aceitação irracional e
absolutamente acrítica de doutrinas que não foram argumentadas; por
isso, se dirige a um público ignorante, supersticioso;  por último,
apresenta contradições, conflitos que opõem tanto os autores dos textos
sagrados como os membros da comunidade.

O pluralismo religioso era necessário, já que
expressava a variedade das formas determinadas nas que o único
princípio de toda realidade podia manifestar"-se e ser reconhecido. Com
relação a este pluralismo, a filosofia atuava como princípio de ordem e
critério de verdade, o mesmo que, no plano político, eram garantidas pelas 
instituções do império com coesão e unidade do organograma do Estado. O
cristianismo, muito mais que o judaísmo, representava uma ameaça para
esta ordem, dado que pretendia substitituir  outra sabedoria religiosa
e reivindicava a exclusividade no culto de seu Deus, recusando
integrar"-se a uma unidade mais ampla.
\bigskip

\hfill Trad. de Josilene S.C.~Bezerra e Maria da Conceição R.~Palanca


\capitulo{A ideia de criação na Teologia do Pseudo"-Aristóteles: um
estudo comparativo entre as teses das \emph{Enéadas} e a concepção do Plotino
árabe}{Jan G.J. ter Reegen}{Uni.~Estadual do Ceará--uece}

\markboth{A ideia de criação na Teologia do Pseudo"-Aristóteles}{j.g.j. ter Reegen}


\section{Introdução}

A \emph{Teologia do Pseudo"-Aristóteles}\footnote{ A partir de
agora indicada como: \emph{TdA.}} pode ser
caracterizada como uma paráfrase das\emph{Enéadas} \textsc{iv}--\textsc{vi}, de Plotino,
elaborada no mundo árabe e que apresenta como assunto principal
a Alma, a Inteligência, o Ser e seus gêneros. A obra visa, sem
sombra de dúvidas, oferecer um referencial teológico"-filosófico
aos estudiosos da iniciante filosofia árabe, mas também promover
a língua árabe como fator de união e unificação do novel
império, fundado a partir das primeiras investidas de Mohamed.
Esta paráfrase é, junto com o \emph{Livro das Causas}; a
\emph{Carta sobre a divina ciência} e as \emph{Palavras do
Sábio Grego}; o que sobrou de uma grande coletânea de
textos que contém as \emph{Enéadas} de Plotino; \emph{Elementos Teológicos} de Proclo 
e \emph{Comentários} de Alexandre de Afrodísia, este conjunto uma
complementação da \emph{Metafísica} de Aristóteles. A 
\emph{TdA} deve ter sido composta ao redor de
800--900/\,200--300H, o que se deduz pela citação feita da obra por
Al"-farabi (870--950/\,259--339H\footnote{2\emph{
    Alfarabis Philosophische Abhandlunge.} Aus dem Arabischen übersetzt
    von{ Dr. Fr. \textsc{dieterici}}. Leiden: E.J. Brill, 1892, nº 23:
    \emph{Wer die Ausprüche des Aristoteles über die Gottherrschaft in seinem
    “Teologia” betitelten Buche betrachtet [\ldots{}]} como também nº 36:
    \emph{Nur finden wir aber, das Aristóteles in seinem Buch von der
    Gottesherschaft, der “Theologia” betitelt ist, geistige Formen annimt und
    erklärt das diesselben in der götlichen Welt existieren”.}
    \textsc{vallat},  em \emph{Farabi et l'École
    d'Alexandrie. Des Premisses de la Connaissance à la Philosophie Politique.}
    Paris: Vrin, 2004 --- apresenta um comentário sobre a presença da TdA em
    Alfarabi, especialmente na seção 6, \emph{Ce qui Farabi lisait dans la
    Theologie du Pseudo"-Aristote} e na \emph{Conclusion et Hipothèses: Farabi,
    Platon et le Néoplatonisme de la Theologie}  Trata, sobretudo, da presença
    do pensamento da \emph{TdA} na “Concordância entre Platão e Aristóteles”,
    nos números  56 e 75\emph{.}})
e a presença de vestígios de seus
textos no famoso círculo de Al"-Kindi.

O grande objetivo do livro, expresso no seu próprio texto, é
apresentar a doutrina primeira a respeito da divindade, a
explicação a seu respeito que Ela é a Causa Primeira.\footnote{
\emph{TdA} Tratado Primeiro, Prólogo\emph{.}} Este objetivo
é desenvolvido em dez livros\footnote{ A versão Latina conta
14 livros, uma vez que chama “livros” algumas divisões e partes
especiais dos tratados do texto  original.} que tentam explicar
um elenco de 141 questões que se apresenta como uma
explicitação\footnote{ Hesitei aqui na escolha da palavra: será
que seria melhor usar \emph{explicação}? Optei pelo termo
usado.} da hierarquia dos seres que procedem do primeiro princípio, as
relações e semelhanças entre o mundo superior e inferior, as
características do mundo inteligível, a alma universal, as
estrelas, o mundo sublunar sensível, a descida da alma para ele
com todas as consequências e implicações envolvidas.

O texto, então, deve ser considerado uma paráfrase, mas, além
disto, constata"-se a existência de trechos que destoam de
Plotino e de suas\emph{Enéadas}, as assim chamadas “acréscimos” que
expõem pontos de vista que apresentam um pensamento pessoal
daquele que é chamado por Adamson de “o Adaptador”\footnote{
Cf. \textsc{Adamson}, \emph{The Arabic Plotinus. A
Philosophical Study of the Theology of Aristotle.} London:
uckworth, 2002, p. 18.} e que é considerado o autor desta
paráfrase, pensamento, aliás que muitas vezes avança num sentido
que é estranho ao texto parafraseado, as\emph{Enéadas}, como por
exemplo afirmando um monoteísmo explicito.

A autoria do texto é bastante complicada, não obstante a famosa
primeira frase do Tratado Primeiro da TdA:\footnote{ Falo do
“primeiro “Tratado Primeiro”, porque a \emph{TdA} apresenta
dois Tratados Primeiros.}

\begin{quote}
Do livro de Aristóteles, o filósofo, chamado em grego Teologia,
que é a doutrina sobre a Divindade. Paráfrase de Porfírio de
Tiro, que foi traduzida para o árabe por Abd al"-Maib ben' Abd
Allah bem Nai'ma, de Emesa, e corrigida por Abu Usuf Ya'qub ben
Ishaq aL"-Kindi;  Deus tenha misericórdia dele, para Ahmad
ben"-al"-Um ‘tassim bil"-Iah. \footnote{ Todas as traduções deste
estudo da \emph{TdA} são da minha autoria e elaboradas a
partir da versão espanhola de \textsc{Rubio}:
\emph{Teologia.} Traducción del árabe, Intruducción e notas.
Madrid, Ediciones Paulinas, 1978.}  
\end{quote}

De modo geral a \emph{TdA} é atribuída a um
“Adaptador”,\footnote{ Cf, nota 6.} isto quer dizer um autor que
a partir da coletânea original elaborou o seu texto e introduziu
nela algumas mudanças muito radicais e significativas que dizem
respeito a Deus; talvez a mais destacada característica desta
paráfrase seja que o Uno de Plotino tem sido transformado num
Deus Criador. Em razão disto foi preciso uma reelaboração ou um
repensar do Primeiro Princípio, aplicando a Deus as regras de
predicação por causalidade e eminência usadas no caso do
Uno.\footnote{ Cf. \textsc{adamson}, op. cit., p. 112ss} 
E mais profundo como mudança, a identificação de Deus com o Ser
e sua apresentação como Criador. 

É importante acentuar que a paráfrase não é influenciada somente
pela tradição grega, seja platônica ou aristotélica.
Deus como Ser parece ser fruto de um debate entre teólogos
contemporâneos como o círculo de Al"-Kindi, desenvolvido no
contexto da Kal\={a}m.

\section{O pensamento plotiniano a respeito da procedência das coisas}

 Para poder avaliar o peso desta mudança,
precisa"-se, inicialmente, procurar o pensamento de Plotino a
respeito destes assuntos. A chave do entendimento está nas
concepções de Plotino a respeito da possibilidade de se atribuir
a Deus determinados predicados, como também na incondicional
negação de Deus como Criador.

\subsection{Predicação em Plotino}

De forma absoluta Plotino afirma a indescritibilidade do
Uno.\footnote{ Cf. \textsc{ter reegen, Jan G.J.} ``Deus não poder
ser conhecido. A incognoscibilidade divina no Livro dos \textsc{xxiv}
filósofos (\textsc{xvi} e \textsc{xvii}) e suas raízes na tradição filosófica
ocidental.'' Em \emph{Mirabilia}, 2, 2003, p. 126--127} Assim
sendo, pode"-se constatar em Plotino três formas de
\emph{Teologia"-Filosofia Negativa}. A primeira é a teologia
negativa da transcendência positiva. O Uno como Causa Primeira
que está acima do Ser é expresso em termos da teologia negativa:
somente por causa de sua realidade não poderia ser expresso em
termos da realidade que conhecemos: frases que esclarecem que  a
realidade transcendente é mais do que aquilo que se nega dela,
são preferíveis:  

\begin{quote}
Cuando se trata, por tanto, de la Natureza más eximia, que no
necessita de ningún auxilio, hay que desechar del todo las demas
cosas. Porque qualquier cosa que le añadas, ya hás aminorado
com la añadidura a quien no necessitaba de nada.\footnote{ Em.
\textsc{vi}, Trat. \textsc{vi}, 7. Tradução de \textsc{Igal}.
Madrid: Editorial Gredos, 1998. Tradução: \emph{Quasndo, então,
se trata da Natureza mais excelente, que não precisa de nenhum
auxílio, deve ser desprezado totalmente o resto das coisas.
Porque qualquer coisa que se lhe acrescenta, já  o diminui, em
acrescentar algo a alguém que não precisa de nada.}} 
\end{quote}

Em segundo lugar vem a teologia negativa matemática que considera
o Primeiro Princípio como uma unidade estando na origem do
número que carrega consigo pouca profundidade de sentimento
religioso. O Uno assim concebido é considerado um princípio de
medida, embora fique claro que ele mesmo não pode ser medido,
porque transcende aquilo que mede. O Uno desta maneira é
entendido como fonte de multiplicação. 

Finalmente, há a teologia negativa do sujeito infinito: o Uno é considerado
como o momento em que toda limitação é negada, a fronteira entre Sujeito e
Objeto colapsa e todas as coisas são fundidas na Unidade.

\subsection{A negação de Deus como Criador}

Negar Deus como Criador deve ser entendido, e
aprofundado, na luz da posição plotiniana a respeito dos
predicados divinos. Afirmar o Uno como o princípio de tudo, é
uma conclusão que se pode tirar entre outros do fato de que a alma
pressupõe uma Inteligência Divina enquanto fonte de sabedoria
que a guia. Essa Inteligência Divina, causa última para os
aristotélicos, pressupõe em Plotino um princípio anterior que a
constitui. 

A constituição das coisas pelo Uno é descrita por Plotino como um
processo de “derivação”.\footnote{ Cf. \textsc{O'Meara,}
\emph{Plotin. Une introduction aux Ennéades.} Fribourg/\,Paris:
Cerf Editions Universitaires, 1992. Outros autores usam
a expressão “emanação” que possui até certa fundamentação em
Plotino que usa imagens de água e luz para descrever a
constituição das coisas a partir do Uno. “Emanação”, porém, tem
certas implicações de cunho material, por isso a preferência de
autores modernos pelo conceito “derivação” ou “processo”.} Quer
se use “processão”, ou “derivação” ou “emanação”, de qualquer
forma o Uno é o poder ou a potência  de tudo e sem ela o Uno
ficaria num solipsismo total:

\begin{quote}
Ficando Ele mesmo, em si mesmo (\textsc{v},4,2,1.20), ele não é um
princípio inerte, “parado em si mesmo” (\textsc{v},4,1,1.34--36). Sendo o
Uno de forma absoluta, sendo o Uno mesmo de acordo com a
expressão retomada por Proclo (\textsc{v},3,12,1.52), o Uno
essencialmente um (\textsc{v},4,1,1.8), o Uno verdadeiramente um
(\textsc{v},
5,4,1.2), o Uno pura e essencialmente um (\textsc{i}.60), o Uno total
perfeitamente um (\textsc{v},3,15,1.5), ele é transbordamento, ebulição
substancial na eternidade imóvel de sua unidade, poder como
consequência de todas as coisas (\textsc{v},1,7,1.9;1,1.24--25; 4,2,1.40),
não poder material como uma matéria é sujeito de uma forma
(\textsc{v},3,15,1.33--35), mas poder formal, como o ponto ou a linha ou o
circulo ou a esfera em Nicolau de Cusa são poder de todas as
figuras geométricas. Ele, consequentemente, somente pode
proceder dele mesmo.\footnote{ \textsc{Muralt}, 
\emph{Néoplatonisme et Aristotelisme dans la Métaphysique
Médiévale.} Paris: Librairie Philosophique J.Vrin, 1995, p.~59.}
\end{quote}

Em Plotino é evitado o uso da palavra Criador, e mesma quando
usada tem ligação com a Inteligência a primeira processão que
cria.  Este transbordar, ou esta processão, é atemporal, não
cria o tempo que só ocorre quando a alma se encontrar com a
matéria, o poder ser, o não formado. Não estamos diante de um
ato explicito de querer, embora ele seja livre enquanto pura
necessidade da própria natureza do Uno.

\subsection{A atitude do Islã diante da teoria de processão}

É claro que o Islã com suas ideias, expressas entre outras nas
\emph{Suratas} \textsc{x} 4,\footnote{ [\ldots{}] é uma promessa de Allah em
verdade. Lo! Ele produz a criação, em seguida recriou"-a, que Ele
recompense aqueles que creem e fazem boas obras [\ldots{}].} ou
\textsc{xxiii}, 12--14,\footnote{ Em verdade, Nós criamos o homem a partir
de um produto de terra molhada. Em seguida o colocamos como uma
gota (de semente) numa morada segura.} ou \textsc{xxxvii},
64\footnote{ Não é Ele o melhor que produziu a criação, em
seguida a reproduziu para vocês do céu e da terra ? Existe algum
Deus ao lado de Allah?}  a respeito da criação, usando a mesma
linguagem que a Sagrada Escritura dos cristãos e judeus, não
pode aceitar a ideia neoplatônica da processão  no que tange à
criação do mundo como algo necessário, e não como uma decisão
explícita de Deus. Isto não conduz à constatação que a
\emph{TdA} defende simplesmente a existência de atributos em
Deus, pois isto seria negar a sua absoluta simplicidade, ponto
defendido incondicionalmente pelo \emph{Adaptador}. Além
disso, ele raciocina: como causa de tudo que tem atributos, Deus
mesmo não pode haver atributo, porque nada existe acima dele que
ele poderia desejar limitar, como também nada abaixo dele que
seria capaz de limitá-lo: Deus é sem limite de qualquer modo. 

Fica evidente que estão em jogo, então, dois tipos de
causalidade: aquela exercida pela Inteligência, que impõe a sua
própria forma a efeitos mais baixos, e aquela do Primeiro
Princípio que não tem forma nem modelo, mas dá origem a uma
forma:

\begin{quote}
Quando a alma vê o seu esplendor e beleza, sabe de onde vem esta
beleza, sem precisar de raciocínio para sabê-lo, porque o sabe
por intermédio da inteligência. A luz primeira não é uma luz que
se dá em alguma coisa, mas ela é somente luz, subsistente na sua
essência. Por isso, essa luz ilumina a Alma por intermédio da
Inteligência sem necessidade de atributos, como os atributos de
fogo e outros, próprios às coisas agentes, pois as ações de
todas as coisas agentes se dão somente por meio de atributos que
nelas existem, não em virtude de sua essência. Por outro lado, o
Agente Primeiro realiza a coisa sem nenhum atributo, porque
nele, indubitavelmente, não existe nenhum atributo, mas ele age
por meio de sua essência. Por isso, é agente primeiro e agente
de beleza primeira que existe na Inteligência e na Alma. O
Agente primeiro é agente da Inteligência que é a Inteligência
perpétua, não da nossa inteligência, porque aquela não é
inteligência participada, nem inteligência adquirida.\footnote{
\emph{TdA}, \textsc{iv}, 62} 
\end{quote}

Mas, se falar de Criador, aplica"-se um atributo ao Primeiro
Princípio que, então, também será o Ser. Como isto pode ser
explicado? O “\emph{Adaptador}” explica isto num trecho do
livro 10 da \emph{TdA} : 

\begin{quote}
Em relação à imponente sabedoria primeira e sua potência, quem é
aquele que poderá vê-la e conhece"-la até às profundezas de seu
conhecimento? Isto em razão de ela ser uma sabedoria em que está
a totalidade das coisas e a potência que criou todas as coisas.
Todas as coisas estão nela, ela é diferente de todas as coisas,
porque é a causa das coisas inteligíveis e sensíveis,
ressalvando que criou as coisas inteligíveis sem intermediário,
enquanto criou as coisas sensíveis por meio das coisas
inteligíveis.
\end{quote}

Isto nos leva a ver que o “\emph{Adaptador}” usa a predicação
para o Primeiro Princípio --- além da aplicação da causalidade
universal, única e absoluta --- também pelo caminho da eminência.
Isto revela que ele não quer comprometer a divina transcendência
com um sistema de predicação positiva:

\begin{quote}
A coisa dotada de vida excelente não se cansa nem nela penetra a
dor, porque nunca deixa de ser completa, desde que foi criada
sem defeito, e por isso não precisa de esforço nem de cansaço.
Aquela sabedoria só foi criada a partir da sabedoria primeira e
a substância primeira a partir da sabedoria. A substância não
foi a primeira e depois a sabedoria, mas a substância é a
sabedoria e o ser primeiro e o ser primeiro, e a substância e a
sabedoria são uma mesma coisa. Por isso, aquela sabedoria é mais
ampla que toda sabedoria, e é a sabedoria das sabedorias [\ldots{}].
\footnote{ TdA 10, 156.}
\end{quote}

Este caminho é expresso pela adição de adjetivos como “puro”,
“acima”, ``primeiro”, “verdadeiro”, entre outros.

\subsection{A ideia da Criação na <<TdA>>}

\emph{}Como já se afirmou, há na \emph{TdA}  -- como
também em outros textos que sobraram da antiga 
\emph{Teologia de Aristóteles,} a saber a \emph{Carta sobre
a Divina Ciência} e as \emph{Ditas do Sábio Grego} --- uma
constante referência ao Primeiro Princípio como um Deus Criador,
o Feitor ou o Originador. Esta identificação de Deus com o
Criador e o Uno de Plotino, celebrada na Kalām do Islã
e na exegese judaica e cristã, fez surgir uma expectativa: O
“Adaptador” mudaria a maneira de Plotino tratar da origem do
cosmo? 

A primeira impressão é que o “Adaptador” simplesmente
transfere o conteúdo do Noūs, da Inteligência. O uso da palavra
“completo” poderia indicar isto, visto que ele aplica a palavra
tanto à Inteligência (Noūs) quanto ao Primeiro Princípio, como
se entre os dois não haveria tanta diferença. Porém, de outros
\emph{loci} é mais do que claro que o Primeiro Princípio é a
origem da Inteligência, que ela foi por ele criada. A leitura da
\emph{TdA}, Livro \textsc{x},  135 indica de maneira inconfundível a
posição do ``\emph{Adaptador}”:

\begin{quote}
A prova de que o Uno puro é uma perfeição\footnote{
\textsc{Adamson},  \emph{opus cit}. p. 121 usa em vez de “perfeição”
a palavra  “completo”, o que, afinal não muda substancialmente o
sentido.}  que está sobre a perfeição, reside no fato de Ele
não sentir necessidade de coisa alguma nem buscar proveito algum
e por causa da intensidade e excesso de sua perfeição só produz
a partir d'Ele outra coisa porque não é possível uma coisa que
está sobre a perfeição seja produtor sem que a coisa produzida
seja perfeita, pois no caso contrário, não estaria sobre a
perfeição. Isto porque se a coisa perfeita produzir alguma
coisa, com maior razão a coisa que está acima da perfeição é
produtora da perfeição, porque produz a coisa tão perfeita que
não é possível que coisa alguma produzida seja mais capaz nem
mais bonita, nem mais elevada do que ela.  
\end{quote}

Para que a coisa seja perfeita --- ou completa --- é necessário não
somente que ela possua todas as coisas, mas que ela o possua
simultaneamente. Sendo assim, à ideia da perfeição estão
associadas algumas outras ideias: 

\begin{enumerate}
\item Tudo que é perfeito deve ser eterno e atemporal;

\item Os atos da perfeição não se sucedem um após o outro, nem têm
começo ou fim, mas tudo ocorre “de uma só vez”;

\item O perfeito --- ou completo --- não pensa discursivamente, deduzindo
uma verdade de outra, porque este \emph{modus procedendi}
implica “transferência'” de uma coisa à seguinte, o que
significa mudança, isto é, movimento;

\item O perfeito --- ou completo --- deve ser um e não muitos, visto a
“deficiência” incluída na realidade de muitos.\footnote{
Cf.~\textsc{Adamson}, \emph{opus cit}., p. 120.}
\end{enumerate}

Para o “\emph{Adaptador}”, então, o Primeiro não é
simplesmente perfeito ou completo, mas perfeito acima da
perfeição, completo acima da completude. A doutrina do
\emph{hyperpleres} aplicado ao Primeiro Princípio, implica a
transferência de todas as características da Inteligência/\,Noūs
plotiniana para o Deus do “\emph{Adaptador}”. É nesta
perspectiva que começa a clarear a afirmação que Deus é Criador.

Fica, entretanto uma interrogação importante: Como esta
criação se concretiza? Ela se realiza de forma mediada ou não
mediada? A dificuldade é sentida na evolução da filosofia árabe,
entre outros por Ibn Sina quando afirma que “o Um como tal
somente produz o um”,\footnote{ Citado em \textsc{Adamson},
\emph{opus cit}., p. 137, nota 50.} em contraste com Plotino que afirma
que o Uno pode produzir --- e de fato produz --- algo múltiplo, isto é,  
o “uno múltiplo” que é a Inteligência. Fato é que existe uma
tensão entre a Criação, como defendida no Al Quran, das coisas
individuais encontradas no mundo sensível e o esquema
neoplatônico das emanações. 

O que se pode apresentar a partir de uma leitura criteriosa e
atenta da \emph{TdA} é que o ato criativo de Deus se realiza
sem intermediário, porque ele cria somente a Inteligência em que
todas as coisas estão contidas: “Ele faz todas as coisas sem
intermediário, tudo de uma só vez, e tudo junto”.\footnote{ TdA
\textsc{viii}, 98} Juntando estas palavras com outro trecho da
\emph{TdA} (\textsc{vii}, 84) depara"-se com uma bela descrição da
majestade do Criador:

\begin{quote}
A prova de que isto é desta maneira encontra"-se na
criatura. Quando é formosa, resplandecente, múltipla e colorida,
perfeita e visível, o contemplador, se for inteligente, não
somente admira a sua beleza exterior, mas também vê o seu
interior, e, como consequência, admira o seu criador e autor e
não duvida que este seja o máximo de beleza e esplendor e que a
sua potência não tenha limite, porque fez algo semelhante a
estas ações repletas de formosura, e de beleza e de perfeição
completa. Assim, então, se o Criador, poderoso e excelso, não
tivesse criado as coisas, mas se tivesse existido somente Ele e
nada mais, as coisas teriam ficado ocultas e a sua formosura e
esplendor não teriam se manifestado nem aparecido. Se este Ente
único tivesse permanecido na sua essência e retido sua potência,
sua ação e sua luz, não existiria nenhum dos entes nem
persistentes nem mutantes e perecíveis, nem existiriam as coisas
múltiplas criadas a partir do Uno, da forma que existem agora,
nem teria havido causas que produzissem os seus causados, nem
elas os teriam introduzido no caminho da geração e dos entes. 
\end{quote}

A citação deixa bem claro a absoluta força criadora de Deus, o
único responsável por tudo que existe e pela beleza que está nas
coisas; sem esta ação criadora, nada teria existido e nada,
então, seria revelado. O Princípio Primeiro, então, cria
tudo, o intermediário é sustentado pelo “Adaptador” no
sentido de que todas as coisas estão no seu poder criativo, mas
este poder é revelado numa emanação/\,processão que divide em ato
o que existe em Deus em potência ou como poder. Afirmar isto é a
mesma coisa que dizer que Deus não cria por intermediários,
porque o “intermediário” é a sua própria potência ou poder: 

\begin{quote}
Digo que na Inteligência está a totalidade das inteligências e do
animal. A razão é que elas estão distribuídas nela e a
distribuição na Inteligência não é porque as coisas nela sejam
subsistentes, nem tampouco que as coisas nela sejam compostas,
mas só porque ela é agente das coisas, exceto que as faz uma após
da outra, em ordem e posição. O Agente Primeiro, entretanto, faz
todas as coisas que realiza simultaneamente sem intermediário e
de uma só vez.\footnote{ TdA, \textsc{viii}, 98}
\end{quote}

O Principio Primeiro, então, é a causa da Inteligência, do Ser,
de forma imediata, estando nele tudo, enquanto na Inteligência
tudo está enquanto criado por Ele; por isso, as coisas proveem da
Inteligência e não de forma direta, mas de forma indireta: em
outras palavras, a Primeira Causa, o Primeiro Principio, cria
tudo sem intermediário.\footnote{ A mesma questão é abordada no
\emph{Livro das Causas}, outro documento que tem sua origem na
original antologia \emph{Teologia de Aristóteles,} 
Proposição \textsc{ii}, \textsc{vi}, \textsc{vii},\textsc{viii},
\textsc{ix}. \textsc{D'Ancona} 
dedica ao assunto um estudo penetrante no seu
\emph{Recherches sur le \emph{Liber de Causis},} Paris: Librairie
Philosophique J. Vrin, pp.~62--95. A grande questão é como conciliar
a afirmação do \emph{Liber} de que Deus é o Criador de tudo e a
presença da Inteligência como mediador universal. Em outra
palavras: \emph{“$[$\ldots{}$]$ équivaut en effet d'établir dans quelle
mesure la doctrine de la procession propre a cette ourage peut
être accueillie”. $[$\ldots{}$]$ cela oblige à examiner la cohérence de
cette doctrine avec l'idée de création, elle aussi soutenue par
le De Causis”}. São Tomás de Aquino no seu comentário, 
\emph{Super Librum de Causis Expositio}, diz o seguinte:
“\emph{Hence the soul is from God as the first cause, but from
na intelligence as the second cause”  (Versão inglesa da
“Exxpositio” Commentary of the Book of Causes translated and
annotes by Vincent a Gualiardo \emph{et al.}). Washington \textsc{dc}: The Catolic University of América
Press, p.~23.}}

E o que dizer de um problema famoso\footnote{ Cf., por
exemplo, santo Agostinho, \emph{Confissões}, Livro  que fala
num contexto parecido com aquele que se encontra na \emph{TdA:}
o neoplatonismo e a fé que Deus é o Criador de tudo.}
da relação entre a Criação e o tempo? O melhor texto a respeito
é, sem dúvida, a famosa “doxologia platônica”,\footnote{ A
expressão é de \textsc{D‘Ancona}, Cristina Costa. ``Per uno
profilo del autore della Teologia del Aristóteles'', Padova:
\emph{Medievo} \textsc{xvi} (1991), p. 83--134.} no fim do Tratado
\textsc{i} da
\emph{TdA}:

\begin{quote}
De que forma bonita e correta o filósofo descreve o Criador,
afirmando que é ele que faz a Inteligência, a Alma, a Natureza e
todas as demais coisas! Porém, não convém que o ouvinte das
palavras do filósofo encare a sua expressão e que, baseado
nelas, se imagine que o filósofo disse que o Criador excelso só
criou as criaturas no tempo, pois mesmo sendo isto imaginável em
razão de sua expressão e sua linguagem, nisto expressou somente,
sem dúvida, a sua vontade de seguir o uso dos antigos. Os
antigos foram obrigados a mencionar o tempo referindo"-se ao
começo da criação somente porque eles queriam descrever a
geração das coisas, e por isso, foram forçados a introduzir o
tempo na sua descrição da geração e da criação que, sem dúvida,
não se realizou no tempo. [\ldots{}] Entretanto, não e desta forma
que as coisas sucedem, quer dizer, nem todo agente realiza a sua
ação num determinado tempo, nem toda causa é antes do seu
causado no tempo. Se quiser saber se este feito temporal ou 
não, olha para seu agente.
\end{quote}

Desta citação fica, portanto, bem claro que nem todas as causas
operam no tempo, como, também que causa e efeito são de
naturezas semelhantes, e finalmente, que a temporalidade é um
fator que diferencia o mais alto do mais baixo.  É neste sentido
que a \emph{TdA} pode afirmar:

\begin{quote}
Unicamente foram criados (os entes verdadeiros) por esta criação
e feitos de tal maneira que não há, entre eles e o Criador,
intermediário algum, sendo incontestável. Como poderia a sua
criação ocorrer no tempo, sendo eles causa do tempo das gerações
temporais, de sua ordenação e de sua nobreza! A causa do tempo
não pertence ao tempo, mas é de outro modo superior e mais
elevado, como é a sombra em relação à coisa que a reproduz.
(\textsc{viii}, 114)
\end{quote}

Este pensamento é completado por \emph{TdA} \textsc{ix}, 130: 

\begin{quote}
A Causa Primeira está firme repousando na sua essência, porém,
não num evo, num tempo, nem num lugar, mas a substância do evo,
tempo e lugar e o restante das coisas existem somente em virtude
dele. [\ldots{}] Assim ocorre também conosco, nossa subsistência e
consistência existem em virtude do Agente Primeiro, dele
dependemos, nele está o nosso desejo e diante dele nos
inclinamos e para ele regressamos.
\end{quote}

Sendo assim, Deus esta acima tanto da eternidade como do tempo!

Uma outra questão a respeito de Deus e a Criação foi ainda
desenvolvida na Kalām, a teologia islâmica: a relação
entre a Criação e a necessidade. Toda a reflexão arvora"-se,
entre outras, na questão sobre a liberdade e necessidade no
pensamento neoplatônico, questão essa que se pode desenvolver em
duas direções. A primeira apresenta a pergunta sobre quem é
livre e sua resposta: livre é aquele que age por sua própria
vontade. A segunda afirma que não é livre aquele cuja relação
com o causado é necessária. Mas, vem logo a interrogação: qual é
a relação entre necessidade e liberdade em se tratar do Primeiro
Princípio, absolutamente simples e Uno, sem sucessão, mas sendo
e tendo tudo ``de uma só vez”?

Nesta questão o “Adaptador” não se afasta muito do texto que
está parafraseando e comentando, as \emph{Enéadas} de Plotino,
mantendo a posição que a Criação é necessária, mas que esta
necessidade não afeta a liberdade:\footnote{ Cf. \textsc{ter
reegen,} ``A Liberdade na Teologia do Pseudo"-Aristóteles'',
\emph{Cadernos \textsc{ufs} --- Filosofia,} vol. \textsc{viii}, Fasc. 6 (2006) ---
são Cristóvão: Editora \textsc{ufs}, 2006, p.~31--61.} 

\begin{quote}
Supondo que não tenham existido as coisas perpétuas nem tampouco
as perecíveis, que são regidas pela lei da geração e da
corrupção, o Uno primeiro não seria a causa verdadeira. Mas,
como teria sido possível que não existissem as coisas, dado que
sua causa é a causa verdadeira, a luz verdadeira e o bem
verdadeiro! Sendo esta a essência do Uno primeiro, quer dizer,
causa verdadeira, seu causado é um causado verdadeiro. Sendo luz
verdadeira, aquele que recebe esta luz é um receptor verdadeiro.
Sendo bem verdadeiro e, dado que este bem transborda, aquele
sobre quem transborda é também verdadeiro. Sendo assim e não
sendo necessário que o Criador exista somente ele, nem que
deixe de criar alguma coisa nobre, receptora de sua luz, quer
dizer, a Inteligência, da mesma forma é tampouco necessário que
exista a Inteligência somente ela e que não forme qualquer coisa
receptora da sua ação, da sua potência nobre e de sua luz
resplandecente, e, por isso, a Inteligência forma a
Alma.\footnote{ \emph{TdA}, \textsc{vii}, 85.}
\end{quote}

O mesmo pensamento é expresso num texto, pertencendo à mesma
grande tradição do pensamento neoplatônico árabe, \emph{As Palavras do
Sábio Grego}:

\begin{quote}
O Primeiro Agente --- Ele é grande e poderoso --- criou todas as
coisas com extrema sabedoria. Ninguém pode compreender as razões
do seu vir à existência, e porque elas estão na condição em que
neste momento se encontram. Ele não pode conhecê-las de forma
completa, ou saber por que a terra está no meio, e por que ela é
redonda e não irregular e quadrada. Ele pode somente dizer que é
assim, a terra tem que ser redonda e colocada no meio, e que o
Criador --- Ele é grande e poderoso --- afez no meio. Ela, por isso,
deve ser redonda e colocada no meio, o único lugar em que pode
estar. [\ldots{}] A mais remota Causa, que é a razão que as coisas se
tornaram o que são agora, não pode ser conhecida e alcançada por
ninguém, porque elas se originaram de acordo com a definitiva
sabedoria, que inclui qualquer sabedoria.\footnote{
\emph{Plotini Opera,} Tomus \textsc{ii}, \textsc{enéadas}
\textsc{iv"-v}. Ediderunt Paul
\textsc{henry} e Hans Rudolf \textsc{schwyzer}. \emph{Plotiniana}
arábica, ad codicum
fidem anglice vertit Geoffrey \textsc{lewis}. Paris/\,Bruxellas: Desclée de
Brouwer et Cie/\,l'Édition Universelle S.A., 1959, p. 484.}  
\end{quote}

No mesmo sentido vai a Carta de Divina Sabedoria, que afirma
que o ato do Primeiro Agente não é precedido por vontade, porque
ela somente age por ser só.

\section{Conclusão} 

Percorreu"-se um caminho nem sempre fácil, guiado pelo estudo de
Peter Adamson e pelas reflexões precisas e profundas de Cristina
D‘Ancona Costa. Muitas coisas ficaram em aberto, outras nem se
revelam com a clareza desejada e intencionada. O conhecimento da
\emph{TdA}  é um objetivo não tão fácil a se alcançar, e exige
uma constante leitura e tentativa de interpretação.\footnote{
Aqui quero deixar claro o meu reconhecimento ao \emph{Grupo de
Estudo de Filosofia Medieval da \textsc{uece}}, que nos anos passados
dedicou quase dois anos à reflexão e estudo deste riquíssimo
documento.} Mas, descobrimos que para o autor da \emph{TdA} o
Uno, Deus, além do Ser, é o Criador de todas as coisas, sem
exceção, das coisas inteligíveis e sensíveis. O caminho
percorrido neste “processo” de criação talvez não seja o mais
claro possível, mas ele está ancorado na fé inabalável do
“Adaptador” que a Divina Sabedoria --- Causa Primeira --- Primeiro
Princípio criou tudo e que tudo que existe participa de sua
imensa beleza e bondade.

\section{Bibliografia}

\begin{description}\labelsep0ex\parsep0ex
\newcommand{\tit}[1]{\item[\textnormal{\textsc{\MakeTextLowercase{#1}}}]}
\newcommand{\titidem}{\item[\line(1,0){25}]}
\tit{~}

\noindent\textsc{Fontes}

\tit{PL0TINI OPERA}. Tomus \textsc{ii}. \emph{Enneades \textsc{iv"-v}}. Ediderunt Paul
\textsc{henry} et Hans"-Rudolf \textsc{schwyzer}. Plotiniana Arabaica. Ad codicum
fidem anglice vertit Geoffrey \textsc{lewis}. Paris/\,Bruxellas: Desclée de
Brouwer et Cie/\,l´Edition Universelle, S.A., 1959

\tit{PLOTINO}. \emph{Enneadi.} Traduzione com texto grego a fronte,
introduzione, note e bibliografia di Giuseppe Faggin.
Prsentazione e iconografia plotiniana di Giovanni Reale.
Revisione finali dei testi, appendici e indici di Roberto Reale.
Milano: Rusconi, 1996 (4ª edizione).

\tit{PLOTINO}, \emph{Eneadas}. Tradução de Jesus Igal. Madrid:
Editorial Gredos, 1998.

\tit{PSEUDO"-ARISTOTELES}. \emph{Teologia.} Traducción del Árabe,
Introducción y notas: Luciano Rubio, O.S.A., Madrid: Ediciones
Paulinas, 1978. 

\titidem. \emph{Liber de Causis. O Livro das Causas.} Uma
tradução e introdução de Jan Gerard Joseph ter Reegen. Porto
Alegre: Edipucrs, 2000.

\tit{THEOLOGIE DES PSEUDO ARISTOTELES, DIE SOGENANNTE}. Aus dem
arabischen übersetztund mit Anmerkungen von Dr. Fr. Dieterici,
Professor an der Universität Berlin. Leipzig: J.C. Hinrichs'sche
Buchhandlung, 1883

\medskip

\noindent\textsc{Outros estudos}

\tit{ADAMSON}. \emph{The Arabic Plotinus. A philosophical Study of
the Theology of Aristotle}. London: Duckworth, 2002.

\tit{COSTA}, Cristina d'Ancona. La Doctrrine de la Création “Mediante
Intelligentia” dans le Liber de Causis et dans sés sources. Em:
\emph{Recherches sur le Liber de Causis.} Paris: Librairie
Philosophique J.Vrin, 1995, 73--95.

\titidem.  Per um profilo filosófico dell'autore della
“Teologia di Aristotele”. Padova: \emph{Medievo,} 17, 1992,
83--134. 

\tit{LOPEZ"-FARJEAT}, Luis Xavier. La inmortalidad del alma en la
Theologia Pseudoaristotélica y su pale en la filosofia de
Al"-Farabi. Em: \emph{Estúdios de Ásia y África \textsc{xl}: 3, 2005, p.
577--606.}

\tit{MURALT}, André de. \emph{Néoplatonisme et Aristotélisme dans la
Métaphysique Médiévale. Analogie, Causalité, Participation}.
Paris: Librairie Philosophique J.Vrin, 1995.

\tit{O'MEARA}, Dominic. \emph{Plotin. Une introductiuon aux Ennéades.}
Fribiourg Suisse/\,Paris: Éditions Universitaires/\,Éditions du
Cerf, 1992

\tit{TOMÁS DE AQUINO}, \emph{Commentary on the Book of Causes.}
Translated by Vincent A. Guagliardo (and others) Washington
D.C., The Catholic University of America Press, 1996.

\tit{TER REEGEN}, Jan G.J. ``Deus não pode ser conhecido. A
Incognoscibilidade divina no Livro dos \textsc{xxiv} Filósofos
(\textsc{xvi} e
\textsc{xvii}) e suas raízes na tradição filosófica ocidental''. Em:
\emph{Mirabilia 2 Alexander Fidora e Jordi Pardo Pastor
(coord.) Expressa lo divino: lenquaje, arte y mística.,} 2003,
p. 121--137.

\titidem.  A liberdade na Teologia do Pseudo"-Aristoteles. Em:
\emph{Cadernos \textsc{ufs} --- Filosofia Universidade Federal de
Sergipe}, vol. \textsc{viii} Fasc. 6 (2006) p. 31--47.. 

\tit{THE CAMBRDIGE COMPANION TO PLOTINUS}. (edited by Lloyd P. Gerson)
Cambridge: University Press, 1996

\tit{ZIMMERMANN}, F.W. The origins of the so"-called Theology of
Aristotle. Em: \emph{Pseudo"-Aristotle in the Middle Ages. The
“}Theology\emph{” and other texts.} Ed. by J \textsc{kraye} and others,
London: The Wartburg Institute, 1986, p. 110--240.
\end{description}



\capitulo{Neoplatonismo e judaísmo medieval: apontamentos sobre a origem do material}%
	 {Cecilia Cintra Cavaleiro de Macedo}
	 {unifesp}

\markboth{Neoplatonismo e judaísmo medieval}{Cecilia Cintra Cavaleiro de Macedo}

A relação entre platonismo e judaísmo não é uma particularidade do
mundo medieval. Podemos considerar Fílon de Alexandria o primeiro
filósofo judeu. Esta designação deve"-se ao fato de ser o primeiro
autor de que temos notícia, a tentar, de modo mais ou menos
sistemático, uma compatibilização entre a filosofia platônica e as
Escrituras. Esta tentativa de compatibilização foi aproveitada e
estendida pelos autores cristãos, judeus e, posteriormente,
influenciou também os filósofos islâmicos dando origem ao
neoplatonismo medieval, tal como conhecemos. 

Fílon toma como ponto de partida o texto das Escrituras, e não as
teorias filosóficas gregas, apresentando uma interpretação filosófica
da revelação, profundamente dependente de sua leitura alegórica e da
consequente rejeição das interpretações literais e históricas. Para
Fílon, é inegável a superioridade da palavra revelada em relação à
filosofia especulativa, sendo esta última, no entanto, considerada
uma ciência propedêutica e um estudo indispensável à ciência de Deus,
como também o farão diversos filósofos judeus medievais. Sua leitura
alegórica consiste numa espécie de hermenêutica mística, que se
propõe a desvendar os mistérios contidos entre as linhas dos textos
bíblicos utilizando a filosofia como poderoso instrumento para o
esclarecimento. Possivelmente, esta opção tenha sido justamente a
responsável por sua rejeição nos círculos judaicos rabínicos
posteriores, mais adeptos de uma leitura tradicionalista, o que fez
com que a influência de sua obra fosse infinitamente superior no
mundo cristão do que nos meios judaicos. Somente em restritos
círculos místicos suas ideias parecem ter sido bem acolhidas, uma vez
que seus comentários ao Pentateuco podem ser enquadrados na
disciplina mística do \emph{Maaseh Bereshit} (Trabalho da Criação).
Efetivamente, Fílon só será citado abertamente nos meios judaicos por
volta do século \textsc{xvi}, mas isto não impede que reflexos de suas ideias
sejam sentidos especialmente no material relacionado à
\emph{Kabbalah} desde o período de sua formação. 

Assim sendo, a primeira questão que se apresenta quanto ao tema que
propomos é, portanto, a possibilidade de transmissão das obras de
Fílon ao judaísmo Medieval. Muito embora inúmeros estudiosos
concordem em que as obras de Fílon tenham sido preservadas
exclusivamente por mãos cristãs, seus ecos podem ser claramente
ouvidos em vozes judaicas medievais e renascentistas. Essa influência
Fíloniana pode ter ocorrido de formas diversas: 

a) A possibilidade de retransmissão ao judaísmo através dos cristãos é
colaborada pelas conexões entre o pensamento cristão e os primeiros
exegetas judeus medievais, especialmente aqueles considerados no
âmbito do Kalām judaico.\footnote{ Isto já foi bastante
explorado, por exemplo, por Sarah Stroumsa. (\textsc{stroumsa}, 1991).} Esta
influência é localizada especialmente durante o nono e décimo
séculos, através das figuras de David Al"-Moqammes, o caraíta
al"-Qirqisani e Saadia Gaón. Por outro lado, autores como Scholem
defendem que,

\begin{quote}
De todos os modos, o fato de que o caraíta Qirqisani (séc. \textsc{x})
estivesse familiarizado com certas citações tomadas dos escritos de
Fílon demonstra que algumas de suas ideias chegaram, talvez por
canais cristãos ou árabes, a membros de seitas judias no Oriente
Próximo. Mas, disso não se pode deduzir que existisse uma influência
contínua até essa época, e menos ainda, até o tempo da formulação da
Cabala na Idade Média. Paralelos específicos entre a exegese de Fílon
e a cabalística deveriam ser atribuídos à semelhança de seu método
exegético que, naturalmente, produziu, em diferentes ocasiões,
idênticos resultados (\textsc{scholem}, 1994, p. 19).
\end{quote}

Entretanto, esta argumentação conduz a uma outra grande dúvida
relacionada à transmissão, uma vez o método de exegese filosófica não
é historicamente um dos métodos usuais adotados pelos rabinos
medievais, e nem mesmo pode ser traçada uma linha na qual este método
tivesse sobrevivido por si, independente do modelo de Fílon. Isso faz
com que a posição de Scholem aparentemente solucione uma das faces da
questão, gerando, em contrapartida, um novo problema praticamente
idêntico.

b) Por outro lado, vale ressaltar aqui a observação de Barthélemy de
que, durante a Idade Média, não se dispunha de uma única coleção de
obras de Fílon, mas uma das fontes que circulava teria sido
“corrigida” ou “revisada”. De acordo com seus estudos, essa revisão
só poderia ter sido realizada por um judeu. Conforme sua
interpretação, esse texto chegara às mãos dos rabinos judeus (até
mesmo por força da utilização das obras do alexandrino por parte dos
primeiros Padres da Igreja como argumento para a defesa da
compreensão cristã do Verbo como Filho de Deus), tendo sido
previamente “censurada” por rabinos judeus na Palestina do século
\textsc{iii}.\footnote{ Discorda assim, dentre diversos estudiosos, de Katz que
defende o inverso. Conforme o autor, “O que impediu o comentário de
Katz de chegar a essa conclusão foi a convicção de que (\ldots{}) para
poder servir de fonte a nosso revisor, as obras de Fílon não foram
certamente conservadas mais do que pelas mãos cristãs.
(\textsc{barthélemy},
1967. p. 57--8).} Se o trabalho de Barthélemy está correto, e sua
argumentação parece bastante coerente com os indícios apresentados
pelas correções ao texto de Fílon, não há como atestar que, ao menos
o \emph{Comentário Alegórico}, fosse totalmente desconhecido dos
judeus no início da Idade Média. Além disso, ao menos por tradição
oral, existe a possibilidade de que algumas de suas ideias tenham
sido incorporadas à especulação mística judaica posterior, vindo a
ressurgir na forma escrita a partir do século \textsc{ix}. 

c) Outro ponto que vale ser considerado, é o fato de que as conclusões
dos estudiosos quanto à impossibilidade de conservação dessas ideias
pelos judeus medievais referem"-se preferencialmente à Palestina. Mas,
à época de Fílon, a população judaica em Alexandria era imensa, e não
temos como avaliar o impacto da transmissão dessas ideias na diáspora
mediterrânea dessa população. “Numericamente é, sem dúvida,
Alexandria e não Jerusalém a metrópole do judaísmo (\textsc{simon}, 1967, p.
19)”. O conhecimento do grego não era também, por sua vez,
particularidade de Fílon ou de alguns poucos privilegiados, mas
gozava de uma grande difusão na população judaica, tendo sido
utilizado, inclusive, em certas escolas rabínicas.\footnote{ “Há todas
as razões para pensar que esse estado de coisas remonta, ao menos
quanto ao que se refere à situação linguística, para além dos 70 e,
em particular que, na época de Fílon há um movimento de intercâmbio
entre a Palestina e uma diáspora da qual Alexandria representava o
foco mais ativo que caracteriza a vida religiosa judia”
(\textsc{simon}, 1967,
p. 20).} “Certamente, uma longa tradição de exegese bíblica existia
em Alexandria e que Fílon deixou sua marca nela ao escrever seus
próprios tratados exegéticos” (\textsc{runia}, 2001, p. 30).

\section{A questão da matéria}

O ponto principal que nos propomos a abordar aqui é a questão da
matéria. Como sabemos, ainda que essa discussão provenha do próprio
texto bíblico,\footnote{ Uma vez que dispomos de textos que conduzem a
visões conflitantes, como a partir do nada (\textsc{ii} Macabeus, 7:28) ou  a
partir da matéria informe (Sabedoria, 11:17).} tanto no Judaísmo
quanto no Cristianismo, a interpretação literal e majoritária do
Livro do Gênesis entende que Deus criou o mundo em seis dias a partir
do nada. Essa interpretação, ao ser desdobrada, implica em três
aspectos: 1) nada existia anterior a Deus e nada pode haver
coexistente a Ele; 2) Deus criou os seres tais como são, de uma só
vez; 3) como nada é comparável a Deus, Ele não pode ter criado, de
forma alguma, a partir de si mesmo. Portanto, esta interpretação da
\emph{creatio} \emph{ex nihilo} representa, a partir de seus três
aspectos apontados, um problema complexo que, por opor"-se
frontalmente às explicações de cunho platônico, não pode ser
desconsiderado na análise de qualquer tentativa de compatibilização
entre o modelo monoteísta e a tradição filosófica Platônica e
Neoplatônica. 

Podemos dizer, \emph{grosso modo}, que para Fílon, o universo é
criado por um \emph{Logos} absolutamente imaterial e se compõe de
um mundo inteligível (expressão que talvez tenha sido cunhada por ele
próprio) e o mundo sensível. Este cosmo inteligível, composto por
Ideias e seres imateriais, como gêneros e espécies, é exemplar ou
modelo para o mundo sensível.\footnote{ “O cosmo noético identificado
com o Logos de Deus no ato da criação serve de modelo ou plano para o
mundo físico. Ele é superior e, portanto, criado primeiro. A relação
entre os dois domínios é indicada especialmente pela metáfora do selo
ou marca, que impõe a inteligibilidade e a ordem na fisicalidade
bruta” (\textsc{runia}, 2001, p. 22).} Logo, o termo matéria utilizado por
Fílon, corresponde à matéria sensível, composta pelos quatro
elementos. O \emph{Logos} criador de Fílon tem sua contraparte
imanente, sendo ela mesma, contudo, imaterial, uma vez que nada mais
é que um outro modo de existência do \emph{Logos}, pertencente ao
mesmo \emph{Logos}, que reúne em si os elementos do cosmo
inteligível. 

Em sua compatibilização com as Escrituras, a criação do mundo sensível
é entendida por Fílon como sendo completada somente a partir do
sétimo dia, uma vez que os seis primeiros parecem se referir muito
mais à ordem da criação das coisas enquanto arquétipos ou modelos, do
que propriamente como seres sensíveis individuais no mundo material.
Assim, temos a criação em dois momentos:\footnote{ Leia"-se aqui o
termo momento como lógico e não temporal.} após a constituição do
cosmo inteligível, num segundo passo, o mundo sensível foi criado e
os seres atualizados; e mesmo esta segunda criação começa antes do
tempo, uma vez que este só passa a existir com o universo. Com esta
criação, o \emph{Logos} torna"-se \emph{encarnado}, não do modo
como foi interpretado pela religião cristã, mas do modo como a alma
imaterial se torna \emph{encarnada} no corpo. A partir deste
momento é também este \emph{Logos} imanente quem permite a
preservação do mundo, sustentando"-o, garantindo sua existência, e
assumindo as funções de Providência Divina (cf. \textsc{wolfson}, 1982). Se
este segundo momento, que se completa ao sétimo dia, é o momento em
que esta matéria em seu conjunto foi ordenada, fica"-nos aqui a
questão: de onde procede esta matéria? 

Notamos, portanto, que, na obra de Fílon, a questão da origem da
matéria está longe de ser esclarecida propriamente e ainda é tema de
contínuas discussões entre os estudiosos. Com o estabelecimento de
intermediários entre Deus e o mundo sensível, Fílon afasta Deus da
matéria. Ao supor o \emph{Logos} como um intermediário Criador,
Deus já é afastado da criação direta da matéria sensível, e, ao
conceber um mundo puramente inteligível que Ele cria em Sua Sabedoria
como modelo para este mundo --- e que, portanto, também é intermediário
--- Fílon afasta"-o ainda mais. Isto ocorre porque aquilo que é sensível
e material, por sua imperfeição, não é considerado digno de uma
origem divina, recebendo a ação de Deus a partir de intermediários
mais perfeitos que ele.

Nesse sentido, com a questão da matéria não resolvida, a obra de Fílon
permite uma interpretação como uma possível adaptação judaica da
ideia do Demiurgo: o \emph{Logos}, potência criadora de Deus,
chamado, entre outras denominações de “filho único de pai incriado”,
cria efetivamente a matéria sensível, ou simplesmente a ordena
conforme a Sabedoria Divina e as ideias exemplares de Deus? Mas, por
vezes, notamos que o filósofo alexandrino parece pender mais para o
judaísmo, no momento em que entende, seguindo a tradição mosaica, que
o mundo foi feito em seis dias, não porque o Criador precisasse de
uma duração, posto que Deus faz e ordena tudo de uma só vez, mas
porque as coisas necessitam de uma ordem (\textsc{philon}, 1961, p. 149--151) .
Ou quando diz que tudo foi constituído ao mesmo tempo, ou seja, ao
menos as Ideias e os gêneros referentes a tudo (\textsc{philon}, 1961, p. 159
e  p. 185). 

Retornando aos três aspectos que apontávamos no início do texto,
podemos verificar que Fílon responderia aqui a duas daquelas três
dificuldades derivadas da crença na criação \emph{ex"-nihilo}. Ou
seja, Fílon responde bem à segunda delas, uma vez que o mundo foi
criado tal como é, de uma única vez, ainda que em dois momentos
lógicos: a criação do inteligível e a criação do sensível; e também à
terceira, uma vez que Deus não cria o mundo a partir de si mesmo, de
sua essência ou substância, já que sua potência criadora e o cosmo
inteligível separam"-no do mundo sensível; vale ressaltar que não há
qualquer alusão a identidade substancial entre Deus e sua Criação, a
não ser através de um \emph{projeto} na Sabedoria de Deus, a partir
do qual o \emph{Logos} ordena e mantém o mundo material, o que se
enquadra perfeitamente nas crenças judaicas. Mesmo a inteligência
humana é entendida como possível de se unir ao \emph{Logos} (mas
não a Deus mesmo) enquanto imagem, impressão, fragmento ou reflexo.
No entanto, a primeira dificuldade permanece, já que a questão da
origem última da matéria sensível não é respondida.

Dificilmente obteremos uma resposta inquestionável à origem da matéria
em Fílon, e vale ressaltar que esta discussão prosseguirá e permeará
todo o judaísmo medieval. Depois de séculos de silêncio em relação à
filosofia propriamente dita, o platonismo despontará novamente no
século \textsc{x}, no Egito, com Isaac Israeli, o primeiro filósofo judeu
medieval. Este prosseguirá na linha inaugurada por Fílon, com os
devidos acréscimos advindos das teorias neoplatônicas e aqueles
provenientes dos mestres islâmicos, especialmente de Al"-Kindi. À
diferença da concepção plotiniana, na qual o \emph{Noūs} provém
diretamente do Uno, Israeli interpõe entre o Criador e o Intelecto
duas substâncias simples: matéria e forma. A matéria primeira é
descrita como:

\begin{quote}
A primeira substância que subsiste em si e é substrato da diversidade
(Mant., §\textsc{i}); a substância universal, que é uma em número, existe por
si e é o substrato de toda diversidade (Subst., iv 5v.); o verdadeiro
gênero primeiro, criado pelo poder do Criador sem mediador (L. Def.,
§ 3, 10--11); É conhecida entre os filósofos como a raiz das raízes
(Mant., §, \textsc{i}) (\textsc{israeli}, \emph{In} \textsc{altmann} e
\textsc{stern}, 1958, p. 159).
\end{quote}

Discute"-se a origem deste pensamento, mas parece seguir a fonte
pseudo"-Aristotélica que ficou conhecida como o “Neoplatônico de Ibn
Hasday”. Aquele texto indicava já a matéria e a forma precedendo o
Intelecto como primeiro ser criado. Para Israeli, a criação da
matéria e da forma primeiras pela Vontade de Deus é a única fase do
processo que pode ser propriamente denominada Criação. O restante dos
seres decorre daquelas por emanação, como as substâncias simples e
universais, e por geração, no caso dos particulares.

Embora não cite seu predecessor, Schlomo Ibn Gabirol (latinizado
Avicebron ou Avencebrol) seguirá, na questão da matéria, quase
literalmente as ideias expostas por Israeli, dando mais ênfase à
questão da Vontade de Deus e detalhando as questões apontadas por seu
predecessor. Para este autor, absolutamente tudo o que existe é
composto por matéria e forma, à exceção de Deus, denominado por ele
Essência ou Substância Primeira, e de sua Vontade Criadora. 

\begin{quote}
E para que fique ainda mais manifesto que todas as coisas se reduzem a
matéria e forma, digo que tudo o que se reduz, é impossível que se
reduza mais do que a uma raiz única ou a mais do que uma. Se tudo se
reduzisse a uma só raiz, não haveria diferença entre esta raiz e o
Único Autor. Além do mais, também seria preciso que esta mesma raiz
fosse só matéria ou só forma (\textsc{ibn} \textsc{gabirol}, 1895,
\textsc{iv}, 6).
\end{quote}

Do mesmo modo como propõe Israeli, a denominação de Criação parece ser
apropriada exclusivamente à saída da matéria e forma primeiras da
Vontade Divina:

\begin{quote}
\textbf{D. ---} Em que sentido se diz que a forma que constitui a
matéria procede da impressão da unidade agente?

\textbf{M. ---} Porque a forma é criada, ou seja, porque a
forma é, e o ser não é impressão, porque a impressão necessita de
algo que a sustente e o ser não precede a substância. E, posto que
não é impressão, é criação, e se o ser se faz por criação e o ser é
próprio da forma, então a forma é criada. (\textsc{ibn}
\textsc{gabirol}, 1895, \textsc{v}, 13)
\end{quote}

Num segundo momento, matéria e forma se reúnem para a formação do
primeiro criado. Este ser difere radicalmente de seu Criador,
especialmente quanto à questão da diversidade e multiplicidade, e
difere também tanto da matéria quanto da forma separadamente, uma vez
que, ao se conjugarem, darão origem a outra coisa. Para Gabirol,
somente o Criador é uno, e todos os seres, até mesmo o mais alto
deles --- a Inteligência --- trazem em si a diversidade, ao serem
compostos por matéria e forma.

\begin{quote}
Deves saber que o ser é de dois modos: o ser em potência, que é
próprio da essência de cada uma delas, ou seja, da matéria em si e da
forma em si, do qual nós já falamos; e o ser em ato, que é próprio da
matéria e da forma quando se unem e se conjugam. (\ldots{}) Isto ocorre
porque, da conjunção da matéria com a forma provém uma outra natureza
composta por elas, que antes não existia em cada uma delas por si;
(\ldots{}) produz"-se, de sua composição e conjunção, um novo significado
que não se encontrava antes em nenhuma delas, porque da conjunção de
coisas diversas surge uma forma que não existia anteriormente em
nenhuma delas. Deves considerar, a partir deste exemplo, o surgimento
de todas as coisas e sua passagem ao ato, quer dizer, ao ser daquilo
que são quando a matéria universal se une à forma universal
(\textsc{ibn}
\textsc{gabirol}, 1895, \textsc{v}.9).
\end{quote}

A Vontade Divina Criadora representa, no sistema de Gabirol, um papel
semelhante ao do \emph{Logos} em Fílon de Alexandria, com a
vantagem de que não oferece riscos à total liberdade e onipotência de
Deus pregada pelo monoteísmo judaico. Enquanto o \emph{Logos}
filônico (deus segundo), de acordo com algumas possíveis leituras,
poderia vir a ameaçar, de certa forma, a figura de Deus enquanto
Criador, a Vontade gabiroliana alcança uma adequação superior entre o
sistema emanacionista neoplatônico e a linguagem bíblica, afastando a
possibilidade de interpretações gnósticas ou dualistas, bem como
reforçando o primado judaico da Vontade de Deus como causa da
existência e duração da realidade criada. 

Com a remoção das demais Potências de seu sistema metafísico e a
substituição das Ideias por um modelo neoplatônico de substâncias
espirituais inteligíveis, seguindo uma terminologia
aristotélica,\footnote{ Algumas notas preliminares sobre o tema da
utilização da terminologia aristotélica por Ibn Gabirol podem ser
encontradas em meu artigo “Neoplatonismo e Aristotelismo no
Hilemorfismo Universal de Ibn Gabirol (Avicebron)” (\textsc{cavaleiro}
\textsc{de}
\textsc{macedo}, 2007).} Gabirol consegue simplificar e organizar o fluxo das
emanações de modo a que se encaixe em qualquer esquema monoteísta,
sem prejuízo de elementos fundamentais à fé judaica, como: a
transcendência absoluta de Deus; o lugar privilegiado do homem na
criação e a negação da eternidade do mundo. A Vontade de Deus, em Ibn
Gabirol, assume um status intermediário entre um atributo unido à
Essência Primeira e uma hipóstase. Nem completamente pertencente a
Ele e indiferenciável do Uno como a entende Plotino, nem hipóstase
totalmente separada como em algumas paráfrases encontradas no
Neoplatonismo Islâmico, a Vontade é um intermediário entre Deus e o
mundo criado que consiste em uma \emph{propriedade} do Criador,
unida a ele em essência, mas separada quanto a seus efeitos, ou seja,
a criação.

\begin{quote}
Logo, a vontade deve ser intermediária entre a essência altíssima e a
forma que deflui da vontade. Mas, quando considerada em si mesma, e
não por sua ação, a vontade então não será nem intermediária nem
finita, mas ela mesma e a Essência são idênticas. (\textsc{ibn}
\textsc{gabirol}, 1895,
\textsc{iv}, 19).
\end{quote}

A filiação neoplatônica de Ibn Gabirol torna"-se bem flagrante pelo
modo como este autor apresenta a sua sequência de seres. O
correspondente gabiroliano do cosmo inteligível é composto por
substâncias simples e espirituais, seguindo a sequência neoplatônica
básica: inteligência, alma tripartida e natureza. A esta sequência
acrescenta"-se a “substância que sustenta a corporeidade do mundo”,
cuja matéria é o inteligível mais baixo que dará origem ao mundo
sensível e corpóreo. Este último é, por sua vez, descrito em
linguagem aristotélica como sendo constituído pela “substância que
sustenta as categorias”. Para Gabirol, nada daquilo que é criado pode
ser absolutamente uno, uma vez que deve diferir radicalmente do
Criador. Portanto, todas as substâncias espirituais, contêm em si a
dualidade, ao serem compostas de forma e matéria, embora esta seja
uma matéria simples, inteligível e incorpórea. Cabe salientar que, a
despeito desta diferença, esta matéria simples e espiritual não se
distingue, em sua substancialidade, da matéria corpórea, sendo a
última simplesmente um grau mais baixo daquela primeira. Toda a
matéria que compõe a Criação é una, concorda e se reduz à matéria
universal primeira. 

\begin{quote}
Mas todas estas matérias e formas inteligíveis concordam quanto à
significação de matéria e de forma. Portanto, tornam"-se universais,
tal como ocorre nas coisas sensíveis, pois, se são matérias
particulares que participam do sentido de matéria, no qual todas são
matérias, é necessário que aquilo no qual concordam seja a matéria
universal. (\ldots{}) e as duas matérias se tornarão uma matéria, e as
duas formas se tornarão uma forma. (\textsc{ibn} \textsc{gabirol},
1895, \textsc{iv}, 7). A
matéria primeira que tudo sustenta é una, reunindo a matéria dos
sensíveis e a matéria dos inteligíveis, até que todas se tornem uma
única matéria (\textsc{ibn} \textsc{gabirol}, 1895, \textsc{iv}, 9).
\end{quote}

Ao buscarmos uma definição de matéria na obra de Ibn Gabirol, uma vez
que esta deveria contemplar tanto a matéria sensível quanto a
inteligível, pouco encontramos, uma vez que ele defende que:

\begin{quote}
Sua definição não é possível; pois não há incidindo sobre elas [a
matéria e a forma] gênero algum que possa servir de princípio para
sua definição. Mas é possível fazer uma \emph{descrição}, em
virtude das propriedades que as acompanham. Assim, a descrição da
matéria primeira, extraída de suas propriedades é a seguinte: é uma
substância que existe por si, que suporta a diversidade, e que é una
em número (\textsc{ibn} \textsc{gabirol}, 1895, \textsc{v}, 22).
\end{quote}

Conforme diversas passagens do \emph{Fons Vitae}, Gabirol aponta que
a matéria e a forma primeiras existem de modo absolutamente simples
na Vontade e são criadas juntas, uma vez que uma não pode existir sem
a outra, em ato, nem por um momento sequer, pois, “A matéria e a
unidade começaram a ser conjuntamente, pois a matéria só é capaz de
ser pela unidade\ldots{}” (\textsc{ibn} \textsc{gabirol}, 1895, \textsc{v}, 9). Este fato poderia
indicar a defesa de Ibn Gabirol da não preexistência da matéria, bem
como sua não coexistência com Deus. Quanto à origem mesma da matéria,
Ibn Gabirol é claríssimo ao afirmar que ela provém diretamente de
Deus, sendo criada diretamente pela essência, enquanto a forma o é
pela propriedade da essência:

\begin{quote}
Da matéria dizemos o mesmo que da forma, isso é, que a matéria é
criada pela essência e a forma pela propriedade da essência, quer
dizer, pela sabedoria e pela unidade, ainda que a essência não seja
determinada por propriedade extrínseca a ela. Esta é a diferença
entre o criador e o criado, porque o Autor é a essência essencial e o
que é criado compõe"-se de duas essências que são a matéria e a forma
(\textsc{ibn} \textsc{gabirol}, 1895, \textsc{v}, 42).
\end{quote}

Uma vez mais, nada pode ser afirmado com absoluta segurança, e ficamos
dependentes da interpretação, já que, conforme o autor antes
afirmara, a essência compõe com sua propriedade uma unidade, sendo
entendida como diferenciada somente a partir dos seus efeitos. Dessa
maneira, 

\begin{quote}
O ser da matéria na sabedoria de Deus é como a existência do conceito
sobre o qual indagas na minha alma, pois, ainda que ele falte em ti,
não se pode concluir disso que ele falte necessariamente também em
mim (\textsc{ibn} \textsc{gabirol}, 1895, \textsc{v},10).
\end{quote}

Portanto, ao analisarmos a filosofia proposta por Ibn Gabirol frente
aos três aspectos apontados --- e arriscamo"-nos a estender aqui também
a Isaac Israeli, pelas semelhanças que foram antes apresentadas ---
concluímos que: a primeira questão, de que nada existia anterior a
Ele e nada pode haver coexistente a Ele, parece estar plenamente
resolvida, uma vez que a matéria é criada por Deus, assim como a
forma, procedendo a primeira da Essência e a segunda da Vontade. Mas,
Ibn Gabirol opta aqui por solucionar a questão da origem da matéria,
em detrimento dos demais aspectos nos quais o dogma judaico se apoia:
quanto ao segundo aspecto, ou seja, à questão da criação dos seres de
uma só vez, tanto Israeli quanto Gabirol filiam"-se completamente ao
modelo neoplatônico de emanações, uma vez que seu entendimento da
questão da criação resume"-se à criação da forma e da matéria
primeiras. Todos os seres inteligíveis, gêneros e espécies dos
sensíveis parecem ter surgido por emanação, sem a concorrência
necessária de um ato criador completo, e se individualizam e
multiplicam por geração. Quanto ao terceiro aspecto investigado, quer
dizer, sobre se Deus cria algo, de alguma maneira a partir de si
mesmo, alguma dúvida pode surgir de passagens como a seguinte:

\begin{quote}
\textbf{D.} --- Já me revelaste estes grandes segredos; Acreditas que
por isso se possa afirmar que o Criador Sublime e Santo está em tudo?

\textbf{M.} --- Certamente por esta razão afirmou"-se isso. Porque a
Vontade, que é virtude Sua, está infusa em tudo e tudo penetra, e
nada existe sem ela, posto que dela procede o ser  e a constituição
de todas as coisas. (\textsc{ibn} \textsc{gabirol}, 1895, \textsc{v}, 39)
\end{quote}

Entendemos, a partir desta passagem, que Gabirol pretende deixar
claro que entende o Criador como \emph{Doador das formas}, o que,
na perspectiva neoplatônica mediada que ele apresenta não quer dizer
necessariamente que Deus crie a partir de si mesmo, mas que Deus
ordena o mundo e confere o ser das coisas que são a partir da força
ou potência criadora que é sua Vontade:

\begin{quote}
\textbf{D.} --- Se a matéria e a forma procedem da essência e da
propriedade, por que se diz que a forma é acrescentada à matéria e de
onde vem a ela?

\textbf{M.} --- A forma vem do superior e a matéria a recebe do inferior
porque a matéria é substrato, dado que possui o ser sob a forma e a
forma está sustentada acima dela.

\textbf{D.} --- E qual é a prova disso?

\textbf{M.} --- A prova disso é que o doador da forma está acima de
tudo; portanto é necessário que sua receptora esteja abaixo dele. E
também como Ele mesmo é o ser verdadeiro, é necessário que o ser
emane d'Ele; e que quanto mais próximo estiver da origem do ser, mais
forte seja sua luz e mais estável em ser. Isso se dá a conhecer pelos
sentidos, porque a substância é um ser mais nobre que o acidente e,
entre os acidentes, a quantidade é mais nobre que a qualidade.
(\textsc{ibn}
\textsc{gabirol}, 1895, \textsc{v}, 42)
\end{quote}

Como complementação a isso, notamos que, Ibn Gabirol insiste na
dessemelhança radical entre o autor e os seres criados, afirmando
que, 

\begin{quote}
A essência primeira não é semelhante à inteligência e não mantém
relação alguma com ela, posto que não está unida a nenhum dos
compostos nem nenhum dos simples; e a relação dos simples para com a
essência, na impossibilidade de conhecê-la, é como a relação do
composto com o simples, na impossibilidade de conhecê-lo (\textsc{ibn}
\textsc{gabirol}, 1895, \textsc{i}, 5). 
\end{quote}

A imagem que se delineia, a primeira vista, é de que Gabirol, nos três
aspectos apontados, concordaria em dois deles com a interpretação
judaica majoritária da Criação e discordaria em um deles, no qual se
apresenta como um neoplatônico. Mas, a questão se revela mais
complicada do que a primeira vista o autor parece defender. Ora, que
a Essência Primeira não mantenha semelhança alguma com os seres
criados fica explicado pela sua absoluta unidade e simplicidade em
contraposição à composição dos seres por duas essências separadas, a
da matéria e a da forma. Mas, e quanto à questão específica da
matéria? Se ela procede diretamente da Essência, não deveria manter
para com ela algum tipo de semelhança, posto que é logicamente
impossível que algo que proceda de outro não mantenha para com este
outro semelhança alguma? E Gabirol parece estar plenamente ciente de
que esta é uma grande dificuldade lógica, tanto que coloca nas
palavras do discípulo a seguinte indagação: 

\begin{quote}
\textbf{D.} --- Se o movimento de todo móvel, e de modo geral, o da
matéria para receber a forma, não se deve senão a seu anseio pelo Ser
primeiro, é necessário que haja alguma semelhança entre eles, pois o
desejo e a união só ocorrem entre semelhantes.
\end{quote}

Mas, nas palavras do Mestre, Ibn Gabirol irá mais uma vez insistir
que,

\begin{quote}
\textbf{M.} --- Entre a matéria e o ser primeiro não há semelhança, a
não ser pelo modo que a matéria adquire a luz e o esplendor do que há
na essência da vontade, que é o que a obriga a mover"-se em direção a
esta e desejá-la; ela não se move, no entanto, para obter a essência
da Vontade, mas a forma que foi criada a partir dela. (\textsc{ibn}
\textsc{gabirol},
1895, \textsc{v}, 32)
\end{quote}

Temos aqui duas questões que se apresentam a partir desta aparente
contradição: se a matéria é criada diretamente da Essência, sua
criação é logicamente independente da forma, que procede da
propriedade criadora; então teríamos que a matéria deva ter sido
criada necessariamente de Deus mesmo, uma vez que nada existe
anterior ou coexistente a Ele. Desse modo, desafiaria a leitura
judaica da Criação, aproximando"-se do neoplatonismo, uma vez que
teríamos aqui a matéria como primeira hipóstase. 

Na via inversa, se o autor realmente crê que a matéria não mantém
semelhança alguma com Aquele que a criou (a Essência Primeira), ela
não pode ter procedido diretamente d'Ele. Portanto, necessariamente,
ou ela deve ter"-se originado de outra fonte, ou ela já existia
previamente junto a Deus. Ao recordarmos a passagem na qual ele nos
adverte que não é possível saber de que modo a matéria já existia na
Sabedoria de Deus, optaríamos provisoriamente por esta segunda
interpretação, pela qual cairíamos num platonismo mais tradicional,
no qual a matéria é preexistente, mas não é propriamente uma
hipóstase. 

Esta opção de interpretação decorre não somente da contradição lógica
que se instala na obra de Ibn Gabirol no que se refere à questão da
procedência da matéria, ou às alusões que ele mesmo faz a esse
respeito, mas também de uma compreensão que era bastante difundida
nos meios judaicos medievais. É sabido que, embora a interpretação
canônica judaica seja a literalista e assuma a criação a partir do
nada, durante todo o período medieval, diversos filósofos judeus
discutiram a questão da Criação frente à Eternidade do Mundo e da
preexistência da matéria. Dentre eles, ao menos Yehudah Ha"-Levi,
Maimônides e Gersônides, ainda que, com reservas, reconheceram a
compatibilidade entre o relato bíblico e a teoria da matéria
preexistente.

Yehudá Ha"-Levi é um pensador tradicionalista, defensor extremo da fé
judaica, inimigo da filosofia aristotélica, e representante do
pensamento rabínico medieval. Escreve sua obra fundamental, \emph{O Cuzari},
para expor a superioridade da fé judaica sobre as demais religiões e
sobre a filosofia. Mas, ainda assim ele não consegue se esquivar de
ter que considerar como válida a teoria da preexistência da matéria.
Ele nos diz que:

\begin{quote}
O problema em torno da criação ou eternidade do mundo é muito profundo
e os argumentos que sustentam ambas as afirmações são equilibrados. O
problema da Criação se resolve pela tradição desde Adão até Noé e é
novamente confirmado por Moisés através da profecia, a qual merece,
de qualquer modo, muito mais crédito que a especulação. Por outro
lado, se um homem amparado na Lei de Moisés admitisse a eternidade da
matéria prima, e acreditasse que houve muitos mundos antes deste
mundo, não haveria por isso defeito em sua fé, pois com isto não nega
que este mundo foi criado há certo tempo e que os primeiros homens
que houve nele foram Adão e Eva (\textsc{halevi}, 2001, \textsc{i}, § 67, p. 39).
\end{quote}

Ha"-Levi não acredita na possibilidade de demonstrar racionalmente
qualquer posição, seja em defesa da criação, seja da eternidade do
mundo, bem como a refutação racional de qualquer dessas
possibilidades. 

Neste ponto é seguido por Maimônides que, no \emph{Guia dos
Perplexos} se propõe a: “Esclarecer os pontos obscuros da Bíblia e
expor explicitamente o verdadeiro sentido de seus fundamentos,
encobertos à inteligência do povo” (\textsc{maimonides}, 1956,
\textsc{ii}, 2). A
Partir do capítulo 13 da segunda parte do \emph{Guia}, Maimônides
se dedica a discutir as teorias sobre a origem do Universo. Aponta
que existem três teorias acerca do tema: a primeira seria aquela que
deriva da Lei de Moisés, e que defende que o Universo foi trazido à
existência da completa não existência, de uma só vez, rejeitando a
eternidade da matéria, do tempo e do movimento; a segunda é a que
advoga a preexistência da matéria, apresentada por ele como “a teoria
de todos os filósofos”, embora ele somente cite Platão; a terceira
afirma a eternidade da matéria, do tempo e movimento e corresponderia
ao pensamento de Aristóteles e de seus seguidores.

A discussão de Maimônides será dirigida à teoria de Aristóteles, uma
vez que, para ele, esta seria a “única merecedora de consideração”
(\textsc{maimonides}, 1956, \textsc{ii}, 14). Passa a refutar os argumentos
apresentados por Aristóteles e os peripatéticos acerca da eternidade
do mundo e a apresentar a teoria da criação como tão plausível quanto
qualquer outra. Seguindo Halevi, não acredita ser possível demonstrar
racionalmente qualquer doutrina sobre a origem do mundo. Para
Maimônides, os argumentos de Aristóteles sobre a eternidade do mundo
não devem ser tomados propriamente como demonstrações, uma vez que
“È, portanto, praticamente impossível inferir, com base na natureza
que uma coisa possui após passar por todos os estágios do seu
desenvolvimento, qual era o estado dela no início do processo”
(\textsc{maimonides}, 1956, \textsc{ii}, 17).

E mais, embora entenda que as teorias de Aristóteles são altamente
adequadas à explicação do mundo sublunar, nega a possibilidade de
compreensão humana sobre qualquer assunto que se refira ao
sobrenatural, e até mesmo à astronomia. Chega a esta conclusão após
discutir as questões obscuras acerca da composição e do movimento das
esferas e dos planetas. A este tipo de especulação o autor se refere
do seguinte modo:

\begin{quote}
É de fato ignorância ou uma espécie de loucura fatigar nossas mentes
com questões que estão fora de nosso alcance, sem a posse dos meios
para nos aproximarmos delas. Devemos nos contentar com aquilo que
está ao nosso alcance e abandonar o que não pode ser alcançado pela
inferência lógica (\textsc{maimonides}, 1956, \textsc{ii}, 24).
\end{quote}

Mais adiante, percebemos que sua avaliação das três teorias e sua
opção por discutir somente as ideias de Aristóteles irá se revelar
providencial. Vemos que, embora Maimônides assim justifique sua
exclusão, a questão não reside na falta de mérito das teorias
platônicas, que não mereceriam consideração. A verdade é que, de
fato, a teoria aristotélica é francamente contraditória com as
doutrinas mosaicas, dado que exclui a possibilidade de milagres,
profecias e intervenção divina, uma vez que o universo como um todo é
fruto das leis naturais, mas, por outro lado, não há como afirmar que
a teoria da preexistência da matéria seja de todo herética frente ao
judaísmo. E Maimônides tem plena consciência disso. Basta ver que, ao
final do capítulo 25 ele nos diz:

\begin{quote}
De todo modo, caso admitíssemos a eternidade do Universo conforme a
segunda das teorias que expusemos --- a de Platão --- de acordo com a
qual os Céus também são transitórios --- não nos oporíamos aos
princípios fundamentais de nossa religião, pois esta teoria não
implica na rejeição de milagres, mas admite a possibilidade
(\textsc{maimônides}, 1956, \textsc{ii}, 25). 
\end{quote}

E conclui afirmando que isto não é necessário, já que nenhuma das
teorias é comprovada; e, uma vez que não existem demonstrações
suficientes, esta é uma mera questão de opinião (\textsc{maimônides}, 1956,
\textsc{ii}, 15), é preferível optar pela tradição.

Por fim, verificamos que, embora a questão da preexistência da
matéria
seja um tema que perpassa todo o pensamento judaico medieval, e que
a
noção platônica da preexistência da matéria jamais tenha sido
refutada ou mesmo considerada totalmente exterior à religião, poucas
são as tentativas efetivas de dar uma resposta inequívoca. Fílon
deixara a questão obscura, optando por afirmações dúbias, que são
interpretadas de modo oposto por seus comentadores;\footnote{ Como
exemplo, Wolfson acredita ser criada; Brehier
discorda.} Halevi e Maimônides optam pela tradição dos Profetas,
cujo argumento de autoridade suplantaria qualquer teoria não
devidamente demonstrada; restam"-nos, então, Isaac Israeli e Ibn
Gabirol, para os quais a questão da matéria constitui componente
essencial de seus modelos metafísicos. 

Retornamos aqui, portanto, a Ibn Gabirol, que nos legou um
pensamento
mais sistemático. Algumas considerações já foram traçadas
anteriormente, como a questão de que dificilmente podemos considerar
a matéria propriamente como uma hipóstase. Especialmente porque não
se pode dizer propriamente que esta possua o ser:

\begin{quote}
Posto que o ser das coisas não é senão pela forma, não é possível que a matéria
sem a forma possua, de modo absoluto, o ser; e se dizemos que a matéria possui o
ser, o que queremos dizer é que possui o ser em potência, ou seja, que ao
receber a forma, passa a tê-lo em ato, e possui o ser em ato. (\textsc{ibn}
\textsc{gabirol}, 1895, \textsc{v}, 8)
\end{quote}

No entanto, para ele, a afirmação, sem mais, que a matéria é
carência
e privação seria incorreta. A matéria é pura potencialidade e,
enquanto potencialidade, podemos dizer que, de algum modo, ela
possui
o ser:

\begin{quote}
\textbf{M.} --- Ainda que se diga que a matéria é privada, não se deve
dizer, no entanto, que não possua ser algum em si mesma --- isso é, o
ser em potência --- embora este seja diferente daquele que possui
quando recebe a forma. Por isso diz"-se da matéria que não possui
absolutamente o ser em ato, porque tem em si o ser em potência e não
o possui em ato a não ser quando se une à forma. E por esta razão o
ser foi descrito como sendo a existência da forma na matéria. Da
mesma maneira, a matéria também não é absolutamente privada, porque
tem em si o ser em potência, isto é, aquele ser que possuía no
conhecimento do Eterno, Excelso e Magno, não composto com a
forma\footnote{ Texto latino: “\emph{similiter etian materia non
est privata absolute quia habet esse in se in potentia, scilicet
illud quod habebat esse in scientia aeterni, excelsi et magni, non
composita cum forma}”.} (\textsc{ibn} \textsc{gabirol}, 1895,
\textsc{v}, 11).
\end{quote}

Podemos compreender que, conforme seu modelo, a concepção da matéria
como privação ou carência seria contraditória com a afirmação de sua
origem divina, pois, como poderia algo que procede da Essência
Essencial, do Mais Alto, do Puro Bem, ser falta, carência e
privação?
Portanto, deve ser entendida como privação do ato, no sentido de que
é pura potencialidade. E esta potencialidade é riquíssima, contendo
em si todas as possibilidades, realizadas ou não. Portanto, nada há
de estranho em afirmar que a matéria existia previamente na
Sabedoria
de Deus e de Deus diretamente procede. A potencialidade absoluta
residia previamente na sabedoria divina, por sua onisciência,
abarcando todos os mundos possíveis. No momento da criação, Deus
escolhe livremente, através de sua Vontade, o melhor modo de
disposição do mundo e ordena aquela potencialidade, atualizando"-a,
ou
seja, conferindo"-lhe as formas. Por isso a matéria deve proceder
diretamente da Essência e a forma da Vontade, porque a forma é
aquela
que confere o ser, ou seja, atualiza o mundo a partir da livre
escolha de Deus.

Há que considerar a contraparte místico/\,religiosa desta
concepção.
Recorrendo à tradição da imagética mística presente na especulação
interna ao judaísmo (\emph{Maaseh Bereshit} e\emph{ Maaseh
Merkavah}), encontramos que, para Gabirol, a forma é pura Luz,
enquanto a matéria é designada ao final do \emph{Fons Vitae} como
o
Trono da Glória, no qual se assenta a Vontade.

Maimônides, no entanto, parece ter dificuldades em explicar as
imagens
utilizadas nas alusões místicas. Examinemos, pois a passagem de
\emph{Pirke Rabbi Eliezer}, que é discutida e interpretada por ele
no \emph{Guia}, reconhecendo sua própria dificuldade:

\begin{quote}
De onde foram criados os Céus? Ele tomou parte da Luz de sua
vestimenta, estendeu"-a como um manto e assim os céus se estendem
continuamente, conforme está dito: “Envolvido em Luz como um manto”
(Salmo 104:2). De onde foi criada a Terra? Ele tomou da neve sob o
seu Trono de Glória e a espalhou, segundo o dito: “Disse à neve:
seja
terra!” (Jó, 37:6). (\textsc{maimônides}, 1956, \textsc{ii}, 26).
\end{quote}

Maimônides considera esta uma passagem estranha por sugerir que a
\emph{Luz de Sua Vestimenta} e o \emph{Trono da Glória} são
elementos que já existiam antes da Criação. Ele defende que esta
interpretação deva ser rejeitada, pois implicaria em assumir a
eternidade do universo conforme a teoria de Platão. O grande rabino
reconhece, no entanto, que não se considera capaz de explicar esta
passagem a contento; e vale ressaltar que a ideia não é uma novidade
medieval, já que a eternidade do Trono é afirmada expressamente nas
Escrituras: “Tu, \textsc{yhvh}, para sempre, sentarás em teu Trono, de
geração
em geração” (Lamentações 5:19). Opta então por entender esta
passagem
como uma alusão a que a substância dos céus (esferas e planetas) é
distinta da substância terrestre, da qual todos os seres
compartilham
no mundo sublunar.

Ora, se olhássemos para esta passagem com os olhos de Ibn
Gabirol
teríamos uma interpretação consideravelmente diferente. Não podemos
esquecer que Gabirol, assim como Israeli foi adepto da leitura
alegórica das Escrituras, em busca de seu significado místico e, a
partir dele renasceu esta tradição exegética na Espanha Medieval.
Assim, podemos sugerir a seguinte interpretação
místico"-especulativa:

A passagem “\emph{Ele tomou parte da Luz de sua vestimenta,
estendeu"-a como um manto}” pode ser interpretada como a saída da
forma universal primeira (a \emph{Luz}) da Vontade Divina (que
aqui
figura como a \emph{Vestimenta}). A referência a “\emph{parte}
da
Luz” deve"-se ao fato de que, ainda que a forma provenha da Vontade,
ela não contém toda a Luz que há na Vontade, conforme Ibn Gabirol
nos
explicara. “Estendeu"-a num manto” refere"-se à constituição das
substâncias inteligíveis, a partir das quais o mundo sensível será
criado.  O conjunto dessas substâncias inteligíveis é precisamente o
significado de “Os Céus”, ou seja, aquilo que há acima deste mundo e
que contém mais propriamente a luz divina, ou seja, o mundo
inteligível, e não os seres celestes corpóreos, como pensara
Maimônides.

Quanto à segunda parte: “\emph{De onde foi criada a Terra? Ele
tomou
da neve sob o seu Trono de Glória e a espalhou”,} assim como “os
céus” podem ser interpretados como o mundo inteligível, “a Terra”
refere"-se ao mundo sensível. Se o Trono é a matéria na qual se
assenta a Vontade/\,Vestimenta, conforme Ibn Gabirol afirma no final
do
\emph{Fons Vitae}, a “neve sob o seu Trono” é precisamente a
porção
inferior e mais densa dessa matéria, indicando que dela foram
criadas
as coisas sensíveis. O que há de mais sutil é criado diretamente do
Trono --- a matéria inteligível. Por outro lado, há que observar que a
imagem da neve sugere que, embora esta seja a matéria mais bruta,
densa e corpórea, ainda nela há luminosidade e brilho provenientes
da
Vontade, o que é sugerido pela imagem da brancura e aspecto
cristalino. Por fim, o elemento de ligação entre as duas partes,
“\emph{e assim os céus se estendem continuamente}” refere"-se
justamente à descida da forma (enquanto Luz segunda defluída da Luz
primeira) do mais alto até o extremo inferior da criação.

Desse modo, acreditamos que, ainda que tenha existido um longo
namoro
entre Judaísmo e Neoplatonismo, será na obra de Ibn Gabirol que a
especulação filosófica judaica e as teorias platônicas e
neoplatônicas sobre a origem do mundo irão atingir seu perfeito
casamento. E, havendo ou não comprovação da transmissão direta das
ideias de Fílon de Alexandria, nota"-se que uma tradição mística
especulativa interna ao judaísmo, embora minoritária e talvez
secreta, atravessou os séculos. Associando as duas tradições e
dispondo da chave de interpretação, que é justamente a leitura
alegórica, a compreensão de certas passagens, tão obscuras ao
judaísmo rabínico, torna"-se clara.

\section{Bibliografia}

\begin{description}\labelsep0ex\parsep0ex
\newcommand{\tit}[1]{\item[\textnormal{\textsc{\MakeTextLowercase{#1}}}]}
\newcommand{\titidem}{\item[\line(1,0){25}]}
\tit{ALTMANN}, A.; \textsc{stern}, S. M. \emph{Isaac Israeli, a Neoplatonic
Philosopher of the Early Tenth Century} .Oxford University Press,
1958.

\tit{AVENCEBROLIS (IBN GABIROL)}. \emph{Fons vitae ---} \emph{ex Arabico
in
Latinum translatum ab Iohanne Hispano et Dominico Gundissalino;} ex
codicis Parisinis, Amploniano, Columbino primum edidit Clemens
Baeumker. Münster: Aschendorff, 1895.

\tit{BARTHÉLEMY}, D. 1967. ``Est"-ce Hoshaya Rabba qui censura le
«~commentaire Allégorique~» ; a partir des retouches faites aux
citations bibliques, étude sur la tradition textuelle du
commentaires
Allégorique de Philon''. \emph{In} Philon D'Alexandrie. Colloques
Nationaux du Centre National de La Recherche Scientifique. Lyon,
11--15 septembre 1966. Paris : Éditions Du Centre National de la
Recherche Scientifique : 57--8.

\tit{CAVALEIRO DE MACEDO}, C. 2007. \emph{Neoplatonismo e Aristotelismo
no
Hilemorfismo Universal de Ibn Gabirol (Avicebron)}. Veritas, 52:
132--148.

\tit{HALEVI}, Y. \emph{El Cuzary},  El Libro de La prueba y de la
demostración en la defensa de la religión menospreciada. Barcelona:
Ediciones Indigo, 2001.

\tit{KATZ}, S. \emph{Mysticism and Language}. Oxford: Oxford University
Press, 1992.

\tit{MAIMONIDES}, \emph{Guide for the Perplexed}, translated from the
original arabic text by M. \textsc{friedlander}. 2nd. Edition. New York:
Dover
Publications, 1956.

\tit{PHILON D'ALEXANDRIE}.\emph{ De Opificio Mundi.} Trad. Roger
Arnaldez,
Paris: Editions du Cerf, 1961\emph{.} 

\tit{RUNIA}, D. \emph{Philo of Alexandria : On The Creation of the
Cosmos
according to Moses}. Introduction, translation and Commentary by
David T. Runia. Leiden: Brill, 2001.

\tit{SCHOLEM}, G. \emph{Desarrollo Histórico e Ideas Básicas de la
Cábala}\emph{,} Barcelona: Riopiedras, 1994.

\tit{SIMON}, M. 1967. \emph{Situation du Ju}\emph{daïsme Alexandrin
dans
la diaspora}. \emph{In} Philon D'Alexandrie, Colloques Nationaux
du
Centre National de La Recherche Scientifique. Lyon, 11--15 septembre
1966. Paris : Éditions Du Centre National de la Recherche
Scientifique: 17--31.

\tit{STROUMSA}, S. 1991 \emph{The impact of Syriac Tradition on Early
judaeo"-arabic bible exegesis}. \textsc{aram}, 3 (1\&2): 89--96. 

\tit{WOLFSON}, H. A. \emph{P}\emph{hilo:} \emph{foundations of
religious philosophy in Judaism, Christianity and Islam.} Cambridge,
Harvard University Press, v. \textsc{ii}, 1982.
\end{description}



\capitulo{Platonismo e cristianismo na doutrina do \emph{logos} de Fílon de Alexandria}%
	{Gisele Amaral}%
	{ufrn}

\markboth{Platonismo e cristianismo na doutrina do logos\ldots{}}{Gisele Amaral}

Desde a primeira metade do século \textsc{xix}, diferentes interpretações
e métodos foram construídos no âmbito da pesquisa sobre Fílon. No
século \textsc{xx}, essa pesquisa orientou"-se de modo mais específico
para o platonismo, para o estoicismo, para o judaísmo
propriamente dito, para a gnose, para a ciência da religião e
assim por diante. Recentemente, no ano 2003, especialistas do
cenário acadêmico internacional realizaram em Eisenach e Jena o
\emph{\textsc{i}. Internationales Symposium zum Corpus
Judaeo"-Hellenisticum}, cuja proposta central foi a discussão
acerca da vinculação entre o cristianismo e o pensamento de Fílon
de Alexandria, ou mesmo de sua dependência. A publicação dos
anais deste evento foi intitulada \emph{Philo und das neue
Testament} (\emph{Fílon e o Novo Testamento}), marcando, desse
modo, esta orientação como a de maior destaque na pesquisa sobre
Fílon dos últimos anos. Tanto no rol das discussões do referido
Simpósio, como na bibliografia filônica produzida a partir do
século \textsc{xix}, a doutrina do \emph{logos} é tratada como
essencial na obra de Fílon, não apenas pela frequência do termo,
que ocorre inúmeras vezes ao longo do \emph{corpus
philonicum,} mas principalmente pelo lugar privilegiado que o
\emph{logos} ocupa na justificação divina empreendida pelo
autor. A repercussão da doutrina do \emph{logos} de Fílon em
autores do cristianismo primitivo vem sendo, portanto, cada vez
mais amplamente investigada, atraindo a curiosidade de alguns,
para os quais ainda são questões cruciais, por exemplo, sobre
como Fílon entendeu o \emph{logos}? Como é possível realizar um
estudo sistemático do \emph{logos} numa obra tão ``eclética'
como a de Fílon de Alexandria? De que modo podemos vincular a
doutrina do \emph{logos} Fílon a uma pré-história do
cristianismo? À luz dessas perguntas, será exposta a presente
comunicação.

Estima"-se que Fílon tenha nascido ca. 25 a.C. e provavelmente
morrido em torno do ano 50 d.C.~e, segundo seu próprio relato em
\emph{De legatione ad Gaium},\footnote{ \textsc{philo}.
\emph{De legatione ad Gaium}, 22, 150, p.~567. \emph{Complete
Works} \emph{In Ten Volumes: and two supplementary volumes},
Cambridge : Harvard University Press, The Loeb Classical
Library; London : William Heinemann \textsc{ltd}, 1987, Vol.
\textsc{x}.}
viveu no mais importante núcleo cultural da Antiguidade tardia,
a cidade de Alexandria, no Egito. Fílon de Alexandria cresceu na
ambiência religiosa do judaísmo de seus pais, mas foi educado na
língua grega e nos padrões da cultura helenística, uma
modalidade de integração que foi denominada judaísmo
helenístico. O helenismo correspondeu a uma época da história da
humanidade de maior interação entre povos e culturas no mundo
antigo. Sob o termo de inspiração grega, entendeu o historiador
alemão Gustav Droysen tratar"-se também de uma época de
sincretismo\footnote{ O termo
\emph{synkretismos} significa originalmente a união política,
a federação. No contexto histórico referido, sincretismo designa
porém um processo típico do período imperial de crença em uma fé
comum a todas as culturas. Posteriormente, o termo foi sendo
substituído por \emph{theokrasia} quando usado para referir a
uma combinação de sistemas teológicos.}, de mistura de cultos
gregos e religiões do Oriente. As famosas expedições
alexandrinas moveram exércitos, anexaram territórios,
conquistaram e edificaram cidades, mas junto com esse processo
de expansão descortinaram, sobretudo, tradições que até então
não se haviam ainda reciprocamente visitado. A Alexandria
egípcia foi, certamente, a principal referência a essa
diversificada e complexa confluência e, no tempo de Fílon, a
cidade ainda mantinha sua condição de centro intelectual,
político e comercial do mundo helenístico. Fílon de Alexandria
não só viveu esse apogeu, como em muito contribuiu para o seu
enriquecimento.

As diversas tradições que ali se encontraram, interagiram e
provocaram um ambiente de influências e misturas refletidas
também nos setores do conhecimento humano: nas artes, na
literatura, na língua, na filosofia etc. Antigas tradições
oriundas do período pré-helênico também puderam ali ser
revividas, como foi o caso, por exemplo, do \emph{orfismo},
numa nova elaboração empreendida pelo \emph{neopitagorismo}. O
encontro entre tradições tão variadas tanto provocou
antagonismos, quanto assimilações e transformações. O
\emph{neopitagorismo} influenciou o \emph{platonismo} em
voga e alcançou o pensamento judeu através de Fílon de
Alexandria, que por sua vez inspirou os primeiros teólogos
cristãos, representados principalmente por Clemente de
Alexandria (c.150--215d.C.), Orígenes (185--253d.C.), dentre
outros.

A filosofia do judaísmo alexandrino, graças a Fílon, não pôde
deixar de associar"-se à tradição da filosofia grega, uma
consequência da proximidade entre a intelectualidade judaica de
Alexandria e as escolas filosóficas gregas difundidas até então.
Essa filosofia judaico"-helenística se somou ao mencionado
sincretismo oriundo também do contato entre filosofias gregas e
religiões orientais. Embora empenhada na absorção da filosofia
ateniense clássica, a filosofia em voga em Alexandria era
orientada por um modo filosófico"-religioso de pensar que
transgredia os limites imanentes da filosofia grega, servindo
como parâmetro não só para o judaísmo em seu esforço de
legitimação intelectual, como também para os cristãos
emergentes. As contaminações mútuas desse período marcaram toda
a época por um intenso
ecletismo.\footnote{ Cf.
\textsc{donini}, Pierluigi. «The history of the concept of
Eclecticism~». In: \emph{The Questions of ``Eclecticism'' ---
Studies in Later Greek Philosophy}, (ed. J.M.Dillon \& A.A.
Long), California : \textsc{ucp}, 1996.} A participação de Fílon nesse
movimento fez com que seus escritos fossem tomados ora por seu
helenismo filosófico, ora pela subordinação de sua filosofia às
palavras da \emph{Sagrada Escritura}. De certo modo, podemos
afirmar que Fílon seguiu a ambiência eclética de seu tempo,
participando da transformação e expansão da cultura clássica no
mundo helenizado.\footnote{ Cf.
\textsc{boys"-stones}, G.R., \emph{Post"-Hellenistic Philosophy.
A Study of its Development from the Stoics to Origen}, Oxford :
\textsc{oup}, 2001.} Tais interseções propiciaram o surgimento de novas
ideias, a reflexão sobre temas antigos e, principalmente, a
exploração de argumentos e conceitos que permaneceram no cenário
filosófico"-religioso ao longo de vários séculos subsequentes.

A exposição dessa ambiência histórica é relevante para a
compreensão do estilo eclético característico da filosofia de
Fílon de Alexandria. Sua obra é marcada, de um lado, pelas
grandes correntes do pensamento grego, tais como o
\emph{estoicismo}, a mística do \emph{neopitagorismo}, o
\emph{ceticismo} e, sobretudo, a academia na versão do chamado
\emph{médio"-platonismo}, todas de certo modo populares no
mundo helenístico. De outro lado, nela também encontramos os
ingredientes essenciais da religião judaica, bem como a
repercussão de uma mística oriental, particularmente de origem
persa.\footnote{ Cf. \textsc{whittaker}. \emph{The
Neoplatonists}. Zurich : Georg Olms, 1912, pp. 218--225.
Whittaker concorda com a teoria de Reitzenstein, segundo a qual
a real origem da gnose remonta até o período do primeiro Império
Persa.} A tal ponto que Fílon é reconhecido como uma importante
fonte de alguns elementos do judaísmo helenístico tomado pelos
gnósticos.\footnote{ \textsc{pearson}, B.A.,~«~Philo and
Gnosticism~», \emph{Aufstieg und Niedergang der römischen
Welt}, (Berlin), 1984 \textsc{ii} :21,1, pp. 295--342.~} Para a
historiografia da religião judaica, Fílon é acima de tudo um
teólogo e chega a ser referido como o primeiro teólogo ao qual
se pode verdadeiramente assim nomear.\footnote{ «~Er ist
Theologe, und was er treibt, ist Theologie~». In :
\textsc{Bousset}, Wilhelm. \emph{Die Religion de
Judentums im Späthellenistischen Zeitalter}, Tübingen : J.C.B.
Mohr, 1966, p.~170.} Justamente como consequência da influência
da filosofia helenística, a teologia que Fílon praticava nos
domínios da diáspora não tinha o mesmo caráter da jurisprudência
palestina, que tradicionalmente entendia a teologia como
orientação dos homens para a lei. A transformação da Bíblia em
um livro helenístico, a partir da tradução da \emph{Torá} para
o grego, a chamada tradução dos \emph{Setenta}, foi de grande
relevância para a expansão do entendimento do mundo judeu e se
tornou um instrumento importante na criação da propaganda
judaico"-cristã. Consequentemente, a interpretação bíblica
helenizada de Fílon influenciou os primeiros Padres da Igreja e
repercutiu na história da formação da
teologia\footnote{ Cf. Verbete
``Theologie'' : \emph{Lexikon philosophischer Grundbegriffe der
Theologie}, Franz Albert, Wolfgang Baum, Karsten Kreutzer.
Freiburg ; Basel ; Wien : Herder, 2003.} cristã, especialmente
na configuração de sua
dogmática. A grande obra exegética de
Fílon tornou"-se paradigma de interpretação da \emph{Sagrada
Escritura}, vinculando de modo inextrincável  paganismo grego e
judaísmo. A exegese de Fílon é um ponto de encontro entre noções
e ideias gregas e judaicas; por isso, Fílon é contemporaneamente
reconhecido como um exegeta do seu
tempo.\footnote{ Cf. \textsc{borgen},
P. \emph{Philo of Alexandria, an exegete for his time}.
Leiden/\,New York/\,Köln : Brill, 1997.} A transição entre a
Antiguidade pagã e o novo referencial religioso do Ocidente,
dirigido pelo advento de Cristo, contou em última instância com
esse cenário de contaminações dentro do qual destacamos a
participação de Fílon de Alexandria. Como é possível identificar
essa transição?

A obra de Fílon é impressionantemente extensa e parece ter sido
preservada pelos cristãos em sua quase totalidade, por isso Fílon
é o autor mais importante em se tratando do Segundo Templo
Judeu. Sua contemporaneidade em relação aos primeiros movimentos
cristãos, aliada ao profundo conhecimento da Lei Judaica e,
ainda, à orientação filosófica de sua exegese bíblica fizeram de
Fílon uma referência atraente, sobretudo para aqueles que
buscavam respaldo teórico para justificar a boa nova do Cristo.
Mas o reconhecimento da importância de Fílon para um melhor
entendimento das origens do cristianismo não está isento de
complexidade e, não raro, de grande controvérsia, pois as
conexões são mais indiretas do que propriamente diretas,
principalmente quando se trata de cotejar tratados de Fílon e
documentos do {Novo Testamento}. A pesquisa
Neo"-Testamentária, em seus vários segmentos, ora admite conexões
indiretas, uma vez que Fílon pertencia ao judaísmo alexandrino,
ora não admite, justamente por isso, conexão alguma, uma vez que
não é possível desvincular o judaísmo de Fílon das escolas
filosóficas de origem grega. Segmentos mais ortodoxos da
teologia cristã combatem a hipótese de qualquer influência da
tradição filosófica grega na fundamentação da fé cristã. Essa
controvérsia, porém, não será aqui examinada. 

Propomos uma análise sucinta acerca do entendimento do
\emph{logos} em Fílon de Alexandria, com o intuito de melhor
compreender essa interseção entre a filosofia grega, a teologia
judaica e o cristianismo nos primeiros séculos de nossa era.
Primeiro, porque se trata de um tema essencial na obra de Fílon,
não apenas pela frequência do termo, que ocorre inúmeras vezes
ao longo de todo \emph{corpus philonicum},\footnote{ Cf.
\emph{Index Philoneus}, verbete \emph{logos}.} mas
principalmente pelo lugar privilegiado que o \emph{logos}
ocupa na sua justificação divina. Além disso, porque o
\emph{logos} é um conceito fundamental tanto na tradição da
filosofia grega, quanto na explicação do mistério do Cristo.
Afinal, o começo de uma nova era significou também o começo de
uma nova experiência do conceito de \emph{logos}. Fílon é um
autor"-chave para a compreensão dessa passagem.

Sintaticamente, \emph{logos} é um substantivo masculino grego
relacionado com o verbo \emph{lego}, o qual, mesmo antes de
ser caracterizado por uma terminologia do pensamento
especulativo, merecia destaque também na linguagem profana dos
gregos. Não foi por acaso que justamente esse termo tenha sido o
termo que melhor definiu a ``novidade'' da postura filosófica dos
gregos ante a tradição do \emph{epos} e do \emph{mythos}. O
\emph{logos} significava ``palavra'' enquanto expressão do
pensamento, ``unidade discursiva'', ``narrativa prosaica'' e,
também, ``enumeração'', ``cálculo'', ``legitimidade'', ``proporção'',
``comprovação'', ``relação'', ``demonstração'', ``explicação'' etc. Essa
abrangência semântica do \emph{logos} grego é, todavia,
justamente o que torna tão difícil delimitar ou privilegiar um,
dentre tantos significados possíveis. A maior evidência desta
dificuldade está, por exemplo, no grande desafio que a língua
grega nos provoca quando arriscamos traduzir o termo
\emph{logos} por outro equivalente. Esta limitação não é uma
precariedade da nossa língua, mas de todas as línguas. Mais do
que isto, trata"-se de uma precariedade intrínseca a todo empenho
de tradução. Por isso, não é possível separar a tradição da
tradução, do mesmo modo que a tradução também não é suficiente
para esgotar toda a experiência de uma tradição já constituída.

Fílon herdou os sentidos e usos mais arcaicos desse \emph{logos}
grego através das correntes filosóficas de Alexandria, no
entanto, sua doutrina do \emph{logos} não está desvinculada do
Deus judaico, nem tampouco de sua profunda devoção ao evento da
criação divina. Conceitos como \emph{logos}, ideia, alma,
força, luz, inteligência, dentre tantos outros, são conceitos
utilizados por Fílon para descrever o que é diretamente
indescritível acerca de Deus, já que para ele Deus não pode ser
diretamente conhecido e apenas percebido extra"-sensivelmente. A
apresentação da doutrina do \emph{logos} de Fílon tem sempre
como referência a essência mesma de Deus. Por isso, todas as
aparições de Deus no \emph{Antigo Testamento} são, de acordo
com a interpretação de Fílon, intermediadas pelos anjos mais
elevados ou pelo \emph{logos}.

Inspirado na doutrina platônica das ideias, Fílon comparou o
\emph{logos} à totalidade das ideias enquanto modelo de toda a
criação. Como modelo do universo, o \emph{logos} é também
imagem de Deus. As ideias, por sua vez, são idênticas ao que
Fílon denomina ``potências'', ou ``forças'' (\emph{dynameis},
\emph{logoi}). Por isso, o \emph{logos} como razão divina é
também a totalidade das forças. Para a criação do universo, Deus
se serve das forças ou ideias incorpóreas e essas forças o
cercam. Dentre elas, duas são fundamentais: a força criadora
(\emph{poietik\={e}}) e a força governadora, dominadora
(\emph{basilik\={e}}). Ambas são dependentes do \emph{logos} e
Fílon esclarece essa relação a partir do seguinte esquema: o
\emph{logos} está entre Deus e as forças, isto é, acima das
forças e abaixo de Deus.

Mas se Deus revela"-se no mundo através das forças, as quais estão
subordinadas à razão divina ou \emph{logos}, pois o
\emph{logos} é a razão divina, então o \emph{logos} pode
significar palavra, na medida em que a palavra é também um modo
de aparição de Deus. No entanto, Fílon esclarece que a razão
divina corresponde ao \emph{logos} \emph{endiathetos}, isto
é, o \emph{logos} da reflexão, do pensamento, enquanto o
\emph{logos} que se exterioriza é o \emph{logos}
\emph{prophorikos}. A palavra proferida é a expressão do
pensamento divino. Essa é uma distinção essencial, pois Fílon
precisa assegurar a eficiência da palavra proferida pelo Deus
judaico. Sabemos que no \emph{Antigo Testamento} não há
diferença entre o querer de Deus e sua ação, Deus quer e assim é
feito. A força do querer e do fazer de Deus é revelada através
de sua palavra. A força da palavra de Deus é criadora da
realidade. Mas como palavra saída de Deus o \emph{logos} não é
nem não criado, como Deus, nem criado, como um homem; um
esclarecimento importante, pois segundo Fílon Deus nunca pára de
criar. Se ele não pára de criar, então o \emph{logos} divino
seria mais uma criação sua, como tantas outras. Para manter a
divindade da razão divina, isto é, para assegurar a
superioridade do \emph{logos} de Deus, Fílon define esse
\emph{logos} como ``o inteligível mais velho'',\footnote{
\emph{Fug}. 101 \emph{ton noeton apax apanton o
presbytatos}} como ``o primeiro nascido de toda a
criação''\footnote{ \emph{Migr}. 6 \emph{o logos o
presbyteros ton genesin eilephoton}} e, ainda, como ``o filho de
Deus''.\footnote{ \emph{Agr}. 51 u.a.}

Com isso, ele pretende evitar que o \emph{logos} seja tomado
como obra de Deus, afinal o \emph{logos} divino é também uma
ferramenta de Deus na criação do universo. Deus é a causa
eficiente do universo, pois é o seu autor; mas quando Deus é
causa de algo (no caso, do universo) através da criação, a
criação ela mesma é uma causa instrumental ou exemplar: “a
sombra de Deus é sua razão, a qual ele utilizou na criação do
universo como ferramenta. Esta sombra, ou imagem, como se pode
chamar é, por sua vez, o modelo (\emph{archetypon,
paradeigma}) de outra coisa”.\footnote{
\emph{Somn.} \textsc{i}, 239} Pois, assim como Deus é o modelo da
imagem, que aqui se chama sombra, a imagem será modelo para
outra imagem. Desse modo, não haveria diferença entre causa
instrumental e causa exemplar.

O \emph{logos} como instrumento na criação não é apenas causa
exemplar. O \emph{logos} é também o instrumento cortante, que
separa e que divide a matéria informe. Por isso, é chamado
\emph{logos} \emph{tomeus}, a ferramenta mais afiada de
Deus. Através dele, Deus distingue a matéria em seus quatro
elementos: fogo, ar, água e terra; e divide os seres vivos em
animados e inanimados.  Em sua função instrumental, o
\emph{logos} é a causa da harmonia do universo, pois através
do \emph{logos} os opostos são corretamente relacionados e
somente nesse equilíbrio é possível unir, por exemplo, corpo e
alma, matéria e forma, sensibilidade e inteligência etc. Fílon
adota esse sentido do \emph{logos} como harmonia dos opostos a
partir de Heráclito, cuja obra \emph{Sobre a natureza} ele
cita, por exemplo, em \emph{Qu.in Gen}., \textsc{iii} 5. Todavia, ele
não adota o \emph{logos} heraclítico sem modificá-lo. Mesmo
porque Fílon considera que Heráclito recebeu de Moisés sua
doutrina do \emph{logos}.

Mas foi a partir da observação estoica do \emph{pneuma} como
força viva que Fílon definiu \emph{logos} como \emph{pneuma},
uma força privilegiada capaz de atravessar e de unir todas as
coisas.  Fílon reconhece no \emph{pneuma} uma força na qual se
baseia a ligação entre coisas inanimadas e ainda como uma força
que preenche tudo, já que o \emph{logos} penetra todo o
universo. “Deus criou o homem da terra e em seu rosto soprou o
hálito da vida, e o homem tornou"-se animado”.\footnote{
(\emph{Leg. All}. \textsc{i}, 31, sobre o {Gênesis} 2,7)} Graças
ao sopro divino, o homem é dotado de uma alma inteligente e
viva, sem a qual ele não seria capaz da virtude. Deus gravou no
homem a inteligência através de forças que podem ser
reconhecidas pelo pensamento. O homem recebeu a inteligência a
partir do \emph{pneuma} divino e por isso ele pode conhecer
Deus. O \emph{pneuma} de Fílon é um princípio intelectual.

Ao final de nosso exame da doutrina do \emph{logos} em Fílon,
não podemos deixar de mencionar a questão sobre a personificação
do \emph{logos} filônico. De fato, uma hipostasia do
\emph{logos} não é declarada em seus textos, ao menos não
explicitamente. Permanece como questão: como é originado? Uma
pergunta que deve ser estendida aos seres intermediários que
estão reunidos sob o \emph{logos}. Diferentemente da
hipostasia do \emph{logos} encontrada no {Evangelho de
são João}, que revela a pessoa conhecida, o \emph{logos}
filônico pode ser entendido como hipostase, não no sentido de
uma existência fora de Deus, mas no sentido de uma
personificação de Deus. Essa tese está ligada à referência de
Fílon a um \emph{deuteros theos}, um segundo Deus, que é nada
menos que o \emph{logos} de Deus --- \emph{ho theou}
\emph{logos}. Todavia, a personificação de Deus no
\emph{logos} não pode significar de modo algum que Fílon tenha
dividido Deus em duas partes, nem tampouco que tenha
multiplicado Deus por dois! A necessidade de se pensar o
\emph{logos} personificado de Fílon decorre das funções
descritas para o \emph{logos}, as quais são consideradas
indignas para Deus e por isso mantidas longe dele. Deus não deve
se ocupar com a matéria, por isso suas ferramentas devem ser
essencialmente diferenciadas. Contudo, a pergunta sobre a
pessoalidade ou impessoalidade do \emph{logos} em Fílon não se
coloca no âmbito de sua própria especulação; seu ponto de
partida é sempre exterior e posterior, ainda que de grande
relevância.

Fílon, em última instância, procura mostrar como os homens se
aproximam de Deus através do \emph{logos}. Esse entendimento o
coloca numa posição intermediária entre o \emph{logos}
impessoal grego e aquele que se tornou o \emph{logos}
antropomórfico dos cristãos. No {Novo Testamento}, o
\emph{logos} não significa unidade ou harmonia entre
contrários, nem tampouco tem função instrumental. Para o
cristão, o \emph{logos} é personificado como Filho de Deus. A
representação do \emph{logos} encarnado no {Evangelho
Segundo são João} explica o mistério do Cristo, o qual é
identificado com o próprio \emph{logos} de Deus, e para nós
supõe o paradigma do \emph{logos} \emph{mesites} (o
\emph{logos} intermediário) inaugurado por Fílon. Ao mesmo
tempo, esse \emph{logos} é reconhecido como intermediário
indispensável para que haja relação entre Deus e os homens. A
sublimação de Deus não permite que ele entre numa relação direta
com os homens, por isso o conhecimento de Deus deve ser mediado
pelo conhecimento do Filho de Deus. São João corrobora esta
condição com o seguinte testemunho: “a Deus ninguém jamais viu:
o Filho unigênito, o que está no regaço do Pai olhando"-o frente
a frente, ele é quem o deu a conhecer”. ({João} 1:16--20)

\capitulo{Ser, unidade e bem em Boécio}%
	{Juvenal Savian Filho}{unifesp}

\markboth{Ser, unidade e bem em Boécio}{Juvenal Savian Filho}

Nosso intuito, aqui, é o de analisar um raciocínio de Boécio,
registrado na \emph{Consolação da filosofia}, pelo qual o
autor romano passa da afirmação de que todas as coisas desejam o
bem à afirmação de que há um único bem desejado por todas as
coisas.\footnote{ Cf. Boécio, \emph{Consolação da filosofia}
\textsc{iii}, prosa 11, n. 38 [95].}

Esse raciocínio apresenta grande semelhança com um trecho da
\emph{Ética nicomaqueia} de Aristóteles,\footnote{ Cf.
Aristóteles, \emph{Ética nicomaqueia} \textsc{i}, 1, 1094a18--22.} o
qual já serviu de base para se dizer que o Estagirita teria
incorrido numa falácia, a
\emph{boy"-and"-girl"-fallacy}.\footnote{ Cf. Peter T. Geach,
“History of a Fallacy”. In: \textsc{geach}, P. T. \emph{Logic Matters}.
Oxford: Oxford University Press, 1972, pp. 1--13.} Com efeito,
ao partir de (a) “todas as nossas ações visam um bem”, e chegar
a (b) “há um bem visado por todas as coisas”, Aristóteles teria
apresentado um raciocínio inválido, como quando se pretende
partir de (a) “se todo menino ama alguma menina”, para chegar a
(b) “então, há uma menina amada por todo menino”.

Do ponto de vista formal, a passagem de (a) para (b) resulta
explicitamente injustificada: (a) ${\forall}$x${\exists}$y(xRy);
(b) ${\exists}$y${\forall}$x(xRy). Essa forma parece ser a mesma
do raciocínio de Boécio.

Não é o caso, aqui, de discutir se Aristóteles cometeu ou não
essa falácia.\footnote{ Para isso, remetemos ao estudo de Marco
Zingano, “Eudaimonia e bem supremo em Aristóteles”. In:
\textsc{zingano},
M. \emph{Estudos de ética antiga.} são Paulo: Discurso, 2007,
p. 98.} O que nos interessa é que a mesma passagem de (a) para
(b), presente na \emph{Ética nicomaqueia}, reproduz"-se na
\emph{Consolação da filosofia}, de maneira que podemos
perguntar pelo significado a ela atribuído por Boécio.

\section{A identificação entre ser, unidade e bem}

A estrutura do texto em que essa passagem ocorre fornece
certamente a sua chave"-de"-leitura, pois é somente após haver
identificado ser, unidade e bem que Boécio afirma a existência
de um único bem desejado por todas as coisas, visto que elas
todas desejam um bem.

Assim, a estratégia argumentativa de Boécio consistirá em dizer
que, ao desejar um bem, todas as coisas desejam o mesmo, de
maneira que, para entendermos o sentido da passagem aqui
analisada (de (a) para (b)), precisamos, antes de tudo,
investigar o modo como Boécio obtém essas duas afirmações.

O núcleo da argumentação de Boécio, conforme o texto, é a
identificação entre ser, unidade e bem, impondo"-nos, portanto,
começar por ela para chegar à compreensão da passagem de (a)
para (b).

Um modo de abordar essa identificação seria enfocá-la a partir da
história da doutrina sobre as características transcendentais do
ente, cujas raízes remontam certamente a Platão e Aristóteles,
embora ela tenha recebido a forma e o vocabulário da
transcendentalidade apenas no século \textsc{xiii}, com a divulgação da
\emph{Metafísica} do Estagirita.\footnote{ Cf. J. A. Aertsen,
“Good as transcendental and the transcendence of the Good”. In:
\textsc{macdonald}, S. (ed.). \emph{Being and Goodness. The concept of
the good in metaphysics and philosophical theology}. Londres:
Cornell University Press, 1991, pp. 56--73.} Boécio, nessa
história, teria sido um autor estratégico, pois teria iniciado
um tratamento eminentemente lógico do tema.

A problemática presente no livro B da \emph{Metafísica} é um
bom ponto de partida para avaliar os inícios dessa história. Com
efeito, discute"-se aí sobre a possibilidade de conceber os
atributos próprios do ente enquanto ente como categorias ou
gêneros, ao que se dá uma resposta negativa, mesmo
considerando"-se a hipótese de tomar tais categorias ou gêneros
como maximamente supremos.\footnote{ Cf. Aristóteles,
\emph{Metafísica} B, 3, 998b22ss.}

Assim, não é porque o ente enquanto ente tenha características
próprias (essência, igualdade, diferença, anterioridade,
posterioridade etc.)\footnote{ Cf. \emph{idem}, G, 2,
1004a35ss.} que poderemos tomá-las por gêneros supremos do ente.
Antes, elas serão como modos de falar do ente, todas logicamente
convertíveis, pois, do contrário, se fossem gêneros, teriam de
ser predicáveis de suas diferenças, mas nenhum gênero é
predicável de sua diferença.\footnote{ Cf. \emph{idem}, K, 1,
1059b30.} Aristóteles teria apontado, assim, para além de todos
os gêneros, visando algo como o conjunto das propriedades
pertencentes a todo ente, e transcendentes à lógica das
categorias.

Na Idade Média (como se observa, por exemplo, já em Tomás de
Aquino\footnote{ Cf. Tomás de Aquino, \emph{De veritate}
\textsc{xxi},
3.}), essa problemática aristotélica de caráter
lógico"-metafísico assume um caráter explicitamente formal, num
tipo de metafísica de segunda ordem que já não remete mais à
ordem das coisas diretamente, mas trata da realidade das coisas
justamente enquanto realidade. De acordo com certa linguagem
neoescolástica, essa problemática caracterizaria uma
\emph{metaphysica generalis} ou \emph{ontologia}, isto é,
uma investigação sobre os princípios e condições gerais do ente.

Ocupando"-se de tais princípios e condições gerais, essa abordagem
do ente, em autores medievais, chegou a desenvolver de tal modo
suas virtualidades conceituais, que seu caráter pareceu, muitas
vezes, mais lógico do que metafísico. Isso explicaria, por
exemplo, a preocupação desses autores, em contexto metafísico,
com a analogia e os predicamentos. Assim, embora ainda não se
possa falar --- pensando"-se na Idade Média --, de uma separação
entre lógica e metafísica tal como ocorrerá na Modernidade,
também não se pode negar o formalismo do tratamento do ente, à
semelhança de um tratamento lógico.

Esse formalismo e suas consequências estão certamente na base das
críticas modernas que não viam neles nada além de abstrações
estéreis. Daí a associação feita das noções e raciocínios
metafísicos com abstrações vazias, como alguns autores modernos
farão, a ponto de David Hume, por exemplo, assombrar"-se com a
tamanha destruição que teríamos de fazer, se quiséssemos
continuar racionais, ao entrarmos numa biblioteca e nos
depararmos com obras de conteúdo metafísico. Nossa melhor
atitude, no seu dizer, seria jogá-las nas chamas, pois estariam
repletas de sofismas e quimeras.

Pesquisadores mais atuais, entretanto, mostraram que, embora
alguns metafísicos tenham chegado, talvez excessivamente, a
abstrações distantes, isso não autoriza a tomar tais abstrações
como independentes da experiência do ente. A própria visão
iluminista da metafísica medieval, como hoje sabemos, é, sem
dúvida, uma visão míope.\footnote{ Cf., entre os estudos mais
recentes: \textsc{nef}, F. \emph{Qu'est"-ce que la métaphysique?} Paris:
Gallimard, 2004; \textsc{zarka}, Y. C. \& \textsc{pinchard}, B. \emph{Y a"-t"-il
une histoire de la métaphysique?} Paris: \textsc{puf}, 2005.}

Na história do tema, então, Boécio seria um autor estratégico,
não apenas por ter sido um dos autores mais lidos da Idade
Média, mas também porque, em sua obra, combinam"-se, ao mesmo
tempo, elementos lógicos e metafísicos na abordagem do ente.
Assim, num texto como o \emph{Contra Eutychen et Nestorium},
que, como se sabe, tem por objeto o debate teológico a respeito
das naturezas de Cristo, e, mais precisamente, no capítulo \textsc{iv}
(versando sobre a diferença entre \emph{ousia} e
\emph{hypostasis}), Boécio lança mão do tema --- que
modernamente será considerado lógico --- da convertibilidade entre
ser e uno.\footnote{ Cf. Boécio, \emph{Contra Êutiques e
Nestório}, cap. \textsc{iv} [295].}

Mais do que enfatizar o caráter lógico impresso por Boécio à
problemática transcendental, ou as virtualidades conceituais da
identificação entre ser, unidade e bem, pretendemos chegar aos
pressupostos metafísicos dessa identificação. Dito de outra
maneira, procuraremos explicitar o modo como Boécio retrata sua
experiência do ente, a fim de encontrar aí as razões para sua
afirmação de que tudo o que é é uno e bom.

\section{Ponto de partida: a experiência do desejo
universal de felicidade} 

O ponto de partida de Boécio, no livro \textsc{iii} da \emph{Consolação
da filosofia}, é a experiência ética, ou o desejo
universal da felicidade. Essa âncora permite a nosso autor
evitar não apenas um distanciamento da vida concreta, mas também
qualquer devaneio por especulações meramente formais. Já uma
primeira leitura do \emph{De consolatione} revela a densidade
dos vínculos estabelecidos por ele entre suas teses filosóficas
e sua experiência de vida.

É certamente por isso que Boécio, ao falar da felicidade, não
parte de uma definição da mesma. Seu início está na experiência
da ilusão vivida pela busca de satisfação por meio dos bens
exteriores e particulares, oferecidos pela Natureza em nossa
vida cotidiana. Esses bens, no seu dizer, podem ser resumidos na
riqueza, nos cargos (hoje talvez pudéssemos falar de trabalho),
no poder (liberdade), na glória (reconhecimento) e no prazer.
Toda forma de satisfação humana ligar"-se"-ia, de algum modo, a um
desses bens.

Todos eles, porém, revelar"-se"-iam relativos e frustrantes, pois a
riqueza, no fim das contas, não traz a autossuficiência; os
cargos não valorizam necessariamente os melhores; o poder é
instável e nutre"-se do temor; a glória é mentirosa; e o prazer
causa escravidão e inquietação.\footnote{ Cf. as prosas 3--7 do
livro \textsc{iii} de \emph{A consolação da filosofia}. Na prosa 8
desse mesmo livro, Boécio sintetiza o desenvolvimento das prosas
3--7.}

Boécio não desvaloriza esses bens por si mesmos, afinal, como ele
afirma nos nn. 14--20 da prosa 2 do livro \textsc{iii}, é natural que
busquemos certa abundância de bens, e não há nenhum mal em
desejar riqueza, cargos, poder, glória e prazer. O que ele faz é
relativizar todos esses bens, mostrando como, se absolutizados,
eles nos dispersam e impedem"-nos de ver que, na verdade,
buscamos, por trás de todos eles, uma satisfação mais completa e
estável.

A busca de satisfação completa e estável seria o que nos faz,
segundo Boécio, desejar sempre mais; experimentar um desejo sem
limites. Esse dinamismo o faz conceber o ser humano como um ser
cujos desejos particulares revelariam um desejo de fundo, no
sentido de alcançar algo capaz de satisfazer em caráter
definitivo. Esse algo definitivo seria, portanto, um pólo que
atrai o ser humano para além de todas as satisfações
particulares. A ele Boécio denomina ora felicidade ora beatitude
(\emph{felicitas/\,beatitudo}), e lhe atribui as notas formais
de finalidade, completude e perfeição.\footnote{ Cf. \emph{A
consolação da filosofia} \textsc{iii}, prosa 2, nn. 2--3.}

Assim, a experiência da insaciabilidade dos desejos particulares
leva Boécio a concluir pela existência do desejo inscrito
naturalmente no ser humano de encontrar um bem verdadeiro
(\emph{est enim mentibus hominum ueri boni naturaliter inserta
cupiditas}\footnote{ \emph{A consolação da filosofia} \textsc{iii},
prosa 2, n. 4.}).

No livro \textsc{ii}, Boécio afirma que os seres humanos, no desejo de ser
feliz, enganam"-se com a ilusão da abundância de bens, e não
buscam esse bem verdadeiro da maneira como deveriam buscar. Para
ele, além de a abundância ser antinatural (pois, como demonstram
os animais, precisamos de poucos bens para viver), os seres
humanos não são apenas corpo, mas também têm mente
(\emph{mens}), e não poderão nunca sentir"-se realizados
completamente caso procurem satisfazer seu desejo apenas com a
abundância. Seria apenas pelo cultivo da mente e pelo
autoconhecimento que o ser humano poderia suplantar a Natureza e
realizar adequadamente a busca da felicidade.\footnote{ Cf.
\emph{A consolação da filosofia} \textsc{ii}, prosa 5, nn. 22--29.}

Boécio chega a dizer que nem a beleza do céu devia ser admirada
por si mesma; o cosmo inteiro tornava"-se um indicador de que a
busca da felicidade devia ir além do transitório e finito, sob
pena de nunca o desejo humano encontrar satisfação.\footnote{ Cf.
\emph{A consolação da filosofia} \textsc{iii}, prosa 8, n. 8.} Sem
desvalorizar o prazer, ele ainda critica Epicuro, dizendo haver
um bem próprio do homem, o qual não pode ser reduzido à
satisfação pelos bens particulares.\footnote{ Cf. \emph{A
consolação da filosofia} \textsc{iii}, prosa 2, nn. 12--13.}

\section{Identificação entre bem e unidade}

É justamente pela contraposição dos bens particulares com a
felicidade --- à qual poderíamos chamar de bem supremo --- que
Boécio chegará a um primeiro resultado “transcendental”, ou
seja, à identificação entre bem e unidade. Ele constata que, no
limite, os bens particulares implicam"-se entre si e compõem uma
unidade, além de remeterem, como dissemos acima, a um polo que
atrai para além de cada bem particular.

A base para essa identificação vem novamente da experiência de
insaciabilidade do desejo humano pelos bens particulares. Por
seu caráter provisório, parcial e mutável, esses bens impelem o
ser humano a buscar satisfação pela posse de um bem final,
completo e perfeito, ou seja, de um bem supremo.

Mas, na busca dessa satisfação, vemos que ninguém visa um bem
particular sem também desejar os outros bens (ninguém deseja uma
forma de riqueza sem também desejar o trabalho, a liberdade, o
reconhecimento e o prazer, e vice"-versa). Isso apontaria,
segundo Boécio, para a unidade que subjaz a todos os bens, não
no sentido de um agrupamento de bens particulares, mas no de um
todo indiviso, amalgamado pela referência a um bem supremo. No
limite, a natureza de todos os bens particulares seria
equivalente; consistiria em serem imagens de um bem verdadeiro,
ou, ainda, em serem bens imperfeitos (\emph{imagines ueri boni
uel imperfecta quaedam bona}). Por isso, desejá-los significará
desejar sua semelhança com o bem supremo.\footnote{ Cf. \emph{A
consolação da filosofia} \textsc{iii}, prosa 9, nn. 22--30. Apesar de
alguns aspectos discutíveis, a análise que Etienne Gilson faz do
pensamento tomista, no capítulo \textsc{xiv} de seu livro \emph{O
espírito da filosofia medieval} (“O amor e seu objeto”), parece
muito útil para compreender essas afirmações de Boécio.} Se
eles são valorizados por si mesmos, é porque o mesmo desejo que
leva a um bem supremo --- a felicidade --- passa por eles, ainda que
com diferenças qualitativas. Do contrário, teríamos de supor um
desejo para cada bem, mas isso contrariaria nossa experiência,
uma vez que é o mesmo impulso de satisfação que nos faz
concentrar"-nos neles.

Boécio apresenta, ainda, duas razões metafísicas para falar desse
bem supremo e levar a acabamento a associação entre bem e
unidade: (a) é necessário afirmar sua existência, porque o
imperfeito supõe o perfeito, e, se temos a experiência da
imperfeição dos bens particulares, é porque deve haver um bem
perfeito por cuja contraposição constatamos o
imperfeito;\footnote{ Cf. \emph{A consolação da filosofia}
\textsc{iii},
prosa 10, n. 3.} (b) é necessário que esse bem supremo seja
único, pois seria contraditório pensar a coexistência de dois
bens desse gênero (a um faltaria o outro, e nenhum seria, então,
supremo).\footnote{ Cf. \emph{A consolação da filosofia} \textsc{iii},
prosa 10, nn. 18--21.}

As dificuldades que podem ser implicadas pela afirmação de um
único bem verdadeiro concomitantemente com bens particulares são
resolvidas por Boécio nos termos de uma teoria da
participação.\footnote{ Cf. \emph{A consolação da filosofia}
\textsc{iii}, prosa 10, n. 22; o texto mais completo para o pensamento de
Boécio a respeito da participação é o \emph{De
hebdomadibus}.} Tudo é sustentado pela participação no bem
supremo, e ser feliz consistirá em atingir o bem supremo assim
como o justo atinge a justiça, ou seja, participando dela
(Boécio fala do homem feliz como um deus, \emph{omnis igitur
beatus deus}\footnote{ Cf. \emph{A consolação da filosofia}
\textsc{iii}, prosa 10, n. 25.}).

Essas dificuldades, numa palavra, confirmam a existência de
apenas um bem verdadeiro, o supremo, a própria bondade, “ápice,
pivô e causa de todas as coisas desejáveis” (\emph{quo fit uti
summa, cardo atque causa expetendorum omnium bonitas esse iure
credatur}\footnote{ Cf. \emph{A consolação da filosofia}
\textsc{iii},
prosa 10, n. 38.}).

\section{Identificação entre unidade e ser}

No quadro da identificação entre unidade e bem, Boécio termina
por também identificar unidade e ser, pois, sendo a morte e o
perecimento uma dissolução da unidade, tudo, então, desejaria
naturalmente o ser, uma vez que nada parece desejar naturalmente
a morte.

O fundamento desse raciocínio de Boécio seria a experiência de
que tudo, na Natureza, tem uma tendência a permanecer na
existência. E permanecer na existência equivaleria a evitar a
dissolução das partes que compõem cada ente.

Assim, observa"-se já nas plantas uma busca do lugar mais
conveniente à sua permanência: umas brotam na planície; outras,
nas montanhas; outras, nos brejos; outras, nas regiões áridas;
outras, enfim, coladas a rochas. Em outras palavras, cada qual,
naturalmente, procura e recebe as condições para permanecer; e,
no conjunto das espécies, vemos um dinamismo sem fim.\footnote{
Cf. \emph{A consolação da filosofia} \textsc{iii}, prosa 11, nn.
18--24.} Boécio exprime essa busca de permanência, da parte dos
seres vivos, como um impulso natural, um \emph{labor} por
conservar"-se e evitar a morte.\footnote{ Cf. \emph{A consolação
da filosofia} \textsc{iii}, prosa 11, n. 16.}

Mesmo os seres ditos inanimados apresentariam um dinamismo como
esse, e a prova viria da busca do lugar natural de cada um (como
o fogo para cima e a terra para baixo). Nessa busca, o que é
mineral dá um testemunho ainda mais eloquente, pois,
naturalmente, evita o rompimento da coerência sólida de suas
partes. Quanto aos fluidos, como o ar e a água, ainda que se
rompam facilmente, buscam imediatamente a recomposição. E tudo
isso graças não a uma escolha livre, mas a uma tensão natural,
ou algo como um impulso (\emph{naturali intentione}).\footnote{
Cf. \emph{A consolação da filosofia} \textsc{iii}, prosa 11, nn.
25--30.}

Apenas no caso do ser humano poderíamos falar, além do instinto,
de um movimento nascido na alma inteligente (\emph{motibus
uoluntariis animae cognoscentis}), com vistas à permanência na
existência. Nesse contexto, Boécio menciona discreta e
respeitosamente a possibilidade de os seres humanos porem fim à
sua existência, mas essa possibilidade, em última instância,
decorreria de constrangimentos externos ao desejo natural de
permanecer.\footnote{ Cf. \emph{A consolação da filosofia}
\textsc{iii},
prosa 11, n. 15.}

E esse impulso ou \emph{labor} equivaleria à conservação da
unidade, evitando"-se a morte. No dizer de Boécio, “tudo o que
existe permanece e subsiste enquanto é uno, mas morre e se
dissolve desde que cessa sua unidade” (\emph{omne quod est tam
diu manere atque subsistere  quamdiu sit unum, sed interire
atque dissolui pariter atque unum esse destiterit}\footnote{
\emph{A consolação da filosofia} \textsc{iii}, prosa 11, n. 10.}).

Essa ênfase de Boécio na unidade torna explícita sua
identificação entre ser, unidade e bem, pois afinal, ele toma o ser
como um bem a partir do movimento natural de busca da unidade.
Em outras palavras, se desejar o ser é desejar a unidade, e se
desejar a unidade é desejar o bem (como foi dito acima), então
desejar o ser será também desejar o bem. Haveria, assim, uma
identificação direta entre eles.

\section{Os bens particulares e o bem supremo}

Resumindo o caminho percorrido até aqui, parece possível
sintetizar o pensamento de Boécio em três grandes momentos:

\begin{enumerate}
\item a identificação entre unidade e bem:

\begin{enumerate}
\item os bens particulares não satisfazem o desejo que eles
despertam;

\item essa experiência aponta para um bem que está além dos bens
particulares (bem supremo);

\item então, os bens particulares só têm o sentido de bens
enquanto remetem a esse bem supremo;

\item no limite, o bem supremo seria o único bem;

\item assim, falar do bem seria falar de unidade, e vice"-versa;
\end{enumerate}

\item identificação entre unidade e ser:

\begin{enumerate}
\item tudo, na Natureza, busca naturalmente conservar"-se na
existência;

\item conservar"-se na existência significa manter a unidade,
isto é, evitar a dissipação das partes;

\item assim, falar de ser é falar de unidade, e vice"-versa;
\end{enumerate}

\item identificação entre ser e bem:

\begin{enumerate}
\item desejar o ser é desejar a unidade;

\item desejar a unidade é desejar o bem;

\item portanto, desejar o ser é desejar o bem.
\end{enumerate}
\end{enumerate}

Esses três momentos permitem esclarecer o modo como Boécio parte
da constatação de que os bens particulares não satisfazem o
desejo e conclui que há um bem supremo para além deles. Pode"-se,
então, retomar o sentido da passagem de (a) a (b), mencionada no
início desse trabalho.

A possível falácia, no texto aristotélico, consistiria em dizer
que, se todas as nossas ações visam a um bem, então haveria um
bem para o qual tendem todas as nossas ações. O texto de
Aristóteles não é muito claro, e poderia ser resumido como
segue:

\begin{enumerate}
\item se há um fim das nossas ações, o qual desejamos por si
mesmo;

\begin{enumerate}
\item de maneira que desejamos todo o resto em função desse fim
de cada ação;

\item e se não escolhemos tudo em vista de outra coisa (para não
proceder ao infinito, tornando o desejo vazio e vão);
\end{enumerate}

\item então há um fim, que é o bem, e o bem supremo.
\end{enumerate}

Dos comentadores clássicos de Aristóteles, parece que um dos
primeiros a notar um possível deslize no texto aristotélico foi
Tomás de Aquino (no comentário à \emph{Ética nicomaqueia}).
Quanto aos comentadores antigos, nenhum o teria percebido, e
Eustrátio, por exemplo, teria interpretado o bem ao modo de algo
transcendente, como se Aristóteles pretendesse que tudo tende a
um único bem, que se encontra acima de tudo.\footnote{ Cf. Marco
Zingano, “\emph{Eudaimonia} e bem supremo em Aristóteles”, \emph{op.
cit.}, p. 98.}

No tocante a Boécio, ele também localiza um bem transcendente,
acima de tudo, e opera uma passagem de (a) a (b) nos seguintes
termos:

\begin{enumerate}
\item todas as coisas desejam o bem (\emph{cuncta igitur bonum
petunt});
\item o bem é precisamente o que é desejado por todas as coisas
(\emph{licet ipsum bonum esse quod desideretur ab omnibus}).
\end{enumerate}

Vista, porém, no seu contexto, isto é, como conclusão da prosa
11, não há nada de falacioso nem de gratuito na passagem operada
por Boécio, pois a afirmação de que há um fim desejado por todas
as coisas, segundo Boécio, seria demonstrada (i) pela
identificação entre unidade e bem, (ii) pela identificação entre
unidade e ser, (iii) e pela identificação entre ser e bem.

Em outras palavras, o que torna legítima a passagem operada por
Boécio de (a) a (b) é o modo como ele obtém (a), partindo do
desejo universal da felicidade --- cuja raiz mais profunda seria o
\emph{labor} universal por conservar a existência ---, e como
ele chega a (b), atribuindo unidade ao bem.

Assim, para manter fidelidade à prosa 11, seria preciso
reproduzir o pensamento de Boécio nos seguintes termos:

\begin{enumerate}
\item os bens particulares não satisfazem o desejo que move a
eles;
\item então, esse desejo só poderá ser satisfeito por um bem
perfeito, completo e final;
\item esse bem perfeito tem de ser uno, pois, do contrário, não
será perfeito, nem completo, nem final;
\item além disso, se não houver um bem desse caráter, não
poderemos sequer falar de bens imperfeitos.
\end{enumerate}

Boécio, assim, identifica como um mesmo movimento o desejo que
faz o ser humano dispersar"-se entre os bens particulares e o
desejo que o faz buscar um bem supremo. Essa atitude, claramente
neoplatônica, permite vislumbrar o alcance de sua afirmação (a)
e da passagem para (b).

Somente isolada da prosa 11 do livro \textsc{iii} essa passagem poderia
dar margem alguma discussão formal, à semelhança do que se fez
com o texto aristotélico. Mas, no seu contexto, ela é
inteiramente justificada. No limite, a forma de (a) seria já a
de que ${\exists}$y${\forall}$x(xRy), pois toda a argumentação
anterior demonstrou a existência de um único bem, uno e idêntico
ao ser, capaz de satisfazer ao mesmo desejo que leva aos bens
particulares e à passe de algo uno e perfeito. Não
haveria, portanto, passagem de (a) para (b), mas um movimento,
sob inspiração neoplatônica, de chegada a (a), ainda que se
possa falar, sob inspiração aristotélica, da importância dos
bens particulares. Assim, (b) seria apenas uma explicitação de
(a).

\section{Conclusão}

Haveria, então, segundo Boécio, uma unidade distributiva de bens
particulares, as finalidades de nossas ações particulares, como
também pretendia Aristóteles ao falar do fim de cada ação,
desejado por si mesmo. Mas o texto aristotélico prevê, ainda na
abertura da \emph{Ética nicomaqueia}, um bem que não figura ao
lado dos outros bens,\footnote{ Cf. Marco Zingano, \emph{op.
cit.}, p. 110.} assim como fará posteriormente Boécio. Por
outro lado, é inegável o caráter platônico da apresentação
boeciana do bem supremo, sobretudo pelo modo como o autor romano
concebe o movimento que leva aos bens particulares e ao bem
supremo como um único movimento.

Boécio parece sentir"-se bastante à vontade para dispor dessa
dúplice herança (sem falar da herança cristã) na elaboração de
sua \emph{Consolação}. O resultado foi o ancoramento de sua
metafísica na experiência concreta do ente, tomando como ponto
de partida o desejo.

Assim, retomando a linguagem da transcendentalidade, parece
possível dizer que, embora Boécio, em outros contextos, dê
preferência a uma abordagem lógica das características do ente
enquanto ente, isso não quer dizer que a identificação
estabelecida por ele entre ser, unidade e bem possa ser tomada
isoladamente, apenas como uma operação gnosiológica ou
simplesmente lógica. Para ele, a tudo o que é deve"-se atribuir
ser, unidade e bem. E somente assim se pode entender a afirmação
de que há um único bem desejado por todas as coisas.

Essa postura seria, ainda, um testemunho límpido da imbricação
entre lógica e metafísica, no seu pensamento. Mais significativa
ainda seria a imbricação da metafísica (e --- por que não? --- da
lógica), ao menos no tocante ao \emph{De consolatione
philosophiae}, com a experiência ética do desejo.

Segundo Claudio Moreschini, o modo como Boécio articula os bens
particulares com o bem supremo e único seria, ainda, uma prova
de sua modernidade entre os neoplatônicos latinos, e mesmo sua
independência com relação a eles, conectando"-se diretamente aos
neoplatônicos gregos.\footnote{ Cf. C. Moreschini, “Boezio e la
tradizione del neoplatonismo latino”. In: \textsc{obertello}, L. (ed.).
\emph{Atti del Congresso Internazionale di Studi Boeziani}.
Roma: Herder, 1981, pp. 297--310.}

Com efeito, Boécio certamente terá conhecido autores como
Apuleio, Calcídio, Macróbio, Marciano Capela e Mário Vitorino,
entre outros. Mas, como diz Moreschini, não se vivia, no mundo
latino, uma vocação especulativa na mesma intensidade vivida
pelos gregos. Talvez somente Agostinho se destacasse nesse
quadro. Por isso, faltaria, por exemplo, a Apuleio, a capacidade
de construir e organizar racionalmente a diferença entre sumo
bem e bens parciais, reconduzindo esses bens parciais a um grau
maior ou menor de participação no bem supremo, e individuando no
desejo humano que leva à dispersão dos bens o único e mesmo
movimento de retorno ao sumo bem.

Boécio, ao contrário, enfrentaria os problemas ligados à
multiplicidade dos bens, e, sem deixar de valorizá-los por si
mesmos, dá uma solução que os concebe em função do bem supremo,
o qual deve ser uno e único. Dessa perspectiva, Boécio parece
mesmo ligar"-se mais ao médio"-platonismo do que ao neoplatonismo,
pois não sobrepõe o uno ao ser, mas os equipara,
identificando"-os, ainda, com o bem.



\capitulo{A influência do neoplatonismo na solução agostiniana do mal}%
	{Marcos Roberto Nunes Costa}%
	{ufpe}

\markboth{A influência do neoplatonismo na solução agostiniana}{Marcos Roberto Nunes Costa}

\section{Primeiros passos rumo à fé: o encontro com  Ambrósio}

Tendo sido aprovado no concurso público para \emph{rector} da
cátedra de Milão,  no outono de 384, Agostinho, com 30 anos de
idade,  chega àquela cidade. 

Poucos dias depois de chegar a Milão, atraído pela fama de orador
do bispo Ambrósio, Agostinho resolveu ouvi"-lo,
motivado, no início, não pela fé, mas pela curiosidade, conforme
ele mesmo diz nas \emph{Confissões}: 

\begin{quote}
Ardorosamente o ouvia, quando pregava ao povo, não com o espírito
que convinha, mas como que a sondar a sua eloquência para ver se
correspondia à fama, ou se realmente se exagerava ou diminuía o
que se apregoava [\ldots{}]. Não me esforçava por aprender o que o
Bispo dizia, mas só reparava no modo como ele falava.\footnote{
\emph{Conf}., \textsc{v}, 13, 23; 14, 24.}
\end{quote}

Agostinho continuou ouvindo os sermões de Ambrósio, quando,
através deste, despertou para um ponto que seria de fundamental
importância na busca de uma solução do problema do mal,  que o
atormentava a tanto tempo: o conceito de “substância
espiritual”, até então desconhecido para ele: 

\begin{quote}
Eu que nem levemente ou por enigma  suspeitava o que era
substância espiritual, contudo alegrava"-me e envergonhava"-me de
ter ladrado, durante tantos anos contra a fé católica
[\ldots{}]\footnote{ \emph{Ibid.,} \textsc{vi}, 3, 4.}
\end{quote}

Com Ambrósio, Agostinho aprendeu que o Deus do cristianismo, uno
e criador, não forma uma substância corpórea ou material, mas
espiritual, iniciando o processo de resolução do seu grande
problema acerca de Deus --- o da natureza ou substancialidade de
Deus. Pois, enquanto fora maniqueu, o problema não consistia na
verdade acerca da existência de Deus, mas quanto à sua
substancialidade e aos caminhos para alcançá-lo. Conforme deixa
bem claro nas\emph{ Confissões,} ao afirmar: 

\begin{quote}
[\ldots{}] quanto a vós, ó Vida da minha vida, também Vos imaginava
como um Ser imenso, penetrando por todos os lados a massa do
Universo e alastrando"-Vos fora dele, por toda a parte, através
das imensidades sem limites, de tal modo que a terra, o céu e
todas as coisas Vos continham e todas elas se acabavam em Vós,
sem contudo acabardes em parte alguma. Mas assim como a massa do
ar, deste ar que está por cima da terra, não se opõe a que a luz
do sol penetre por ele, atravessando"-o, sem o rasgar nem cortar,
mas enchendo"-o inteiramente, assim julgava que não só as
substâncias transparentes do céu, do ar e do mar, mas também as
da terra, eram, por Vós, penetradas em todas as suas partes,
grandes e pequenas, para receberem a Vossa presença,
governando"-as interiormente com Vossa oculta inspiração e
exteriormente dirigindo tudo o que criastes. Assim conjeturava
eu, pois não Vos podia conceber de outra maneira.\footnote{
\emph{Ibid.,} \textsc{vii}, 1, 1--2.} 
\end{quote}

Desse panteísmo materialista, desdobrava"-se uma resposta ao
problema do mal, ponto que atormentava desde jovem e que o levou
a abraçar o maniqueísmo: 

\begin{quote}
O principal e quase único motivo do meu erro inevitável era,
quando desejava pensar no meu Deus, não poder formar uma ideia
dele, se não lhe atribuísse um corpo, visto parecer"-me
impossível que houvesse alguma coisa que não fosse material.
Daqui deduzia eu a existência de uma certa substância do mal que
tinha a sua massa feia e disforme --- ou fosse grosseira como a
que chamam terra ou tênue e sutil como o ar --- a qual eu julgava
ser o espírito maligno investindo a terra.\footnote{
\emph{Ibid.,} \textsc{v}, 10,19--20. Maiores informações sobre a
doutrina e a moral maniqueia ver nossa obra: \textsc{costa}, 2003,
especialmente os capítulos  3 e 4, intitulados, respectivamente,
“A Cosmologia Dualista Maniqueia” e “A Moral Maniqueia”, p.
39--111.}
\end{quote}

Rompendo com a tradição ontológico"-cosmológico"-materialista
maniqueia, a qual afirmava a existência de duas forças coeternas
criadoras do mundo: o bem, simbolizado pela luz,  e o mal,
simbolizado pelas trevas, ambas de natureza corpórea e em eterna
luta, Agostinho aprendera, com Ambrósio, que Deus --- Ser uno e
criador de todas as coisas, não forma uma substância corporal ou
material, mas uma “substância espiritual”. 

Além disso, Ambrósio, alicerçado na tradição de Orígenes, adotava
o “método alegórico” como meio de banir das Sagradas Escrituras
as contradições,  os escândalos, os enganos, os mistérios que
envolvem as histórias de fé. Assim, sem negar o sentido literal
da Bíblia, afirmava que, para melhor compreendê-la, é preciso
distinguir “a letra” e  o “espírito”. Ou seja, mostrar que por
baixo das aparências materiais da letra, existem as verdadeiras
intenções dos autores inspirados e que deve ser posto em
evidência o espírito ou o sentido profundo dos 
ensinamentos.\footnote{ Cf. \textsc{vannini}, 1989, p. 34--35: “O bispo de
Milão tomou de Orígenes e dos outros mestres gregos a arte de
interpretar a  Escritura não ao pé da letra, mas 
alegoricamente, de modo que a  página bíblica  não choca mais
com a simplicidade e rudeza de suas  descrições. Em particular,
Ambrósio insiste sobre a concepção espiritual de Deus e da alma:
longe daquela concepção antropomórfica de Deus e materialista do
homem que por tanto tempo repugnava o maniqueu Agostinho e que o
levou a repudiar o Velho Testamento”.} 

Isso trazia grandes progressos ao pensamento de Agostinho. A
partir daí, muitas coisas que pareciam inaceitáveis nas
Escritures,  como, por exemplo, em relação à afirmação de que
“\emph{o homem foi criado à imagem de Deus}”,  que tinha sido
a pedra de tropeço  ao ler a Bíblia pela primeira vez, logo após
a leitura do \emph{Hortensius,} levando"-o ao maniqueísmo,
agora passariam a ter sentido. Conforme diz: “A partir daí  nos
Livros Santos já não é absurdo o que me parecia absurdo, podendo
ser interpretado de um modo diferente e mais
aceitável”.\footnote{ \emph{Conf.} \textsc{vi}, 11, 18.}

Entretanto, os sermões de Ambrósio ajudaram Agostinho a
compreender a natureza espiritual de Deus, por outro lado
criaram um problema em relação ao problema do mal, pois, se
todas as coisas corpóreas e incorpóreas foram criadas por um
único Ser --- Deus, criador, imutável e incorruptível,  totalmente
espiritual, isto é, que não forma uma substância  corpórea, mas
espiritual e do qual não pode vir senão o bem, como explicar o
mal presente no homem e no mundo?

\begin{quote}
Qual a origem do mal, se Deus, que é bom, fez todas as coisas?
Sendo o supremo e sumo Bem, criou bens menores do que Ele; mas
enfim, o Criador e as criaturas todas são bons. Donde, pois, vem
o mal?\footnote{ \emph{Ibid.,} \textsc{vii}, 5, 7. Esta é justamente a
questão que dá início, mais tarde, à obra  \emph{Sobre o
Livre"-Arbítrio,} onde se encontra a resposta definitiva ao
problema do mal, que reaparece nos seguintes termos:
“\emph{Unde male faciamus? ---} Donde vem o praticarmos o mal?”
(\emph{De lib. arb.} \textsc{i}, 2,4).}  
\end{quote}

Ou seja, se, do ponto de vista religioso, as pregações de
Ambrósio causaram grandes mudanças no coração de Agostinho, do
ponto de vista filosófico, ele continuava angustiado,
principalmente em relação ao problema do mal, conforme relata:
“Já me tínheis libertado daquelas prisões, Ajudador meu, e
ainda, sem êxito, buscava a origem do mal”.\footnote{
\emph{Conf.,} \textsc{vii}, 7,11.} Teria de esperar mais um pouco, de
passar por outras experiências, ou adquirir novos conceitos para
alcançar o  ponto final. 

\section{A consolidação do novo princípio ontológico no
neoplatonismo}

Mas, em meio a tão grandes angústias, eis que nasceu uma luz, que
Agostinho atribui como sendo a “mão oculta” de Deus, e que, em
\emph{As Confissões,} chama de “O colírio das dores''
(\emph{collyrio dolorum}).\footnote{ cf. \emph{Ibid.,} \textsc{vii}, 8,
12.}  É que,  em 386, chegaram às suas mãos “alguns
livros platônicos”, conforme relata o próprio  Agostinho: 

\begin{quote}
E primeiro, querendo Vós mostrar"-me primeiramente como
``\emph{resistis aos soberbos e dais graças aos humildes''}
[\ldots{}], me deparastes, por intermédio de um certo
homem,\footnote{ Conforme se  vê, Agostinho não diz expressamente
o nome do homem que fez chegar até ele os referidos livros.
Entretanto, é comum atribuir"-se tal feito a Flávio Mânlio
Teodoro, a quem Agostinho dedicou uma de suas primeiras obras ---
\emph{De beata vita}, importante personalidade
literária e política, que chegou ao cargo de cônsul do Império.
Um homem culto, amante da filosofia neoplatônica, com quem
Agostinho travou  amizade em Milão ao integrar"-se ao grupo de
Ambrósio e Simpliciano. O'\textsc{meara}, 1954, p. 161,  por exemplo,
seguindo a linha de Courcelle (1950) diz que “foi Ambrósio quem
apresentou Agostinho a Teodoro e que aquele lhe deu algumas
\emph{Enéadas} de Plotino que Victorino havia traduzido para o
latim”.} intumescido por monstruoso orgulho,\footnote{ Aqui,
vemos claramente que, apesar de reconhecer a importância que os
neoplatônicos tiveram na sua formação intelectual,  Agostinho os
trata como “orgulhosos”, “soberbos”,  por  pensarem poder
alcançar a verdade (Deus), com seus próprios esforços, pela
razão natural, sem a ajuda da graça divina. Mais adiante
voltaremos ao assunto.} alguns livros platônicos,\footnote{ Por
Agostinho ter  usado uma expressão um tanto vaga “alguns livros
platônicos” (\emph{Conf.,} \textsc{vii}, 9, 13),  há uma série de
controvérsias sobre quais livros  realmente Agostinho teria lido
naquele momento, se as obras do próprio  Platão, ou dos
neoplatônicos. Ora, o próprio Agostinho, em uma outra obra,
aumenta  a discussão, quando diz: “Li entrementes algumas
poucas obras de Platão” (\emph{De beat. vit.}1, 4).
Entretanto, como Agostinho afirma nos primeiros livros
das\emph{ Confissões}, e em outras obras, não dominar a língua
grega, para alguns autores, dentre eles, \textsc{jolivet}, 1932, p. 84,
“não parece que Agostinho tenha conhecido diretamente as suas
obras [\ldots{}]. Pode"-se supor que um copista tenha, por engano
aplicável, colocado o termo ``Platão'', em vez de ``Plotino''.  É
possível  também que o termo ``Platão'' designe aqui, não os
escritos, mas a doutrina do filósofo da Academia, de quem os
neoplatônicos se titularam discípulos e herdeiros”. Tais
hipóteses são perfeitamente justificáveis quando lemos o
\emph{Contra Acadêmicos,} onde Agostinho, ao se referir a
Plotino, diz: “Plotino,  filósofo platônico tem tal semelhança
com o mestre, que se poderia crer serem contemporâneos, mas, já
que tão largo intervalo de tempo os separa, que Platão esteja
revivido naquele” (\emph{Contra acad.} \textsc{iii}, 18, 41).
Entretanto, é possível que Agostinho tenha conhecido  algumas
obras de Platão, de forma indireta através de citações em latim
contidas nos autores clássicos, ou traduções da época, conforme
nos diz \textsc{rincón} \textsc{gonzález}, 1992, p. 39: “Concernente aos diálogos
de Platão, Agostinho deve ter lido, em traduções, o
\emph{Timeu,} o \emph{Fedón} e o \emph{Menón”.}}
traduzidos do grego para o latim.\footnote{ \emph{Conf}.,
\textsc{vii},
9, 13. Aqui, apesar de Agostinho não dizer o nome do tradutor
dos livros platônicos, é comum atribuir"-se tal tarefa a Mário
Victorino, um famoso orador que gozava de tal prestígio que os
romanos ergueram uma estátua em Roma em sua homenagem, quando
ainda estava vivo. A esse respeito diz \textsc{roque}, 1955, p. 94: “É
ainda no texto de Marius Victorinus que Agostinho leu em meados
de 386--387 o que ele chama de ``livros platônicos'' e que tiveram
em sua conversão uma influência decisiva. Parece que se tratava
das \emph{Enéadas} de Plotino”.} 
\end{quote}

Afora as controvérsias acerca de quem Agostinho leu,  o certo é
que o  neoplatonismo exerceu uma influência tal que, não só seu
espírito foi esclarecido e seu pensamento transformado, senão
que, desde então e por toda sua vida, as  teorias neoplatônicas
estiveram sempre na base de sua  própria doutrina.\footnote{ Mais
do que uma simples descoberta, Agostinho se apaixonou pela
filosofia neoplatônica e ficou fiel a esse amor através da sua
vida cristã, até ao derradeiro suspiro. Basta vermos a forte
influência indireta desta filosofia ao longo de seus escritos,
além das citações diretas que faz dos filósofos neoplatônicos.
Entretanto, como bem observa \textsc{rostagno}, 2000, p. 48,  “a
filosofia neoplatônica lhe fornecerá um suporte de pensamento ao
qual não renunciará jamais [\ldots{}]. Por outro lado, Agostinho
inicia uma decisiva crítica ao próprio neoplatonismo e deste
modo torna"-se criador da teologia que nós conhecemos”. Por isso,
para \textsc{jolivet}, 1932, p. 8, “não era Agostinho quem se convertia
ao neoplatonismo, na leitura de seus livros, como estamos
acostumados a ouvir, mas era Plotino quem era convertido por ele
ao cristianismo”.} 

Convém, portanto, que mostremos, em linhas gerais, as principais
teses do neoplatonismo (na versão de Plotino), mais
especificamente aquelas que influenciaram o pensamento de
Agostinho, complementando as recentes aquisições advindas de
Ambrósio, mostrando em que sentido as mesmas ajudaram"-no no
processo de conversão ao cristianismo; e também, apontando
alguns sinais de mudanças no seu pensamento futuro. Por outra,
em que sentido Agostinho supera o neoplatonismo.



\subsection{A ontologia neoplatônica plotiniana}

O ponto de partida da filosofia de Plotino\footnote{ Segundo
\textsc{sesé}, 1997, p. 64, Plotino, filósofo neoplatônico,  nasceu no
Egito em 204 e faleceu em Minturno, na Campânia, em 270. Foi
discípulo de Amônio Sacas, que fundara o neoplatonismo em
Alexandria. Plotino dispensara em Roma os seus ensinamentos com
imenso sucesso. Sua única obra, as \emph{Enéadas},  foi assim
intitulada e publicada por seu discípulo e biógrafo Porfírio
(232/\,3--305). Plotino teria grande influência, não só sobre
Agostinho, mas sobre diversos Padres da Igreja, como, Basílio de
Cesareia (330--379), que se inspira nele, ao explorar a região
dos espaços inacessíveis de Deus, servindo"-se daquelas asas, no
opúsculo sobre o \emph{Espirito Santo;} Ambrósio, apoia"-se
nele, para falar do \emph{êxtase} de são Paulo e para definir
o \emph{mal} como carência do \emph{bem,} e muitos outros.}
está assentado no velho problema da filosofia grega, 
especialmente de Platão e Aristóteles, da relação entre mundo
sensível e mundo inteligível.

Segundo Jolivet,\footnote{ Cf. \textsc{jolivet}, 1932, p. 86.} apesar de
sua aproximação com Platão, quando adota, pelo menos num
primeiro momento, ou por questões metodológicas, o seu dualismo,
Plotino aponta para um sistema monista, onde, tentando superar o
mestre, procura mostrar que seu dualismo não é composto de dois
elementos opostos e independentes ontologicamente como naquele, 
mas, pelo contrário, ambos os mundos --- inteligível e sensitivo ---
têm o mesmo princípio e fim ontológico. Tudo deriva e volta ao
\emph{Uno}. Ou seja, para o supracitado autor o dualismo
plotiniano é mais metodológico que ontológico.

Para tanto, Plotino apresenta uma explicação metafísica do cosmo
pela teoria da emanação, onde tudo é explicado a partir de um
único ponto ontológico --- o \emph{Uno}. Ou seja, no Princípio,
o \emph{Uno} é tudo o que existe (monismo) e dele procedem
todas as coisas por processão (emanatismo).\footnote{ O fato de
afirmar que todas as coisas derivam do \emph{Uno} por emanação
levaria alguns comentadores a acusarem Plotino de panteísmo,
como, por exemplo, \textsc{sciacca}, 1966, p. 141, que diz: “Panteísmo
acósmico: a eternidade e a infinitude do ser envolvem as coisas
num solene e misterioso silêncio” e \textsc{klimer}; \textsc{colomer}, 1961, p.
111, que diz: “Ainda que sua intenção não é panteísta, mas dada
a ausência da noção de criação fica  difícil manter  a diferença
entre Deus e mundo”. Já \textsc{copleston}, 1977, p. 458, busca um
caminho intermediário, quando diz: “Trata"-se de uma linha
intermediária entre a criação teísta por uma parte e, por outra,
uma teoria plenamente panteísta”. Posição esta que julgamos mais
acertada e que chamamos de panenteísta, ou seja, que “o
\emph{Uno} seja ao mesmo tempo transcendente e imanente ao
todo”, pois, como diz o próprio Plotino nas \emph{Enéadas}: “É
necessário que exista um algo anterior a tudo, algo que deve ser
simples e diferente de tudo posterior; existente por si próprio,
transcendente ao que dele procede e, ao mesmo tempo, de maneira
típica, capaz de estar presente nos outros seres” (\emph{En.}
\textsc{v}, 4,1).}  Por outra parte, Plotino procura mostrar que a passagem
do \emph{Uno}  à multiplicidade dos seres não é direta, mas
que tudo deriva do \emph{Uno} por desdobramentos ou
processões,\footnote{ Segundo \textsc{alsina} \textsc{clota}, 1989, p. 53, é
possível que Plotino tenha despertado para a ideia de processão
a partir do conceito emanatista de criação do pensador
judeo"-helenístico Fílon de Alexandria: “Efetivamente em Fílon,
Deus, que é inteiramente transcendente, cria a partir da
superabundância de sua perfeição. O emanatismo filoniano
reaparecerá em Plotino, ainda que em forma completamente
distinta. O processo através do qual se produz a criação se
chama, na terminologia plotiniana, \emph{proodos,} que os
modernos têm traduzido por \emph{processão”.} Nas\emph{
Enéadas,} Plotino fala da processões como de uma sucessão de
círculos concêntricos, surgidos a partir de um único ponto:
“Existe qualquer coisa que poderia dizer"-se centro: ao redor
deste, há um círculo que irradia o esplendor emanante daquele
centro; ao redor deste (centro e primeiro círculo), um segundo
círculo, luz da luz” (\emph{En.} \textsc{iv}, 3, 17).} que
compreendem graus diversos ou intermediários hierarquicamente
dispostos da perfeição. 

\subsection{A cosmologia plotiniana}

Mantendo seu aparente dualismo, pelo menos por questões
metodológicas, Plotino, seguindo o sistema platônico, divide a
realidade (o Todo) em dois mundos: mundo inteligível e mundo
sensível, onde, internamente, cada um  deles sofrerá
desdobramentos, obedecendo à disposição hierárquica da
perfeição.

\subsection{A multiplicidade do mundo inteligível}

Começando pelo mundo inteligível e incorpóreo, que como um todo é
superior à realidade sensível, Plotino apresenta a famosa tríade
composta pelas três hipóstases primordiais.

No ápice de tudo está o \emph{Uno}\footnote{ \textsc{quiles}, 1981, p.
18, diz que “o nome \emph{Uno} é dado por Plotino após ter
buscado em vão outro com que expressá-lo”. E o próprio Plotino
chega a dizer que é impossível conceituar devidamente o primeiro
princípio, ele está além do nosso pensamento, é algo inefável:
``O \emph{Uno} não é objeto de conceito nem de ciência'
(\emph{En.} \textsc{v}, 4,1)”.}  --- Deus\footnote{ Sabemos que Plotino é
um pagão, e como tal não usa explicitamente em suas
\emph{Enéadas} a palavra Deus. Entretanto, é comum se traduzir
o \emph{Uno} como sinônimo de Deus. De qualquer forma, como
observa \textsc{bussola}, 1990, p. 32, “o \emph{Uno}, evidentemente, é
Deus, mas não o Deus que as Escrituras judaico"-cristãs reduzem a
um ponto indefinido nalgum lugar do céu, com características
antropomórficas, e sim o Deus de Spinoza, ou, talvez, na
linguagem da mentalidade moderna, a Energia Eterna, infinita,
inexplorável, única e incomensurável de que fala Einstein”.} ---,
o Super"-Bem,  que é transcendente, perfeito, eterno, infinito e
necessário. Deste primeiro Princípio, emana a segunda processão,
a \emph{Inteligência ou Noūs}, que é uma cópia do \emph{Uno;}
e, embora tenha sido engendrada imediatamente pelo \emph{Uno,}
e, portanto,  é a mais perfeita de todas as processões, esta
não tem a unidade perfeita. Ela marca o início da
multiplicidade, pois, não obstante ser a processão mais próxima
do primeiro Princípio\emph{,} a \emph{Inteligência} ou
\emph{Noūs} traz em si uma divisão interna; por um lado, ela
contempla diretamente o \emph{Uno}\footnote{ Conforme vemos
nas \emph{Enéadas,} o \emph{Noūs} ou \emph{Inteligência}
tem em si as verdadeiras realidades, que são as essências. Por
conseguinte, o \emph{Noūs} não conhece indiretamente por
imagens, mas a coisa em si: “[\ldots{}] ele mesmo é aquilo que
conhece” (\emph{En.} \textsc{v}, 9, 5).},  do qual é parte, e,
por outro lado, ela contempla a si mesma, é razão consciente de
si mesma. Ou seja, ela é, ao mesmo tempo, a \emph{Inteligência}
que pensa, e é, por outro lado, Ser, enquanto é pensada. 

Por fim, encerrando o mundo inteligível, temos a terceira
emanação,  a \emph{Alma} \emph{universal} ou \emph{Alma do
mundo} (substância espiritual), princípio animador do universo,
que dá vida a todos os corpos (seres), da qual falaremos com
maiores  detalhes  mais adiante, quando de sua relação com o
mundo sensível.

É claro que entre as três hipóstases inteligíveis de Plotino e a
Trindade bíblica, que Agostinho defenderia com tanto fervor mais
tarde, depois de convertido, há enormes diferenças. Em primeiro
lugar, como observa Fraile,\footnote{ Cf. \textsc{fraile}, 1956, p. 733:
“As três hipóstases, o \emph{Uno}, a \emph{Inteligência} e a
\emph{Alma universal}, é o que se  têm chamado a trindade
plotiniana. São três hipóstases distintas, inferiores uma às
outras e não consubstanciais e idênticas em essência [\ldots{}], são
completamente distintas da Trindade do dogma católico”.} em
Plotino, apesar do monismo, onde tudo deriva e volta ao
\emph{Uno}, há uma subordinação hierárquica entre as três
hipóstases, sendo as duas últimas uma emanação da primeira. E,
mais do que isso, a terceira hipóstase, a \emph{Alma,} não
emana diretamente da primeira, mas indiretamente, através da
segunda. Portanto, há degradação hierárquica ou
despotencialização, não no \emph{Uno}, que continua perfeito,
pois este pode dar sem perder, mas nas processões
sucessivas.\footnote{ Segundo \textsc{fraile}, 1956, p. 708,  é
graças a essa degradação ou despotencialização da perfeição nas 
processões sucessivas, onde “a Causa  não se identifica com seus
efeitos”(\emph{En.} \textsc{v}, 2, 1), que podemos considerar Plotino
um filósofo panenteísta;  caso contrário, cada processão seria
idêntica ao \emph{Uno}, e Deus seria tudo e tudo seria Deus, 
o que seria um puro panteísmo. Há sim emanatismo, pois cada
coisa  contém parte de Deus, por derivar dele por emanações
sucessivas, mas cada coisa não é Deus, bem como o conjunto de
todas as  coisas não é Deus. O \emph{Uno} é ao mesmo tempo
imanente e transcendente --- panenteísmo.} Assim sendo, as três
hipóstases não formam uma única substância ou essência. 

Na Trindade bíblica, ao contrário, não há separação nem
degradação hierárquica entre as três Pessoas; o Pai, o Filho e o
Espírito Santo formam um único Ser --- Deus. Ou seja, as três
Pessoas formam uma só substância ou essência, uma só perfeição. 

De qualquer forma, apesar de toda diferença entre as três
hipóstases e a Trindade bíblica, a noção plotiniana de
hipóstases inteligíveis, enquanto realidade que está acima do
mundo sensível, deu grande contribuição para a evolução
intelectual de Agostinho. Primeiro, porque, através delas,
Agostinho confirmaria racionalmente  a ideia cristã, aprendida
com Ambrósio, de que Deus (o \emph{Uno} ou o Supra"-Bem, para
Plotino) é um ser único, de “substância espiritual” ou
transcendente, que não tem corpo ou extensão. 

Mais tarde, em \emph{As Confissões,} utilizando"-se do conceito
neoplatônico"-cristão de “substância espiritual”, Agostinho
denunciaria os maniqueus que, segundo ele, cometeram um grave
erro ao pensar toda natureza e toda substância, inclusive Deus e
a alma, como se fossem coisa corpórea.\footnote{ Cf.
\emph{Conf.,} \textsc{iii}, 7:  “Ignorava eu que Deus é espírito e não
tem membros dotados de comprimento e de largura, nem é matéria
porque a matéria é menor na sua parte do que no todo. Ainda que
a matéria fosse infinita, seria menor em alguma das suas partes,
limitada por um certo espaço, do que na sua infinitude. Nem pode
estar toda inteira em qualquer parte, como o espírito, como
Deus”.} 

Em segundo lugar,  Agostinho, que já havia lido o Evangelho de
são João, ou pelos menos antes já tinha tomado conhecimento
dele, agora,  ao ler em Plotino que do \emph{Uno} emana o
\emph{Intelecto} ou o \emph{Logos}, conclui, pelo menos num
primeiro momento,  que há uma estreita relação entre o Logos ou
\emph{Noūs} de Plotino e o \emph{Verbo} do Evangelho de são
João,\footnote{ É claro que entre a \emph{Inteligência}
ou\emph{ Logos} de Plotino e o \emph{Verbo} do Evangelho de
são João há uma grande diferença, e isto Agostinho perceberá
mais tarde, conforme veremos mais adiante ao tratarmos de sua
passagem do neoplatonismo ao cristianismo. Entretanto, como
observa \textsc{sciacca}, 1956, p. 7: “Agostinho, por um momento,
apressado por suas dúvidas e ansiedades de as resolver, não vê
que o Intelecto de Plotino é muito diferente do Verbo
cristão”.} conforme narra explicitamente em \emph{As
Confissões}: 

\begin{quote}
Nele li, não com estas palavras, mas provado com muitos e
numerosos argumentos, que ``\emph{no princípio era o}
\emph{Verbo e o Verbo existia em Deus e Deus era o Verbo: e
este no princípio existia em Deus. Todas as coisas foram feitas
por Ele, e sem Ele nada foi criado. O que foi feito,  n'Ele é
vida e a vida era a luz dos homens; a luz brilhou nas trevas e
as trevas não a compreenderam''.} A alma do homem, ainda que dê
testemunho da \emph{luz,} não é, porém a \emph{Luz;} mas
o\emph{ Verbo} --- Deus --- é a \emph{Luz verdadeira que ilumina
todo o homem que vem a este mundo}.\footnote{ \emph{Conf.,}
\textsc{vii}, 9, 13.} 
\end{quote}

\subsection{A multiplicidade do mundo sensível}

Mas o maior problema  na  doutrina emanatista de Plotino estava
na relação entre as três hipóstases primordiais inteligíveis e
as substâncias corpóreas --- os seres corpóreos ou materiais, ou
seja, na passagem do mundo inteligível ao mundo sensível (o
velho problema da filosofia grega, conforme anteriormente).
Pois, como conciliar a unidade perfeita, espiritual, eterna,
infinita, imutável e necessária do \emph{Uno} com  a  natureza
finita, corporal, múltipla e contingente dos seres corpóreos,
sem  abandonar o monismo?\footnote{ A esse respeito diz
\textsc{turrado},
1995, p.  754: “A questão mais difícil para o monismo plotiniano
é  a  matéria  ou  \emph{Hylê}, que ele introduz em sua
teoria do egresso"-regresso (processão"-ascensão), se bem que a
reduz ao nada, ao mal ou ``privação de bem'' (\emph{st\={e}resis}),
ao mesmo tempo que é considerada como necessária para a 
existência da alma”.} 

Para  resolver tal problema, Plotino enquadra, também, a matéria
e os seres corpóreos (mundo sensível) em sua teoria da
“processão”, onde tudo está no \emph{Uno}. Ou seja, toda a
multiplicidade no  mundo sensível deriva, também, em última
instância, do \emph{Uno} por um processo sucessivo de
processões que compreendem graus diversos ou intermediários da
perfeição. Há, portanto, uma continuidade ou unidade (monismo)
entre os dois mundos, o inteligível e sensível, e não um
dualismo radical como querem alguns. 

Plotino diz que isso é possível, porque a multiplicidade dos
seres corpóreos, por emanação, em nada diminui a essência do
\emph{Uno}, pois este pode dar sem perder. O \emph{Uno}
continua sendo transcendência, Super"-Bem. Pois, do ponto de
vista da quantidade, um ser diminui cada  vez que  se tira
qualquer uma de suas partes, mas, do ponto de  vista da
qualidade, é muito diferente: assim como a  luz se  difunde por
todas as partes, sem perder nada do seu ser, e  o amor maternal
se multiplica, sem esgotar"-se nunca, o \emph{Uno} pode
expandir"-se por todas substâncias, espirituais e materiais, sem
perder nada.\footnote{ Segundo, \textsc{quiles}, 1981, p. 18, quando
Plotino diz que Deus dá sem perder, estamos  diante de uma
verdadeira posição panenteísta, onde Deus é, ao mesmo tempo,
imanente e transcendente: “O \emph{Uno} é existente por si
próprio, transcendente ao que dele procede e, ao mesmo tempo, de
maneira típica, capaz de estar presente nos outros
seres”(\emph{En.} \textsc{v}, 4,1). Deus está presente em todas as
coisas, por participação, mas nenhuma coisa é Deus, nem mesmo o
conjunto das coisas. Também é graças a esta afirmação que
Plotino salva a incorruptibilidade e imutabilidade de Deus, pois
mesmo que todas as coisas saiam de Deus, e mais do que isto, que
se degradem as processões sucessivas que Dele procedem, Ele
continua, uno, perfeito e imutável na sua transcendência.}

A irradiação, a luminosidade do \emph{Uno,} perpassa tudo, até
ao grau mais ínfimo (a matéria), que é o extremo oposto ao
primeiro Princípio (\emph{Uno}). Assim, tanto as três
hipóstases do mundo inteligível como as substâncias materiais,
serão consideradas como expressões ou emanações  de sua 
plenitude infinita, onde, de forma deterministicamente
hierárquica, a \emph{Inteligência} procede do \emph{Uno},
como os raios emanam do sol; por sua vez, a
\emph{Inteligência} que é fecunda, engendra a \emph{Alma do
mundo}, com a qual chegamos aos limites do mundo inteligível e
tocamos na origem do mundo sensível. A \emph{Alma do mundo},
por sua vez, engendra o ser ou a matéria, último grau da
processão, lugar da multiplicidade e portanto, princípio, ou
melhor, possibilidade do mal.

Portanto, como se vê, à \emph{Alma} \emph{universal} cabe o
papel de fazer a transição ou garantir a unidade entre o mundo
inteligível e o mundo sensível. Ou seja, 

\begin{quote}
por ocupar um grau intermediário entre os seres, embora
pertencendo ao convívio divino, a \emph{Alma} está, não
obstante, no último grau do reino do espírito e, confinando com
o ser sensível, dá algo de si mesma a esse nosso mundo e, em
contrapartida, recebe algo dele.\footnote{ \emph{En.} \textsc{iv},
8,7.}
\end{quote}

Para Plotino, isso só é possível, porque a \emph{Alma}
\emph{universal}\emph{,} analogamente ao que acontece com a
segunda hipóstase (a \emph{Inteligência}), traz em si uma
divisão interna, ou uma dupla natureza;  por um lado, ela é
atividade intelectiva (alma superior), voltada a contemplar o
\emph{Uno,} sua principal vocação, embora não o conheça
diretamente, mas através da segunda hipóstase\emph{,} ou seja,
das imagens, dos conceitos ou formas existentes no mundo das
ideias da \emph{Inteligência.} Por outro lado, ela 
relaciona"-se com o mundo sensitivo (alma inferior), onde, assim
como no caso do \emph{Uno}, multiplicando"-se, sem dividir"-se
ou perder sua unidade,\footnote{ Cf. \emph{Ibid.,} \textsc{iii}, 1, 8,
1.} dá forma à massa informe --- à matéria, gerando os
seres corpóreos.

Desse modo, paralelamente ao que acontece no mundo das hipóstases
inteligíveis, há, também, no mundo sensível, um descenso de bens
a partir do \emph{Uno.}  Ou num sentido inverso, de modo
hierarquicamente ascendente, todas as coisas que existem no
mundo sensível recebem seu ser ou são engendradas pelo seu grau
imediatamente superior. 

Aqui, estamos diante de um segundo elemento que Agostinho
absorvera do pensamento neoplatônico, e que influenciaria 
profundamente seu pensamento futuro, a saber: a noção de
participação. 

Assim, no sistema monista plotiniano,  a matéria,  que na sua
união com a \emph{Alma} \emph{universal} dá origem aos seres
corporais, é a última processão do \emph{Uno}, e como tal, é
eterna e necessária.\footnote{ \textsc{fraile}, 1956, p. 719: “A matéria é
necessária porque o Universo consta de contrários, e não poderia
haver contrários  se não existisse a matéria”. E é claro que,
sendo a matéria algo necessário,  também o é o mal.} Esta,
apesar do extremo distanciamento do Bem, não forma um princípio
ontológico independente (como no maniqueísmo). Pois, apesar de
ser, juntamente com a \emph{Alma universal}, um dos princípios
originantes da multiplicidade dos seres corpóreos, isso não
significa que a matéria se constitua em um princípio ontológico
originante de si mesma, pois, em nenhum momento das
\emph{Enéadas} Plotino fala de um segundo princípio
ontológico originante.  A matéria, assim como  a \emph{Alma do
mundo}, faz parte, em última das instâncias, do
\emph{Uno},\footnote{ Cf. \textsc{bussola}, 1990, p. 41: “As emanações
diminuem e se degradam à medida que se afastam do \emph{Uno};
logo, num momento pode não haver mais luz, somente escuridão
[\ldots{}] mas, cuidado!, pois essa escuridão está ainda no
\emph{Uno}. O último lampejo antes  da escuridão  é apenas a
última forma de emanação do \emph{Uno}, dentro do
\emph{Uno}”.} ela é,  assim, o extremo limite do \emph{Uno}
(para além dos limites da matéria não há mais processão alguma,
ou não existe mais nada)\emph{,} lugar da obscuridade, da
multiplicidade e, portanto, fonte ou possibilidade do mal. Por
isso, Plotino fala da matéria, quando de seu estado de natureza
pura, ou seja, sem que esteja unida à \emph{Alma} \emph{do
mundo}, para com ela formar o ser, como privação ou defecção ---
falta de forma, indeterminação,  distanciamento do Bem --- o
não Ser, a que Plotino dá o nome de “nada”.\footnote{ Vale 
salientar aqui, que o termo “nada”  utilizado por Plotino
significa tão"-somente o estado mutável em que se encontra a
matéria no seu  eterno e contínuo movimento que dá origem aos
seres múltiplos. Entretanto, apesar de ser mutável, fluida,
informe, indeterminada, a matéria é algo real.  Não é o nada
absoluto do cristianismo,  onde se diz que Deus criou o mundo do
nada, ou melhor, se precisar de nenhuma matéria preexistente.} 

Finalmente, aqui estamos diante do terceiro elemento plotiniano,
que influenciaria profundamente na solução agostiniana do mal; a
noção de “nada”, como equivalente  ao conceito de “não Ser”.

Durante o tempo em que fora maniqueu, Agostinho não conseguia
conceber realidade alguma senão como substância material ou
corpórea, incluindo não só  Deus, mas também o “nada”. Para ele,
substância era sinônimo de coisas corpóreas.\footnote{ \textsc{jolivet},
1932, p. 76, apresenta este ponto como o principal obstáculo
para uma solução do problema do mal durante o período em que
Agostinho foi maniqueísta, ou seja, a dificuldade que tinha em
conceber o mal como um “nada”,  que ele chama de “nulidade
metafísica”, um problema filosófico, que só seria resolvido a
partir de agora com seu encontro com o neoplatonismo. Aqui vale
relembrar, conforme comentamos no início do capítulo \textsc{i}, quando
apresentamos os motivos que levaram Agostinho a entrar no
maniqueísmo, que este havia lido \emph{As Categorias} de
Aristóteles, e a partir dali, não conseguia ver nada que não
encerrasse em si as categorias aristotélicas da substância, ou
seja, que não fosse um corpo com tais predicados. Por isso,
tanto para o maniqueísmo, e consequentemente para Agostinho,
assim como para Aristóteles a noção de “vácuo” era simplesmente
inconcebível. Já \textsc{ricoeur}, 1988, p. 31--2,  mostra que o conceito
neoplatônico de “não ser”  iria ajudar Agostinho, não só na
concepção do mal como privação, defecção --- não ser, mas na
aceitação da ideia bíblica de criação a partir do nada, que se
apresenta como um nada absoluto: “Se Agostinho pôde opor"-se à
visão trágica da gnose é primeiramente porque ele pode pedir
emprestado da filosofia, do neoplatonismo, um aparelho
conceptual capaz de arruinar a aparência conceptual do mito
racionalista. Dos filósofos, Agostinho sustenta que o mal não
pode ser entendido como substância, pois pensar o ``ser'' é pensar
``inteligivelmente'', pensar o ``\emph{Uno}'', pensar o ``bem''.
Então, a filosofia exclui todo o fantasma do mal substancial.
Por outro lado, nasce uma nova ideia de nada, a do \emph{ex
nihilo,} contida na ideia de uma criação total e sem excesso”.}  Ou seja,
enquanto fora maniqueu, quando tentou definir  o “nada”, só pôde descrevê-lo  em
termos corpóreos, como um
“espaço vazio” (\emph{locus inanis}), como diz: 

\begin{quote}
Pois, tudo o que concebia como não ocupando espaço, me parecia um
nada, um \emph{nada} absoluto, e não um vazio como sucederia,
se um corpo é removido de um lugar, e o lugar fica vazio de todo
corpo, seja terrestre, úmido, aéreo ou celeste\emph{,} mas,
enfim, um lugar vazio, como um \emph{nada}
\emph{espaçoso}.\footnote{ \emph{Conf.} \textsc{vii}, 1,1.} 
\end{quote}

Para o maniqueísmo o “nada” era equivalente a um “espaço vazio”,
mas um espaço corpóreo, que é uma coisa ou um ser, e não um 
“nada absoluto”, que era inconcebível. Por isso, tudo é
concebido como substância corpórea, e o “nada” a rigor, não
existia para o maniqueísmo.

Entretanto,  apesar de Plotino ter definido o não ser (ou o nada)
como o “ilimitado”, o “informe”, o “indeterminado”, isso, para
Agostinho, ainda não resolvia plenamente o problema do mal, por
tratar"-se ainda de uma explicação natural, quando o coloca no
universo físico ou na matéria;  não que  esta seja o mal em si,
como pensavam os maniqueus, mas  o lugar onde o mal acontece, já
que ela é a possibilidade do mal, ou seja, a matéria é o mal
enquanto potência e não enquanto ato. Além disso, o mal em
Plotino, exerce  uma função necessária no cosmo, diferentemente
de Agostinho, conforme palavras de Turrado: 

\begin{quote}
Para  Agostinho, os males não podem ser naturais; para Plotino, 
sim. Sem eles, para Agostinho, a ordem seria  perfeita; para
Plotino, ela não existiria.  Para Plotino, o mal e o pecado não
têm que ajustar"-se à ordem, formam parte  necessária dela e a
compõem. Para Agostinho, perturbam a ordem, da qual não formam
parte natural; se se ajustam  à ordem,  é  exercendo uma função
acidental, não essencial.\footnote{ \textsc{turrado}, 1995, p. 762.} 
\end{quote}

Assim, em Plotino o mal é algo necessário,\footnote{ \textsc{mondolfo},
1973, p. 215, falando sobre a necessidade do mal em Plotino, diz
que, por questões lógicas, da mesma forma como o \emph{Uno} é
necessário, visto que sem ele não haveria o segundo ou os
múltiplos, assim também é o mal, última processão, sem a qual os
anteriores ou intermediários não seriam o que são. Igualmente
diz \textsc{reale}, 1994, p. 488: “Se é necessário que exista o que
vem logo depois do primeiro (o \emph{Uno}); consequentemente
haverá também termo último, a matéria, que nada mais guarda
dele: e essa é a necessidade do mal”.} pois, para que as
coisas participem, é necessário que o mal seja também em si,
conforme palavras do autor da \emph{Enéadas}: “É preciso que
tenha um ilimitado em si, um informe em si, e assim pelas outras
propriedades que caracterizam a natureza do mal\ldots{}”.\footnote{
\emph{En.} \textsc{i}, 8, 3. Aqui, mais uma vez, voltamos a observar
que, mesmo  sendo algo necessário, pelo fato de a matéria
derivar, em última instância do \emph{Uno}, o mal em nada
compromete a incorruptibilidade e imutabilidade de Deus, pois
Deus pode dar sem perder. Deus é ao mesmo tempo imanente e
transcendente --- ou panenteísta (omnipresente).} Por outra parte, “o
mal não é senão um corolário da diversidade essencial e
necessária dos seres criados e da essencial e necessária
limitação do ser contingente”.\footnote{ \textsc{jolivet}, 1932, p. 94.
Igualmente diz \textsc{quiles}, 1981, p. 24: “O mal para Plotino está nos
seres, como algo que é o oposto ao bem absoluto. Tanto que,
assim como existe o bem absoluto, este mal deve ser também um
mal absoluto”. Ver, também, \textsc{sciacca}, 1956, p. 17, que diz: “Para
Plotino o mal é natural, ele é uma necessidade inerente à ordem
natural, porque as almas caíram na miséria (matéria), da qual os
vícios decorrem ‘necessariamente'\,”.}

Assim sendo, o mal aparece como elemento necessário dentro da
degradação da perfeição expressa na multiplicidade das
essências. Como nenhuma delas pode realizar todo o Ser, há
inevitavelmente, ou necessariamente, uma desigualdade das
essências emanadas. O mal faz parte da “ordem” do universo, ou 
está  determinado  aprioristicamente, já que, conforme observa
Cirne"-Lima,\footnote{ Cf. \textsc{cirne}-\textsc{lima}, 1997, p. 85.}  ``se tudo
está pré-programado no ovo inicial, assim também o é o mal”. Ele
não é nada mais que o limite, a  negação do bem maior e, nesse
sentido, em nada incomoda a beleza do universo, e de certa forma
ele é um “bem” necessário,\footnote{ Cf. \textsc{jolivet}, 1932, p. 94 e
97 que diz: “A  matéria, em si mesma, é o mal, mas no conjunto
tem o seu papel e, com relação ao todo, tem também sua bondade e
sua beleza. De fato, o mal não existe, então, segundo Plotino,
senão relativamente às essências mais perfeitas, não é senão um
bem menor”.} reduzindo o mal a um problema puramente de
estética na ordem natural do universo.  E tal  posição ainda não
satisfazia plenamente o coração inquieto de Agostinho. Observa
Vannini: 

\begin{quote}
O catecúmeno  da Igreja católica  não estava instigado  só por 
questões estéticas, mas, bem mais, por problemas existenciais. É
um sentido para sua vida que ele busca. A verdade não tem para
ele só uma dimensão  noética, mas também, prevalentemente, ética
e religiosa.\footnote{ \textsc{vannini}, 1989, p. 38.}  
\end{quote}

De qualquer maneira, se a ontologia plotiniana ainda não resolvia
plenamente  o problema do mal,  pelo  menos Plotino serviu"-lhe
de ponto de partida, ou estimulador, para uma solução
definitiva, conforme diz Sciacca, com muita precisão:

\begin{quote}
Os ``platônicos'' ajudaram, com certeza, Agostinho a esclarecer o
problema do mal que, desde muitos anos, e mais que qualquer
outro, o atormentava. Antes de tudo, Agostinho viu, neste
momento, a ``maneira'' de resolver o problema; porém, não viu a
``solução'' [\ldots{}]. Os ``platônicos'' serviram"-lhe somente de ponto
de partida. Desde então Agostinho se engaja numa via autônoma,
que o conduzirá a resolver o problema de maneira totalmente
pessoal.\footnote{ \textsc{sciacca}, 1956, p. 14. Igualmente diz
\textsc{jolivet},
1932, p. 99--100: “Plotino teve a imensa vantagem de colocá-lo no
caminho das soluções que ele buscava obscuramente e, antes de
tudo, de arrancá-lo de seu materialismo. Este foi o grande
benefício que tirou da leitura das obras neoplatônicas e isso
basta para justificar toda a importância que Agostinho
reconheceu sempre à influência destas obras, para seu progresso
para a fé, já que o materialismo era a raiz de todos os seus
erros. Entretanto,  Agostinho se serviu da doutrina plotiniana
como uma espécie de trampolim para saltar mais longe e mais alto
[\ldots{}]. Plotino representava, a seus olhos só um meio, de nenhum
modo um fim”. Da mesma opinião é \textsc{bettetini}, 1994, p. 76: “Para
Agostinho a noção de nada dará a chave para o problema do mal,
que ele buscava resolver apoderando"-se da concepção maniqueia de
um mal substancial [\ldots{}]. Em termos ontológicos, o mal se revela
agora como privação do ser e portanto de beleza e de bondade, 
tendência ao nada, enquanto, em termos éticos, o mal se
configura como a livre escolha da vontade que prefere ir contra
a ordem das coisas e tender para o corpóreo muito mais que para
o divino”. Igualmente diz \textsc{dolby} \textsc{múgica}, 1989, p. 443: “Que
descobre santo Agostinho no Círculo de Milão acerca do mal?
Fundamentalmente, sua falta de entidade ontológica. O mal,
repetirá continuamente ao longo de seus escritos, é uma
carência, ausência de substância”.}
\end{quote}

Ao ler as \emph{Enéadas,}  Agostinho despertou para 
possibilidade de haver um contrário à substância, seja ela
material, ou espiritual, ou seja, a não substância --- o não ser
ou o “nada”. A partir daí, Agostinho daria um importante passo
na busca de uma solução para o problema do mal, que começou a
parecer como que “um tirar fora”, uma privação, ou seja, o mal
não é uma substância, não forma um ser, mas, pelo contrário, é
ausência, defecção, do Bem --- o não ser: 

\begin{quote}
Vi claramente que todas as coisas que se corrompem são boas: não
se poderiam corromper, se fossem sumamente boas, nem se poderiam
corromper, se não fossem boas. Com efeito, se fossem
absolutamente boas, seriam incorruptíveis, e se não tivessem
nenhum bem, nada haveria nelas que se corrompesse. De fato, a
corrupção é nociva, e se não diminuísse  o bem, não seria
nociva. Portanto, ou  a corrupção nada prejudica --- o que não
seria aceitável --- ou todas as coisas que se corrompem são
privadas de \emph{algum bem.} Isso não admite dúvida. Se,
porém, fossem privadas de todo o bem, deixariam inteiramente de
existir. Logo, enquanto existem, são boas [\ldots{}]. Portanto, todas
as coisas que existem são boas, e aquele mal que eu procurava
não é uma substância, pois, se o fosse, seria um bem.\footnote{
\emph{Conf.} \textsc{vii}, 12, 18.}
\end{quote}



Tal conceito, mais tarde,  ajudaria Agostinho a passar do sentido
ontológico"-estético"-natural  a um sentido
ontológico-ético"-moral, no cristianismo, que significa falta,
ausência, defecção, privação de essência ou de substância --- a
não substância, que acontece, não no mundo natural, como em
Plotino, mas no livre"-arbítrio do homem, ou melhor, na má
vontade do livre"-arbítrio no homem, uma realidade puramente
% Verificar nota
imaterial.\footnote{ A esse respeito \textsc{sciacca}, 1956, p. 17,  diz:
“No momento mesmo que lia Plotino, Agostinho transpunha o
conceito plotiniano de ``não ser'' e o traduzia em termos
diferentes. Isso porque ele já era um cristão, porque o
cristianismo lhe ensinou que o mundo não é uma emanação, mas uma
criação de Deus; que, desde então, a matéria, enquanto tal, é
também criação de Deus, não é um mal, e que não existe o não ser
substancial”. Igualmente \textsc{de} \textsc{capitani}, 1994, p. 63: “Quando lhe
perguntavam, como ele repete muitas vezes nos escritos
antimaniqueus: \emph{unde malum?} (Conf. \textsc{iii}, 7,12), ele não
sabia ainda e não podia saber que o mal não é senão
\emph{privação} de bem --- \emph{usque ad quod omnino non est.}
Este aprofundamento doutrinal foi fruto nele da descoberta do
neoplatonismo”. E noutro texto: \textsc{de} \textsc{capitani}, 1985, p. 69, diz:
“Agostinho toma do neoplatonismo de Plotino ou de Porfírio
substancialmente  a ideia do mal como negatividade, como falta,
como privação, como deficiência de ser e, portanto, de todo
bem”.} 



\section{Bibliografia}


\begin{description}\labelsep0ex\parsep0ex
\newcommand{\tit}[1]{\item[\textnormal{\textsc{\MakeTextLowercase{#1}}}]}
\newcommand{\titidem}{\item[\line(1,0){25}]}

\tit{agostinho}, Santo. \emph{Confissões.} 9. ed. Trad. de J.
Oliveira Santos e A. Ambrósio de Pina.  Petrópolis: Vozes, 1988.
367 p. (livro)

 \titidem. \emph{O
livre"-arbítrio}\emph{.} Trad.,  introd. e notas de Nair de
Assis Oliveira. São Paulo: Paulus, 1997. 296 p. (Coleção
Patrística, n. 8).

 \titidem. \emph{Contra os
acadêmicos --- A ordem --- A grandeza da alma --- O mestre}. Trad. de
Agustinho Belmonte. São Paulo: Paulus, 2008. 414 p.
(Coleção Patrística, n. 24).

 \titidem.  \emph{A vida
feliz}\emph{:} diálogo filosófico. Trad. e notas de Nair de
Assis Oliveira. São Paulo: Paulinas, 1993. 109 p..

 \tit{alsina clota}, José. \emph{El neoplatonismo}: síntesis del
espiritualismo antiguo. Barcelona: Anthropos, 1989. 160 p.


\tit{bettetini}, Maria. \emph{La misura delle cose}:\emph{
s}truttura e modelli dell'universo secondo Agostino d'Ippona.
Milano: Rusconi, 1994. 268 p.

 \tit{bussola}, Carlo.
\emph{Plotino}\emph{:} a alma no tempo. Vitória:
\textsc{fcaa}/\textsc{ufes},
1990. 61  p.

 \tit{cirne"-lima}, Carlos.  \emph{Dialética para
principiantes}\emph{.} Porto Alegre: \MakeUppercase{Edipucrs},
1997, 236 p.

 \tit{copleston}, Frederick. El neoplatonismo
plotiniano. \emph{In:} \emph{Historia de la
filosofía}\emph{:} Grecia y Roma. José Manuel Garcia de la
Mora. Barcelona: Editorial Ariel, 1977. vol. \textsc{i}, p. 455 ---
465.

 \tit{costa}, Marcos Roberto Nunes. \emph{O problema do mal
na polêmica antimaniqueia de santo Agostinho}\emph{.} Porto
Alegre: \MakeUppercase{Edipucrs}; Recife: \textsc{unicap}, 2002. 429 p.


 \titidem. \emph{Maniqueísmo:} história,
filosofia e religião. Petrópolis: Vozes, 2003. 175 p.

 \tit{de capitani}, Franco. Il problema del male nell'\textsc{viii} trattato della
prima ``Enneade'' di Plotino. \emph{In:} \textsc{angelli}, Franco (Org.).
\emph{Sapienza antica:} studi in onore di Domenico
Pesce. Parma : Facoltà di Magistero --- Università degli Studi di
Parma, 1985, p.~68 -98.

 \tit{dolby mugica}, Maria del Carmen. El
problema del mal en san Agustín y la racionalidad de lo real.
\emph{Revista Agustiniana}\emph{.} Madrid, v. 93, 1989, p.
437--454.

 \tit{fraile}, Guillermo. Plotino. \emph{In:}
\emph{Historia de la filosofía}\emph{:} Grecia y Roma. 
Madrid: La Editorial Catolica, 1956. vol \textsc{i}, p. 703 --- 735. 


\tit{jolivet}, Regis.  \emph{San Agustín y el neoplationismo
cristiano}\emph{.} Trad. de G. Blanco ; 0. Iozzia ; M. Guirao
; J. Otero ; E. Pironio y J. Ogar. Buenos Aires: Ediciones
C.E.P.A, 1932. 219 p.

 \tit{klimer}, Federico ; \textsc{colomer}, Eusebio.
Plotino. \emph{In}: \emph{Historia  de la
filosofía}\emph{.} Madrid: Editorial  Labor, 1961. p. 108 ---
115.

 \tit{mondolfo}, Rodolfo. Plotino. \emph{In:} \emph{O
pensamento antigo:} história da filosofia grego"-romana
(\textsc{ii}): desde Aristóteles até os neoplatônicos. Trad. Lycurgo
Gomes da Motta. São Paulo : Mestre Jou, 1973, p. 199 --- 226. 


\tit{O'meara}, John J. \emph{La jeunesse de saint Augustin}\emph{:}
son évolution intérieure jusqu' à l'époque de la conversion.
Trad. de Jeanne Henri Marrou. Paris: Librairie Plon, 1954. 277
p.

 \tit{plotino}. \emph{Enneadi}\emph{.} 4. ed. Trad., Int. e
noti di Giuseppe Faggin. Milano: Rusconi, 1996. 1602 p.


\tit{quiles}, Ismael. \emph{Plotino}: a alma, a beleza e a
contemplação. Trad. de Ivan Barbosa Rigolin e Consuelo
Colinvaux. São Paulo: Centro Editor --- Associação Palas Athena,
1981. 103 p.

 \tit{reale}, Giovanni. Plotino e o neoplatonismo.
\emph{In:} \emph{História da filosofia antiga:} as
escolas da era imperial. Trad. Marcelo Perine e Henrique C. de
Lima Vaz. São Paulo : Loyola, 1994, vol. \textsc{iv}, p. 399 --- 533.


\tit{ricoeur}, Paul. \emph{O mal:} um desafio à filosofia e
à teologia. Trad. Maria da Piedade Eça de Almeida. Campinas :
Papirus, 1988, 53 p. 

 \tit{ricón} \textsc{gonzález}, Alfonzo. \emph{Signo
y lenguaje en san Agustín}\emph{.} Bogotá: Centro Editorial,
Universidad Nacional de Colombia, 1992. 215 p. 

 \tit{roque}, Maria
Luiza. “De Doctrina  Christiana” de Santo Agostinho e a
Filosofia de Platão. \emph{In:} \textsc{vv}.\textsc{aa}. \emph{Atualidade de
Santo Agostinho}\emph{.} Sorocaba : Faculdade de Filosofia,
Ciências e Letras de Sorocaba, 1955, p. 91 --- 99.


\tit{rostagno},
Sergio. Libertà e servitù dell'uomo. \emph{In:}
\textsc{perissinotto},
Luigi (org.). \emph{Agostino e il destino
dell'Occidente.}  Roma: Carocci Editore, 2000. p. 47 ---
59.

 \tit{sciacca}, Michele Federico.  Plotino. \emph{In:}
\emph{História da filosofia}\emph{:} Antiguidade  e Idade
Média. Trad. de Luis Washington Vita.  São Paulo: Mestre Jou,
1966. vol \textsc{i}, p. 137--145.

 \titidem. \emph{Saint
Augustin et le néoplatonisme}\emph{:} la possibilité d'ume
philosophie chrétienne. Paris: Éditions Béatrice"-Nauwelaerts,
1956. 67 p.

 \tit{sesé}, Bernard. \emph{Agostinho}\emph{:} o
convertido. Trad. de Magno Vilela. São Paulo: Paulinas, 1997.
158 p.

 \tit{turrado}, Argimiro.  El problema del  mal y  la
responsabilidad moral de las personas  especialmente  en  la
“Ciudad de  Dios” de  s. Agustín. \emph{Revista
Agustiniana}\emph{,} 1995. p. 733--789.

 \tit{vannini}, Marco.
\emph{Invito al pensiero di sant'Agostino}\emph{.} Milano:
Mursia Editore, 1989. 200 p.

\end{description}


\capitulo{Ser e bem"-ser em Dionísio Areopagita}%
	 {José María Nieva}%
 	 {unt"-unsta/\,argentina}

\markboth{Ser e bem"-ser em Dionísio Areopagita}{José María Nieva}


``A vida é ação'', afirma de uma maneira surpreendente, mas, de certo
modo enigmática, Aristóteles (\textit{Política} \textsc{i} 1254a) 

Que tal afirmação seja enigmática está confirmado porque, segundo o 
Estagirita, toda \textit{praxis}, toda ação implica certa tensão do \textit{telos}, rumo a um fim, 
em direção a um certo bem (cf. \textit{Ética a Nic.} \textsc{i} 1094a); não obstante, 
alcançar este \textit{telos} por parte daquele ser que é o único capaz da
\textit{praxis} exige a tomada de decisão, e em consequência disso, a tarefa 
de deliberar para obtê-lo: é necessário exercitar"-se como um bom arqueiro, 
isto é, ser capaz de atingir a marca branca.

Ao fim, então, trata"-se não apenas de um termo referente a uma ação, senão 
também da plena realização do sujeito que atua. Este é porque trata"-se não
apenas do viver, mas do viver bem (cf. \textit{Política} \textsc{i} 1253b; 1258a; III 1280a)
e, o Estagirita entende por isso algo que possui um excedente de valor. 
Algo, portanto, que é melhor e que o torna preferível. Em \textit{Tópicos} 118a
ele afirma que ``viver bem é melhor que viver; viver bem é fruto de um
excedente, enquanto viver é necessário. O excedente existe quando, estando
disponível o necessário, procura"-se outras coisas de valor''.\footnote{
  Algumas linhas antes da passagem, em \textit{Tópicos} 116a, Aristóteles
  determina os lugares do preferível e afirma que o mais desejável é o mais
  duradorou e o mais estável, e entre esses lugares observa ``o bem, pois 
  todas as coisas tendem ao bem''. O bem como \textit{telos} postula certos ecos
  platônicos que se farão sentir, com maior claridade, no neoplatonismo.
  Cf.~\textsc{Moustsopulos}, 2007, p.257--262.}

Esta célebre distinção já havia sido estabelecida pelo Sócrates platônico, 
quando afirmava, com uma clara preocupação com a saúde da alma no exercício da
virtude, que ``o que vale não é viver, mas o viver bem'' (\textit{Críton} 48a)
e, para ele, isto não era mais do que viver como exigia a probidade e a justiça. 

De uma maneira ou de outra, esta inquietude platônico"-aristotélica -- com seus 
próprios matizes e seus próprios interesses -- está presente no debate
contemporâneo sobre a reflexão moral, quando a pergunta acerca da natureza do bem 
se transpõe à pergunta perturbadoramente e torna"-se ``por que queremos ser
bons?''. Pergunta que pode se abrir em outras tantas, mas que, ao fim e ao cabo,
poderiam se reduzir ao essencial: ``como podemos ser melhores, isto é, bons?''.

Tal pergunta é assumida a partir de uma perspectiva de revalorização da
filosofia moral grega, por R.~Kraut em seu instigante livro \textit{What is Good
and Why: The Ethics of Well"-Being} (\textsc{hup}, 2007). 

Portanto, uma retrospectiva do passado não tem por que ser necessariamente
histórica, arqueológica. O passado pode ter, para não dizer ser, uma fonte
inesgotável de respostas latentes para aquele que queira exercitar
plenamente o sentido do humano, o ser e fazer"-se homem. De tal maneira que a
atenção no passado pode abrir o espaço ao novo, ao estímulo para criar o novo 
ao nutrir"-se do já ocorrido e do que já foi dito. 

%1
Em decorrência disso, o propósito das páginas que se seguem é desentranhar 
o significado e o papel que possui noção de ``bem"-ser'' em um autor que,
graças a um pseudônimo e atenção ao pensamento grego, marcou profundamente a
filosofia e a teologia medievais: Dionísio Areopagita.

%2
Tal noção não recebe, no marco de suas obras, uma clara e explícita tematização,
isto é, nosso autor faz alusão a ela como se fosse já algo bem conhecido, e o
destinatário de seu pensamento, portanto, supera com certeza o que ele pretende
fazer entender. 

Mas, tal noção possui um relevo singular no pensamento neoplatônico no processo
anagógico que a alma realiza em busca de seu verdadeiro ser, e que a dispõe para
alcançar a meta última do processo: a união com o Uno. Por isso, para
compreender o que Dionísio tem em mente com esta noção é necessário voltarmos
para o pensamento sobre o verdadeiro sentido que confere. 

No tratado 9 da \textit{Enéada} \textsc{vi}, Plotino traça, com nitidez e força
espiritual inigualável, não só a preparação intelectual que a alma deve produzir
para encontrar a si mesma, como também a superação que há de realizar em si
mesma para unir"-se com o Uno. De fato, todos os tópicos abordados em tal tratado
convergem ao que é principal: o tema da unificação interior, isto é, da unidade.
Afirmação fundamental desde o início, quando adverte que tudo o que é \textit{é}
em um tanto uno. Isto implica para a alma, consequentemente, não viver em
dispersão e multiplicidade, na região da não semelhança, no \textit{topos} 
da alteridade, onde mantém"-se esquecida do Pai e, por sua vez, esquecida de si
mesmo. O retorno até si mesmo é o retorno à unidade, até a unificação interior,
rumo à meta definitiva da verdadeira vida: voltar"-se para o Uno.

%3
Plotino afirma, então, que ``tender a ele [o Uno] é ser em maior grau, onde está
o \textit{bem"-ser}; distanciar"-se dele é ser meramente, ser em menor grau.  Nele
é onde a alma descansa e se livra de males instalando"-se na região limpa de todo
mal\ldots{} viver neste lugar é viver de verdade; porque a vida presente, a vida
sem Deus, é um rastro de vida e um arremedo daquela'' (\textit{En.} \textsc{vi}
9, 9, 12--16).

A tensão ao Uno não é outra coisa que o pleno exercício do que se poderia 
chamar de uma atitude \textit{epistrófica}, isto é, para ser realmente,
plenamente, é preciso retornar"-se ao Uno; mas este retorno significa tomar
consciência que, apesar de estar presente em nós, somos nós que estamos
presentes nele. A conversa é, então, um acréscimo, um aprofundamento do sempre
presente. Nesta atitude a alma vai descobrindo"-se a si mesma e, por sua vez,
adquire sua verdadeira identidade, realizando desse modo o que Hadot chama de
\textit{perfeição existencial} (Hadot, 1994, p.192).\footnote{
     Cf. \textsc{Schniewind}, 2003; \textsc{Mc Groarty}, 2006; \textsc{Wallis}, 2002}
Ela é mais na medida em que
se volta para as realidades superiores e nele obtém seu bem"-ser. Com outras
palavras, na tensão para o Uno se demonstra verdadeiramente o que um ser é,
expressando em outras palavras, em sua conversão para o Uno, um ser que adquire
cada vez
mais valor ao recuperar com o movimento o que perdeu a causa do distanciamento
da Fonte originária de seu ser. Podemos dizer, então, que por este movimento de
conversão, tão fundamental para todo o neoplatonismo, um ser adquire forma e
uma qualificação cada vez maior por sua aproximação do Bem. 

%4
A conversão é desejo do Bem e o Bem que se deseja coincide com a plena
realização de todo ser, enquanto todo ser -- segundo Plotino -- deseja sempre
algo que não tem, e o que não tem é o ser pleno, ou melhor, o \textit{bem"-ser}.
Todo desejo tende, portanto, ao aperfeiçoamento, ao cumprimento ou realização, à
definição do que é indefinido e, desse modo, implica um progresso que, para cada
um, consiste na recuperação de seu próprio ser e na obtenção da felicidade
através de sua própria autorrealização.\footnote{
  \textsc{Di Pasquale Barbanti}, 1994, p. 193
  } Uma vez que o Bem se identifica com o
Uno, o desejo do bem e da realização torna"-se desejo de unidade e de unificação
enquanto todo ser tende à própria unidade uma vez que tende à Unidade absoluta,
com base no princípio segundo o qual alcançar a si mesmo significa, de certo
modo, alcançar o Uno. Na \textit{Enéada} \textsc{vi} 5, 1 Plotino afirma que
``este princípio é o mais firme de todos e o que formulam, por assim dizer,
nossas mentes não como recapitulação de todos os princípios particulares, mas
como premissa anterior a todos os princípios particulares, incluindo o que
estabelece e enuncia que todos os seres aspriram ao bem. Este princípio  se
verifica, efetivamente, apenas se supusermos que todos os seres tendem à unidade e
constituem uma unidade e de que a unidade seja objeto de sua aspiração\ldots{}
Porque nisto consiste o bem desta natureza una: em ser de si mesma e em ser ela
mesma, isto é, em ser una. E por isso se diz também com razão que o bem é
próprio''.

%5
Esta perspectiva metafísico"-espiritual perdura em Proclo quando, na proposição
31 dos \textit{Elementos de Teologia}, afirma: ``Todas as coisas desejam o Bem,
e cada
uma o alcança através da mediação de sua causa próxima; por isso cada uma tem
desejo de sua própria causa. Por aquilo que lhe fornece o ser obtém seu
\textit{bem"-ser}; a fonte de seu \textit{bem"-ser} é o objeto primário de seu
desejo; e o objeto primário de seu desejo é aquilo que tende à conversão.''

Conversão e desejo se misturam tão fortemente que pode ser apenas na medida em
que retorna para o Princípio. Isto permite afirmar, ademais, que o ser revela
uma estrutura desiderativa, portanto, uma nota distintiva o perpassa em sua
totalidade: a carência, a saudade, a falta, em termos mais
positivos, o desejo de plenitude. 

%6
Igual a Plotino (cf. \textit{En.} V 5, 12, 8; VI 2, 11, 25; VI 7, 31, 17) o
desejo do Uno, do Princípio que faz todas coisas serem, é constitutivo de todo
ser e para que este seja é necessário que retorne, converta"-se, volte"-se para o
ser plenamente. Na mesma proposição citada, Proclo o estabelece ao começar
dizendo: 
``Porque se procede, mas não ser converte em causa de sua processão, não tem
desejo da causa, uma vez que todo o desejo está voltado para o objeto de seu
desejo'' (cf. \textit{El.} \textit{Th.} Prop.~38). O desejo com sua nota
de falta e de carência constitui todo ser que procedeu, todo ser que mostra em
si a dependência de sua causa. Esta demonstração não é senão o signo de não ser
autossuficiente, portanto, na busca de completude, de perfeição, de acabamento,
de realização.

%7
Contudo, segundo Proclo, cada ser volta a seu princípio, segundo seu modo de ser
específico ou próprio, isto é, já essencialmente, já vitalmente, já
cognoscivelmente. Enquanto a processão faz com que o ser seja, com o surgimento de
sua fonte primordial, a conversa ou retorno permite  que este mesmo ser adquira
uma identidade, um valor, uma bondade qualitativa que o transforma
significativamente porque neste movimento de regresso obtém uma unificação cada
vez maior, uma semelhança cada vez mais perfeita, \textit{kata to dynaton}, com
seu Princípio.\footnote{Cf. \textsc{Lloyd}, 1998, p. 127, \textsc{Di Pasquale
Barbanti}, 1993, p. 91--92;
\textsc{Beierwaltes}, 1990, p. 161--198; \textsc{Gersh}, 1973, p. 25--48.}

De fato, Proclo se pergunta: ``Se permanece e procede, mas não se
converte, como então acontece de cada coisa ter um desejo natural por seu 
\textit{bem"-ser} e pelo Bem, e haver uma tensão ascendente ao seu Princípio?'' 
(\textit{Elem.~Teol.} prop.~35). Ideia magistralmente explicitada no
\textit{Comentário ao Alcibíades} \textsc{i} quando afirma que ``aí onde está o
ser, aí mesmo está o \textit{bem"-ser}'' (317, 1), dizendo de uma maneira mais clara
e contundente no \textit{Comentário ao Parmênides} \textsc{vii} 1210, 
que ``quando Platão nas \textit{Leis} \textsc{iv} 716c diz que o divino é a
medida de todas as coisas, e muito mais que o homem, como um de seus
predecessores diz que é preciso entender qual sentido se diz que é uma medida, 
temos então que reconhecer que ele está falando no sendido de proporcionar a
todos os seres tanto seu ser como seu \textit{bem"-ser}, ao ser a causa
originária da medida em cada um deles''.

O Uno"-Bem é a medida de todas as coisas, uma vez que é o objeto de desejo, de
saudade e de esforço para todos os seres. (Cf. \textit{In Parm.} \textsc{vi
1194}.
Para o ser que é capaz de ação, como o homem, o movimento de conversão não é
tanto uma troca de um lugar mas uma consciência ou intencionalidade cada vez
mais clara de seu Princípio, em outras palavras, é a direção da vontade; uma
tensão que sustenta a Fonte de seu bem através do exercício das virtudes (cf.
\textit{Comentário à República} \textsc{i} 206, 26--207, 1; \textsc{i} 236, 21; \textsc{i} 271, 1). Tensão
que Proclo explicita bem ao diferenciar amor e desejo, porque embora 
``amor e desejo tenham o mesmo objeto, eles não se distinguem um do outro senão
no relaxamento e na tensão do desejo'' (\textit{Comentário a Alcibíades
\textsc{i}} 329, 24).

Esta tensão se manifesta de uma maneira muito clara no \textit{Comentário à
Rep.}~Diss.~\textsc{vii}, onde mostra que a virtude aperfeiçoa a natureza graças
à educação. A virtude aperfeiçoa porque outorga o bem"-ser a cada ser segundo a
natureza constitutiva de cada ser. A virtude é aquilo que aperfeiçoa a vida de
quem a possui e, em consequência, é fonte do bem"-ser, não do ser. A virtude pode
fazer melhor do que é e já existe. Nesse sentido é um aperfeiçoamento
existencial que realiza no seu ser a completude. Na seguinte
\textit{Dissertação} \textsc{viii} do mesmo \textit{Coment.}~Proclo explica de
uma maneira precisa que este aperfeiçoamento obtém sua força do Bem:
``\textit{ten ek tagathou teleioteta}.

Se entende desse modo que \textit{Panta gar ta onta tou agathou ephietai} 
(ibidem 329, 14; cf. \textit{Teologia Platônica} \textsc{i} 22), a universalidade desta
proposição, que funde suas raízes na tradição platônico"-aristotélica, cobra um
relevo singular quando é pensada para o homem, uma vez que este, quando busca o
Bem, ao buscá-lo com esforço sustentado e com clara consciência de que aí
encontra seu repouso, deve manter constantemente esta tensão se quiser obter seu 
\textit{bem"-ser}, tensão que por fim, é mantido pelo Bem mesmo enquanto ``atrai
todas as coisas para si mesmo, está pleno dele mesmo e dá a todos os seres o
\textit{bem"-ser}'' (ibidem 153, 18).

Estas ideias reaparecem de uma maneira muito difusa no tratado dionisiano
\textit{Sobre os nomes divinos}. Repetir que ela deriva diretamente de Proclo
pode, até certo ponto, amainar a inquietude do investigador, mas é necessário
reconhecer que isto proporciona, por outro lado, uma série de problemas, como
por exemplo: quando o autor talvez o tenha lido? 

Neste caso preciso, como afirmamos acima  quanto à noção que buscamos elucidar
sobre o \textit{bem"-ser}, Dionísio não se detém para explicá-la ou torná-la 
compreensível. Pelo contrário, parece já ser uma noção forjada e compreensível
para seu destinatário.

De fato, nos capítulos quatro e cinco do tratado mencionado, capítulos que
podemos considerar nevrálgicos na trama argumentativa dionisiana, surge a noção
de \textit{bem"-ser} em um claro contexto metafísico"-espiritual. 

Referindo a divina denominação de Bem, que enquanto Bem essencial estende sua
bondade a todos os entes, o autor precisa pontualmente com relação às
inteligências angélicas que ``elas têm morada na Bondade, e disso se dá o
cimento e a coesão, a proteção e a casa dos bens e, guardando"-a, mantém também o ser e o \textit{bem"-ser},
e configuradas de acordo com ela, são também conformadas ao bem e, segundo o
guia da norma divina, às que estão com elas lhes comunicam os presentes que a
partir do Bem se expandem até elas'' (\textit{Sobre os nomes divinos}
696a).\footnote{ 
  Cito os textos dionisianos segundo a edição crítica \textsc{Corpus
  Dionysiacum} \textsc{i--ii}, Berlin: W. de Gruyter, 1990--1991. A tradução do
  grego é nossa.}

Como diz o parágrafo, a noção de \textit{bem"-ser} está aplicada somente àqueles
seres que não só possuem uma capacidade intelectiva, mas também uma capacidade
desiderativa eletiva, enquanto todos os meios utilizados para explicá-la, de
certo modo, aludem à estabilidade. O Bem é estabilidade autossuficiente e
plenitude absoluta para todo ser. Contudo, nesta tensão, estes seres encontram
seu \textit{bem"-ser}, o qual revela com certa claridade que ser -- para Dionísio
-- é movimento, movimento espiritual. 

De fato, este capítulo demonstra muito bem que o ser é movimento, porque nele
Dionísio se dedica a ressaltar as diferentes classes de movimento: circular,
retilíneo, helicoidal. Esses são diversos modos de ser, o primeiro
caracterizando especialmente as inteligências angélicas, mas que não lhe
impossibilita de alcançar as almas racionais, se estas forem capazes de retornar
a si e desse modo voltar"-se para seu Princípio imitando o movimento
constantemente uniforme daquelas. Embora estes movimentos se diferenciem muito
bem entre si segundo a conversão, isto é, segundo o retorno que se opera a fim
de alcançar o Bem. 

Mas antes de compreender melhor as consequências destas ideias, vejamos como
aparece a noção de \textit{bem"-ser} neste capítulo cinco. Aqui nosso autor está
centrado na divina denominação de ser e, embora de um modo diferente, volta a
marcar que as essências angélicas estão cimentadas ao redor da Tríade mais que
essencial, obtendo desse modo não só o ser, mas também o ser deiforme. Mas na 
linha seguinte nos diz que ``as almas e todos os demais entes, segundo o mesmo
princípio, têm não só o ser, mas também o \textit{bem"-ser} Nele, existindo e
sendo boas, começando [a ser] desde Ele e conservando"-se Nele e encontrando seu
limite Nele (\textit{Sobre os nomes} 821d).

Como entender, então, a noção de \textit{bem"-ser}? Como compreender que todos os
entes, tudo o que é, tendendo ao Bem, obtém seu \textit{bem"-ser}? O que
significa tender ao Bem e em que consiste?

Tudo tende para o Bem, mas cada ser segundo seu modo próprio. Tudo busca o fim,
mas segundo o modo de sua estrutura desiderativa. A conversão ou retorno ao
Princípio se opera de diversos modos: já essencialmente, já vitalmente, já
cognoscivelmente (cf.~\textit{Sobre os nomes divinos} 700b). Esta conversão 
é o movimento constitutivo de todo ser, nela começa a revelar a verdadeira
perfeição ou o \textit{bem"-ser} de tudo o que é. Deus é o \textit{bem"-ser} de
tudo o que é porque tudo o que é o busca e o deseja, tudo lhe tem saudade,
encontrando Nele sua medida, seu limite, sua perfeição. 

O ser é movimento para o \textit{bem"-ser}. Isto se revela com clareza se
repararmos no texto grego, no qual o uso das preposições põe em alta relevância
o movimento \textit{epistrófico} que o ser precisa realizar a fim de alcançar
sua perfeição. Tudo começa \textit{desde} o que Pré-existe, se conserva, se
mantém \textit{Nele} e encontra sua perfeição \textit{Nele}. Esta última frase
desdobra de certo modo a ideia sobre a qual pretendo insistir, já eu no texto
grego lemos \textit{eis auton peratoumena}. No particípio verbal está a noção de
limite, de perfeição, de medida e, Deus, para Dionísio, é a medida de todos os
seres, a perfeição de todos os seres (cf. \textit{Sobre os nomes divinos} 684c,
697c, 824a. Deus é o limite de tudo enquanto tudo tende a Ele. 

Em consequência, a mencionada frase pode ser traduzida também: encontrando sua
perfeição quando se volta para Ele ou, são aperfeiçoados quando seu movimento se
dirige para Ele. 
Penso que ambas opções não desvirtuam o pensamento dionisiano, pelo contrário,
põem ênfase no que, para ser verdadeiramente, é preciso retornar para o
Princípio doador de aperfeiçoamento, porque Ele consuma definitivamente a
\textit{ephesis} de todo ser, porque produz a saudade da perfeição de todo o ser
que, de um modo ou de outro, reconhece sua própria carência ao bastar"-se a si
mesmo, uma vez que ``toda a bondade nos seres encontra sua origem no Bem
transcendente. Todo ser existente encontra em Deus seu princípio originário e
sua perfeição superabundante transcendentalmente, não só como a Fonte para
chegar a ser, mas também como a Fonte de sua existência contínua e a realização
de toda sua potencialidade'' (Kharlamov, 2009, p. 163). Por outro lado, não pode
deixar de se ressaltar o valor da preposição \textit{eis}. Ela marca
enfaticamente, ao que me parece, o ponto final do movimento que enquanto objeto
aperfeiçoa tal movimento. Ela marca, então, o sentido da obtenção de um repouso
enquanto alude a um ``entrar em''  ou, se quiser, a um ``estar em''.

Assim o expressa nosso autor ao dizer: converte como fazia o limite próprio de
cada um, e ao qual tudo tende: o intelectual e racional, cognoscivelmente\ldots{} 
\textit{kai tagathon estin\ldots{} eis ho ta panta epistrephetai kathaper eis
oikeion hekasta peras kai hou ephietai panta\ldots{}} (\textit{Sobre os nomes
divinos} 700b). A noção de limite leva dentro de si duas acepções: o fim de um
movimento e, em tal sentido, o objeto ou aperfeiçoamento do mesmo, enquanto
acepção de definição. O Bem é fim de tudo enquanto define verdadeiramente tudo o
que é, sendo ele mesmo indefinível. ``Pois a Divindade que está além do ser é o
ser de todas as coisas (\textit{Sobre a hierarquia celestial} 177d).

``Até'', \textit{eis}, esta preposição mostra significativamente, de fato, que o
movimento não é tanto um puro processo ou trânsito do potencial para o atual e,
desse modo, \textit{ateles}, como em Aristóteles. O movimento é desenvolvimento
espiritual, é tensão de plenitude, saudade de realização, desejo do Bem. O ser
enquanto movimento é desejo de \textit{bem"-ser}. 

Contudo, este retorno de volta para o Princípio é um ato livre, um ato de
resposta próprio e autodeterminado, especialmente para os seres dotados de
inteligência; embora Dionísio pareça empregar amplamente esta liberdade
estendendo"-a a todos os seres, já que tudo o que fazem é para o Bem. 

Ainda assim, Dionísio se mostra um tanto confuso em sua consideração do
movimento como nota distintiva do ser, neste caso, do ser espiritual. O
movimento pode ser já uma caída ou uma elevação, digamos, um distanciamento ou
um acréscimo cada vez maior no caminho do Bem.\footnote{
  Cf. \textsc{Kharlamov}, 2009, p. 185; \textsc{Golitzin}, 1994, p. 90.}  
Que o movimento seja um
distanciamento, uma queda, aparece com claridade nas reflexões que dedica ao
mal, no capítulo quatro de \textit{Sobre os nomes divinos}, onde ressalta que é
possível o mal porque um ser espiritual cai de sua tensão em direção ao Bem,
isto é, há um afrouxamento na busca do Bem. O Bem atrai tudo para si, mas no
caso dos seres livres, esta atração é uma chamada à qual se pode responder ou
não. 

Como um eco do que temos afirmado nas páginas anteriores a respeito de Plotino,
isto é, acerca da presença espiritual do Uno em tudo, mas de maneira mais aguda
e penetrante naquele que está voltado para ele, Dionísio nos diz que ``a Tríade
princípio de bem e mais que bondosa\ldots{} está junto de tudo, não tudo está
junto dela (\textit{Sobre os nomes divinos} 680b).

Estar junto a ela é voltar"-se, estar junto dela é se converter, estar junto 
dela é ordenar o próprio movimento interior, estar junto dela é despertar a
tensão para ela.

O mal para um ser espiritual é um movimento desordenado e inarmônico,
debilidade e defeito, afrouxamento na tensão para o Bem. 

Contudo, como é possível corrigir esta desordem?

O universo dionisiano é hierárquico não somente na ordem do ser, mas também no
de atuar. A hierarquia ontológica dos seres recebe um valor superior quando é
pensada para aqueles que são livres.\footnote{
  Cf. \textsc{Roques}, 1983}

De fato, em \textit{Sobre a hierarquia celestial} 373d, Dionísio diz que ``o
princípio desta hierarquia é a Fonte da Vida, a Essência da bondade, a única
Causa dos entes, a Tríade, a partir da qual têm os entes o ser e o
\textit{bem"-ser} por bondade. Precisamente a esta diviníssima Beatitude que tudo
transcende, à Monada que é verdadeiramente Tríade segundo uma vontade que é para
nós incompreensível e conhecida somente para Ela, se deve a salvação definitiva
de nossa essência e das essências superiores; e esta salvação não pode ter lugar
sem a divinização daqueles que são salvos. 

Cobra aqui todo seu significado espiritual a noção de \textit{bem"-ser}.  Ela é
divinização e a salvação dos seres que são capazes de alcançá-la. 

Contudo, é necessário afirmar que quando dizemos  ``capazes'', não estamos
dizendo que seja uma realização puramente humana, uma pura aquisição por direito
próprio ou por mérito próprio. Pelo contrário. O esforço, melhor, a tensão
sustentada na consecução do Bem é o signo mais claro de um bem"-atuar, de um
acréscimo no \textit{bem"-ser}. O homem se mostra dessa maneira como um
co"-operador com o dom que recebeu e que, graças ao esforço de conversão, de
dirigir e manter sua tensão para o Bem, vai transformando sua vida e seu modo de
ser para aproximar"-se cada vez mais do firme cimento que obtém as inteligências
angélicas do Bem.

Enquanto homens nossa conversão, nossa \textit{epistrophe}, não está totalmente
realizada. Enquanto analogia, ou seja, capacidade receptiva significa nossa
capacidade ou potencial de cada um para Deus, aquele que em nós é capaz e que
está aberto a responder à \textit{palavra} de Deus dirigida e não a que,
enquanto resposta, se manifesta em nosso próprio ser que é movimento em direção
a Ele. Se realiza assim o dinamismo do ser criado que enquanto \textit{ousia}
expressa o \textit{logos} divino, enquanto \textit{dynamis} declara sua
capacidade para a realização de sua participação em Deus e enquanto
\textit{energeia}, como o processo ou ação pela qual se realiza esse projeto
plasmado nos \textit{logoi} divinos.

Poucas linhas depois do texto que acabamos de citar, Dionísio escreve que ``o
fim comum de toda hierarquia consiste no afeto sustentado por Deus e pelas
coisas divinas'' (\textit{Sobre a hierarquia eclesiástica} 376a), para expressar
logo de uma maneira nítida de que modo se traduz este afeto para Deus:
``Como disse nosso mestre, enquanto o primeiro movimento da inteligência para as
coisas divinas está representado pelo afeto para com Deus, a fase mais
originária da manifestação deste afeto, que visa à prática sagrada dos
mandamentos, se identifica com a inefável produção de nossa existência divina'' 
(\textit{Sobre a hierarquia eclesiástica} 393b). 

O contexto de ambas passagens é claramente cristão, eclesiástico e litúrgico. Do
que se trata é do ingresso em um novo modo de ser, em um novo modo de existir,
em um novo modo de orientar o movimento que anima ao ser dotado de liberdade.
Isto se torna claro ao notarmos que Dionísio fala de afeto; o texto grego diz
\textit{agapesis}. Contudo, segundo o capítulo quatro de \textit{Sobre os nomes
divinos} \textit{eros} e \textit{agape} não se diferenciam quase em nada, ou
melhor, em nada, tanto que nosso autor mencionará o divino nome de \textit{eros}
aplicado a Deus, e selará essa reflexão aludindo aos \textit{Hinos eróticos} de
seu ilustre mestre. Assim se vê bem que, seguindo o velho adágio da tradição
platônico"-aristotélico: \textit{Panta gar ta onta tou agathou ephietai}, o
\textit{eros}
está aplicado somente às essências angélicas e às almas racionais.\footnote{
  Para a concepção de \textit{eros} em Dionísio, cf.  Perl, 1997; Osborne, 1994.}
Desse modo se
entende bem que o primeiro movimento da inteligência na direção de Deus é o
afeto para Ele, mas, como acreditamos havê-lo mostrado em páginas anteriores,
este afeto há de ser sustentado, intenso, firme, carente de debilidade se quiser
chegar verdadeiramente ao fim, ao termo, à plenitude de seu desenvolvimento
espiritual. 

Desenvolvimento espiritual que é o nascimento para a vida cristã. Embora
Dionísio faça uso de um vocabulário de rica tradição filosófico"-teológica, o que
verte com eles é uma nova significação que transcende o plano meramente humano. 

A tensão do afeto para com Deus não é outra coisa que a aceitação livre da
chamada de Deus, que convida a voltar"-se para Ele, abandonando o contrário à
verdadeira vida e reduzindo a multiplicidade dos próprios movimentos débeis para
a vida uniforme e divina, dispondo"-se desse modo para a capacidade e atividade
adequadas a Deus. 

A verdadeira identidade de todo ser se alcança na conversão para o Princípio Bom
de tudo. 

Isto não é outra coisa que a plena consciência e a clara resposta que põe em
direção vetorial para o Infinito, ``nosso eros pela Beleza que eleva até Ele e
nos faz elevar para Ele'' (\textit{Sobre a hierarquia eclesiástica} 372b).

Repercute nesta frase toda a tensão platônica do \textit{eros} na busca da
Beleza infinita, do oceano imenso da Beleza, mas, em Dionísio esta zona pelágica
imensa que procura obter o \textit{bem"-ser} não é outro que o Deus da Escritura
que é também \textit{eros} por suas criaturas.\footnote{
  Dionísio utiliza a imagem do oceano no \textit{Sobre a hierarquia celestial}
  261a. Para a noção de infinito em Dionísio cf. \textsc{Lilla}, 2005,
p. 187--197 e \textsc{Hager}, 1993, p. 45--69}

Assim podemos compreender que a divinização, pela qual se ingressa na salvação,
e que Deus oferece a seus seres espirituais começa na atividade destes seres
que retornam para Ele por seu verdadeiro movimento erótico, para que ao fim
deste desenvolvimento espiritual a criatura torne"-se tão unida a Deus que sua
atividade não é mais do que a atividade divina, o \textit{Eros} divino fluindo
nela.\footnote{
  Cf. \textsc{Louth}, 2001, p. 39 r 107} 

Deus se outorga como \textit{bem"-ser} porque o ser que se move para Ele
aperfeiçoou a aptidão dada por Ele para recebê-lo, aptidão que o faz cada vez
mais partícipe do dom divino por sua resposta livre, generosa e supremamente
erótica.\\ 
\bigskip
\hfill Trad.~Jorge Sallum

\section{Bibliografia}

\begin{description}\labelsep0ex\parsep0ex
\newcommand{\tit}[1]{\item[\textnormal{\textsc{\MakeTextLowercase{#1}}}]}
\newcommand{\titidem}{\item[\line(1,0){25}]}
\tit{Beierwaltes}, W. \textit{Proclo. \textsc{i} fondamenti della sua
metafisica}, Milano: Vita e Pensiero, 1990

\tit{{Corpus Dionysiacum} \textsc{i"-ii}}. Ed. B. Schula, G. Heil, A.
Ritter, Berlin: W. de Gruyter, 1990--1991

\tit{Di Pasquale Barbanti}, M. 1994. Piacere, bene, felicitá in
Plotino. In L. Montoneri (a cura), \textit{\textsc{i} Filosofi Greci e il
Piacere.} Bari: Laterza

\tit{Di Pasquale Barbanti}, M. \textit{Proclo. Tra filosofia e
teurgia}. Catania: Bonanno, 1993

\tit{Gersh}, S. \textit{Kinêsis Akinêtos. A Study of Spiritual Motion
in the Philosophy of Proclus}. Leiden: Brill, 1973

\tit{Golitzin}, A. \textit{Et Introibo ad Altare Dei. The Mystagogy of
Dionysius Areopagita}. Thesalonik: Analecta Vlatadon, 1994

\tit{Hadot}, P. \textit{Plotin, Traité 9. \textsc{vi} 9.} Paris: Cerf, 1994 

\tit{Hager}, F. 1993. \textit{Infinity and Simplicity of God in
Plotinus, Proclus and Pseudo"-Dionysius}, \emph{The Journal of
Neoplatonic Studies} \textsc{ii} (1)

\tit{Kharlamov}, V. \textit{The Beauty of the Unity and the Harmony of
the Whole. The Concept of Theosis in the Theology of
Pseudo"-Dionysius the Areopagite}. Oregon: Wipf\&Stock, 2009

\tit{Lilla}, S. \textit{Dionigi l´Areopagita e il platonismo
cristiano}. Brescia: Morcelliana, 2005

\tit{Louth}, A. \textit{Denys the Areopagite}. London: Continuum, 2001

\tit{Llyod}, A. \textit{The Anatomy of Neoplatonism}. Oxford: Clarendon
Press, 1998

\tit{Mc Groarty}, K.  \textit{Plotinus on Eudaimonia}. Oxford:
\textsc{oup},
2006

\tit{Moutsopoulos}, E. De Aristóteles a Proclo: movimiento y deseo del
Uno en la Teología Platónica, \textit{Filosofía de la Cultura
Griega}, Zaragoza: \textsc{puz}, 2007

\tit{Osborne}, C. \textit{Eros Unveiled. Plato and the God of Love}.
Oxford: Clarendon Press, 1994

\tit{Perl}, E. 1997. \textit{The Metaphysics of Love in Dionysius the
Areopagite}. \emph{The Journal of Neoplatonic Studies} \textsc{vi} (1)

\tit{Roques}, R.  \textit{L´univers dionysien. Structure hiérarchique
du monde selon le Pseudo"-Denys}. Paris: Cerf, 1983

\tit{Schiewind}, A.  \textit{L´Éthique du sage chez Plotin}. Paris:
Vrin, 2003

\tit{Wallis}, R. \textit{Neoplatonism}. London: Duckworth, 2002
\end{description}



\capitulo[Tradição e contemporaneidade gnósticas no neoplatonismo: Proclo,
   Schelling e Sampaio Bruno]{Tradição e contemporaneidade gnósticas\\ no
   neoplatonismo: Proclo, Schelling e\\ Sampaio Bruno}%
	 {Edrisi Fernandes}%
	 {ufrn}

\markboth{Tradição e contemporaneidade gnósticas\ldots{}}{Edrisi Fernandes}

\section{Introdução}

Partindo do Medioplatonismo, o Gnosticismo e o Neoplatonismo
tiveram por algum tempo um crescimento paralelo entre os séculos
\textsc{ii} e \textsc{iv}, entrando em contato em situações e proporções que mal
se começaram a mapear e dimensionar e sofrendo transformações
que refletem atitudes hostis ou amistosas, mas nunca
indiferentes, revelando interdependências que permitem apontar
que essas duas grandes correntes de pensamento --- complexas e
certamente separáveis em uma grande diversidade de agrupamentos
secundários --- se construíram a partir de uma crescente percepção
de alteridade e diferença que, no entanto --- e apesar da oposição
“fundadora” do pensamento plotiniano em relação aos Gnósticos
(\emph{Enéades}, \textsc{ii}.8, \textsc{ii}.9, \textsc{v}.5 e
\textsc{v}.8) --- preservou momentos
de fertilização mútua até o final da era pagã. Na opinião de
Benjamin Walker,

\begin{quote}
Embora alguns Neoplatonistas tenham sido hostis aos Gnósticos,
todos eles tiveram pupilos Gnósticos, e na verdade o movimento
inteiro estava saturado pelo Gnosticismo. Plotino (m. em 268)
opôs"-se fortemente a certas crenças Gnósticas e escreveu um
tratado contra elas [\emph{Enéades},
\textsc{ii}.9],\footnote{\textsc{notas} 
Foi Porfírio (\emph{Vita Plot.}, 16.11) quem identificou como
Gnósticos aqueles que Plotino ataca na \emph{Enéade} \textsc{ii}. 9.}
mas seu trabalho reflete certas ideias Gnósticas,\footnote{ John
D. Turner [1992. Gnosticism and Platonism: The Platonizing
Sethian texts from Nag Hammadi in their relation to later
Platonic literature. \emph{In}: R. T. \textsc{wallis} e J.
\textsc{bregman}
(Eds.), \textit{Gnosticism and Neoplatonism}. Albany: State
University of New York Press, p. 425--459; p. 455--6] sumarizou um
artigo de R. T. Wallis, “Plotinus and the Gnostics: The Nag
Hammadi Texts”, apontando quais foram as ideias dos tratados dos
Gnósticos “sethianos” que Plotino aceitou ou rejeitou. Turner
menciona em sua conclusão que “o débito de Plotino para com os
metafísicos Gnósticos deve ser reconhecido, como H. J. Kraemer
(\textit{Der Ursprung der Geistmetaphysik: Untersuchungen zur
Geschichte des Platonismus zwischen Platon und Plotin}, 2 ed.
Amsterdã: B. R. Greener, 1967, p. 223--264) apontou bastante
detalhadamente há muito tempo”.} e através dele essas ideias
entraram na corrente principal da filosofia religiosa ocidental.

A filosofia do Neoplatonismo foi uma filosofia de revelação e
desdobramento divinos. A criação do mundo seguiu de perto a
teoria Gnóstica das emanações.\footnote{ Os vocábulos empregados
em grego para “emanação” são, para os Gnósticos,
\emph{probolê} e \emph{proerchestai}.} (\ldots{})

O trácio Proclo (m. em 485), além de uma preocupação com o
oculto, também reafirmou a filosofia privativa dos Gnósticos,
afirmando que Deus é não ser, e apenas pode ser entendido, se é
que pode sê-lo, por negativas. Diz"-se que o escritor
não identificado que tomou o nome do converso ateniense de Paulo
(\emph{Atos} 17: 34), Dionísio o Areopagita --- melhor conhecido
como o Pseudo"-Dionísio (c. 500) ---, foi seu discípulo. Suas obras
introduziram na Igreja muito do Neoplatonismo e do Gnosticismo,
incluindo ideias como a do Deus incognoscível, a \emph{via
negativa} e a hierarquia celeste (Walker, 1983, p. 167--8). 
\end{quote}

Nesta investigação, algumas instâncias do pensamento de Proclo,
de Schelling e de Sampaio Bruno, notadamente no que respeita aos
problemas da teodiceia e da relação entre a criação e seu
fundamento, são tomadas como exemplos de recepção,
transvaloração e assimilação de doutrinas achegadas àquelas do
Gnosticismo em sistemas que, no entanto, preservaram uma
afinidade bem maior com o Neoplatonismo. Apesar de não ser da
alçada desta investigação discutir sobre se houve realmente uma
“filosofia privativa dos Gnósticos”, pensamos que os exemplos
que serão analisados a seguir sugerem fortemente que a palavra
“privativa” não é muito apropriada para retratar a situação da
ideologia Gnóstica em relação às aproximações e apropriações que
lhe foram feitas pelo Neoplatonismo, por darem a impressão de
uma restritividade ou exclusividade de ideias que está muito
longe de caracterizar a multifária realidade das interações
culturais ao longo dos séculos.

\section{\textsc{i}. PROCLO}

Um hino intitulado pela posteridade “Hino a todos os deuses” ou
“Hino ao deus dos Oráculos Caldeus”\footnote{ \emph{Hymnos
koinos eis theoús} para Abel, Ludwich, Vogt e R. M. Berg;
\emph{Hymne aux dieux des ``Oracles Chaldaïques''} em Proclus,
1994.} diz o seguinte

\begin{quote}\begin{verse}
Escutai, deuses que comandais o timão\footnote{ Platão,
\emph{Político}, 272e.} da sagrada sabedoria e que,\\ 
acendendo nas almas a chama do desejo do retorno, as atraís até
os imortais, \\
permitindo àquelas [almas], purificadas pelas inefáveis
iniciações dos hinos, \\
evadir"-se da caverna tenebrosa (\emph{skótion
keuthm[1ED7?]na})!\\
Escutai, grandes salvadores, e concedei"-me pela compreensão dos
livros divinos,\\
uma luz santa, dissipando a treva para que eu possa distinguir
entre o deus imortal (\emph{ámbroton}) e o homem!\\
Que um \emph{daimon} malfazejo (\emph{ouloà} \emph{rhézôn
daímôn}) jamais me retenha submerso nas marés do
esquecimento,\footnote{ No “Hino às Musas” de Proclo, as almas
que estão no reino da matéria, de onde esperam escapar com a
ajuda das Musas, são representadas como estando aprisionadas num
mar de esquecimento.}\\
longe dos [deuses] abençoados (\emph{makárôn}), e que uma
Punição (\emph{Poin[1EBF?]}) enregeladora não aprisione, nos
grilhões da vida (\emph{bíou desmoí}),\footnote{ Aqueles que
prendem a alma ao corpo (\emph{Timeu}, 73b3).} \\
minh'alma precipitada nas geladas ondas da geração,\footnote{ “A
punição da alma que não viveu de acordo com o \emph{Noūs}, mas
sim, ao invés disso, com o corpo, toma a forma da reencarnação
compulsória” (Berg, 2001, p. 235).} \\
nas quais não quisera ondear demasiado tempo!

Ó deuses, [vós que sois os] guias na direção da fulgurante
sabedoria,\\
escutai"-me, e revelai àquele que se apressa no caminho ascendente\\
os ritos e iniciações transmitidos pelas palavras sagradas! \\
(Proclus, 1994, p. 36 [grego, ed. E. Vogt] e 37 [trad.]).
\end{verse}\end{quote}

Nesse hino, a menção indireta a dois mundos (dos deuses/\,da
imortalidade/\,da luz/\,da compreensão e da lembrança
\emph{versus} do homem/\,das ondas da geração/\,da treva/\,da
incapacidade de distinguir e do esquecimento), a imagem do mundo
do homem como lugar de desterro (a caverna tenebrosa) associado
a uma entidade malévola e a ideia da salvação pelo conhecimento
poderiam levar um leitor ignorante de sua autoria pelo
neoplatonista Proclo a cogitar a possibilidade de tratar"-se de
um hino gnóstico, uma vez que os elementos citados --- o dualismo
de mundos, a origem da alma do homem em um lugar luminoso (para
onde aspira retornar) e seu banimento para um lugar trevoso onde
atua uma potestade má, a existência de um conhecimento
experiencial\footnote{ Desse conhecimento faz parte não apenas a
compreensão dos livros divinos, mas também e principalmente a
iniciação nos mistérios.}  salvífico --- são mais amiúde
identificados com o Gnosticismo do que com o Neoplatonismo,
apesar de ambos partilharem de diversos elementos que não são
incomuns no Platonismo anterior (antigo e médio) e em algumas
religiões da civilização helenista, e mesmo de épocas e locais
mais remotos.

Apesar de Rudolphus Berg enxergar nos “livros divinos” a que
Proclo faz menção nesse hino os “livros sagrados em geral”
(Berg, 2001, p. 221), alguns comentadores\footnote{ Como é o
caso de Westerink e Saffrey.} querem ver nesses livros os
\emph{Oráculos Caldeus} --- uma coleção de crípticos versos
místico"-religiosos pesadamente influenciados pelo
Medioplatonismo --- notadamente aquele de Numênio de Apaméa, que
revela importantes similaridades de crenças e atitudes com o
Gnosticismo (Elsas, 1975, p. 238 e ss.). O primeiro Platonista a
se interessar seriamente pelos \emph{Oráculos Caldeus} parece
ter sido um certo Antonino, contemporâneo mais jovem de Plotino
e do qual Proclo faz menção no seu \emph{Comentário ao Timeu}.
Entre os Neoplatonistas, Porfírio foi o pioneiro do uso
filosófico dos \emph{Oráculos}, e ele ou provavelmente algum
membro de sua escola, através do anônimo \emph{Comentário ao
Parmênides} estudado por Hadot, parece ter levado aos Gnósticos
o interesse pelos \emph{Oráculos} --- os textos “Sethianos
platonizantes” de Nag Hammadî\footnote{ \emph{Alógenes},
\emph{Zôstrianos}, \emph{As Três Estelas de Seth},
\emph{Marsanes}, conforme John Turner, autor de diversas
publicações sobre esses tratados.} parafraseiam em forma de
revelação ou como alegoria diversas passagens que aparecem no
anônimo \emph{Comentário ao Parmênides} --- por exemplo, a
doutrina da tríade noética Existência"-Vida"-Intelecto, implícita
em Plotino (\emph{Enéades}, \textsc{vi}.7.17, 13--26) e concebida ---
parece que com mediação da exegese dos \emph{Oráculos Caldeus}
(Majercick, 1992) --- para explicar a derivação do Segundo
Princípio do \emph{Parmênides} em relação ao Primeiro, aparece
no \emph{Comentário ao Parmênides} (Fragmento \textsc{xiv}) e, além
disso, sob a forma do “triplamente potente\footnote{ No
\emph{Alógenes} e no \emph{Marsanes} as três potências da
Existência, Vida (ou Vitalidade) e Intelecto (ou Mente)
subsistem independentemente como um tipo de hipóstase chamada de
“o triplamente potente”, mediador entre o Espírito Invisível (a
Deidade Transcendente) e o eão Barbelô. No \emph{Zôstrianos},
o Espírito Invisível desdobra"-se como três poderes, Existência,
Vida e Intelecto; nesse tratado, o Intelecto é consubstantial
com Barbelô. Em \emph{As Três Estelas de Seth}, uma tríade
similar encontra"-se presente no próprio Primeiro Princípio
inefável e hiperôntico como uma espécie de “prefiguração” da
Existência, Vida e Benção (aparecendo esta no lugar do
Intelecto), aparecendo essa tríade novamente, mas já de modo
substancial, em Barbelô.}”, no \emph{Alógenes} e no
\emph{Marsanes}. Fechando o círculo, Proclo se inteirou do
conteúdo dos tratados Gnósticos, mas propôs uma leitura dos
\emph{Oráculos} que os (re)aproxima da tradição do
Neoplatonismo. 

Uma trama de influências como a esboçada no parágrafo anterior
deve servir para exemplificar o quanto é difícil “separar o joio
do trigo” quando se trata de mostrar em que medida
Neoplatonistas e o Gnósticos, ao mesmo tempo em que consolidavam
experiências culturais distintas, emprestaram uns dos outros ao
mesmo tempo em que se inspiravam em temas e fontes anteriores
comuns. Parece"-me oportuno recordar aqui uma observação de
Francis MacDonald Cornford: “Nenhuma disputa pode ocorrer exceto
se ambas as partes tem alguma posição fundamental sobre a qual
concordam. Essa base comum é a última coisa de que provavelmente
estão conscientes. Logo, no debate filosófico ela pode passar
quase completamente sem menção” (Cornford, 1950, p.
29).\footnote{ “\emph{No dispute can be carried on unless both
parties have some fundamental standpoint on which they agree.
This common basis is the last thing of which they are likely to
be aware. Hence in the philosophic debate it is apt to pass
almost wholly unmentioned}”.} Na polêmica entre Neoplatonistas
e Gnósticos, contudo, não faltou menção à base comum Platônica
por Plotino, que acusou os Gnósticos\footnote{ A “polêmica
anti"-gnóstica” ocupa as \emph{Enéades} \textsc{iii}.8, \textsc{v}.8,
\textsc{v}.5 e
\textsc{ii}.9
[tratados 30, 31, 32 e 33 na ordem cronológica de Porfírio, que
nomeou (\emph{Vita Plot}., 5.34 e 16.11) o último como “Contra
os gnósticos”].} por sua falta de rigor filosófico e sua
irresponsabilidade hermenêutica diante dos ensinamentos de
Platão --- traduzida num desprezo odioso ao mundo sensível, numa
atribuição da causa do Mal ao mundo inteligível e numa antipatia
em relação ao Demiurgo. Simplificadamente, podemos dizer que, se
o Platonismo desenvolve uma apropriação filosófica dos mitos,
encontramos no Gnosticismo incontáveis e complexas instâncias de
mitologização de temas e ensinamentos filosóficos. 

Resenhando o livro \emph{Order from Disorder: Proclus' Doctrine
of Evil and its Roots in Ancient Platonism}, de John Frederick
Phillips, Todd Krulak propôs (Krulak, 2009) alguns pontos que
Proclo parece ter valorizado ao redigir o tratado \emph{Perì
tês tôn Kakôn Hypostaseôs}/\emph{De malorum subsistentia}, uma
sinopse de suas opiniões sobre o problema do Mal em contraste
com a posição de Platonistas como Plutarco, Numênio e Ático, bem
como de vários Gnósticos, a favor de uma visão de mundo
dualista, propositora da existência de uma alma do mundo maligna
que seria coeterna e independente da deidade governante deste
mundo. Entre as opiniões que Krulak destaca como características
da posição de Proclo, podemos apontar três que, em nosso ver,
passam com certa dificuldade no teste de uma caracterização como
estritamente antignósticas: 1) “a Causa Primária é inteiramente
boa”; 2) “o Mal não é uma entidade desejada pelo Divino, mas vem
a existir como resultado de uma desarmonia em níveis ontológicos
inferiores”; 3) o substrato do Mal que pode ser encontrado nos
corpos naturais é “um proto"-corpo introduzido [induzido] na
matéria por traços vestigiais das Formas [platônicas]”. Vejamos
separadamente cada uma dessas opiniões, a começar pela última,
avaliando suas diferenças em relação a posições distintas de
outros Platonismos, com particular atenção para o Gnosticismo. 

Plotino entendeu a matéria como o princípio do Mal que se
encontra nos corpos naturais, enquanto Porfírio, não se
afastando muito de seu mestre, atribuiu esse Mal a corpos
irracionais precósmicos. Para o Gnosticismo, as almas são
fragmentos da substância divina, caídas neste mundo, unidas à
matéria corporal e mescladas ao Mal. Na descrição que Irineu de
Lyon fez do Gnosticismo de Ptolomeu, da “escola” valentiniana,
lemos (Layton, 2002, p. 335--6): “Ao [indagar sobre a raiz sem
começo\footnote{ Irineu de Lyon (Santo), \emph{Contra as
Heresias}, “O Mito Gnóstico na Versão de Ptolomeu” (Layton,
2002, p. 332--358), 1.2.1.} e] tentar o impossível e o
incompreensível, ela [Sophia] produziu essência sem forma (\ldots{}).
A essência da matéria --- dizem eles --- teve sua primeira fonte na
falta de conhecimento, desgosto, medo e terror\footnote{ Isto é,
insatisfação, temor, e profunda perturbação e insegurança ---
“desgosto porque não entendera [a raiz sem começo], medo de que
a vida a deixasse [i.e., Sophia] como o fez a luz, insegurança
diante de tudo isso” em “O Mito Gnóstico na Versão de Ptolomeu”,
1.4.1; “medo, desgosto, incerteza” em 1.5.4.}”.\footnote{ “O Mito
Gnóstico na Versão de Ptolomeu”, 1.2.3.} Os quatro elementos
constituintes do mundo são associados nesse relato\footnote{ “O
Mito Gnóstico na Versão de Ptolomeu”, 1.5.4.} a modalidades ou
atributos negativos:

\begin{quote}
Do terror e desespero foram gerados os elementos que
compreenderam o mundo, da mesma forma que as coisas corpóreas
foram geradas do que é mais estacionário, como dissemos acima; a
terra (foi gerada) pela fixidez do terror; a água, pela
atividade do medo; o ar, pela fixação do desgosto. Mas o fogo
está naturalmente presente em todos esses, como (um princípio
de) corrupção e morte, do mesmo modo como falta de conhecimento
--- assim ensinam eles --- está escondida nas três paixões acima
mencionadas (Layton, 2002, p. 346).
\end{quote}

Nesse mito gnóstico, os males do terror, medo, desespero,
desgosto e ignorância aparecem como princípios originadores dos
elementos da matéria corruptível e perecível. Tais males
costumeiramente mostram"-se associados no Gnosticismo a um ato de
imprudência, soberba ou arrogância rebelde de uma deidade
secundária,\footnote{ No mito gnóstico valentiniano na versão de
Ptolomeu, Sophia e seu pensamento (aspecto) demiúrgico, Achamôth
(do hebraico \emph{hokhmah}, \emph{hokhmôth}, “sabedoria”),
espécie de “sabedoria inferior”. No \emph{Evangelho de Filipe}
(34), a distinção se dá entre Ekhamôth (= Sophia), a Sabedoria,
e Ekh"-Môth, a “Sabedoria da Morte” (do aramaico e hebraico
mishnaico \emph{‘ek môth}, “como [a] morte”).}
subentendendo"-se que a matéria espiritual de seus pensamentos
transtornados e revoltosos em relação à ruptura com o
\emph{Plerôma}\footnote{ Grego
\emph{pl}\emph{é}\emph{r}\emph{o}\emph{ma}
(“plenitude; totalidade; inteireza”), o indiviso nível
ontológico mais superior e originário.} diversifica"-se em
sentimentos desagradáveis que se materializam na forma do mundo
que habitamos. Misturada à matéria do nosso mundo, no entanto,
sobrevivem resíduos de uma matéria não corrompida, divina e
passível de reintegrar"-se à sua fonte supramundana. Podemos
dizer, portanto, que no Gnosticismo como em Proclo os corpos
naturais remontam a traços vestigiais das Formas, sendo que o
monismo de Proclo preserva a bondade da criação, embora nela
reconheça distintos níveis de ordem, enquanto no Gnosticismo as
Formas que deixam seus traços no mundo são de natureza
predominantemente corrompida, resultando numa criação má --- não
obstante sobreviver no mito gnóstico clássico a ideia ---
remontando em última análise ao \emph{Fedro}, ao
\emph{Timeu} e ao \emph{Fédon} --- de “uma centelha divina no
homem, derivada do reino divino, caída nesse mundo de provação,
nascimento e morte, e necessitando ser despertada pela
contraparte divina do eu de modo a ser finalmente reintegrada”
(Bianchi, 1967, p. \textsc{xxvi}).\footnote{ No que se refere a Plotino, a
encarnação da alma pode em certa medida ser vista como uma
“queda”, tendo em vista que a alma perde a sua plenitude
espiritual e a sua autonomia (\emph{Enéades}, \textsc{iv}.8.5.16), mas
o descenso da alma, hipóstase do divino, ocorre  por deliberação
livre, com o propósito de auxiliar os seres situados no mundo
inferior (\textsc{iv}.8.7.1).} 

 De modo semelhante ao que se vê em Proclo, também no
Gnosticismo o Mal não é uma entidade desejada pelo Divino, vindo
a existir como resultado de uma desarmonia em níveis ontológicos
inferiores. No Gnosticismo, contudo, o retrato dessa desarmonia
é muito mais trágico. Acompanhado Henri"-Charles Puech, podemos
dizer que abaixo do Deus bom absolutamente transcendente, ou em
oposição a Ele --- dependendo de tratar"-se de um dualismo mitigado
ou absoluto ---, e distinto desse Deus alheio ao mundo,
desconhecido, oculto, inefável, 

\begin{quote}
existe [no Gnosticismo] um outro Deus, inferior ou essencialmente
mau, que criou e que domina o mundo. Esse é um Demiurgo, fraco,
de mente estreita se não ignorante, ou ele pode ser o próprio
Diabo, princípio não engendrado ou Príncipe das Trevas e
encarnação do Mal como tal, culpado por ter produzido o universo
e o homem carnal (\ldots{}). A imperfeição e iniquidade desse Deus
são conhecidas --- e bastante bem --- por esses produtos [as Trevas
e o Mal] e pelas leis tirânicas que ele impõe sobre o curso dos
eventos e sobre suas desgraçadas criaturas (Puech, 1957, p. 59).
\end{quote}

Segundo Hugo Bianchi, a cosmologia Gnóstica, indissociável de sua
teodiceia, “está ontologicamente baseada na concepção de um
movimento descendente do divino cuja periferia (frequentemente
chamada \emph{Sophia} ou \emph{Ennoia}) teve de submeter"-se
à provação de enfrentar uma crise e produzir, ainda que apenas
indiretamente, este mundo” (Bianchi, 1967, p. \textsc{xxvi}). Para
Proclo, um movimento descendente do divino produz o mundo sem
que haja uma crise, e o mundo e o Mal não são produtos do
divino, mas hipóstases deste --- Proclo entende o Mal como uma
espécie de oposição incompleta ao Uno/\,Bem, valendo"-se de um
vocábulo platônico,
\emph{hyp}\emph{e}\emph{nantion}/\,“subcontrário”,\footnote{
\emph{Teeteto}, 176a.} e de um neologismo helenista,
\emph{parypostasis}/\,“existência parasítica”,\footnote{ “Um tipo
de defeito, uma indeterminação e uma privação” (\emph{DMS},
49, 10--11). Conforme Jan Opsomer e Carlos Steel (tradutores de
Proclus. \textit{On the Existence of Evils}. Londres: Duckworth,
203, p. 51, n. 81), antes de Proclo esse vocábulo foi usado por
Porfírio, Siriano, Juliano, Gregório de Nissa e Jâmblico; este
último parece ter sido o primeiro a aplicá-lo à existência do
mal.} para conceber a natureza do Mal: conforme John Frederick
Phillips, para Proclo o Mal “existe como um tipo especial de
privação que excede aquilo que é mera ausência de ser, tal que
ele é uma oposição ao bem; não obstante, sua oposição ao Bem não
é tão completa a ponto de resultar numa doutrina dualista”
(Phillips, 2007, p. 89--90). 

Para concluir nosso apanhado das opiniões sobre o problema do Mal
que seriam características da posição de Proclo vejamos agora
aquela que diz ser inteiramente boa a Causa Primária. Ora, já
vimos que no mito Gnóstico a origem do mundo e do Mal radica"-se
em um ato de soberba ou rebeldia de uma deidade secundária, não
se podendo, numa análise aprofundada, atribuir o Mal à Causa
Primária da existência --- mesmo porque, como apontou Bianchi, o
Divino não pode dar as costas a este mundo, pois é-Lhe
necessário recuperar o \emph{pneuma} repartido com a cisão
primordial (Bianchi, 1967, p. \textsc{xxvii}). Vimos também que a “causa
material” do Mal e do mundo no Gnosticismo é uma materialidade
nefasta derivada da corporificação de pensamentos transtornados
e revoltosos da deidade secundária, enquanto em Proclo podemos
dizer que o Mal tem por essência uma degradação do Bem, uma
diferenciação melhor vista como quantitativa do que como
qualitativa, mais subtrativa que ontologicamente transmutativa.
Para Proclo, Deus (noutra chave, o Uno/\,Bem), ademais de ser
Causa Final, como para Aristóteles, é também Causa
Eficiente\footnote{ \emph{In Parm}., 922 (Proclus, 1992, p.
278): “Como poderia o Divino permanecer inativo e sem efeitos
enquanto o céu, imitando"-O como ele o faz, exibe um tal poder
criativo em relação àquilo que está abaixo dele, de modo que
todo seu movimento e arranjo age sobre o inteiro reino da
geração, e o faz girar e ordena por meio de uma legião de
princípios racionais imanentes, até chegar às plantas e mesmo
aos objetos inanimados?”. Conforme Morrow e Dillon (Proclus,
1992, p. 205), Proclo contesta em 921.10 e seguintes “concepções
inadequadas da relação das Formas [Platômicas] com os
particulares, tanto aquela depois apropriada pelos Estoicos, que
[entende que] a Forma é imanente na matéria, quanto aquela dos
Peripatéticos, que [entende que] o Divino não está
providencialmente preocupado com o mundo sensível”.} (Proclus,
1992, p. 278) --- “ao desencadear o efeito, a causa permanece sem
diminuição e inalterada”\footnote{ Frase de Dodds, interpretando
a Proposição 26 dos \emph{El. Theol}.: “Toda causa produtiva
produz a seguinte e todos os princípios subsequentes, enquanto
ela própria permanece inalterada” [Proclus, 1963, p. 30 (grego)
e 31 (inglês)]. Cf. Platão, \emph{Tim}. 42e, Plotino,
\emph{En}. 5.4.2, Proclo, \emph{Theol. Plat.} \textsc{v}.18.283, e
uma vasta coleção de referências apontadas por Dodds.} (Proclus,
1963, 214). Para conciliar esse ponto de vista com a realidade
da existência (parasítica, que seja) do Mal, se chegou a falar
de uma “Causa Deficiente” (\emph{causa deficiens}) mais que
uma causa eficiente (\emph{causa efficiens}) daquele em
Proclo. Na verdade, como aponta Lucas Siorvanes, “para Proclo o
Criador [do mundo]\footnote{ Equivalente ao Demiurgo do
\emph{Timeu}; o terceiro componente da tríade de “Pais” que
regem a essência, o poder (ou processão e vivificação) e a
atividade criativa do Intelecto. O Criador equivale ao
“intelecto do Intelecto”.} não é apenas a causa eficiente
(\emph{poietikós}; literalmente ``fazedora''), mas também a
formal (\emph{eidetikós}) e final (\emph{telikós}) do mundo
temporal e físico.\footnote{ \emph{Theol. Plat.}, \textsc{v} cap. 20,
\textsc{v}.55, \textsc{v}.61.10--15; \emph{In Tim.}, \textsc{i}.266.25--30 (referências
apontadas por Siorvanes, 1996).} Com essa atribuição radical
Proclo rejeita os Aristotélicos, para quem o primeiro Intelecto
é a causa final, mas não uma causa eficiente, e os Estoicos,
para quem o intelecto criador é inseparável da matéria”
(Siorvanes, 1996, p. 151).

A ideia da permanência integral da causa no efeito é anterior a
Proclo, aparecendo frequentemente em textos Gnósticos, como no
\emph{Tratado Tripartido},\footnote{ \emph{Nag Hammadi Codex}
(\textsc{nhc}) \textsc{i}, 5, um tratado “Valentiniano” heterodoxo, atribuído à
primeira metade do século \textsc{iii}. Nossa tradução baseia"-se naquela
de Attridge e Mueller, 1988.} 52, 12--25, onde lemos em relação
ao “Pai, que é a raiz da totalidade” (51, 3--4): “Nem Ele irá se
retirar daquilo pelo qual Ele é, nem ninguém O forçará a
produzir um fim que Ele jamais desejou. Ele não teve quem desse
início à Sua existência; logo, Ele é em Si imutável, e ninguém
pode removê-Lo de Sua existência e de Sua identidade, daquilo em
que Ele existe” (Attridge e Mueller, 1988, p. 61). Esse
raciocínio do Gnosticismo em relacão à Causa Eficiente leva"-nos
a entender nosso mundo, deserdado do Bem no Gnosticismo
clássico,\footnote{ O \emph{Tratado Tripartido}, em sua
heterodoxia tardia, inverte essa perspectiva apresentando o
Demiurgo e o mundo como sendo bons.} como produto de uma “causa
deficiente”, ao menos no domínio do tempo e da matéria, já que,
como recorda Serge Hutin, para os Gnósticos “o tempo é mau e
constitui uma fonte de angústia; a gnose se opõe tanto à
doutrina estoica do tempo cíclico, circular, como à doutrina
cristã de um tempo linear que se estende irreversivelmente desde
a criação”. Assim, 

\begin{quote}
até os gnósticos que não admitem a reencarnação se acham
obcecados pelo \emph{tempo}. Igualmente ao mundo físico, o
tempo --- que subjaz, por outra parte, em todas as manifestações
do cosmo visível --- é “mescla” e “mancha”: o ciclo do tempo não
é outra coisa que a Fatalidade; o tempo pertence ao mundo
material, enquanto o mundo superior é atemporal (e se acha
separado do primeiro por um limite que, em princípio, é
absoluto) (Hutin, 1964, p. 14).
\end{quote}

No Gnosticismo é marcante o dualismo entre o mundo imaterial
e atemporal e o mundo material e temporal, e podemos perceber
traços de um dualismo desse tipo em Proclo, como por exemplo na
seguinte passagem (\emph{Theol. Plat.}, \textsc{v}.6.24,23--25,9):

\begin{quote}
É natural que esse universo (\emph{to pân}) tenha dois
tipos de vida, período e revolução
(\emph{z}\emph{o}\emph{às kaì periódous kaì
synkyklés}\emph{e}\emph{is}), um de Kronos e outro de
Zeus,\footnote{ Chamado por Proclo de “o Demiurgo da totalidade
(\emph{demiourgós toû pantòs/ toû pantòs demiourgós})”.} como
afirma o mito do \emph{Político} [269e1--270a8]. Em um dos
períodos [= de Kronos], ele [o universo] gera por si mesmo
(\emph{autómata}) todas as coisas boas e tem uma vida livre de
miséria e desgaste,\footnote{ \emph{Político}, 271c8--272b2.}
mas no outro [= de Zeus], ele participa da discordância material
(\emph{tes hylikes
plemm}\emph{e}\emph{l}\emph{e}\emph{ías}) e da natureza
multicambiante (\emph{tes polymetabólou
phýse}\emph{o}\emph{s}). Há duas modalidades, então, de vida
no mundo; uma inaparente/\,invisível e mais intelectiva
(\emph{aphanoūs kaì
no}\emph{e}\emph{r}\emph{o}\emph{t}\emph{é}\emph{ras}),
a outra mais física e aparente/\,visível
(\emph{physik}\emph{o}\emph{t}\emph{é}\emph{ras kaì}
\emph{e}\emph{mphanoūs}); uma é definida pelo cuidado da
Providência (\emph{prónoian aphorizom}\emph{é}\emph{nes}),
a outra procede desordenadamente conforme o Destino
(\emph{kath' h}\emph{e}\emph{imarm}\emph{é}\emph{nen
atakt}\emph{o}\emph{s proïoúses}). A que é secundária,
multiforme (\emph{poly}\emph{e}\emph{ides}) e objetivada
através da natureza (\emph{dià tes phys}\emph{eo}\emph{s}
\emph{e}\emph{pit}\emph{e}\emph{loum}\emph{é}\emph{ne})
depende do ordenamento de Zeus (\emph{diías exértetai
táx}\emph{eo}\emph{s}), aquela que é mais simples,
intelectiva e inaparente/\,invisível
(\emph{aploust}\emph{é}\emph{ra kaì
no}\emph{e}\emph{rà kaì aphanès}, [depende do ordenamento]
de Kronos (Proclo, 2005, p. 654/\,656 [grego] e 655/\,657
[italiano]; Phillips, 2007, p. 160).\footnote{ Cf. ainda a
\emph{Theol.} \emph{Plat.}, \textsc{v}.25.}
\end{quote}

Um outro ponto aproxima o Gnosticismo do tratamento procleano
da relação entre o deus soberano do reino invisível e o deus do
reino visível: a intermediação de uma figura feminina, que no
Gnosticismo é amiúde Sophia e na \emph{Teologia Platônica} de
Proclo é Rhea. Nas palavras de Siorvanes, “Proclo\footnote{
\emph{Theol. Plat}., \textsc{v}.36.8--37.21.} identifica essa poderosa
entidade causal, que é superior ao próprio Criador [do mundo],
com o princípio feminino do universo. Ela é a mãe do Criador, e
de seu útero nasceram tanto ele quanto o mundo. Ela é a deusa
Rhea como ``fluxo''\footnote{ Esclarecimento de Siorvanes (1996,
p. 151): “em grego, \emph{rhôe}; v. Platão, \emph{Crátilo},
402”. Na \emph{Teologia Platônica}, Rhea é o segundo
componente da tríade de “Pais” que regem a essência, o poder (ou
processão e vivificação) e a atividade criativa do Intelecto.
Está encarregada do poder vivificante do Intelecto. No
\emph{Aporíai kaì lýseis perì t}\emph{o}\emph{n
pr}\emph{ó}\emph{t}\emph{ô}\emph{n
arkh}\emph{ô}\emph{n}/\emph{De Principiis} de Damáscio,
284 (frags. 304--5 Kern), Rhea, “a que flui”, representa a
natureza material.}” (Siorvanes, 1996, p. 151). Mas Rhea não se
rebelou como Sophia, atuando como princípio conector, e não
separador, entre o inaparente e o aparente. Apesar de seu
dualismo, Proclo tem, ademais, o cuidado de dizer
(\emph{Theol. Plat.}, \textsc{v}.6.25,11--19) que 

\begin{quote}
Certamente também Zeus é causa da vida invisível do universo,
dispensador do intelecto e líder da perfeição intelectiva
(\emph{noû khoregòs kaì t[1EBD?]s no}\emph{e}\emph{râs
t}\emph{e}\emph{l}\emph{e}\emph{iótetos
hégem}\emph{ó}\emph{n}), mas ele conduz acima todas as
coisas, até o reino de Kronos e, sendo líder com seu pai, faz
existir todo o Intelecto pericósmico. E se é necessário falar a
verdade explicitamente, cada um desses dois períodos --- quero
dizer, o aparente/\,visível e o inaparente/\,invisível --- participa
de ambos esses deuses, mas um é mais de Kronos, enquanto o outro
está sujeito ao reino de Zeus (Proclo, 2005, p. 656 [grego] e
657 [italiano]; Phillips, 2007, p. 160).
\end{quote}

Para Proclo, nem o Criador nem a criação são maus, apesar do
\emph{status} inferior desta última, e não há oposição ou
conflito entre a Deidade suprema e o Demiurgo.

Dentre as opiniões que Krulak sugere serem características da
posição de Proclo existe, não obstante tudo o que dissemos, uma
que passa no teste de uma caracterização como estritamente
antignóstica: “o mal não é essencial à alma”; o Mal representa
uma fraqueza da alma, e não resulta (como querem intérpretes
dualistas de Platão, \emph{Leis}, 896c) de uma Alma do Mundo
ou Demiurgo malignos. 

\section{Schelling}

Em 1215, o quarto concílio lateranense estabeleceu como dogma
a doutrina da criação a partir do nada,\footnote{ “(\ldots{})
\emph{qui sua omnipotenti virtute simul ab initio temporis
ultramque, de nihilo condidit creaturam, spiritualem et
corporalem}. (\ldots{})” (\emph{Conciliam Lateranensis \textsc{iv} 1215,}
\emph{Constitutiones, \textsc{i}. De fide catholica}, linhas
10--12. \emph{In}: Josepho Alberigo \emph{et al}.,
\textit{Conciliorum Oecumenicorum Decreta}, 3 ed. Bolonha:
Istituto per le Scienze Religiose, 1973, p. 230.} tendo listado
certos pontos cosmológicos essenciais em resposta a desafios
heréticos. Os principais pontos foram os seguintes (Riggs, 1998,
p. 161--162): 1- a natureza espiritual e também a corpórea foram
criadas [= não emanaram] a partir do nada; 2- a criação é
temporal [= não é contínua]; 3- a humanidade é uma unidade de
alma e corpo [igualmente bons]; 4- a criação é boa; 5- o pecado
resulta da tentação do Diabo [e o Mal não é uma privação do
Bem]. Antes do 4º concílio de Latera, abundaram controvérsias
sobre como definir uma única cosmogonia cristã normativa diante
da herança pelo Cristianismo de dois conceitos distintos, a
criação a partir do nada e o emanacionismo. Conforme Cheryl
Riggs, prevaleciam nessa época três sobre posições criação a
partir do nada (Riggs, 1998, p. 151): 1- o nada é, na verdade,
algo (\emph{aliquid}); 2- o nada é, de algum modo, Deus; 3- o
nada é, na verdade, um não algo. A segunda e a terceira dessas
posições situam"-se mais perto da posição adotada pelo 4º
concílio lateranense, na medida em que parecem indicar que só
Deus existe desde sempre e que, de um modo absolutamente livre,
um Deus onipotente criou de Si\footnote{ Posição atacada por
Agostinho (\emph{Confissões}, 12.7).} (onipresente que é), ou
da inexistência (\emph{e}\emph{k tou m}\emph{e}\emph{
ontos})/\,desde o nada (\emph{e}\emph{x ouk ontos}),\footnote{
Posição abraçada por Eriúgena (\emph{De divisione naturae},
1.12, 2.24 e 3.5 e ss.), Anselmo (\emph{Monologium}, 5--8 e
9--12), Tomás de Aquino (\emph{Summa Theol}., 1.44) e Bernardo
de Clairvaux (\emph{De consideratione}, 5.6.14)} mas através
de Si, a existência. A doutrina conciliar entendeu, ademais, que
o mundo foi feito inteiramente de uma \emph{substantia}
distinta daquela de Deus\footnote{ Opinião contrária a de
Boaventura (\emph{Itinerarium mentis in Deum}, 1.14, 2.1 e
6.2), por exemplo.} e que este não é Causa Material do mundo. A
posição que propunha que o nada é \emph{aliquid} (alguma
coisa) convinha perigosamente às crenças e heresias que
sustentavam que existe eternamente algo que não é Deus ou que a
\emph{substantia} de que o mundo é feito pode em certa medida
ser entendida como uma forma degradada da \emph{substantia}
divina. Algumas crenças e heresias entendiam, além disso, que a
deidade suprema não é Causa Material do mundo, e que esta deve
ser buscada precisamente em algo independente de Deus e que
precede a criação, merecendo por isso o nome de “nada”
(\emph{nulla res nata}). A criação, ademais, não teria
resultado de uma decisão livre, e essa doutrina se aproxima de
entendimentos Neoplatonistas de que do Uno Inefável emana, por
necessidade de Sua natureza e a partir de Sua essência, “a
grande cadeia do ser” que se estende desde as Formas eternas até
a matéria informe.

Vivendo numa época muito distante e distinta daquela do 4º
concílio de Latera --- cerca de meio milênio depois (1775--1854),
em circunstâncias moldadas pela pulsão revisionista da história
herdada da Reforma protestante e do Iluminismo e pelo
embevecimento Romântico com os estudos cabalistas\footnote{ “Dos
pós"-kantianos, Schelling é o pensador que maior parentesco
parece revelar com a Cabala” (Serrão, 1960, p. 15).}
(notadamente através de Jacob Böhme) ---, Schelling veio a abraçar
em \emph{Filosofia e Religião} (1804) e no
\emph{Freiheitsschrift},\footnote{ Título abreviado comumente
empregado para indicar o texto \emph{Philosophische
Untersuchungen über} \emph{das Wesen der menschlichen Freiheit
und die damit zusammenhängenden Gegenstände} (1809).} em seu
tratamento do problema da cosmogonia e da teodiceia, uma
concepção da relação entre Deus, o nada e a criação que, em sua
heterodoxia, aproxima"-se da ideia de que o nada é, na verdade,
algo --- é como que uma protomatéria resultante da decisão da
Divindade (enquanto pura potencialidade) em “outrar"-se” como
Fundamento, num ato originário que, se por um lado poderíamos
chamar de instância de autocriação de Deus como Criador, cria na
mesma ocasião um substrato, a “natureza de Deus”, a partir do
qual se constrói a criação. 

O mapeamento de influências platônicas e neoplatônicas na obra do
jovem Schelling já foi feito por Michael Vater\footnote{
\textsc{vater},
M. G. Schelling's Neoplatonic System"-Notion:
``\emph{Ineinsbildung''} and Temporal Unfolding. 1976.
\emph{In}: R. \textsc{baine} \textsc{harris} (Ed.), \textit{The Significance of
Neoplatonism}. Norfolk, Virginia: International Society for
Neoplatonic Studies, Old Dominion University, p. 275--299.} e,
antes dele, por Werner Beierwaltes. Para este último, “há
estruturas de pensamento de Schelling e Plotino que se tocam e
mutuamente se ilustram” (Beierwaltes, 1972, p. 109).\footnote{
Cf. ainda \textsc{beierwaltes}, W. 1973. \textit{Absolute Identität,
Neuplatonische Implikationen in Schellings “Bruno”}.
\emph{Philosophisches Jahrbuch}, 80: 242--266 (republicado em
\textit{Identität und Differenz}, Frankfurt"-sobre"-o"-Meno:
Vittorio Klostermann, 1980, p. 204--240).} Sem querermos nos
aprofundar aqui sobre a questão da contribuição Neoplatonista ao
pensamento schellinguiano, pensamos que a literatura cabalística
--- apropriadora de tradições Gnósticas e Neoplatônicas\footnote{
No cabalismo do \emph{Sepher Yezirah} Leo Baeck (1873--1956)
enxergou uma forte influência procleana, enquanto Gershom
Scholem (1897--1982) viu na tradição cabalista como um todo um
aporte tanto do Neoplatonismo quanto do Gnosticismo --- para
Scholem, o esquema cabalista da Emanação é uma versão da
estrutura Neoplatonista, modificada pela sobreposição de uma
hierarquia Gnóstica. Para Charles Mopsik (em \textit{Les Grands
Textes de la Cabale: les rites qui font Dieu}. Paris: Verdier,
1993), a Cabala provençal, que tem Isaac o Cego (1165--1235) como
representante mais destacado, estaria na base do “casamento” do
pensamento religioso judaico com o pensamento Neoplatonista.} ---
é o lugar onde melhor podem ser encontradas argumentações
assemelhadas àquelas da teocosmogonia schellinguiana (mais
explícita no \emph{Freiheitsschrift}). Majoritariamente
através de Böhme, mas também através de Reuchlin e de outros
cabalistas cristãos, Schelling herda da Cabala (especialmente a
luriânica\footnote{ De Isaac Luria (1534--1572).}) a preocupação
com a cisão originária do Absoluto.

Para Schelling, a dissimilaridade dos entes em relação a Deus
revela que, em seu devir, em paralelo ao ato autorrevelador do
Absoluto --- Ser por excelência ---, opera uma outra modalidade de
ser, cuja atividade dimana daquilo que, em Deus, não é
exatamente Ele mesmo (Schelling, 1860, p. 359). Essa “oposição
interna” em Deus é considerada por muitos como um traço Gnóstico
fundamental.\footnote{ Cf. Epifânio, \emph{Adversus haereses},
\textsc{xxxiii}, 7. Em seu ensaio “The Gnostic Syndrome: Typology of its
Thought, Imagination, and Mood” (\textit{Philosophical Essays}.
Englewood Cliffs, 1974, p. 263--276), Hans Jonas descreve o
conflito interno no Divino como um momento estrutural do
Gnosticismo.} Entendido separadamente da criação, Deus é
“desfundamentação” (\emph{Grundlosigkeit}), pura vontade
(\emph{Wollen}) independente do tempo enquanto protoexistência
(\emph{Urseyn}) (Schelling, 1860,
p. 350). Schelling distingue entre um ser enquanto mero
fundamento da existência (Schelling, 1860, p. 357)\footnote{
“\emph{dem Wesen, sofern es existirt, und dem Wesen, sofern es
bloß Grund von Existenz ist}”.} e um ser enquanto existente, e
considera o Absoluto em Deus como sua existência
(\emph{Existenz}), o ser real (\emph{das aktuale Seyn}),
enquanto o fundamento (\emph{Grund}), por outro lado, é Sua
natureza, que Lhe é inseparável, se bem que distinta
([\emph{ist eine}] \emph{Unterscheidung}).\footnote{ “Essa é
a incompreensível base da realidade das coisas (\ldots{}), aquilo que
(\ldots{}) sempre permanece no fundamento (\emph{Dieses ist an den
Dingen die unergreifliche Basis der Realität\ldots{} das\ldots{} ewig im
Grunde bleibt})” (Schelling, 1860, p. 359--60).} Para Schelling,
uma pura potencialidade precede o real, e a realidade é
apresentada como determinada por --- ou tensionada entre --- dois
princípios (complementares mais que antagônicos): um princípio
“desfundamentado” e, no entanto, fundamentador (identificado com
a unicidade, se bem que dipolar) e um princípio que é o
fundamento que, apesar de ser o \emph{locus} espaciotemporal
onde seres se fundam e crescem (na multiplicidade), é também
onde afundam e se decompõem. 

A pura potencialidade que precede o real é uma espécie de nada
que, paradoxalmente, é condição de possibilidade de tudo. Em
frases escritas por Paulo Borges em relação ao filósofo luso
José Marinho (1904--1975), mas que podemos levar em conta para um
bom entendimento de Schelling, “a re"-velação/\,geração de Deus
a/\,por si mesmo, enquanto emerge do sem"-fundo do Nada (\ldots{}), é,
simultaneamente, a re"-velação/\,geração de todos os seres,
possibilitando"-os a partir da \emph{a"-possibilidade}
primordial do Nada” (Borges, s/\,d, p. 10).\footnote{ Conforme
Paulo Borges (s/\,d, p. 15--16 n. 50), a palavra \emph{nada} não
se relaciona apenas à “mera negação e privação de existência,
ser e sentido, sendo apenas o termo menos desadequado para dizer
uma \emph{transcendência}, ou plenitude intrinsecamente
excessiva de todas as determinações positivas e negativas”, e
isso parece ser sugerido pela etimologia do seu equivalente
latino: ``\emph{nihil} advém da aposição da partícula negativa
\emph{ne} ao substantivo neutro \emph{hilum,[hil] i}, o qual
já tem o sentido de ``quase nada, nada, um nada'', sendo assim
sugerido, pela negação deste, o \emph{algo} misterioso (ou
seja, etimologicamente, \emph{inefável}) do que não é coisa
alguma nem a sua negação”.} Abissal, o Absoluto, “\emph{não
sendo, não"-não sendo nem sendo para ser ou não ser}, se intui
como a transcendência pura do \emph{Nada} virgem de qualquer
afirmação"-negação”. A cisão do Absoluto constitui para José
Marinho “a indeclinável condição de possibilidade de toda a
manifestação” (Borges, s/\,d, p. 10), seja ela de Deus, seja de
todos os seres, e assim também pensou Schelling.

A concepção schellinguiana do Começo como momento em que o “caos
das potências” --- um outro nome do Nada --- se realizou na Criação
entende que no processo de produção do real a partir do
Absoluto, na transformação do Fundamento infinito na existência
finita (por uma “contração do Fundamento”),\footnote{ A ideia
schellinguiana de uma “contração do Fundamento”
(\emph{Anziehen des Grundes}) faz recordar paralelos na
concepcão luriânica da \emph{tzimtzum} (“contração”)
originária do \emph{‘Eyn Sof} (o “Sem"-limite”, Ilimitado,
Infinito).} ocorre transitoriamente uma coincidência do
Infinito com o finito (Žižek, 1996), que por
um outro ponto de vista equivale a uma indistinção originária
entre o Bem e o Mal.  Este último não é concebido como um
apêndice “patológico” do finito (i.e., experiências ou
coisas contingentes), como algo que distrai os indivíduos de sua
busca pelo Bem Infinito. Ao invés disso, o Mal é entendido não
como um distanciamento ou degradação do Bem, mas como a
“subjetividade cindida” do próprio Deus,\footnote{ Cf., por
exemplo, \emph{Isaías}, 45, 7, e \emph{Provérbios}, 16, 4.
Cf. Ainda \textsc{scholem}, G. 1977. Gut und Böse in der Kabbala. In. G.
\textsc{scholem}, \textit{Von der mystischen Gestalt der Gottheit.
Studien zu Grundbegriffen der Kabbala}. Frankfurt"-sobre"-o"-Meno:
Suhrkampf, 1977, p. 49--82.} ou, noutros termos, como um resíduo
irredutível (\emph{nie aufgehende Rest}) do próprio Infinito
(Schelling, 1860, p. 359--60). Essa mácula possui no
\emph{Freiheitsschrift} o estatuto “ontológico” do
desregramento (\emph{Regellose}) e da treva
(\emph{Finsterniß}) ---, e conforme Slavoj Žižek testemunhamos a
emergência desse “Mal ontológico” naquelas ocasiões em que o
Infinito se faz ato ou é invocado enquanto finitude. O Mal não é
considerado como uma imperfeição metafísica; o desregramento, a
desordem e a amorfia originais (\emph{ursprüngliche}), a “base
incompreensível (\emph{unergreifliche}) da realidade”
(Schelling, 1860, p. 359), são na criação apenas um traço
vestigial, podendo sinalizar um Bem não efetivo (\emph{ein
unwirksames Gutes}) ou a ausência do Bem em ato
(\emph{aktuell}) (Schelling, 1860, p. 400), mas faticamente
não devem ser vistos como prelúdios do Mal. Ora, dizer do Mal
que ele é a ausência do Bem em ato ou um Bem não efetivo é
chegar muito perto de qualificar o Mal como privação do Bem,
como o faz boa parte dos neoplatonistas, ou particularmente como
uma antítese incompleta do Bem, como faz Proclo, e isso sugere
que Schelling assimila e dá novo significado a muita coisa
daquilo que critica, sem romper completamente com seus
antecessores.

Do mesmo modo como o Mal é entendido por Schelling como uma
espécie de resíduo proveniente de uma cisão do próprio Deus, o
mundo sensível é pensado como tendo início através de uma
ruptura do Absoluto.\footnote{ Ademais do
\emph{Freiheitsschrift}, consulte"-se ainda de Schelling a esse
respeito a obra \emph{Philosophie und Religion} (Sämmtliche
Werke\emph{,} ed. K. F. A. Schelling\emph{,} \textsc{vi}, p. 11--70),
p. 38 e ss.} Esse entendimento é tomado por muitos como sendo
de herança gnóstica. O conflito interno no Divino é retratado
como “\emph{Einbildung}” do Absoluto em um
“\emph{Gegenbild}”, no qual Ele contempla a Si mesmo. Não há
nessa concepção, contudo, qualquer ideia de culpa ou de
ignorância como determinantes da cosmogênese, e isso que
distancia Schelling dos Gnósticos. Enquanto o pensamento
Gnóstico se caracteriza geralmente por uma visão pessimista da
criação e enfatiza a necessidade de libertar o homem do mundo,
Schelling apresenta uma compreensão otimista da criação e da
liberdade no mundo.

\section{Sampaio Bruno}

José Pereira da Sampaio, que desde seus 14 anos se autonomeou
Bruno (doravante, Sampaio Bruno;
1857--1915) e que por conta de
diversas semelhanças de seu pensamento com aquele de Schelling
foi qualificado de “um Schelling português” (Chacon,
1995),\footnote{ Chacon acompanha (e cita) Joel Serrão, que
disse: “é precisamente com o de Schelling que o sistema do
portuense Bruno apresenta semelhanças mais flagrantes” (Serrão,
1960, p. 15).} é provavelmente o mais importante pensador
heterodoxo português,\footnote{ Os dados biográficos que se
seguem podem ser conferidos em Rodriguez, s/\,d, e Torres, 1975,
p. 141--152.} com uma obra que conforme Ricardo Vélez Rodriguez
“não se circunscreve ao campo da filosofia, da literatura ou da
história do pensamento. Circulou por todos esses terrenos, sem
se confinar em nenhum deles. A sua meditação constitui uma
tentativa libertária de saber místico, alicerçado em fontes
gnósticas\footnote{ Essas fontes são muito difíceis de mapear.
Em \emph{A Ideia de Deus} (1902) Sampaio Bruno oferece um
breve resumo da doutrina gnóstica da criação mediante emanações
a partir de uma citação de Pedro Amorim Viana (\textit{Defeza do
Racionalismo ou Analyse da Fé,} 3 ed. Porto: Casa de A. R. da
Cruz Coutinho, 1885, p. 152), e oferece numa breve sentença a
diferença fundamental entre seu pensamento e a “velha gnose”:
“Nesta [minha] nova concepção, a matéria não é eterna como Deus
e as emanações divinas não vão prevaricando à medida que se
afastam da origem” (Bruno, 1902, p. 465).} e da Cabala
especulativa” (Rodriguez, s/\,d). Seu pai, José Paes de Sampaio,
foi maçom da Loja Firmeza Portuense, e dele Bruno pode ter
recebido as primeiras influências indiretas do pensamento de
matriz Gnóstica ou Neoplatônica. Com apenas 17 anos publicou o
livro \emph{Análise da Crença Cristã --- Estudos críticos sobre
o cristianismo, dogmas e crenças} (1874), abraçando desde cedo
uma atitude revisionista em relação à tradição do cristianismo
normativo. Após o fracasso da revolução republicana de 31 de
janeiro de 1891 exilou"-se de Portugal, aonde retornaria em
fevereiro de 1893, trazendo vultosa quantidade de livros
adquiridos na Espanha, França, Bélgica e Holanda, e carregando
na mente “diversos trabalhos (\ldots{}), sobretudo acerca de judeus
portugueses” (Torres, 1975, p. 145).\footnote{ Torres, 1975, p.
145. Para Joel Serrão, as fontes predominantes de Sampaio Bruno,
a partir da crise do exílio, “são essencialmente esotéricas, e
direta ou indiretamente cabalísticas” (Serrão, 1960, p. 14).}
Quando estava no exílio teve em Salamanca “uma espécie de êxtase
místico (``o sobressalto profundo da minha consciência'') que o
conduziu a buscar a unidade primordial de onde tudo provém”
(Rodriguez, s/\,d). Teve outras experiências místicas de que deu
testemunho. Em virtude de suas leituras e suas vivências
místicas, Joel Serrão postula que torna"-se necessário buscar
para o pensamento de Sampaio Bruno “um quadro explicativo
diverso do habitual, um quadro (\ldots{}) que remonta ao
neoplatonismo, à filosofia dos gnósticos, à cabala judaica
(\ldots{}). O pensamento de Bruno evidencia raízes esotéricas,
filia"-se a uma tradição acroamática que, a par e passo da
filosofia moderna, racionalista e crítica, subrepiticiamente
continuou o ensino da cabala judaica, mística e emanacionista”
(Serrão, 1958, p. 138).

Em 1898 Sampaio Bruno publicou \emph{O Brasil Mental}, obra com
dois temas principais --- o positivismo e o monismo (abordados
inclusive em suas repercussões científicas, sociais e políticas)
--- onde se delineia seu pensamento antipositivista aspirante a
“uma filosofia mais inexata e menos terrestre” (Bruno, 1989, p.
282).\footnote{ O positivismo, “positivo demais” por sua vez, “é
rigidamente limitado; tem as fronteiras próximas e claramente
visíveis; é uma curva fechada; é um polígono cujas arestas não
toleram que as ultrapassem a conjectura”.} 

Bruno pretendeu compatibilizar a filosofia com o mistério, como
proclamou José Marinho:

\begin{quote}
Bruno não nos diz tão somente que o Mistério está na origem e no
princípio \emph{da sua filosofia}, mas nos diz que o Mistério
é o seminal princípio \emph{de toda a filosofia} (\ldots{}).
Aceitar o Mistério em filosofia é sempre contingência grave,
perigosa, suspeita para a maioria dos filósofos enquanto tais.
Quanto mais grave não é para modernos aceitar, com a noção de
mistério, a de queda originária!
(\ldots{}) Quando a queda misteriosa se aceita, todavia, não nos
seres imperfeitos que nós somos, mas em Deus, no seio da própria
Perfeição, nesse caso as pessoas de bom senso e os filósofos
para quem a razão não quer abjurar do bom senso tem o ímpeto
irreprimível de repudiar tal visão. Se no entanto se lhes diz,
como por exemplo disseram intérpretes respeitados como lógicos,
louvados por [serem] claros, tais [como] Teixeira Rêgo ou J.
Serrão, que a tese de Sampaio Bruno (\ldots{}) está fundamente
correlacionada com uma via antiga e moderna de famosos
visionários, poetas geniais e filósofos de raros caminhos,
talvez haja alguma esperança de se chegar enfim a tomar a sério
o que é mais sério (Marinho, 1976, p. 84--5; itálicos originais).
\end{quote}

Rui de Abreu Torres enxergou no pensamento de Sampaio Bruno
traços que vão do racionalismo deísta, anticlerical e
progressista dos Iluministas até o misticismo “hermético” de
inspiração maçônica/\,martinista e cabalista (Torres, 1975, p. 148
e ss.). Essa trajetória pode ser acompanhada desde a
\emph{Análise da Crença Cristã} (1874)\footnote{ Obra
complementada, de certa forma, por \emph{A Questão Religiosa}
(1907).}  até \emph{O Brasil Mental} (1898), \emph{A Ideia
de Deus} (1902) e \emph{O Encoberto} (1904).\footnote{ Obra na
qual ecoam as predições messiânicas de Gonçalo Annes Bandarra, o
sapateiro de Trancoso, do mesmo modo que no
\emph{Freiheitsschrift} de Schelling são escutados ecos das
prédicas de Böhme, o sapateiro de Görlitz.} Já para José
Teixeira Rêgo, “Sampaio (Bruno), que na sua mocidade se
inclinara para o ateísmo (\emph{Análise da Crença Cristã})
veio, pela reflexão e até por fatos que pretendia serem
revelações (\emph{A Ideia de Deus}, p. 146, 148 e ss.) a crer
em Deus” (Rêgo, 1931, p. 43) --- um Deus que, “embora de origens
bíblicas, passou pelo cadinho do Gnosticismo, da Cabala
especulativa --- em suma, de todo misticismo ocidental heterodoxo”
(Serrão, 1960, p. 8). Conforme Joel Serrão o próprio Sampaio
Bruno “caracterizou sucessivamente o seu pensamento por estas
formas: ``panteísmo idealista'', espiritualismo transcendente'',
``misticismo idealista'', expressões insuficientes, todavia, para
caracterizar o núcleo de intuições e vivências do qual partia.
Se o quisermos exigir um pouco mais, há que ir"-se para uma
adjetivação redundante: \emph{monismo
místico"-teológico"-teúrgico}, talvez” (Serrão, 1960, p. 13).

A atitude de Sampaio Bruno diante do mistério da origem é tanto
filosófica quanto mística, e abertamente heterodoxa. Em
\emph{O Brasil Mental}, Bruno ataca a concepção de que 

\begin{quote}
há um Deus único, que possui, ele só, naturalmente, a verdade
toda e que a revela em códigos fundamentais. O embuste humano
elaborou esses livros por excelência e deu a obra do seu tosco
toutiço como a emanação da mesma divindade. (\ldots{}) Nestes termos,
todo o trabalho de investigação filosófica cessa ou não se
enceta, porque a verdade se conhece --- integral --- sem os labores
da busca e sem as necessidades árduas da demonstração (Bruno,
1898, p. 145).\footnote{ Cf. as p. 199--200.}
\end{quote}

O mesmo pode"-se dizer da atitude anti"-mística da ortodoxia:
estando a verdade toda revelada, normatizada e
institucionalizada, desprezam"-se os méritos da busca\footnote{
Sampaio Bruno admite que a interação entre o humano e o divino
reúne elementos do esforço e da graça: “A oração é a aspiração
do espírito alterado para o espírito puro; subordina"-se a uma
lei transcendente de atração. O milagre é a emanação que
impulsiona o espírito alterado a avançar na libertação” (Bruno,
1902, p. 465). A profecia, por sua vez, é uma dádiva ao homem de
“seres espirituais superiores a nós (\ldots{}) mas que, quando o
queiram ou lhes seja cometido, no"-la podem comunicar,
fazendo"-nos conhecer com antecedência o futuro” (p. 174).} e o
valor singular do encontro individual com o divino.

Para Alexandre Fradique Morujão, em sua tentativa se solucionar o
problema das relações entre o mundo e seu fundamento Sampaio
Bruno encaminha"-se para uma solução 

\begin{quote}
ligada a certas especulações de raiz grega, Neoplatônica, que nos
primeiros séculos do Cristianismo alimentara diferentes sistemas
gnósticos heterodoxos, que teve importância na mística
especulativa germânica dos séculos \textsc{xvi}-\textsc{xvii}, especialmente em
Jacob Böhme, que se esforça para conciliar o Absoluto com a
existência da realidade finita e com a presença nesta do Mal,
concebido como algo de radical e de efetivo. Esta conciliação só
se tornaria possível se em vez de tomarmos como ponto de partida
um Ser perfeito partíssemos de uma geração intemporal de Deus.
Este misticismo atravessa a cultura ocidental e vai ser
recolhido pelo filósofo alemão Schelling, nos inícios do século
\textsc{xix} em pleno romantismo, na obra \emph{Filosofia e Religião}
de 1804 e traduz"-se na ideia da geração do mundo como uma queda
cósmica (Morujão, 1987, p. 12--13 [236--237]).
\end{quote}

Em \emph{A Ideia de Deus} pode"-se perceber uma influência do
pensamento alexandrino (seja ele de Fílon, de Plotino ou dos
Gnósticos) mesclada com elementos do pensamento
schellinguiano/\,hegeliano: 

\begin{quote}
No princípio era a Perfeição, o espírito homogêneo e
puro.\footnote{ Para Sampaio Bruno, esse espírito homogêneo e
puro é simultaneamente Deus e o Tempo (com maiúscula, para
distingui"-lo do tempo alterado ou espaço) --- cf. Bruno, 1902, p.
389. Essa concepção faz recordar a distinção zurvanista entre
Zervan (Zurvân) Akarana, o tempo ilimitado, e Zervan
dareghô-[k]hvadâtha (ou simplesmente Zervan), o tempo de longa
duração.} No segundo momento, mercê do efeito de um mistério,
temos o espírito diminuído\footnote{ Esse espírito diminuído
(“em quantidade, não na qualidade; na potência, não na
essência”) é, não obstante, Tempo puro.} e a seu par, a
diferença que se tornou homogênea, isto é, o mundo.\footnote{
Isto é, tempo alterado.} No terceiro momento, reintegrar"-se-á o
espírito puro, pela absorção final de todo o homogêneo. Assim,
três são os instantes supremos do crescimento. Um: é o espírito
homogêneo e puro, que foi e há de voltar a ser. Eis o ponto de
partida e eis o ponto de chegada. Outro: é o espírito puro, mas
diminuído atualmente, pelo destaque separativo do universo.
Enfim, o outro ainda: é esse universo, que aspira a regressar ao
homogêneo inicial. 
\end{quote}

Nós não podemos compreender como foi esse mistério da
diferenciação da parte do espírito puro. Porém, que ele dado se
houvesse é necessário: para que, um tanto inteligivelmente, o
enigma universal nos seja, ainda que em seu limiar, acessível
(Bruno, 1902, p. 460).

Aqui, o destaque que separa de Deus o universo refere"-se àquela
cisão da plenitude ou Perfeição primordial homogênea e pura ---
pura, inclusive, de qualquer determinação racional e, portanto,
recebedora da designação de “mistério”. Em \emph{O Brasil
Mental} Sampaio Bruno sugeriu\footnote{ Bruno, 1898, p. 364; cf.
p. 365 (trata"-se de uma sugestão indireta). Na p. 366 e ss.
Sampaio Bruno aborda a “impossibilidade orgânica” de se
construir na intuição sensível a ideia de infinito --- o que não
implica, contudo, numa impossibilidade lógica (p. 368).} que
qualificar até mesmo a Deus (o espírito puro, mas diminuído pela
cisão primordial) de infinito seria, a rigor, inapropriado,
posto que o infinito de Deus não é, “como na matéria, sucessão,
dimensão, variação, duração etc.”; não é, enfim, o infinito
matemático no tempo"-espaço, existindo propriamente “fora da
sucessão, fora da dimensão, fora da variação, fora da duração”
(Bruno, 1898, p. 364).

No entendimento de Sampaio Bruno, podemos associar a
tendência desprezadora do caráter mistérico da origem à
pretensão de qualificar concretamente a Deus. Com efeito, [nas
palavras de \emph{O Brasil Mental}] “Deus, para o monoteísta,
é um ser concreto e tão concreto tem de ser que, por qualidades
ou atributos, tratam de afastá-Lo dos outros concretos para
deles O distinguir. O monoteísta possui o irrespeito intrínseco.
Nutre"-o com a pretensão definidora, categórica e integral. Por
isso, estabelece (pensa que com rigor) a essência divina, e ousa
por as grosseiras mãos ao de sobre o mistério inefável” (Bruno,
1898, p. 143--4). O mesmo que se dá com os monoteístas dar"-se"-ia
com os politeístas; “aos seus tipos antropomórficos ambos os
consideram como superiores (nos seus atributos e em sua
essência) ao homem, que deles depende, mas concretos, reais,
existindo conjuntamente com [mas separadamente d']a
fenomenalidade dos seres” (Bruno, 1898, p. 144).\footnote{
Prossegue a sentença: “No intimo, a matéria de que são feitos os
deuses não é jamais a massa rude de que afeiçoado foi o
homem”.}

Para Eduardo de Soveral, “a dificil e fundamental questão de
conciliar a existência do Mal com a Perfeição e Onipotência
divinas, a que nenhum monoteísmo pode furtar"-se, (\ldots{}) no caso
específico da cultura portuguesa, sempre ocupou um posto
nuclear” (Soveral, 1987, p. 166). Pensadores portugueses, como
Antero de Quental, Sampaio Bruno, Teixeira de Pascoaes, Leonardo
Coimbra, Fernando Pessoa, José Marinho, Eudoro de Sousa e
Agostinho da Silva, acolhendo aquilo que Isaque Pereira de
Carvalho chamou de “uma tendência ao paradoxo lógico"-ontológico”
(Carvalho, 2007, p. 166 n. 6), com umas poucas diferenças entre
si entenderam Deus como “ausência de determinação
ôntico"-ontológica, um não"-um, uma não entidade, um não algo, um
vazio enquanto desprovido de qualquer qualificação delimitadora,
positiva ou negativa, e, por isso mesmo, ``tudo'', uma plenitude
que se pode entender quer como um todo, simples, indiferenciado
e indeterminado, quer como um tudo propriamente dito, que contém
em ato todas as determinações e antinomias possíveis, porém sem
que nelas se determine” (Borges, 2004, p. 122). Para ser todo,
Deus não pode prescindir do nada, do mesmo modo como para ser
completo tem de conter em Si, primordialmente, aquilo que
aparece ao homem como o Bem e o Mal. Tranquilo em seu
entendimento da relação do problema do Mal com o mistério da
origem, Sampaio Bruno pronuncia [em \emph{A Ideia de Deus}]
que “nem Deus é indiferente à nossa dor nem a sua maldade
possível nos alucina. Ele não goza duma plena felicidade
egoísta; também ele sofre da diminuição do espírito puro e do
mal da criatura, espírito alterado” (Bruno, 1902, p. 460).
Conforme Fradique Morujão, para Sampaio Bruno a “desdivinação”
origina o Mal, que progride com a diversificação da
matéria;\footnote{ Afonso Rocha entende que o entendimento de
Sampaio Bruno acerca da diversificação da matéria pode ser
relacionado com o judaísmo místico da Cabala (Rocha, 2006, vol.
\textsc{ii}, p. 150 e ss.).} contudo, os átomos da matéria conservam
frações mínimas da Consciência Divina, e nesse mínimo de
consciência opera, por disposição natural (como direção do
movimento nos átomos, instinto nos animais, razão no homem) “um
esforço por uma abertura para Deus, ou seja, para se integrar no
homogêneo” (Morujão, 1987, p. 14 [238]). O homem, “espírito
ascendendo na sua convergência do regresso” (Bruno, 1902, p.
460) e a criatura melhor dotada de consciência, tem por missão
libertar a criação inteira através da própria autolibertação
(Bruno, 1902, p. 470).

Resumindo a doutrina bruniana sobre o Mal em \emph{A Ideia
de Deus}, Eduardo de Soveral a considera “frontalmente
antiaugustiniana”. Considerando “inaceitável a existência
simultânea do Mal e de Deus”, Sampaio Bruno, que outrossim
considera empiricamente demonstrável, através da realidade da
harmonia no mundo, a existência de Deus, conclui pela
não onipotência atual d'Este, admitindo que, para isso (conforme
Soveral), ocorreu em Deus “uma misteriosa degradação ou
enfraquecimento.\footnote{ Para Sampaio Bruno, a perda da
onipotência ou “enfraquecimento” (voluntário) de Deus
corresponde à substituição progressiva do determinismo
(“fatalidade”) pela liberdade, da necessidade pela
contingência.} A partir daí admite que passou a haver uma
profunda aliança entre Deus, que conservou sua onisciência, e os
homens, a quem competirá a luta contra o Mal, iluminados por
Deus e mediante um progressivo aumento do saber (\ldots{})” (Soveral,
1987, p. 167 n. 11). 

Do mesmo modo como a cosmogonia tem três momentos para
Sampaio Bruno --- espírito homogêneo ou puro, espírito diminuído
ou alterado, espírito homogêneo reconstituído --- a redenção da
criação também apresenta em Bruno três instantes ---
autolibertação, libertação dos irmãos e libertação
universal.\footnote{ “O fim do homem neste mundo é libertar"-se a
si, libertando os outros seres” (Bruno, 1902, p. 468).} Esse
retrato da libertação do homem e da criação (incluindo animais e
coisas) em três momentos (Bruno, 1902, p. 469--70) acompanha a
progressão na ordem das três formas de hominidade no tratado 38
das \emph{Enéades} de Plotino (\textsc{vi}.7.6), podendo também
corresponder a uma progressão na hierarquia das três partes da
alma conforme Platão e Plotino ou, na Cabala,\footnote{ J. Serrão
opinou que a teoria da emanação em Sampaio Bruno “só pode ser
levada a bom termo se tivermos presentes os pressupostos
místicos"-metafísicos da Cabala” (Serrão, 1960, p. 15).} à
reciprocidade humana às finalidades de ‘\emph{Eyn Soph} (“[o]
sem limite”), a Causa Primeira, conforme o \emph{Zohar} ---
Bendição (\emph{Berachá}), Santificação (\emph{Kedushá}) e
União (\emph{Yehud}). De todo modo, ao completar"-se a
reconstituição do espírito homogêneo e do Tempo puro terá fim o
Mal, podendo"-se dizer que ocorre, nessa \emph{apokatástasis}
ou \emph{frashokereti}, uma absorção final no Bem.

\section{Conclusão}

Proclo, Schelling e Sampaio Bruno podem ser estudados como
exemplos de acolhimento, modificação e apropriação de doutrinas
Gnósticas ou assemelhadas em sistemas filosóficos que, contudo,
lograram preservar uma maior afinidade com o Neoplatonismo ---
esses três autores abraçaram ideias acerca da criação e seu
fundamento, e apresentaram soluções ao problema da teodiceia, ao
longo de linhas que são habitualmente consideradas como
Neoplatonistas. O pensamento procleano aproxima"-se do
Gnosticismo principalmente por meio da influência, comum a
ambos, de Numênio de Apameia e dos \emph{Oráculos Caldeus}.
Ideias de Proclo parecem ter influenciado o desenvolvimento da
Cabala --- tanto da judaica quanto da cristã --- e é amplamente
reconhecida a aproximação de Schelling e de Sampaio Bruno aos
ensinamentos cabalistas acerca da cosmogênese e da origem do
Mal. Correntes não cabalistas do Neoplatonismo também chegaram a
Schelling e, através deste ou por outras vias, até Sampaio
Bruno, pensador heterodoxo que “tanto se assumirá como gnóstico
com se declarará seguidor de uma outra concepção gnóstica que
não a da ``velha gnose'' (Rocha, 2006, vol. \textsc{ii}, p. 147).\footnote{
Cf. ainda p. 148 e ss.} Apesar de Afonso Rocha ter opinado que
a “outra concepção gnóstica” seguida pelo portuense é aquela da
Cabala judaica, não podemos esquecer que o Neoplatonismo também
propõe uma gnose salvífica que busca conciliar o Absoluto com a
finitude e com a existência do Mal radical e efetivo. Esses são
apenas uns poucos exemplos das complexidades e dificuldades que
aguardam os pesquisadores que se dispuserem a palmilhar com
seriedade o fascinante campo das investigações acerca da questão
da tradição e da contemporaneidade gnósticas no Neoplatonismo.


\section{Bibliografia}

\begin{description}\labelsep0ex\parsep0ex
\newcommand{\tit}[1]{\item[\textnormal{\textsc{\MakeTextLowercase{#1}}}]}
\newcommand{\titidem}{\item[\line(1,0){25}]}

\tit{ATTRIDGE}, H. W., \textsc{mueller}, D. (Tradutores) 1988. The Tripartite
Tractate (\textsc{i}, 5). \emph{In}: J. M. \textsc{robinson} (Ed. geral),
\textit{The Nag Hammad\=\i Library}, 3 ed. revista. S.
Francisco: Harper \& Row, p. 58--60 (introdução) e 61--103
(tratado).

\tit{BEIERWALTES}. W. \textit{Platonismus und Idealismus}.
Frankfurt"-sobre"-o"-Meno: Vittorio Klostermann, 1972.

\tit{BERG}, R. M. \textit{Proclus' Hymns: essays, translations,
commentary}. Leiden/\,Boston/\,Colônia: Brill, 2001.

\tit{BIANCHI}, H. \textit{Le Origini dello Gnosticismo.} Leiden: E. J.
Brill, 1967.

\tit{BORGES}, P. Sem data. \textit{Nada, Espírito e Saudade em José
Marinho}. Disponível em: www.pauloborges.net

\tit{BORGES}, P. “Do Nada que é Tudo”. A Poesia Pensante e Mística de
Agostinho da Silva. \emph{In}: \textsc{autores} \textsc{vários},
\textit{Agostinho da Silva. Um Pensamento a Descobrir}. Torres
Vedras: Cooperativa de Comunicação e Cultura, 2004, pp. 121--126.

\tit{BRUNO (José Pereira de Sampaio)}. \textit{O Brazil Mental}. Porto:
Livraria Chardron, 1898.

\tit{BRUNO (José Pereira de Sampaio)}. \textit{A Ideia de Deus}. Porto:
Livraria Chardron, 1902.

\tit{CARVALHO}, I. P. de. 2007. \textit{Teologia e mitopoiética da
história em Agostinho da Silva}. \emph{Revista Convergência
Lusíada} (\textsc{rj}), 23 (Número especial --- Centenário de Agostinho da
Silva, 1906--2006): 159--169.

\tit{CHACON}, V. 1995. Sampaio Bruno: um Schelling Português.
\emph{In}: L. A, \textsc{barreto} (Org.), \textit{Colóquio Antero de
Quental Dedicado a Sampaio Bruno} [= \textit{Colóquio Antero de
Quental, Anais}] (1º módulo 14 e 15/\,09/\,1993; 2º módulo 17, 20 e
21/\,09/\,1993). Aracaju: Secretaria de Estado da Cultura de
Sergipe, 1995, p. 275--287.

\tit{CORNFORD}, F. McD. \textit{The Unwritten Philosophy and other
essays}. Cambridge: Cambridge University Press, 1950.

\tit{ELSAS}, C. \textit{Neuplatonism und gnostische Weltablehnung in
der Schule Plotins}. Berlim/\,N. Iorque: Walter De Gruyter, 1975.

\tit{HUTIN}, S. \textit{Los Gnósticos} (1958, 2 ed. 1963), trad. T.
Moro Simpson. Buenos Aires: Editorial Universitaria de Buenos
Aires, 1964 (reimpr. 1976).

\tit{KRULAK}, T. 2009. \textit{John Frederick Phillips}\textit{\emph{
Order from Disorder: Proclus' Doctrine of Evil and its Roots in
Ancient Platonism}}. \emph{Bryn Mawr Classical Review}, 02
(40). Disponível em: http:/\,/bmcr.brynmawr.edu/
2009/\,2009--02--40.html.

\tit{LAYTON}, B. \textit{As Escrituras Gnósticas: nova tradução com
anotações e introdução} (1987), trad. M. Oliva. São Paulo:
Loyola, 2002.

\tit{MAJERCICK}, R. 1992 \textit{The Existence"-Life"-Intellect Triad in
Gnosticism and Neoplatonism}. \emph{Classical Quarterly}, 42:
475--488.

\tit{MARINHO}, J. \textit{Verdade Condição e Destino no Pensamento
Português Contemporâneo}. Porto: Lello e Irmão, 1976, p. 84--5.

\tit{MORUJÃO}, A. F. 1987. \textit{O itinerário filosófico de Sampaio
Bruno}. \emph{Revista Portuguesa de Filosofia}, 3--4: 1--18
(225--242).

\tit{PHILLIPS}, J. F. \textit{Order from Disorder: Proclus' Doctrine of
Evil and its Roots in Ancient Platonism}. Leiden/\,Boston: Brill,
2007.

\tit{PROCLO}. \textit{Teologia Platonica} (ed. bilíngue), trad. M.
Abate. Milão: Bompiani, 2005.

\tit{PROCLUS\emph{}.} \textit{The Elements of Theology}, trad. E. R.
Dodds, 2 ed. Oxford: Clarendon Press, 1963 (reimpr.
2000).

\tit{PROCLUS}. \textit{Proclus' Commentary on Plato's}
\textit{\emph{Parmenides}}, trad. G. Morrow e J. Dillon.
Princeton, N. Jérsei: Princeton University Press, 1992 (reimpr.
da ed. de 1987, com correções).

\tit{PROCLUS}. \textit{Hymnes et Prières}, trad. H. D. Saffrey. Paris:
Arfuyen, 1994.

\tit{PUECH}, H.-C. 1957 (reimpr. 1973). \textit{Gnosis and Time}
(publicado originalmente nos \emph{Eranos Jahrbücher}, \textsc{xx},
1951). \emph{In}: R. \textsc{manheim} (Ed. e trad.), \textit{Man and
Time} (\emph{Papers from the Eranos Yearbooks}, 3). Princeton:
Princeton University Press, 1957, p. 38--84.

\tit{RÊGO}, J. T. \textit{Estudos e Controvérsias, 2ª Série}. Porto:
Faculdade de Letras, 1931, p. 39--56 [ed. original: \textsc{rêgo}, J. T.
1915. \textit{José Pereira de Sampaio (Bruno)}. \emph{Águia},
nº 45, 2ª série].

\tit{RIGGS}, C. A. 1998. Prophecy and Order: Mysticism and medieval
cosmologies in the twelfth and thirteenth centuries.
\emph{In}: A. \textsc{ferreiro} (Ed.), \textit{The Devil, Heresy, and
Witchcraft in the Middle Ages: essays in honor of Jeffrey B.
Russell}. Leiden/\,Boston/\,Colônia: Brill, p. 149--163.

\tit{ROCHA}, A. \textit{O Mal no Pensamento de Sampaio (Bruno)}, 2
vols. Lisboa: Imprensa Nacional"-Casa da Moeda, 2006.

\tit{RODRIGUEZ}, R. V. Sem data. \textit{Um liberal português
heterodoxo: Sampaio Bruno (1857--1915)}. Disponível em
www.ecsbdefesa.com.br.

\tit{SCHELLING}, F. W. J. von. \textit{Philosophische
Untersuchungen über das Wesen der menschlichen Freiheit und die
damit zusammenhängenden Gegenstände} (1809). \emph{In}:
\textit{Sämmtliche Werke} (ed. K. F. A. Schelling),
\textit{\textsc{vii}
Band, \textsc{i} Abteilung, 1805--1810}. Stuttgart \& Augsburg: J. G.
Cotta, 1860, pp. 333--416.

\tit{SERRÃO}, J. \textit{Sampaio Bruno: o homem e o pensamento}.
Lisboa: Livros Horizonte, 1958.

\tit{SERRÃO}, J. (Org.) 1960. Apresentação. \emph{In}:
\textit{Sampaio Bruno, Prosa}. Rio de Janeiro: Agir, p. 5--15.

\tit{SIORVANES}, L. \textit{Proclus: neoplatonic philosophy and
science}. New Haven/\,Londres/\,Edinburgo: Yale University
Press/\,Edinburgh University Press, 1996.

\tit{SOVERAL}, E. A. de. 1987. \textit{Algumas notas em torno da noção
de pecado: recordando as “Confissões” de Sto. Agostinho}.
\emph{Revista da Faculdade de Letras: Filosofia}, 4: 151--176.

\tit{TORRES R}. de A. 1975. \textit{Historiadores e Eruditos} (Cultura
Portuguesa, 16). Lisboa: Empresa Nacional de Publicidade/\,Editora
Notícias, pp. 141--152.

\tit{WALKER}, B. \textit{Gnosticism: Its History and Influence --- a
survey of gnostic thought from their pre"-Christian origins to
their modern manifestations}. Wellingborough, Northamptonshire:
The Aquarian Press, 1983.

\tit{ŽIŽEK}, S. Selfhood as Such Is Spirit: F. W. J. Schelling on the
Origins of Evil. \emph{In}: J. \textsc{copjec} (Ed.), \textit{Radical
Evil}. N. Iorque: Verso, 1996, p. 1--29.
\end{description}



\capitulo{Heidegger e Agostinho. A propósito das lições sobre Agostinho
		e o Neoplatonismo (1920--1921)}%
	{Bento Silva Santos}
	{ufes/\,cnp\textnormal{q}} 


\markboth{Heidegger e Agostinho. A propósito das lições\ldots{}}{Bento Silva Santos}


A relação de Martin Heidegger com o pensamento de Agostinho, na
década de 1920, pode ser caracterizada de três modos possíveis.
Em primeiro lugar, a apropriação parcial do pensamento do bispo
de Hipona, especialmente de suas descrições da vida da alma
inquieta (\emph{facticidade hermenêutica}) e do conceito do
tempo nos livros \textsc{x} e \textsc{xi} das \emph{Confissões}. Em segundo
lugar, a rejeição da reflexão agostiniana a partir do momento em
que ela se associa à esfera do transcendente, de Deus, da
eternidade. Esta pesquisa fora considerada por Heidegger como
abandono da facticidade e retorno às concepções metafísicas
clássicas. Nesta postura, Heidegger proclama, no semestre de
1921--1922, um “ateísmo de princípio” como dever da filosofia, ou
seja, para ver e captar a vida fática em suas possibilidades
ontológicas decisivas, a filosofia há de ser fundamentalmente
ateia, renunciado a uma revelação, a qualquer certificação
prévia de haver assegurado a posse do sentido próprio da vida.
Em terceiro lugar, o não reconhecimento de sua dívida em face de
Agostinho: enquanto a facticidade continua a exercer sua função
básica para a analítica existencial, a reflexão agostiniana é
cada vez mais agonizante até tornar"-se objeto de observações
exclusivamente negativas. 

Neste artigo eu me limitarei ao exame da apropriação da tradição
cristã representada pelo pensamento de Agostinho por parte de
Heidegger em sua releitura do Livro \textsc{x} das \emph{Confissões},
procurando explicitar em minha análise as insuficiências desta
leitura e algumas omissões significativas.

\section{O interesse de Heidegger pela experiência religiosa
do cristianismo das origens}

É no contexto das tendências filosóficas mais representativas da
segunda metade do século \textsc{xix} --- a neoescolástica de Franz
Brentano, o neokantismo de Paul Natorp e de Heinrich Rickert, a
filosofia da vida de Wilhelm Dilthey e a fenomenologia de Edmund
Husserl\footnote{ Sobre a formação do jovem Heidegger, ver
especialmente \textsc{v}. \textsc{sassi}, 2007, p. 21--59.} --- que Heidegger
esboçará uma forma de pensar extremamente original. Heidegger
questionará toda a abordagem tradicional da filosofia
(especialmente em seu contexto histórico com o \emph{retorno à
origem} empreendido pela filosofia neoescolástica e pela
filosofia científica dos neokantianos) afirmando que tais
filosofias são determinadas fundamentalmente por uma
\emph{tradição} (isto é, por uma filosofia historicamente dada
na tradição) e não por um retorno originário às coisas
interrogadas. Diante da situação da filosofia em sua época,
Heidegger se norteará pela seguinte preocupação: como se vive a
realidade antes mesmo de se organizá-la em uma postura
epistemológica, valorativa ou de concepção de mundo? Heidegger
examinará a postura fundamental do vivenciar (\emph{die
Urhaltung}), dando um passo atrás na abordagem de Husserl que
falou apenas de \emph{vivência} (\emph{Erlebnis}) sempre já
orientada por uma consciência.\footnote{  “O vivenciar
(\emph{das Er"-leben}) não passa diante de mim, como uma coisa
que eu coloco como objeto, mas eu mesmo me aproprio (\emph{ich
selbst} \emph{er"-eigne es mir}), e ele acontece conforme a sua
essência (\emph{es ereignet} \emph{sich seinem Wesen
nach}). E se compreendo olhando para isto, então eu o
compreendo, não como processo, como coisa, objeto
(\emph{Vor"-gang, als Sache, Objekt}), mas como algo totalmente
novo, um evento (\emph{ein Ereignis})”. E ainda comenta
Heidegger: “Evento (\emph{Er"-eignis}) não significa que eu me
aproprio da vivência a partir do exterior ou a partir de
qualquer ponto; externo e interno \emph{(aussen und innen})
tem aqui assim pouco sentido como físico e psíquico
\emph{(physisch und psychisch}). As vivências são eventos
(\emph{die Er"-lebnisse sind Er"-eignisse}) enquanto elas vivem
do que lhes é próprio (\emph{aus dem Eigenen}) e é somente
assim que vive a vida (\textsc{heidegger}, 1919, p. 75).} 

O aprofundamento crítico com a tradição filosófica se verifica
especialmente nas preleções que ocorrerão após da primeira
guerra mundial, a partir de 1919. A característica deste
aprofundamento consiste fundamentalmente no fato de que o ponto
de partida de toda pesquisa sobre a vida é a apreendida em sua
originalidade: o da destruição crítico"-fenomenológica. Isto “não
é um aspecto secundário, mas antes pertence necessariamente à
fenomenologia” (\textsc{heidegger}, 1920, p. 186). Portanto, o
aprofundamento sempre se faz a partir da experiência fática da
vida enquanto mobilidade própria da existência. 

Ora, a relação entre Heidegger e Agostinho, enquanto objeto deste
artigo e fruto de minha pesquisa sobre a recepção de Agostinho
na tradição filosófica,\footnote{  A minha pesquisa
intitulada “A filosofia como diálogo em Agostinho  Um estudo
sobre a relação entre  Metafísica neoplatônica e Cristianismo
nos diálogos filosóficos (386--391)” inclui uma seção sobre a
recepção de Agostinho na tradição filosófica (Heidegger,
Gadamer, Wittgenstein, Ricouer etc).} emerge inequivocamente
do curso de Heidegger do semestre de inverno de 1920--1021
(“Introdução à fenomenologia da Religião”), no qual encontramos
reflexões de caráter metodológico\footnote{   As temáticas
que são abordadas devem consideradas à luz de um contexto mais
amplo, uma vez que elas exercem um papel fundamental no projeto
de uma “Hermenêutica da Facticidade”, que se desenvolve durante
todo o ciclo de lições que Heidegger dará em Freiburg de 1919 a
1923. O fenômeno da vida na questão metodológica assume uma
conotação diversa, que lhe é conferida pela introdução do
conceito de \emph{Faktizität}. Heidegger falará assim de
experiência fática da vida.} acerca do “fenômeno da experiência
fática da vida” e uma “interpretação fenomenológica do
cristianismo das origens”. É a experiência de vida que descreve
são Paulo em suas cartas, a saber: a carta aos Gálatas e as duas
cartas aos Tessalonicenses, através da redefinição em termos
fenomenológicos desta experiência de vida protocristã, que
Heidegger assume a tarefa de uma destruição da tradição
teológica que perdeu o seu caráter originário tentando explicar
a estrutura da vida religiosa com a introdução da filosofia
platônico"-aristotélica, que cria um sistema doutrinário
desvinculado de todo liame com a vitalidade autêntica que se
expressa na experiência de fé. “Quando Deus é concebido
primariamente como objeto de especulação (\emph{als Gegenstand
der Spekulation}) há uma queda da compreensão autêntica”
(\emph{Abfall vom eigentlichen Verstehen}).\footnote{ 
\textsc{heidegger}, 1920--1921, p. 97.}  Neste sentido Heidegger
procurará, a partir do diálogo com são Paulo e Agostinho,
identificar momentos puramente cristãos, isto é, a experiência
fática da vida fundada na descoberta da temporalidade originária
do ser do homem cristão. A religiosidade cristã vive a
temporalidade como tal.

Mas, afinal, de onde deriva para Heidegger o interesse pela
experiência religiosa do cristianismo das origens? Durante o
ciclo das lições dadas em Freiburg o uso sistemático do método
fenomenológico que Heidegger herdara de Husserl se relaciona
sempre mais frequentemente com motivos que provêm do âmbito
religioso. Esta atenção particular deve"-se a um confronto
contínuo entre tradição cristã e tradição metafísica. O projeto
heideggeriano de uma re"-fundação da metafísica se desenvolve
paralelamente ao re"-pensamento da tradição cristã que, segundo
Heidegger, perdeu o seu caráter originário ao longo da história
do ocidente. Paradigmático neste sentido é o interesse pela
“mística”, que se desenvolve de modo concreto já na tese de
habilitação de 1916 sobre Duns Scotus e que teria prosseguimento
no curso projetado, mas jamais ministrado, sobre \emph{Mística
Medieval}. Da mística Heidegger deriva uma nova ideia de
\emph{intencionalidade} que prescinde da cisão entre sujeito
cognoscente e objeto conhecido. Todavia, o que mais preocupa o
filósofo é ver se é possível configurar a experiência mística em
si, no modo em que expressa a sua religiosidade, como
experiência fática. Ora, a mística é fundada em uma experiência
pessoal, na imediaticidade com a qual a divindade se comunica
com o homem. Na intensidade desta experiência, em primeiro
momento, Heidegger procura a manifestação do caráter originário
da experiência religiosa, uma vez que nela vê aquele caráter de
vitalidade que o cristianismo dogmático e racionalista herdado
da tradição (patrística e, sobretudo, escolástica) não possuiria
mais. Mas esta atenção dada ao pensamento místico em Heidegger
logo se atenuará, uma vez que, embora este coloque uma ruptura
com a tradição, não será mais útil para fornecer os instrumentos
para uma fenomenologia da experiência religiosa. Isto se deve ao
fato de que a experiência mística é “não histórica” e
“não conceitual”, ao passo que emerge como central para
Heidegger o inserção da vida fática em um âmbito histórico"-real.
 

\section{As lições de Heidegger: <<Augustinus
und der Neuplatonismus>> (1920--21)}

O interesse forte e paradoxal pelo cristianismo se constata ainda
mais no curso \emph{Agostinho e o Neoplatonismo} dado por
Heidegger no semestre de verão em 1921, que vem a ser uma
leitura incomum para a época de alguns textos de Agostinho,
particularmente das \emph{Confissões}. Estas reflexões
constituem a segunda parte das lições pronunciadas por Heidegger
no quadro de uma leitura fenomenológica da religião. A
hermenêutica fenomenológica de Heidegger tinha como objetivo
fundamental a vida fática. Esta hermenêutica da facticidade se
aprofundou durante o semestre de 1921--1922 no contexto de uma
interpretação fenomenológica de Aristóteles. É no horizonte da
“experiência fática da vida” que Heidegger reencontra Agostinho,
o que significa que os textos do bispo de Hipona participam
\emph{hic et nunc} na hermenêutica da facticidade e, portanto,
na criação do conceito mesmo da facticidade. G. Agamben
sublinhou até mesmo o fato de que a origem do próprio conceito
de “faciticidade”, em sua acepção heideggeriana, deve ser
procurada não em Fichte ou em Husserl, mas em Agostinho. Este
escrevera o seguinte: “\emph{facticia est anima}”, isto é, “a
alma humana é factícia, no sentido de que ela foi ``feita'' por
Deus” (\textsc{agamben}, 1988, p .63--84; 66; \textsc{capelle}, 2001, p. 184--185).
Agostinho opõe \emph{facticius} e \emph{nativus}, em
conformidade ao uso clássico do latim; esta observação acerca da
origem do vocábulo “facticidade” é fundamental para reforçar a
“dívida agostiniana” no pensamento de Heidegger. A vida não tem
lugar “naturalmente”, mediante um nascimento segundo as regras
previamente estabelecidas pela natureza. A vida é “fabricada” a
todo instante novamente para além das regras gerais. Se o homem
gera (segundo a natureza), Deus cria (em sua soberania): a
facticidade. Heidegger cita esta frase de Agostinho como
\emph{leitmotiv} de todo o seu pensamento: \emph{quaestio
mihi factus} (!) \emph{sum}. Esta \emph{facticidade} reflete
a obra de “\emph{fazer} a verdade” diante de Deus
(\emph{coram Deo}), o que equivale à exploração do abismo da
consciência humana para além da cisão entre exterioridade e
interioridade.  

Tal deve ser, segundo Heidegger, a relação com todo pensador do
passado, enquanto ele é determinado pelo pensamento filosófico
autêntico (ele mesmo fático). No caso em questão, Agostinho fez
uma verdadeira “hermenêutica da facticidade”. É Agostinho que,
identificando o tempo com a alma, descobre o essencial mesmo da
vida fática. Assim, em vista de mostrar sua apropriação
não objetivante da tradição cristã representada por Agostinho,
Heidegger, no curso intitulado \emph{Agostinho e o
Neoplatonismo}, afasta as concepções de E. Troeltsch, de W.
Dilthey ou de A. Von Harnack, que se orientam segundo três
pontos de vista distintos, herdeiros de uma problemática em voga
na época, a da \emph{helenização do cristianismo}. O que
interessa Heidegger não é o papel que exerceu o pensamento
agostiniano na história das ideias, mas a situação da realização
histórica do fenômeno da vida fática na época de Agostinho e o
impacto deste cumprimento em nossa própria situação, na vida
fática que nós mesmos somos. Neste sentido, a leitura
heideggeriana de Agostinho coloca"-nos novamente, segundo os
dizeres de O. Pöggeler, “diante de uma escolha dessas opções que
formam o pensamento ocidental” (Pöggeler, 1967, p. 59), a saber:
a apreensão fundamental da vida que nós mesmos somos, de um
lado, e, de outro lado, a falta ou a fuga, em suas diversas
facetas, desta vida diante dela mesma. O fato de Heidegger
escolher S. Agostinho como interlocutor significa que certamente
os textos deste \emph{manifestam} esta dupla dimensão da
facticidade, em sua apreensão e em seu abandono, ao passo que na
maior parte dos outros textos que nossa tradição filosófica
salvaguardou, o fenômeno da facticidade permanece fechado. 

Em suma: a crítica heideggeriana sobre a história objetivante e,
neste caso, sobre o Neoplatonismo, completa nossa abordagem:

\begin{quote}
O neoplatonismo e Agostinho não são casos fortuitos, mas em sua
concepção a historicidade deve justamente elevar"-se a uma
dimensão de efetividade, dimensão singular, que constitui o
próprio do “aí” onde nós nos mantemos hoje. A história nos toca
e nós mesmos a somos. É justamente isto que não vemos hoje
enquanto cremos ter a história e possuí-la em uma visão
histórica objetivante até aí inigualável (\textsc{heidegger}, 1921, p.~%
173). 
\end{quote}

A relação entre Neoplatonismo e Agostinho determina a posição
teórica de Heidegger face à história objetivante. Heidegger fala
de uma delimitação negativa consistente em Agostinho, como
também na filosofia neoplatônica, que se estruturariam a partir
de motivos essenciais da dogmática filosófica e teológica. A
delimitação à qual Heidegger está mais ligada consiste em elevar
ao nível da historicidade o caráter próprio da relação entre
Agostinho e o neoplatonismo. Em outras palavras: ele deseja
elevar ao nível da experiência da vida fática a realização do
neoplatonismo de Agostinho. É preciso aproximar"-se de um
“cristianismo” livre de vínculos dogmáticos e de uma
“metafísica”, cujo mais íntimo sentido e cuja mais íntima
essência emergem a partir de uma obra de “destruição” e
“desconstrução”. Da objetivação da experiência originária cristã
na forma de uma “grecização da consciência cristã da vida”,
realizada pelo encontro entre cristianismo e filosofia grega,
deve"-se chegar “à experiência originário"-cristã da vida, que
encontra o seu aspecto fundamental no ``sentido de realização'' da
existência, no ``como'' se vive na situação de espera. O teólogo
cristão é aquele que atende às suas obrigações nas situações
cada vez mais diversas que constituem a vida fáctica. A relação
pessoal com Deus pode realizar"-se a partir de e no empenho
fáctico"-existencial na vida histórico"-concreta, sem que aconteça
algum discurso sobre ele. A religiosidade cristã é para ser
assumida, portanto, a partir e permanecendo na experiência
fáctica da vida. A religiosidade cristã é uma expressão da
experiência fáctica da vida”.  (\textsc{sassi}, 2007, p. 193--207;
\emph{passim}) (\textsc{brito martins}, 1998, p. 295). 

\section{A \emph{memória} agostiniana no \emph{Augustinus und der Neuplatonismus}}

No curso de 1921, a leitura do livro \textsc{x} das \emph{Confissões},
onde Agostinho medita sobre a alma e a memória, ocupa um lugar
central. A reflexão agostiniana sobre a memória introduz a
problemática sobre do tempo do Livro \textsc{xi}. A identificação da alma
e da memória liberta esta última da referência exclusiva ao
passado e faz dela o lugar onde o presente, o futuro e o passado
sucedem simultaneamente. Esta simultaneidade é a famosa
\emph{distensio animi}, o tempo agostiniano explorado no Livro~\textsc{xi}. 
Com esta concepção de Agostinho, já estamos diante do
fenômeno do tempo; por essa razão Heidegger, que procura nesta
época aceder tanto a temporalidade como à essência da vida
fática, concentra todo seu interesse no fenômeno da
\emph{memória} agostiniana. Este agudo interesse exercerá um
papel crucial tanto na elucidação do essencial da facticidade
como na descoberta da temporalidade própria do \emph{Dasein}. 

No § 9 do curso, intitulado “O estupor a propósito da memória”,
Heidegger chama a atenção para a infinidade da \emph{memória}
agostiniana: \emph{penetrale amplum et infinitum}. Esta
infinitude da memória significa ultrapassar a si mesmo, o
ultrapassar de uma identidade pessoal, “visto que por sua
própria infinitude, o que a memória encerra se estende para além
de tudo o que a alma pode apreender” (\textsc{barash}, 1996, p. 107).
Depois de ter constado esta não coincidência da alma consigo
mesma (“Não chego, porém, a apreender todo o meu ser”:
Agostinho, \emph{Confissões} \textsc{x}, 8,), Heidegger pergunta: “Onde
deve estar, o que partir dela mesma a alma não apreende?”
(\textsc{heidegger}, 1921, p.~182), o que nos evoca a própria pergunta de
Agostinho: “Então onde está o que de si mesmo não encerra?”
(\emph{Confissões} \textsc{x}, 8). Esta pergunta de Heidegger põe em
evidência uma dupla pesquisa na qual Agostinho se empenhou. De
um lado, estando em cisão consigo mesma, a alma se manifesta
como o cuidado (\emph{curare}) de si mesmo. Estando em cisão
com ela mesma, a alma está dispersa e à mercê das tentações
assim compreendidas: \emph{concupiscentia carnis,
concupiscentia oculorum, ambitio saeculi}. Sem o socorro divino
a alma não terá outra alternativa senão a de ceder, consumindo
sua queda que duplicaria a cisão inicial. De outro lado, a alma,
em cisão consigo mesma, procura justamente o socorro que não
seria limitado por ela mesma. A procura da alma conduz,
portanto, a partir dela mesma, a uma realidade que estaria fora
dela, a fim de que esta realidade responda à “situação” própria
da alma. Heidegger discerne aqui os fenômenos de
intencionalidade e de transcendência. O socorro procurado só
pode vir de Deus, e Heidegger empreende então uma reflexão sobre
o liame entre a facticidade da alma e a figura de Deus: “Na
procura de Deus alguma coisa em mim não somente vem à
`expressão', mas isto constitui minha facticidade e a maneira
pela qual eu cuido dela” (\textsc{heidegger}, 1921, p. 192). Assim a
procura de Deus por parte da alma pode ser vivida como um
elemento constitutivo da vida fática que manifestaria também as
perspectivas de compreensão da intencionalidade e da
transcendência. No curso de 1921, esta questão colocada por
Heidegger sobre a relação entre a facticidade e a figura de Deus
foi rapidamente abandonada. 

Mas, no contexto de explicitar o pensamento de Heidegger em
relação à tradição filosófico"-cristã representada por Agostinho,
poderíamos indagar: qual é, portanto, a resposta que Heidegger
fornece à pergunta sobre a relação entre a figura humana e Deus?
É conhecida esta resposta já esboçada no curso de 1918 sobre a
mística medieval: a figura de Deus cuja relação com a alma
finita constitui a vida fática, faz parte igualmente da vida
fática e é também, por conseguinte, radicalmente temporal. Em
filosofia, Heidegger rejeita qualquer outra concepção desta
relação, especialmente aquela tradicional do cristianismo
segundo a qual é possível conceber, filosoficamente, a relação
entre a finitude humana e Deus transcendendo absolutamente esta
finitude. Para Heidegger, o afã de integração da eternidade em
si atemporal na finitude temporal da alma não tem nenhum sentido
filosoficamente pertinente: só o tempo constitui o horizonte
autêntico da filosofia; a eternidade não teria lugar algum na
esfera temporal.\footnote{   Entretanto, é precisamente pela
memória, tal como é examinada no Livro \textsc{x}, que vemos em Agostinho
a eufonia entre a finitude misteriosa do mundo e do mistério de
Deus que não cessa de habitar seu lugar. Os capítulos 8--13, onde
se articula a reflexão de Agostinho sobre a \emph{memoria
Dei}, a \emph{memoria mundi} e a \emph{memoria sui}, mostram
inequivocamente o quanto tudo o que, pelos sentidos, eu realizo
no ``imenso palácio de minha memória'' pertence ao interior de mim
mesmo e que é na travessia da memória de si que eu descubro a
memória de Deus (cf. \textsc{capelle}, 2005, p. 165).} 

Assim manifesta"-se a rejeição de Heidegger em relação ao
pensamento de Agostinho quando este prossegue suas reflexões
sobre a figura de Deus"-eternidade e alma finita. Após ter aberto
o caminho para as análises da vida fática até sua dimensão
teológica, Agostinho, segundo Heidegger, teria traído este
caminho verdadeiramente filosófico\footnote{   Soa estranho
o uso simultâneo aqui dos termos “teológico” e “filosófico” para
explicitar o pensamento de Heidegger. Acontece, porém, que o
próprio recorreu ao mesmo tempo a duas afirmações aparentemente
contraditórias: de um lado, ele reivindicou um “ateísmo de
princípio” para o exercício do filosofar; de outro lado,
enquanto filósofo da facticidade, ele se define como um “teólogo
cristão” (\emph{Carta endereçada a K. Löwith} em 19 de agosto
de 1921). Como explicar esta conjunção original? Ora, o
filósofo, tendo metodologicamente como fonte e objeto da
filosofia exclusivamente a esfera temporal, pode fazer um
discurso ao mesmo tempo sobre coisas divinas enquanto estas são
consideradas, em suas próprias raízes (radicalmente), como
temporais. Quanto à teologia que se ocupa com algo da esfera
supra"-temporal, o filósofo dela se afasta; a filosofia é
metodologicamente ateia. Somente a partir de e permanecendo no
próprio contexto mundano"-fático pode se abrir ao “divino
pessoal”.}  em favor de especulações metafísicas
estimuladas pelo sistema das relações entre o tempo e a
eternidade. A alma temporal que procura a Deus, procura a
eternidade, algo que está além do tempo. Em vez de concentrar"-se
mais longamente no sentido que Agostinho confere à noção de
infinito no seio da expressão \emph{penetrale et infinitum},
Heidegger acusa Agostinho de ter feito um compromisso com o
neoplatonismo, inaceitável do ponto de vista da filosofia
fática. Os Padres da Igreja teriam, portanto, submergido sua
concepção original da \emph{memória} no poço platônico da
\emph{anamnesis},\footnote{ De um lado, convém observar que
Agostinho critica a teoria platônica da reminiscência no
\emph{De Trinitate} \textsc{xii}, 15,24, o que manifesta, por sua vez,
a insuficiência da leitura heideggeriana. De outro lado,
Agostinho remete sua concepção da\emph{ memória} não tanto à
filosofia neoplatônica, mas, antes de tudo, aos escritos de
Paulo. Por conseguinte, a noção agostiniana de eternidade deve
ser compreendida primordialmente a partir das Escrituras
Sagradas, e só secundariamente, até mesmo não obrigatoriamente,
a partir da tradição platônica. Se Heidegger rejeita o conceito
de eternidade forjado pela metafísica grega, devemos constatar a
ausência, em sua reflexão, de toda explicação com o conceito de
eternidade bíblico, do qual Agostinho é herdeiro. (cf. \textsc{barash},
1996, 109, n.12; \textsc{capelle}, 2005, 164--165). Além disso, quando
Agostinho discorre sobre a relação entre \emph{memória e
disciplinas liberais}, há uma expressão que sugere a rejeição da
doutrina da reminiscência: “\emph{sed in meo recognovi}'' (mas
na minha [mente] as reconheci) (trad.: “mas reconheci"-as
existentes em mim”; \emph{Confissões} \textsc{x}, 17, linhas 16--17),
isto é, reconheci de certo modo (as noções $=$ as noções mesmas e
não as imagens provenientes da percepção sensorial.) Se ao
prefixo {\emph{re}} do verbo \emph{recognovi} dermos
o sentido iterativo (isto é, no sentido de que “reconhecei” uma
segunda vez), o verbo parece indicar que Agostinho acolha ainda
a teoria da reminiscência de Platão; mas, segundo \emph{Conf}.
\textsc{vii}, 17, 23, Agostinho renunciou a tal teoria; o prefixo
{\emph{re}}, portanto, tem valor intensivo: “reconheci
de certo modo” as noções.} na cisão filosoficamente mortal
do tempo e da eternidade. O fenômeno da facticidade está ausente
desde que remetamos o tempo, a inquietude original da alma, a
uma entidade imutável, tranquila que seria sensato obter para a
alma o repouso e a fruição. Esta fruição, \emph{fruitio Dei},
o repouso no \emph{Summum bonum}, equivale à supressão da
angústia fática, da vida fática como tal. Assim, em referência à
eternidade, é o fenômeno do próprio tempo que é omitido. Através
da crítica heideggeriana a Agostinho e ao platonismo está
estabelecida inequivocamente a necessidade de abordar o tempo a
partir dele mesmo e não partir da eternidade.  

\section{O fenômeno da <<tentação>> (§§ 13 até 15) como
expressão da mobilidade originária da vida fática}

Não há melhor exemplo para ilustrar a leitura heideggeriana do
texto agostiniano do que a análise do “fenômeno da tentação”:
“Numquid non tentatio est vita humana super terram sine ullo
interstitio?” (Agostinho, \emph{Confissões} \textsc{x},28,39) As três
tentações de que fala Agostinho --- a saber:
\emph{concupiscentia carnis, concupiscentia oculorum} e
\emph{ambitio saeculi} --- são objeto de discussão por parte de
Heidegger que as caracteriza como fenômenos que se relacionam
antes de tudo com a compreensão do “eu sou” (\emph{Ich bin}).
Nas análises da memória e dos tipos de tentação Heidegger
apreende uma perspectiva ontológico"-existencial, ressaltando os
elementos estruturais da vida fática que emergem da
autobiografia de Agostinho. Eis como Heidegger expressa o
caráter fundamental da facticidade na leitura que faz de
Agostinho: “O caráter de tentação --- não em sentido religioso;
para experiênciá-lo não é necessário que se viva qualquer
experiência fundamental de tipo religioso. Diga"-se, porém, que
somente através do cristianismo se tornou visível a tentação
como característica da mobilidade. Visível, experienciável na
vida fática, possível de ser assumida no ‘eu sou'\,”
(\textsc{heidegger}, 1921, p.\,154). 

Embora comente o “fenômeno da tentação” a partir do § 13 até o §
15, o § 12 (“O cuidado'' [cura] como caráter fundamental da vida
fática) ocupa lugar central na medida em que permite reunir em
só um momento o fenômeno da dispersão da vida e o do cuidado. O que
Heidegger designa por fenômeno da dispersão da vida é o fenômeno
da “distentio animi” ao nível da temporalidade, e da “in multa
defluximus” ao nível da existência fática. A ambivalência da
vida é vivida como provação; ela é ao mesmo tempo a dispersão e
a diversidade das possibilidades da existência. Mas transporta
igualmente com ela a possibilidade de uma unidade desta
dispersão e desta diversidade. Esta unidade é produzida por
Deus. O fenômeno da tentação é apresentado por Heidegger segundo
quatro grupos de problemas: 1º) O problema da própria tentação
se referindo essencialmente à facticidade do eu; 2º) O problema
do “in multa defluximus”; 3º) O sentido do questionamento
“quaestio mihi factus sum”; 4º) O problema axiológico que diz
respeito à hierarquia dos valores relativamente à tentação. A
expressão “in multa defluxere” tem uma afinidade conceitual com
o que Heidegger chama, em \emph{Sein und Zeit}, a
\emph{Geworfenheit}. Em \emph{Sein und Zeit}, o ser"-lançado
pertence à constituição existencial do “aí”. 

Em nosso contexto atual, porém, devemos verificar se “in multa
defluximus” manifesta a possibilidade de expressar o fenômeno da
\emph{tentatio} enquanto existencial. A frase de Agostinho
explica a força existencial de sua interrogação. Heidegger, por
sua vez, parafraseia mais de uma vez a passagem de Jó 7,1 da
\emph{Ítala} insistindo no caráter fundamental no qual
Agostinho experimenta a vida fática: “Numquid non tentatio est
vita humana super terram sine ullo interstitio?”. Heidegger
fornece uma dupla significação da tentação no contexto de uma
inspiração bíblica: “Diversos sentidos da tentação. [No primeiro
sentido] 1. ``tentatio deceptionis'': com a tendência de induzir à
queda; [No segundo sentido] 2. ``tentatio probationis'' com a
tendência à provação. No primeiro sentido: tentação significa
unicamente a tendência a cair; no segundo sentido significa que
Deus nos põe à prova”  (\textsc{heidegger}, 1921, p.~273).

Em \emph{Sein und Zeit} o cuidado revela a estrutura originária
total do \emph{Dasein} e, de outro lado, revela também a
multiplicidade fenomenal da constituição do todo estrutural. O
cuidado é “destinado a preparar a problemática
fundamental"-ontológica, a questão do sentido do ser em geral”
(Heidegger, 1927, p. 172). A tentação, interpretada por
Heidegger em suas lições expressa duas experiências do “si”
partilhado entre “in multa defluximus” e a “continentia”
enquanto dois fenômenos autênticos que estruturam o ser do
homem. É em uma linguagem ontológica que Heidegger faz esta
releitura designada como “eigentliche Verstehensrichtung”. Neste
sentido, o pensamento de Heidegger se desdobra a partir de
motivos existenciais agostinianos à luz de conceitos ontológicos
de Aristóteles. Se, de fato, é o caráter polissêmico do sentido
do ser (\emph{ousia} e \emph{tode ti}) permite entender “in
multa defluximus”, as três formas de tentação, em compensação,
expressam o sentido da origem da existência enquanto realização
histórica da facticidade. É preciso ainda distinguir nesta
hermenêutica da facticidade, de um lado, entre o que pertence
propriamente dito ao sentido categorial desta existência fática
--- portanto, o sentido formal das categorias, a saber o
\emph{legein} --- e, de outro lado, a hermenêutica histórica
realizada, isto é o ser em vir"-a"-ser. (\textsc{brito martins}, 1998, p.
322--323).

Na linguagem ulterior de \emph{Sein und Zeit} a questão do
sentido do ser em geral deixa"-se explicitar ontologicamente
segundo três existenciais: a afeição, a compreensão, a linguagem
enquanto abertura do \emph{Dasein}. O cuidado, enquanto
articulação total originária, designa a totalidade do
\emph{Dasein} em um horizonte do que está diante de si mesmo.
Ele é a conexão que religa o homem à sua própria definição.
Agostinho expressa esta totalidade como experiência existencial
da relação entre o eu e o Deus pessoal: “Quando estiver unido a
Vós com todo o meu ser, em parte nenhuma sentirei dor e labor. A
minha vida será então verdadeiramente viva, porque estará toda
cheia de Vós” (Agostinho, \emph{Confissões} \textsc{x},28). A adesão ao
``Vós” divino resume afinal a totalidade de “meu ser” que não
pode existir verdadeiramente sem que Deus o encha de sua
presença. “Agora, porém, visto que Vós libertais do seu peso
aqueles que encheis, não estando cheio de Vós, sou ainda peso
para mim” (Agostinho, \emph{Confissões} \textsc{x}, 28). É somente
quando eu tiver aderido a Deus que não serei mais para mim dor
(\emph{dolor}) e labor (\emph{labor}), o que faz peso para
mim quando vivo longe de Vós.

\begin{quote}
Minha vida é minha própria vida, eu existo. Quando me uno a Vós,
do fundo de mim mesmo, eu me apóio total e radicalmente em Vós ---
“a vida estará cheia de Vós” --- toda relação à vida, a
facticidade inteira será penetrada de Vós e será realizada de
maneira a que toda realização se cumpra diante de Vós
(\textsc{heidegger}, 1921, p. 249). 
\end{quote}

É porque não estou cheio de Vós que eu sou um peso para mim, Esta
situação de sentir"-me --- “oneri mihi sum”, “um peso para mim
mesmo” --- expressa uma experiência que poderíamos chamar a face
“afetiva” da provação ao “mihi quaestio factus sum”, “eu me
tornei para mim mesmo uma questão”, o que corresponderia ao
existencial do \emph{compreender}, tal como Heidegger o define
em \emph{Sein und Zeit} (\textsc{brito martins}, 1998, p. 323).  

Na leitura heideggeriana verifica"-se, portanto, como a tentação é
expressão de mobilidade originária da vida fática, daquele ser
que não pode mais chegar a um estado de repouso, de imobilidade
definitiva. A vida em si mesma é uma prova, uma tentação, no
sentido ontológico"-existencial. Não faz sentido, portanto, a
discussão, quer sobre sua possibilidade, quer sobre a chance de
furtar"-se a ela.  

\section{Da interpretação de Agostinho ao abandono definitivo
da religião} 

Sem dúvida Agostinho é uma presença constante e documentável no
caminho que Heidegger percorre --- desde os primeiros textos em Freiburg,
a partir de 1919, até à obra \emph{Sein und Zeit} --- nas fases
de elaboração daquela analítica ontológico"-existencial que tem
em vista recolocar o problema esquecido sobre o sentido temporal
do ser. Uma presença não só implicada com determinadas
problemáticas desta pesquisa sobre o ser em geral, mas também
envolvida com a própria ideia desta pesquisa enquanto tal, e com
a sua reformulação radical enquanto estrutura do \emph{Dasein}
e abertura do ser. 

Mas este encontro com Agostinho é uma relação complexa. Neste
encontramos certamente uma leitura insuficiente de Agostinho e
algumas omissões significativas.  

 Em seu comentário da conferência de 1924 de Heidegger, Ph.
Capelle (2005, p. 161--164), indica ao mesmo tempo a
distância e a dívida de Heidegger em relação à reflexão
agostiniana. Heidegger apoia sua interpretação do conceito de
tempo de Agostinho no binômio \emph{distentio/\,intentio}.
\emph{Distentio} significa, para Agostinho e Plotino, uma
temporalização da alma assim que esta é mergulhada no fluxo
temporal do quotidiano. É aqui que se enraíza a conotação
negativa do tempo. Mas a noção de \emph{intentio}, típica de
Agostinho, valoriza a dimensão positiva da temporalidade, que
diz respeito ao momento escatológico da alma, o gesto de
apropriação da vinda de Deus em seu seio e de sua conversão
incessante, \emph{em tensão} para as coisas divinas
irredutíveis à ordem cronológica quotidiana. Heidegger,
apropriando"-se do movimento da \emph{intentio} agostiniana
(através da identificação do tempo e da alma), subtraindo"-a de
sua referência à eternidade (Deus) e colocando"-a em uma
perspectiva do ser"-para"-o"-fim, a qual se tornará posteriormente
o ser"-para"-a"-morte. Ora, o que é abandonado por este
deslocamento é a ideia agostiniana segundo a qual a
\emph{intentio}, acolhendo a eternidade, não suprime a
\emph{distentio}, a finitude temporal própria ao
homem,\footnote{   O que é eliminado pela conversão, segundo
Agostinho, é a \emph{aversio}, o tempo do pecado, que não
podemos confundir com a \emph{distentio}. É esta confusão que
Heidegger não soube evitar. (Cf. \textsc{capelle}, 2005, p. 163--164).} 
mas a enriquece da dimensão inaudita da presença da
transcendência absoluta. Mas a identificação do tempo com a
própria alma parece abrir o que ultrapassa o tempo, ou até mesmo
muito mais: o que ultrapassa o tempo constitui esta
identificação. Realizando uma leitura restritiva de Agostinho,
Heidegger jamais pensou em explicar a possibilidade desta
habitação da eternidade no seio da temporalidade. Tal
procedimento lança uma sombra em um ponto essencial de seu
pensamento, a saber: a reivindicação de pensar o tempo a partir
do tempo e não em referência à eternidade.  

Embora as referências aos textos de Agostinho sejam raros em
\emph{Sein und Zeit}, os seus temas essenciais, porém, devem
muito à reflexão do bispo de Hipona. Indiquemos, por exemplo,
fenômenos tais como “opacidade ôntica e pré-ontológica do
Dasein”,\footnote{ \textsc{heidegger}, 1927, p.~44.} que Heidegger
evidencia citando Agostinho,\footnote{  “Agostinho se
interroga \emph{Quid autem propinquius meipso mihi?} E se vê
na obrigação de responder: \emph{ego certe laboro hic et
laboro meipso: factus sum mihi terra difficultatis et sudoris
nimii} (\emph{Confissões}, lib. 10, cap. 16)” (\textsc{heidegger}, 1927,
p. 43--44). Agostinho descreve este fenômeno em união com a
questão da memória e do esquecimento; portanto, em união com a
questão do tempo, ao passo que Heidegger omite este liame. Em
compensação, Heidegger fala da memória no § 68 de \emph{Sein
und Zeit} sem se referir a Agostinho. “O curso de 1921 pode
fornecer uma razão para esta ausência: aos olhos de Heidegger,
Agostinho não chega a resolver o enigma do esquecimento, porque
ele não domina o ternário intencional do \emph{Gehalt-,
Bezugs-, Vollzugssinn}. Segundo o direcionamento intencional que
se privilegia, o \emph{non praesto} \emph{est}, que define o
fenômeno do esquecimento, se reveste de um sentido diferente
(\textsc{ga}
60, 188)”. (\textsc{greisch}, 2000, p. 228--229; cf. também
\textsc{barash}, 1996, p.
111--112).} a transcendência em união com a
intencionalidade,\footnote{   Para compreender como o binômio
transcendência/\,intencionalidade, essencial na analítica
existencial, se refere à reflexão de Agostinho, ver \textsc{ga} 60, p.
191--192. Heidegger, porém, cala"-se quanto a esta relação em
\emph{Sein und Zeit}. (cf. \textsc{greisch}, 2000, 230).} a
verdade,\footnote{   \emph{Sein und Zeit}, § 44, em liame
com \textsc{ga} 60, p. 192--204. Em seu \emph{Sein und Zeit},  Heidegger
não se refere, porém, a Agostinho no que diz respeito à questão
da verdade. (cf. \textsc{greisch}, 2000, p. 230--232).} a afeição
(\emph{Befindlichkeit}) e o compreender
(\emph{Verstehen}),\footnote{   § 29 e § 31 de \emph{Sein
und Zeit}. No curso de 1921, Heidegger já descobrira esses
fenômenos a partir da famosa expressão agostiniana: “Mihi
quaestio factus sum”.} a curiosidade\footnote{   § 36.
Heidegger se refere explicitamente a Agostinho.} ou ainda
autenticidade/\,a inautenticidade.

Não obstante os pontos comuns com Agostinho, Heidegger considera
este, em \emph{Sein und Zeit}, como figura importante no
processo metafísico do esquecimento do ser. Agostinho
privilegiou, no pensar, o \emph{ver}: o pensamento deve
conduzir à \emph{contemplação} das coisas divinas. Desta
maneira o ser, identificado com Deus, torna"-se o que
“está-constantemente"-sob"-os"-olhos”. Esta concepção do ser como
“estar"-sob"-os"-olhos” é a submissão do ser à medida humana; por
essa razão, tratar"-se"-ia do “esquecimento do ser”. Ora, este
caminha paralelamente com a omissão do fenômeno originário do
tempo: “estar"-sob"-os"-olhos” remete para “a presença constante”;
portanto, para o tempo presente, e este privilégio do presente
não permite mais apreender o tempo de uma maneira originária.
Considerar o fenômeno do tempo a partir do presente significa
medir cronologicamente, segundo o relógio, “vulgarmente”. A
concepção originária do tempo se elabora a partir do futuro, a
partir do ser"-para"-a"-morte. No § 81, a concepção do tempo de
Agostinho é assimilada, com a de Aristóteles, à “experiência
vulgar do tempo”. Trata"-se, aliás, da única referência, em
\emph{Sein und Zeit}, a Agostinho, no que diz respeito à
questão do tempo. Como na conferência de 1924, sobre o
\emph{conceito de tempo}, Heidegger sublinha o liame que
Agostinho e Aristóteles elaboram entre o tempo e a alma. Ele não
se preocupa, porém, de mostrar a diferença eventual entre as
concepções aristotélica e agostiniana do problema. Designando
este liame entre a alma e o tempo como um horizonte da
“interpretação do Dasein como temporalidade”, o autor de
\emph{Sein und Zeit} passa imediatamente às análises deste
horizonte em Hegel. Agostinho não atrai mais a atenção de
Heidegger! Este retorna também na mesma ocasião em que apareceu
\emph{Sein und Zeit} à concepção de tempo de Agostinho em
\emph{Os Problemas fundamentais da fenomenologia} (1927). Sem
explicar uma afirmação emblemática --- “Agostinho percebe de
maneira mais original certas dimensões do fenômeno do tempo” ---
Heidegger entra imediatamente em debate com Aristóteles. Quanto
a Agostinho, Heidegger lastima em ter que “renunciar a uma
interpretação detalhada [\ldots{}] do tratado agostiniano”. É desta
maneira frustrante que o estudo da concepção de tempo de
Agostinho é definitivamente abandonado por Heidegger.  

Mas sem aprofundar essas observações sobre a recepção de
Agostinho no contexto deste artigo, retorno ao ponto principal
da minha proposta. No curso do semestre de verão de 1921 sobre
\emph{Agostinho e o Neoplatonismo} Heidegger analisa o Livro
\textsc{x}
das \emph{Confissões}. Desta análise encontramos duas
características fundamentais da figura de Agostinho. Em primeiro
lugar, ele esboça uma forma autêntica de religiosidade, a partir
de uma compreensão do mundo"-de"-si (que determina
\emph{interioridade}) e da definição de conceitos como
\emph{graça} e \emph{pecado}. Em segundo lugar, sublinha o
caráter da imperfeição do ser humano e de sua necessidade
constitutiva sobre o ser de Deus. Para responder a tal desejo,
Agostinho fornece à religiosidade uma base conceitual que coloca
as premissas para o nascimento de uma teologia dogmática. De
fato, ao tentar explicar o sentido da religiosidade Agostinho
recorre às categorias derivadas da metafísica grega, no seu
caso, do Neoplatonismo, e dá ensejo à “helenização” do
cristianismo, assim como o tinha configurado são Paulo, que perde
o seu caráter de experiência de vida originária.

Para Heidegger, perdeu sentido todo o projeto de uma
fenomenologia da religião e, portanto, de uma destruição da
tradição teológica da qual se propunha partir para aportar à
redefinição de toda a metafísica. Assim ele modifica sua
estratégia e se empenha diretamente na destruição metafísica,
dirigindo sua atenção para Aristóteles. Através de uma
reconstrução da filosofia aristotélica, Heidegger coloca as
bases para a sua nova reflexão que faz diretamente referência à
questão do ser. Todavia, ele não abandona ainda por completo as
temáticas religiosas. A inversão de tendência em relação às
concepções filosóficas do tempo concerne ao fato de que ele não
considera a religiosidade à luz da filosofia aristotélica,
definindo entre ambas uma expressão do sistema metafísico.
Heidegger procura, ao contrário, uma expressão original da
filosofia aristotélica que seja reconduzível à dimensão da
historicidade e facticidade provenientes da experiência
protocristã. Assim os conceitos fundamentais tais como
\emph{vida e temporalidade}, que serviram como fio condutor da
interpretação de Paulo, são aqui novamente propostos na mesma
determinação, mas em um contexto diverso. 

O que é de notável importância é que a partir deste curso surge
na filosofia heideggeriana um caráter de “ateísmo de princípio”,
o qual só é detectado no modo pelo qual Heidegger coloca as
questões metodológicas. O fato de que a tradição teológica se
volte para a filosofia grega não determina, portanto,
simplesmente a sua perda de originalidade. É, antes, do
mal"-entendido da metafísica grega que se constrói o atual
sistema dogmático, que se reflete também na história da
filosofia. Não é simplesmente a metafísica que começa a fazer
parte do cristianismo, mas é uma interpretação errônea dela que
constitui os fundamentos do mesmo; por isso, a passagem de
Heidegger para além das considerações de tipo religioso não deve
ser considerada como uma superação, mas como um passo atrás que
permita indagar sobre questões preliminares (a tradição
metafísica) com as quais se pode chegar à verdadeira redefinição
da teologia. Com a destruição da metafísica, Heidegger se propõe
obter novos pressupostos pelos quais seja possível definir
concretamente \emph{o ser aí do homem} (\emph{Dasein}) e a
sua reconsideração à luz do fenômeno religioso. Todavia,
Heidegger para durante a sua especulação em procura de
determinação existencial do ser, e não haverá nenhum retorno à
religião. Apesar do abandono das temáticas religiosas, Heidegger
não se define ateu, uma vez que ele não tomou nenhuma decisão
sobre o \emph{ser de} \emph{Deus} ou sobre seu
\emph{não ser}. Este remeter a questão, porém, fará com que
Heidegger permaneça longe da fé em Deus por toda a sua vida, em
uma atitude de indiferença que se expressará de modo peculiar no
interesse pelos gregos, e pela ideia de divindade que provém
deles. Por fim, concluo com a observação pertinente de
Costantino Esposito: “para considerar profundamente a presença
de Agostinho em Heidegger deve"-se chegar a compreender, na
impostação e nas possibilidades mesmas abertas pelo questionar
heideggeriano, que peso tenha a \emph{ausência} de Agostinho
nele” (\textsc{espósito}, 1993, p .259).  

\section{Bibliografia}

\begin{description}\labelsep0ex\parsep0ex
\newcommand{\tit}[1]{\item[\textnormal{\textsc{\MakeTextLowercase{#1}}}]}
\newcommand{\titidem}{\item[\line(1,0){25}]}
\tit{AGABEM}, G..1988. ``{La passion de la facticité}''.
\emph{In}: \textsc{aa}.\textsc{dd}. \emph{Questions ouvertes. Heidegger}.
Paris: Osiris, p. 63--84.

\tit{BARASH}, J.-A.1996. \emph{Le temps de la memoire. A propos de la
lecture heideggérienne de Saint Augustin}.
\emph{Transversalités}, 60: 103--112.

\tit{BRITO MARTINS}, M.1998. \emph{L'herméneutique originaire
d'Augustin en relation avec une ré-appropriation
heideggerienne}. \emph{Mediaeval}\emph{ia. Textos e Estudos}
13/\,14 (1998):17--477.

\tit{CAPELLE}, Ph..\emph{Philosophie et Thélogie dans La pensée de
Martin Heidegger}.Paris: Cerf, 2001.

\titidem, \emph{Finitude et mystére}.Paris:
Cerf, 2005

\tit{ESPOSITO}, C.1993. ``\emph{Quaestio mihi factus sum.}
Heidegger di fronte ad Agostino.'' \emph{In}\emph{:}
L. \textsc{alici} T \textsc{alii} (a cura di), \emph{Ripensare Agostino:
interiorità e intenzionalità}.Roma: Institutum Patristicum
“Augustinianum”, p. 229--259.

\tit{GREISCH}, J. \emph{L'Arbre de Vie et l'Arbre Du Savoir}.Paris:
Cerf, 2000.

\tit{HEIDEGGER}, M. \emph{Die Idee der Philosophie und das
Weltanschauungsproblem} (\textsc{kns} 1919)\emph{.} In: Zur Bestimmung
der Philosophie. Gesamtausgabe Band 56/\,57.Frankfurt a.M.:
Klostermann, 1987, 1--117.

\titidem, \emph{Einleitung in die
Phänomenologie der Religion} (\textsc{ws} 1920--1921). In: Die
Phänomenologie des relisiösen Lebens. Gesamtausgabe Band
60.Frankfurt a.M.: Klostermann, 1995, 1--156.

\titidem, \emph{Augustinus und
Neuplatonismus} (\textsc{ss} 1921). In Die Phänomenologie des relisiösen
Lebens. Gesamtausgabe Band 60.Frankfurt a.M.: Klostermann, 1995,
p. 159--298.

\titidem, \emph{Sein und Zeit}.Tübingen: M.
Niemeyer, 1927.

\tit{SASSI}, V. \emph{A questão acerca da origem e a apropriação
não objetivante da tradição no jovem Heidegger}. Tese de
Doutorado. Pontifícia Universidade Católica do Rio Grande do Sul
--- PUCRS, 2007.

\tit{ FLETEREN}, F.2005. \emph{Martin Heidegger's
interpretations of Saint Augustine : Sein und Zeit und
Ewigkeit}. Lewiston (\textsc{ny}): The Edwin Mellen Press, 2005.

\tit{PÖGGELER}, O., \emph{La pensée de Martin Heidegger}.Paris:
Aubier/\,Montainge, 1967
\end{description}

%%%%

\capitulo[Dionísio Pseudo Areopagita e a construção política
da estrutura hierárquica do mundo cristão]{Dionísio Pseudo Areopagita
e a construção política da estrutura hierárquica do\\ mundo cristão}{Cícero 
Cunha Bezerra}{\textsc{ufs}}
\markboth{Dionísio Pseudo Areopagita
e a construção política\ldots}{Cícero Cunha Bezerra}

\epigraph{Os sistemas antigos, na maior parte, foram filosofias da ordem. A alma,
a cidade visível e a sociedade dos inteligíveis deveriam reproduzir,
cada uma a sua maneira, a ordem representada no universo.}{\textsc{roques}, 1983, p.~35}

\section{Considerações iniciais}


A obra de Dionísio Pseudo Areopagita é tão misteriosa quanto o seu
autor. Misteriosa no sentido de que atraiu e influenciou autores que
vão desde o século \textsc{vi} d.C., período tido como o mais seguro para
situá-la,\footnote{ Sobre a datação da obra, ver os estudos de \textsc{ruh},
\emph{Storia della mística occidentale}, vol.~\textsc{i}, trad.~Michele
Fiorillo, Milano: Vita e Pensiero, 1995 p.~35; \textsc{saffrey}, ``{Un
lien objetif entre le Pseudo"-Denys et Proclus}'' Em: \emph{Recherches sur le
Néoplatonisme après Plotin}, Paris: J.~Vrin, 1990, pp. 226--234;
\textsc{saffrey}.~\emph{Nouveux liens objetivifs entre le Pseudo"-Denys et
Proclus}, \emph{ibid}, pp.~235--247; \textsc{corsini}.~``La questione
areopagitica. Contributi alla cronología dello Pseudo"-Dionigi'' Em:
\emph{Atti della Accademia delle Scienze di Torino, II. Classe di Scienze
Morali, Storiche e filologiche}, 93, 195801959, Torino 1959, pp. 5--75.
\textsc{bezerra},  \emph{Mística e Neoplatonismo em Dionísio Pseudo
Areopagita}, São Paulo: Paulus, 2009.} até o início da
modernidade, sem, no entanto, ter"-se nenhuma informação segura sobre a
sua autoridade apostólica, com exceção das várias citações presentes no
próprio texto que identificam o seu autor com um grego convertido por
Paulo de Tarso conforme a passagem de Atos 17.34. 

Embora tenhamos diversas traduções e comentários que justificaram,
por \emph{escolha deliberada ou por incompetência} (\textsc{ruh}, 1995, p.
36), o conteúdo do \emph{Corpus Areopagiticum} com os primeiros
escritos apostólicos, hoje sabemos tratar"-se de uma verdadeira síntese
entre a teologia cristã e o pensamento filosófico neoplatônico baseado
na estrutura hipostática plotiniana e, particularmente, na sua
“teologização” realizada por Proclo. Para esse nosso estudo, nos
contentaremos somente em expor, a partir das \emph{Enéadas} de
Plotino, a base triádica das hipóstases e suas consequências éticas e
políticas dentro da estrutura hierárquica dionisiana.

Sabemos que um dos aspectos centrais do neoplatonismo está na
relação entre a união (\emph{henosis}) com o uno como uma
consequência da compreensão da vida em seu aspecto total. W.
Beierwaltes no seu estudo sobre a \emph{Enéada} V 3 afirma: ``{a
contemplação é idêntica ao contemplado, e o Espírito é o pensado: o
Espírito pensa o seu ser como si mesmo. Nesta autorrealização pensante o
Espírito é ato puro e vida}'' (1995, p.~63). Esse desnivelamento segue uma
interpretação que, como o próprio Plotino expõe, remonta aos primórdios
da filosofia grega e sua busca de explicação do real a partir de um
princípio (\emph{arkhé}) capaz de justificar a geração infinita dos
seres ou a multiplicidade em relação ao uno que permanece sempre o
mesmo. Para Plotino, o \emph{Parmênides} de Platão é o texto que
melhor revela o sistema henológico em suas três naturezas: ``O
Parmênides de Platão é o mais exato; ele distingue o primeiro uno, ou o
uno em sentido próprio (\emph{kyrióteron hén}), o segundo
uno, que é uma unidade múltipla (\emph{hén pollá}), e o
terceiro que é unidade e multiplicidade (\emph{hén kaì pollá})''
(\emph{En}. V,1--8"-25).

Seguindo essa perspectiva, Proclo nas \emph{Dez questões sobre a
providência} recupera o sistema hipostático plotiniano para justificar
a existência de um “conhecimento” superior ao intelectual:

\begin{quote}
Além de toda forma de conhecimento está o saber da providência, superior
também ao intelecto e próprio do uno, em virtude do qual toda divindade
existe e, como se diz, se ocupa de toda coisa; tal realidade é um ato
anterior à inteligência. A inteligência, graças a este único uno
subsiste e conhece todas as coisas (\emph{Prov}.1, 4--5).
\end{quote}

Consequentemente, cabe à terceira hipóstase (a alma) a dupla tarefa
de ao contemplar o espírito (inteligível) fundar o mundo e com isso
permitir que a alma individual, do homem, ao contemplar a inteligência
e a si mesma alcance, nesse movimento de conhecimento e de
auto"-conhecimento, a iluminação que, em um primeiro momento, se dá pelo
pensamento discursivo e depois pela superação de toda multiplicidade,
inclusive conceitual. 

Chegamos, assim, ao núcleo do sistema neoplatônico, ou seja, o
dinamismo que funda o que W. Beierwaltes nomeou de \emph{identidade
dinâmica}. Temos na formulação procleana da tríada \emph{permanência}
(\emph{moné}), \emph{processão} (\emph{próodos}) e
\emph{conversão} (\emph{epistrophé}),  herança da expressão de uma
unidade diferenciada que permite uma compreensão da geração e definição
de tudo o que provém do uno como ato originário,\footnote{ Não podemos
esquecer que quando Plotino nomeia o uno de “ato puro” não o faz no
mesmo sentido aristotélico, mas compreendendo"-o como superior a toda
essência. Nesse sentido, teríamos que pensá-lo como \emph{àneuousía},
ou seja, como o que produz a partir do nada. Sobre a reinterpretação
das noções de ato e potência, como já indicamos, ver o esclarecedor 
trabalho de \textsc{pigler}, 2002, pp.~46--47; Ver também: \textsc{narbonne},
\emph{La métaphysique de Plotin}, Paris: Vrin, 2001, pp 13--17.}
preservando cada natureza individual. O movimento geracional pode ser
exemplificado na \emph{Enéada} V, 2, 5: ``A processão
(\emph{próeisin})\footnote{ As formas proposicionais “epi”, “pró”
apontam para uma “direção a ser seguida” e que, segundo Beierwaltes,
constitui a relação dinâmica e ativa de toda realidade. Cf.
\textsc{beierwaltes}.~\emph{Autoconoscenza ed esperienza dell’Unità,
Plotino, Enneade V3}, trad. Alessandro Trotta, Milano: Vita e Pensiero,
1995, p.~205.} se realiza do primeiro ao último; cada coisa
permanece em seu lugar próprio; e o engendrado ocupa um posto inferior
ao seu gerador.'' Como se pode notar nessa passagem, o movimento
geracional do uno ao múltiplo revela os dois movimentos iniciais, isto
é, de \emph{permanência} e de \emph{processão}. Tudo permanece em
si mesmo enquanto participa do que lhe antecede, mas, “decai” em função
da diferença estabelecida pela distinção entre o gerador e o gerado.
Uma outra passagem revela de modo bastante claro o que estamos aqui
expondo: ``{Se existe algo após o primeiro, é necessário ou bem
que provenha imediatamente dele, ou bem que se remonte a ele através
dos intermediários'' (\emph{En}. \textsc{iv},7,1). Com o termo
\emph{intermediários} Plotino refere"-se aos dois níveis posteriores
ao uno (\emph{deutéron kaì tríton}).

\begin{quote}
Se existe algo posterior ao primeiro, não será simples; será uma unidade
múltipla. De onde vem ela? Do primeiro, pois não será por casualidade,
do contrário, ele (o uno) não seria o princípio de todas as coisas.
\end{quote}

É importante perceber que, ao contrário da tradição cristã, a
geração não é um ato voluntário. E o mais importante é que abarca todos
os níveis da natureza. Desde os seres inanimados até os providos de
vontade e razão. Diz Plotino:
\begin{quote} 
Vemos que todas as coisas que alcançam sua perfeição engendram não se
contentando em permanecer em si mesmas, mas produzem outras coisas; não
somente os seres dotados de vontade, mas os que vegetam sem vontade ou
os seres inanimados que comunicam o que podem do seu ser.
\end{quote} 

\section{A \emph{epistrophé} como caminho e posse da
felicidade}

E como se daria o retorno? No movimento de regresso ao uno está o
fim (\emph{télos}) de todo esforço filosófico plotiniano. A filosofia
neoplatônica, em todas as suas vertentes, é uma filosofia do retorno à
unidade primeira. Em seus vários sentidos \emph{epistrophé} aponta
sempre para um desejo de regresso, no caso plotiniano, ao fundamento
originário de toda realidade. Em sendo assim, processão e conversão são
partes intrínsecas de uma relação na qual a geração infinita dos seres
é pensada sempre sob a ótica de uma \emph{dinamicidade} que mantém e
produz o real em suas múltiplas formas.

A imagem mais própria e que tem uma larga história dentro do
pensamento filosófico grego é a de um “círculo” ou “esfera” que revela,
metaforicamente, o movimento de \emph{transcendência"-imanência} de um
uno que é \emph{totalidade} e, ao mesmo tempo, \emph{parte}
fundante enquanto centro de toda circunferência. Quando dizemos que o
retorno é o fim do filosofar estamos ressaltando a tarefa própria do
ser humano que é, socraticamente, o conhecer"-se a si mesmo
(\emph{gnothi sautón}). 

Nesse ideal de autoconhecimento estaria a distinção ou a
radicalização levada a cabo por Plotino de uma visão de mundo
completamente voltada para o interior. Segundo esta perspectiva, não
haveria um sentido “político” no neoplatonismo já que a \emph{pátria}
almejada não seria mais a \emph{pólis} grega, mas a \emph{Ítaca
interior} tão bem representada na interpretação alegórica realizada por
Plotino da figura e retorno de Ulisses. Essa não seria uma
característica unicamente da filosofia plotiniana. Dominic O’Meara nos
chama atenção para o fato de que o \emph{médio platonismo} no século
\textsc{ii} d.C.~ser marcado por um lado pelo aspecto “escolástico” e conservador
das ideias platônicas tais como a separação entre o mundo
\emph{sensível }e \emph{inteligível} e, por outro, por um profundo
sentido \emph{literário e religioso} (O’Meara, 1975, p.~19). Como
consequência, pese a pluralidade de interpretações do texto platônico,
os neoplatônicos subordinaram a vida\emph{ prática} à vida
\emph{contemplativa}. 

Não que isso signifique uma separação radical entre teoria e
prática, já que a vida contemplativa, seguindo a estrutura da alegoria
da caverna de Platão, é fruto do esforço de compreensão intelectual da
realidade em seu aspecto total,\footnote{ Dominic O’Meara observa que
em Plotino temos uma relação hierárquica não dualista, como encontrada
no gnóstico, mas um sistema que preserva a continuidade do uno que se
dá na multiplicidade dos seres. Uma hierarquização, portanto, de
continuidade e dinamismo. Cf. O’Meara, 1975, p. 122--123.} mas,
sem dúvida, há uma mudança com relação ao papel ou finalidade da
filosofia na construção de uma vida feliz que passa, necessariamente,
pelas ideias de sabedoria e de sábio. Dada a complexidade da temática,
nos restringimos somente a evidenciar o aspecto autárquico que
caracteriza a felicidade de uma vida virtuosa. Na \emph{Enéada}
\textsc{i},4,20--25 após indagar sobre o bem próprio do homem, Plotino responde:
``Ele é seu próprio bem graças a vida perfeita que possui.''

A discussão gira, portanto, em torno da \emph{indiferença }do
sábio frente ao mundo. Se o bem que lhe é próprio, derivado de uma vida
contemplativa baseada no exercício das virtudes, basta para uma vida
feliz, o caminho de uma ação política externa parece ser relegada a um
nível inferior de atividade. Na distinção entre \emph{sensação}
(\emph{aísthesis}) e \emph{conhecimento }(\emph{gnosis}) subjaz a
herança aristotélica de diferenciação entre o que corresponde aos três
níveis de atividades anímicas: vegetativa, sensitiva e racional. A
felicidade (\emph{eudaimonía}), como fruto do exercício do
pensamento, passaria, necessariamente, pelo abandono de tudo
(\emph{aphele pánta}) em direção à união com o uno. Observa W.
Beierwaltes:

\begin{quote}
O voltar"-se à interioridade e ascese transformante no Espírito devem ser
intensas também como
\emph{aphairesis}: um
movimento de abstração no qual o pensamento se retrai sempre mais das
suas múltiplas relações,
\textit{abstrai} de si, para
tornar"-se livre para uma intensiva unidade, ativando e desenvolvendo o
potencial de unidade presente em si mesmo, o Espírito e o Uno que funda
e que lhe precede (1993, p.~51).
\end{quote}


  Na \emph{Enéada} \textsc{iii}, 8, 6--35 vemos Plotino referendar o que foi
dito acima: ``Não somente ele tende a se unificar e a se isolar
das coisas exteriores, mas volta"-se para si mesmo e tem em nele todas
as coisas''; a vida consiste em um contínuo exercício contemplativo de
uma \emph{vida originária} que é definida como intelecção primeira,
(\emph{nóesis}) a qual se segue uma \emph{vida segunda} que é
intelecção segunda (\emph{deutéra nóesis}) e uma \emph{terceira
vida} que é última intelecção (\emph{eskháte nóesis}) (\emph{En}.~%
\textsc{iii}, 8, 20). No fundo, regressamos ao sistema das hipóstases na qual o
uno é potência de tudo (\emph{dúnamis ton pánton}) e sem o qual
nenhuma vida pode existir.

Dissemos anteriormente que o caminho que conduz à felicidade
consiste no exercício das virtudes. É interessante observar que o
\emph{tratado} \textsc{i},2, 19 começa exatamente com uma associação curiosa
entre uma “fuga do mundo”, espaço dos males, e o \emph{assemelhar"-se
a deus}. Plotino cita Platão no \emph{Teeteto}\footnote{ Diz Sócrates
no Teeteto:``É certo, Teodoro. Porém não é possível eliminar os
males – forçoso é haver sempre o que se oponha ao bem – nem mudarem"-se
eles para o meio dos deuses. É inevitável circularem nesta região, pelo
meio da natureza perecível. Daqui nasce para  nós o dever de procurar
fugir quanto antes daqui para o alto''. \textsc{platão},
\emph{Teeteto}, 176a, trad. Carlos Alberto Nunes, Universidade
Federal do Pará, 1988, p.~49.} (176a) como ponto
de partida para a sua reflexão: ``Uma vez que os males residem aqui
e por necessidade circulam a região do mundo e uma vez que a alma deseja
fugir dos males, é necessário fugirmos daqui''. E em que consiste tal
“fuga”? A resposta é exatamente a platônica: \emph{assemelhar"-se a
deus} (\emph{homoiothenai the}\emph{o}).\footnote{ Michele Abbate,
baseando"-se no testemunho de Marino, discípulo e biógrafo \mbox{de Proclo,}
ressalta a influência que o \emph{Teeteto} desempenhou no pensamento,
não apenas de Proclo, mas de neoplatônicos do seu entorno que
interpretaram a “dinivização” platônica como expressão da máxima
espiritualização do homem. Cf. \textsc{abbate}, 2008, p.~5.} A questão que
se segue é: como se dá tal assimilação? Adentramos assim na análise das
virtudes cívicas e as superiores. As chamadas virtudes cívicas
(\emph{politikàs legoménas aretàs}) são as mesmas presentes na
\emph{República}: prudência, coragem, temperança e justiça. Essas
virtudes são tidas como “reguladoras” dos temperamentos e apetites e,
portanto, são louvadas como algo a ser perseguido pelo sábio, no
entanto, recorrendo uma vez mais à Platão, Plotino as define como
“purificadoras”, isto é, necessárias à união, sem que ocorra, no entanto,
por elas a assimilação a deus (\emph{En}.\textsc{i},2,3--10). 

A contemplação dos inteligíveis, entendida como ato da alma, é a
virtude que se revela como fim de todas as virtudes purificadoras. Dito
de outro modo, a contemplação é uma atividade orientada à inteligência.
Esse é o ponto para o qual convergem todas as virtudes. Talvez nenhuma
passagem tenha influenciado tanto uma visão “apolítica” do pensamento
de Plotino do que a passagem final da \emph{Enéada} \textsc{i},2,7 25: 

\begin{quote} 
Separando"-se o máximo possível e vivendo inteiramente, não a vida do
homem de bem que é a aquela estimada pela virtude cívica, mas
abandonando"-a e optando por outra, pela dos deuses. Pois ele busca
assemelhar"-se aos deuses, não aos homens de bem. O assemelhar"-se com os
homens de bens é uma semelhança de uma imagem com outra imagem, copias
de um modelo, enquanto que o outro é assemelhar"-se a um ser distinto
como a um modelo (\emph{parádeigma}).
\end{quote} 

\section{Dionísio Pseudo Areopagita: diferenciação e unificação
hierárquica}

Salvatore Scimè, no seu trabalho \emph{Studi sul
Neoplatonismo} (1953) já afirmava o que hoje sabemos muito claramente,
isto é, que a concepção do absoluto em Dionísio é herdeira direta de um
mundo que se estruturou a partir de fatores culturais e religiosos
graças à incorporação da metafísica ateniense ao espírito místico
oriental (p.~65).  Das filosofias presentes nas escolas de Atenas e
Roma, ao somarem"-se ao ambiente religioso cristão alexandrino, não se
poderia esperar algo mais do que um verdadeiro sincretismo típico de um
helenismo cristianizado. Nesse contexto, o pensamento dionisiano é o
melhor exemplo de construção de uma reflexão em que o estritamente
filosófico fundamenta o teológico sem, no entanto, comprometer o que
ambos têm de específicos. 

Na base comum das \emph{Enéadas} e do \emph{Corpus }dionisiano
está o Uno como absoluta transcendência a todos os seres. Esse
postulado, reinterpretado à luz do Deus \emph{absconditus}
veterotestamentário, foi o que permitiu não somente o diálogo com a
tradição metafísica platônica, mas possibilitou uma justificação
teórica capaz de explicar como Deus, tomado em seu aspecto de causa de
toda criação, permanece distante e, ao mesmo tempo, interno ao processo
geracional, já que o mesmo é, segundo o pensamento cristão, a garantia
de toda subsistência.

Dionísio comparte a estrutura das tríadas neoplatônica associando"-a
aos elementos cristãos que estruturam e garantem a ordem e, ao mesmo
tempo, a comunicação entre os dois níveis da realidade:
\emph{sensível} e \emph{inteligível.} Diz ele:

\begin{quote} 
Porque não é de fato possível que a nossa mente se eleve de maneira
puramente espiritual, imitando e contemplando as hierarquias celestes
sem ajuda de meios materiais que nos guiem como requer nossa natureza
(\emph{\textsc{hc}}, 1,§3, 121c).
\end{quote} 

Estamos assim, diante de um modo de pensar a realidade a partir da
inteligibilidade arquetípica e que encontra na própria estruturação da
natureza um caminho de ascensão às realidades inteligíveis. No fundo, é
uma resposta à pergunta sobre a natureza da participação, isto é, de
como é possível que as coisas múltiplas sejam fundadas e participem da
unidade originária, estando essa completamente desprovida de relações. 

\section{Hierarquia celeste: purificação, iluminação e aperfeiçoamento}

Dionísio encontra no sistema neoplatônico uma saída capaz de somar"-se ao
modelo cristão de comunidade na qual o processo de ascensão se dá como
um movimento, em primeiro lugar, hermenêutico do texto bíblico e,  em
segundo, como adesão e comunhão com o espírito divino que se revela no
próprio mundo visível.  Não podemos esquecer que as \emph{Sagradas
Escrituras} são, para Dionísio, uma \emph{forma poética} \mbox{do que} em si
mesmo é sem forma e sem figura (\emph{\textsc{ch}}, 2, 137 b).

\begin{quote} 
As belezas visíveis são imagens da beleza invisível, os cheiros
sensíveis são imagens da difusão inteligível, as luzes materiais são
imagens de uma imaterial efusão de luz, as sagradas disciplinas
discursivas são imagens da plenitude contemplativa da inteligência e os
graus das ordens terrenas simbolizam a ordem harmoniosa e divina [\ldots{}]
(\emph{Idem}, \textsc{i}, 3, 121d).
\end{quote} 

Como se pode observar, estamos diante de um modelo de participação em
que cada aspecto dos ritos possui uma correspondência espiritual com um
objetivo: auxiliar na contemplação. Bem ao modelo procleano, a
filosofia é \emph{mystagogia}, isto é, um processo de assimilação
pelos símbolos ao fundamento ininteligível que supera toda
representação, mas que, por um ato de bondade, se manifesta livremente
aos que buscam contemplar. Trata"-se da deificação, ou seja, de um
processo de elevação das imagens sensíveis às inteligências simples
representadas nas hierarquias celestes.

É importante sublinhar que, apesar da preocupação dionisiana de que os
mistérios revelados sob as formas das imagens (leões, águias, bois,
etc.) não sejam tomados em seus sentidos literais, há uma justificação
para o uso dessas: preservar os mistérios propiciando um nível de
compreensão compatível com cada indivíduo partícipe da contemplação. As
\emph{Escrituras} permitem, de maneira proporcional a cada natureza,
uma ascensão paulatina aos mistérios espirituais que implicam em uma
gradual passagem das formas sensíveis à verdade secreta da inteligência
que permanece obscura para os não iniciados. Dionísio cita uma passagem
de 1 Coríntios, 8,7 que diz:  ``mas nem todos têm a
ciência. Alguns, habituados, até há pouco, ao culto dos ídolos, comem a
carne dos sacrifícios como se fosse realmente oferecida aos ídolos, e
sua consciência, que é fraca, mancha"-se''.\footnote{ \textsc{biblia de
jerusalém}, São Paulo: Paulus, 2004.}

Há, de fato, um duplo aspecto quando Dionísio aborda a temática da
manifestação divina. O primeiro seria o que ele nomeia de
\emph{imagens adequadas} e o segundo, são nomeadas de \emph{figuras
dessemelhantes}. No primeiro caso temos algo bem próximo, uma vez mais,
da tradição neoplatônica\footnote{ Sobre as tríadas ver: \textsc{hadot}.~%
\emph{Porphyre et Victorinus}, Paris: Études Augustinennes, pp
213--344; \textsc{bezerra}.~\emph{Compreender Plotino e Proclo},
Petrópolis: Vozes, 2006.} que são as várias definições mediante
conceitos também encontrados nos textos bíblicos tais como: Razão,
Inteligência, Substância, Luz, Vida. Nesse caso, embora inadequados do
ponto de vista da natureza divina, posto que Deus não pode ser reduzido
a nenhum conceito ou ideia, são \emph{representações veneráveis} (\emph{\textsc{ch
}}3, 140c). As imagens ou figuras dessemelhantes são as já abordadas
anteriormente. 

Para um melhor entendimento do aspecto simbólico, teríamos que tratar
da linguagem em sua forma catafática (positiva) e apofática (negativa),
no entanto, dado o limite e objeto do tema aqui abordado, diríamos,
como resumo geral do que foi dito, que a linguagem é sempre antinômica,
isto é, expressa a paradoxal transcendência e imanência de Deus e sua
possibilidade de comunicação com as criaturas (\textsc{scazzoso}, 1976, p. 195).
Pontuemos melhor a estrutura hierárquica celeste. Em que consiste?
Quais suas mediações e funções? 

Dionísio define hierarquia como: ``uma ordem sagrada
(\emph{táxis hierá}), uma ciência e uma operação que se
conforma, na medida do possível, ao Divino, e que participa
proporcionalmente segundo as iluminações comunicadas por Deus mesmo''
(\emph{\textsc{ch}} \textsc{iii}, 164d). O fim da hierarquia é, portanto, a assimilação
pela via anagógica a Deus. Mais do que uma estrutura fixa, a hierarquia
possui o aspecto de uma “disposição” sagrada que expressa, em última
instância, a beleza divina que opera em todos os níveis do real. R.
Roques observa que o termo \emph{táxis} possui dois sentidos bem
específicos: militar (disposição da tropa) e constituição política de
uma cidade ou Estado. Esse último sentido foi absorvido pela Igreja
significando uma sociedade organizada por Deus e para Deus (1983,
p.~36). 

 Uma ideia importante e que está na base do modelo dionisiano é o de
``colaborador de Deus'', ou seja, mediante a atividade contemplativa o homem alcança
progressivamente os estágios de purificação, iluminação e
aperfeiçoamento (\emph{\textsc{ch}} \textsc{iii}, 2, 165c). Nesses três movimentos
estão três virtudes: pureza, poder contemplativo e sabedoria perfectiva
que permitem ao homem colocar"-se em uma situação de perfeita cooperação
divina. Não é por casualidade que o sábio será pensado por Dionísio
como \emph{cosmopolita}, ou seja, como cidadão de todo
\emph{cosmo}. 

Seguindo o modelo neoplatônico de uma processão contínua do uno ao
múltiplo interconectada pela ideia de uma providencia que emana do
princípio theárquico e que atrai para si todas as coisas, Dionísio
propõe uma participação na divindade \emph{hipersubstancial} mediante
a comunhão dos elementos que lhes são comuns. Em sendo assim, os seres
vivos participam de um ser e de uma vida que lhes transcendem. Do mesmo
modo, a racionalidade é fruto de uma participação na sabedoria superior
a toda razão, mas que, enquanto tal, é o que fundamenta toda
inteligibilidade. Estamos tratando de uma concepção de cosmo que tem
sua origem na grande tradição de comentadores do \emph{Timeu} de
Platão. O postulado de uma harmonia universal sob o signo da vida e da
razão permeia o neoplatonismo, assim como, o estoicismo, duas correntes
que penetraram de forma decisiva na cosmologia cristã. 

Temos, portanto, o seguinte esquema participativo:

\renewcommand{\caption}[1]{{\centering#1}\par}

\begin{table}[h]
\caption{Deus (unidade \emph{trihipostática})}
\begin{tabular}{lll}
(a) & Tronos, Querubins e Serafins & Primeira tríada\\\hline
(b) & Dominações, Potências e Poderes & Segunda tríada\\\hline
(c) & Principados, Arcanjos e Anjos & Terceira tríada 
\end{tabular}
\end{table}



A primeira tríada, sem intermediação, permanece ligada diretamente a 
Deus. Seus nomes em hebraico são evocados como sinônimos de suas
“funções”: Serafins, como aqueles que “ardem”, são símbolos da efusão de
sabedoria; Querubins, como atitudes de conhecer e contemplar, são símbolos
da beleza theárquica e, finalmente, os Tronos altíssimos que são
símbolos das alturas e estabilidade elevada acima de toda baixeza. É a
expressão da \emph{imutabilidade do amor divino} (\textsc{\emph{ch} vii}, 2, 208b). 

À segunda tríada correspondem os seguintes sentidos: elevação não servil
(Dominações) e livre de todo desejo tirânico; indicação de uma força
inerente a toda atividade divina (Potências) e uma disposição bem
ordenada (Poderes) que permitem a conexão entre as tríadas da primeira
e da terceira ordens. A última disposição triádica, também chamada de
hierarquia angélica, significa a força divina de comandar e guiar
(Principados). Aos Arcanjos cabe a tarefa de mediação entre os
Principados e os Anjos que, por sua vez, fazem a intermediação com o
mundo sensível. Como se constata há um perfeito ordenamento em que as
primeiras tríadas regem as segundas e essas às terceiras. E como se dá
a conexão com a realidade sensível? Chegamos assim, a segunda
hierarquia: Eclesiástica.

\section{Hierarquia eclesiástica}

Seguindo a lógica de que os inferiores seguem os superiores, isto é,
que é necessário ascender do nível \emph{baixo}, pelo
\emph{mediano} até o \emph{alto,} Dionísio associa a hierarquia
eclesiástica à hierarquia celeste. Com isso temos um elo entre o
inteligível e o sensível capaz de explicar a participação na natureza
divina e fundamentar uma visão de uma comunidade marcada pela
\emph{deificação da vida, dos hábitos e das disposições}
(\emph{\textsc{eh}}, 1, 372b).

Ao contrário da hierarquia celeste, a hierarquia humana ou
eclesiástica, tem como marca a multiplicidade de símbolos que conduzem,
na medida da capacidade de cada um, à unidade:

\begin{quote} 
Esse é, pois, o comum fim de toda hierarquia, o amor contínuo de Deus e
dos mistérios divinos que produzem, santamente em nós, a presença
unificante de Deus.  Mas para isso é necessário o abandono de tudo o
que é obstáculo, contemplar e compreender a santa verdade, participar,
na medida do possível, graças a uma perfeita e deificante união,
daquele que é unidade mesma [\ldots{}] (\emph{\textsc{eh}}, \textsc{i},3 376a)
\end{quote} 

Como dissemos anteriormente, embora Dionísio não discuta a natureza do
ato criador em seu aspecto volitivo, há um evidente acento, no que
concerne ao movimento expansivo da criação, no aspecto de “doação” a
toda substancia racional e intelectual, de um amor que é expressão do
poder criador, mas também, de retorno do homem a sua origem primeira.
Só o amor associado à observância às santas prescrições, é que
permitirá a \emph{re"-união}: 

\begin{quote} 
O primeiro movimento, no plano intelectual, para coisas divinas é o
amor de Deus, mas a primeira condição para que o sagrado amor possa
santamente manifestar os divinos mandamentos é a nossa inefabilíssima
operação de deificação. Ser deificado é fazer nascer Deus em si (\emph{\textsc{eh}},
\textsc{ii}, 392b).
\end{quote} 

\emph{Fazer nascer Deus} ou tornar"-se divino é realizar, em vida, o
projeto moral"-unificador que perfaz todo o sistema hierárquico
dionisiano. A passagem acima citada é uma referência ao rito batismal
como símbolo do nascimento divino no homem e que, segundo o que
Dionísio nomeia de \emph{santa tradição,} guarda em si um significado
que vai além do simples ritual uma vez que implica uma decisão e desejo
em assumir e conduzir a vida em função dos princípios divinos. Não
trataremos aqui dos significados atribuídos aos sacramentos,
interessa"-nos somente pensar a estrutura política derivada do modelo
hierárquico que assumiu sua forma concreta na Igreja. Recorrendo uma
vez mais à tradição triádica, Dionísio coloca a hierarquia eclesiástica
entre duas outras: a celeste e a legal. A celeste, como vimos,
corresponde às realidades inteligíveis que permanecem diretamente
ligadas a Deus, a legal, tem em Moisés sua forma originária: a lei. No
entanto, a eclesiástica, sendo \emph{celeste e legal} (\emph{\textsc{eh}},
\textsc{v}, 501c), assume o estatuto de mediação entre os dois extremos.

A distinção entre imagem e verdade que encontramos na passagem da
\emph{Hierarquia Eclesiástica} \textsc{iii},1,428 é mais uma alusão ao
processo anagógico e revela, fundamentalmente, a raiz simbólica que
norteia a relação entre o homem e Deus graças aos elementos
neoplatônicos configurados na estrutura da realidade em Dionísio.
Baseada nos princípios de purificação, iluminação e perfeição, a Igreja
assume a imagem de uma grande \emph{ordem} que teria como finalidade
a promoção do ideal platônico de sabedoria somada à virtude. 

É importante observar que Dionísio, longe de postular uma mera adequação
do modelo pagão na forma de institucionalização de uma comunidade na
Igreja, reinterpreta e o projeto sob o aspecto da transcendência.
Ressalta Michele Schiavone: ``É inegável a herança da tradição
filosófica grega, mas se trata de uma exegese vivificada por um
espírito novo da mensagem cristã''(1963, p. 233). Que espírito é esse
se não o de construção, na terra, do reino dos céus? Um reino marcado
pelo amor e pelo conhecimento que, no caso de Dionísio, significa a
junção da filosofia e da teologia na direção do bem supremo. A
hierarquia eclesiástica fecharia, assim, a estrutura das processões:


\begin{table}[h]
\caption{Thearquia (\emph{Hierarquia celeste})}
\begin{tabular}{lll}
(d) &Tronos, Querubins e Serafins & Primeira tríada\\\hline 
(e) &Dominações, Potências e Poderes & Segunda tríada \\\hline
(f) &Principados, Arcanjos e Anjos & Terceira tríada
\end{tabular}
\end{table}


\begin{table}[h]
\caption{(\emph{Hierarquia eclesiástica})}

\begin{tabular}{lll}
(g) &Bispos, Padres e Ministros & Primeira tríada dos iniciadores\\\hline
(h) &Monges, Povo santo e Penitentes & Segunda tríada dos iniciados
\end{tabular}
\end{table}

A dignidade ontológica de cada participante das tríadas inteligíveis
(\emph{tríades noetaí}) é definida a partir da sua aproximação
contemplativa com o Uno originário. Em sendo assim, não estamos diante
de um “posto” definido por valores externos, mas toda a ordem se
sustenta numa conformação radical da natureza de cada ser com aquilo
que lhe compete. Em outras palavras, estamos diante da definição da
justiça como realização própria de cada \emph{phýsis} naquilo que lhe
é próprio. O sistema hierárquico dionisiano influenciou profundamente o
medievo cristão ao propiciar uma visão da Igreja como uma forma
perfeita de sociedade com inúmeras consequências. O fundamental a ser
observado no sistema eclesiástico proposto por Dionísio são os valores
que norteiam e garantem o lugar de cada membro como parte constitutiva
da harmoniosa ordem universal divina. Se o sistema, como um todo, é
questionável enquanto uma proposta autônoma e explicitam de ética (idem,
p. 2) o seu aspecto moral e o seu senso de justiça, baseado no amor
caritativo, o alinha perfeitamente a um projeto de comunidade na qual
cada indivíduo é responsável pelo grau de crescimento espiritual dos
seus membros e, consequentemente, pela ordenação da comunidade frente à
desordem (\emph{ataxía}). A cada membro corresponde uma função:
bispos (perfeição), sacerdotes (iluminação) e ministros (purificação).
Os sacramentos e ritos são ferramentas estético"-alegóricas que ajudam
na condução de uma vida regrada por um lado, pelo conhecimento
paulatino, na medida das potencialidades de cada um, das verdades que
iluminam, purificam e aperfeiçoam o caráter do homem.

\section{Considerações finais}

Após essa breve exposição das estruturas inteligíveis e
sensíveis das hierarquias dionisianas, cumpre indagarmos, de uma
maneira mais pontual, sobre as possíveis implicações política delas
derivadas.  Dominic O’Meara é, dentre os comentadores da obra
dionisiana, quem melhor postula uma recuperação do pensamento político
no neoplatonismo.\footnote{ \textsc{o’meara}.~``Vie politique et divinisation
dans la philosophie néoplatonicienne, dans Chercheurs de sagesse''.
\textsc{goulet"-cazé} (org), Paris: 1992, p. 501--510.} No seu artigo
\emph{évêques et philosophes"-rois: philosophie politique
néoplatonicienne chez le Pseudo"-Denys} (1997, 75--88) encontramos a
seguinte questão: ``em que medida pode"-se considerar a
eclesiologia de Pseudo"-Dionísio como a transposição cristã de uma
filosofia política neoplatônica?'' A pista tomada para uma resposta a
essa questão está na visível aproximação com o modelo de cidade
proposto por Platão na \emph{República }e, principalmente, a
interpretação neoplatônica da já citada passagem do \emph{Teeteto}
176b na qual se evidencia o ideal de assimilação ao divino mediante a
ascensão gradual e possível \emph{(homoiosis theo kata dynaton})
(Idem, p.~76).\footnote{ René Roques observa que, em Platão, a natureza
comanda a função e a função ao adequar"-se a natureza permite o
surgimento da justiça na cidade ideal. Dionísio, ainda que postulado em
um nível sobrenatural, fundamenta sua “cidade” nos mesmos princípio.
Cf. \textsc{roques}, 1983, p.~90.}

O papel que as virtudes cívicas desempenham no pensamento platônico
como exercício da alma é imprescindível para uma compreensão política
do pensamento dionisiano. Uma vez mais recorrendo a O’Meara, lembramos
que é importante perceber o posto do Uno como fundamento último do
projeto platônico identificado com o Bem que, no caso de Dionísio,
passa a ser associado ao próprio Deus (idem, p. 77). É evidente, na
obra dionisiana, o caráter de uma lei natural ordenadora de todos os
níveis da natureza. O termo utilizado por Diónisio é \emph{thesmós} e
aponta para o aspecto transcendente dos princípios reguladores.

Pese o rigorismo da estrutura hierárquica não podemos reduzi"-la a
uma ordem estática e sem dinamismo. As hierarquias são \emph{topoi}
nos quais as naturezas se adéquam em função das suas capacidades, isto
é, não se trata de uma imposição da lei, mas uma cooperação em que
todos, juntos, mantém uma perfeita unidade na multiplicidade do
mundo. Observa R. Roques: ``Dionísio confia à iluminação
hierárquica, que não é outra coisa que a transmissão gradual da luz
divina, a realização do velho ideal grego de \textit{epistéme} pela
\textit{theoría}, entendida como o fim último do esforço catártico e
gnoseológico exclusivamente humanos''(1983, p.~95).

Um aspecto fundamental que constitui o dinamismo hierárquico
dionisiano diz respeito à relação entre as partes integrantes dos
movimentos de processão (\emph{próodos}) e conversão
(\emph{epistrophé}). Em perfeito acordo com o pensamento de Proclo de
que todo ser procede de um princípio para o qual se converte
(\emph{\textsc{et}}, §31),\footnote{ \textsc{proclo}, \emph{Éléments de Théologie},
trad. Jean Trouillard, Paris: Aubier, 1965.} Dionísio postula,
quase que nos mesmo moldes, a ideia de uma atividade cíclica
(\emph{\textsc{et}} §33) em que cada ordem busca retornar, ou melhor, fazer
retornar as coisa à unidade. Nesse sentido, a dialética, em suas duas
faces (descendente e ascendente) constitui o pano de fundo que permite
uma aproximação do sistema dionisiano ao modelo platônico expresso no
livro \textsc{x} Da República platônica sob a forma da alegoria da caverna
(passo). 

Se como ressalta Casertano (2011, p. 31) a justiça, como tema
central na ordenação da sociedade, em Platão, implica em uma associação
ou ligação entre a perspectiva política e à do indivíduo, podemos
dizer, pese ao aspecto “espiritual” dado à questão, que Dionísio
preserva a motivação comum ao pensamento platônico de uma transformação
completa do homem como condição para a instauração da ordem justa. É
bem verdade que, o que é dialético em Platão no sentido do conhecimento
do \emph{logos }ou razão das coisas (idem, p. 105) é, para Dionísio,
compreensão hermenêutico"-simbólica das verdades reveladas nas
Escrituras. 

Compreensão que, como vimos, supera a simples adesão\emph{,}
característica da fé, posto que é um exercício constitutivo de
\emph{elevação espiritual do sensível para o inteligível e das
imagens sagradas e simbólicas aos cumes simples das hierarquias
celestes} (\emph{\textsc{ch}}, 125a). Nisso consiste uma vida moralmente de
acordo com a lei e, enquanto tal, ordenada em função de um projeto
político de organização da sociedade humana em virtude de um princípio
originário fonte de toda harmonia.



\section{Bibliografia}
\begin{description}\labelsep0ex\parsep0ex
\newcommand{\tit}[1]{\item[\textnormal{\textsc{\MakeTextLowercase{#1}}}]}
\newcommand{\titidem}{\item[\line(1,0){25}]}

\tit{abbate}, M. \emph{Il divino tra unità e molteplicità, Saggio sulla
Teologia Platonica di Proclo}, Milano: Edizioni dell’Orso, 2008.

\tit{andia}, Y. \emph{Henosis, l’union à Dieu chez Denys l’Aréopagite}, New
York:Leiden, 1996.

\tit{roques}, R. \emph{L’univers dionysien, structure hiérarchique du monde
selon le Pseudo"-Denys}, Paris: Éditions du \textsc{cerf}, 1983.

\tit{bezerra}, C.C. \emph{Dionísio Pseudo"-Areopagita, Mística e
Neoplatonismo}, São Paulo: Paulus, 2009.

\tit{brisson}, L\emph{. Leitura de Platão}, trad. Sonia Maria Maciel, Porto
Alegre: \textsc{edipucrs}, 2003.

\tit{casertano}, G. \emph{Uma introdução à República de Platão}, trad. Maria
das Graça G. de Pina, São Paulo: Paulus, 2011.

\tit{dionisio}, P.A. \emph{Tutte le opere}, traduzione Piero Scazzoso,
Milano: Rusconi, 1997.

\titidem. \emph{Œuvres complètes du Pseudo"-Denys
L’Aréopagite}, traduction par Maurice de Gandillac, Paris: Aubier,
1943.

\titidem. \emph{Corpus Dionysiacum}: Berlin: De
Gruyter, 1990/\,1991 in: Thesaurus Linguae Graecae, \textsc{CDR}om,
Univ.California, Invine, 1999.

\tit{o’meara}, D. \emph{Éveques et philosophes"-rois: philosophie politique
néoplatonicienne chez le Pseudo"-Denys} in: Denys l’Aréopagite et as
postérité en Orient et en Occident, Actes du Colloque Internacional,
Ysaber de Andia (org.), Paris: Institut d’Études Augustiniennes, 1997,
p.~75--88

\tit{proclo}, De decem dubitationibus circa Providentiam in: Tria Opuscula,
Traduzione di Francesco D. Paparella, texto Greco a cura di Alberto
Bellanti, Milano: Vita e Pensiero, 2004.

\titidem. \emph{Éléments de Théologie}, trad. Jean Trouillard,
Paris: Aubier, 1965.

\tit{ruh}, K, \emph{Storia della mistica Occidentale}, v. \textsc{i}, traduzione di
Michele Fiorillo, Milano: Vita e Pensiero, 1995.

\tit{saffrey}, H.D. \emph{Recherches sur le Néoplatonisme après Plotin},
Paris: J.Vrin, 1990.

\tit{scazzoso}, P. \emph{La teología antinómica dello Pseudo"-Dionigi},
Milano: \textsc{aevum}, fasc. \textsc{iii}, \textsc{iv}. 1976.

\titidem. \emph{Ricerche} \emph{sulla struttura del
linguaggio dello Pseudo"-Dionigi Areopagita}, Milano: Vita e Pensiero,
1967.

\tit{schiavone}, M. \emph{Neoplatonismo e Cristianesimo nello Pseudo
Dionigi}, Milano: Marzorati, 1963

\tit{scime}, S. \emph{Studi sul Neoplatonismo}, Messina: Educare, 1953.

\tit{trouillard}, J. \emph{La mystagogie de Proclos}, Paris: Les Belles
Lettres, 1982.
\end{description}
%%%%

\capitulo[A idealização amorosa na obra de Guimarães Rosa]{A idealização
  amorosa na obra\\ de Guimarães Rosa}%
	{Jacqueline Ramos}{ufs/\,ita}

\markboth{A idealização amorosa na obra de Guimarães Rosa}{Jacqueline Ramos}

\epigraph{“O amor tenteia de vereda em vereda, de serra em
serra\ldots{}”}{\textsc{Rosa}, \emph{Corpo de baile}, 1969: 234}

O amor também \emph{tenteia} de estória em estória, de livro a livro na obra
de Guimarães Rosa. Apesar de cambiante, a representação do amor
mantém uma característica comum que é a idealização. Há, digamos, uma
variação de tonalidades nas representações da idealização amorosa,
resultante de modos diversificados no manejo do foco narrativo. Essa
alternância de perspectivas sobre o mesmo tema parece estar vinculada
a um debate maior que o autor trava com as ideias filosóficas.

Leitor contumaz, a vasta erudição de Rosa foi construída sob estudo
sistemático,\footnote{ O espólio literário do autor, sob a guarda do
Instituto de Estudos Brasileiros --- \textsc{ieb}/\textsc{usp}, conta com inúmeros
cadernos de anotações e estudos, os volumes de sua biblioteca são
todos anotados segundo um código de notas de leitura desenvolvido
pelo autor, além dos manuscritos das obras, da correspondência com
amigos, tradutores, diplomatas, escritores, familiares etc. Por ser
material que desvela preocupações do autor, ou tomada de posição
frente a determinados temas e problemas, tem se constituído fonte
inestimável tanto para os estudos que se ocupam da intertextualidade
promovida na obra rosiana, quanto para aqueles de crítica
genética.} e a filosofia estava entre suas predileções. Não é
sem razão que os estudos de seu espólio literário constituem muitas
vezes ponto de partida para aqueles que se ocupam do diálogo de Rosa
com as ideias filosóficas.

É o caso de Sperber (1976) que coteja as passagens de vários
pensadores da biblioteca"-espólio de Rosa, passagens assinaladas pelo
autor e muitas vezes comentadas na margem do exemplar, identificando
a incorporação dessas ideias em sua obra ficcional. Sperber recorta
para esse seu estudo o esoterismo paulista, a \emph{Bíblia} e os
evangelhos, Upanishad, Platão, Sertillanges, Romano Guardini e
Plotino. Para Sperber, “os principais conceitos platônicos
assinalados por Rosa, aparentemente, referem"-se ao mito da caverna,
ao conceito do amor que, decaído, perde suas asas e à crença na alma
antes do nascimento e depois da morte” (1976: 65). A estudiosa passa,
então, a identificar nas várias estórias de \emph{Corpo de baile}
essa visão das essências e do primordial, que, em “Cara de Bronze”,
estória em que mais se detém, seria a própria poesia a que o
protagonista se lança em demanda. Com relação à Plotino, Sperber nota
em \emph{Corpo de baile} o aproveitamento da interpretação mítica
da natureza que promoveria a “solidariedade mágica”. Também o tema do
mal, que nos textos plotínicos assinalados por Rosa é estranho à
alma, ao indivíduo e à realidade, pertencendo ao não ser; seria
retomado em \emph{Grande sertão: veredas}:

\begin{quote}
O senhor não vê? O que não é Deus, é estado do demônio. Deus existe
mesmo quando não há. Mas o demônio não precisava de existir para não
haver --- a gente sabendo que ele não existe, aí é que ele toma conta
de tudo. O inferno é um sem"-fim que nem não se pode ver. Mas a gente
quer Céu é porque quer um fim: mas um fim com depois dele a gente
tudo vendo (1986: 59).
\end{quote}

Em suas análises, Sperber percebe não uma aderência ao texto e ideias
de Plotino ou de outros pensadores, mas seleção de \emph{topoi}:
“Guimarães Rosa, porém, pôde isolar partes que lhe parecessem
especialmente sugestivas para a imaginação criadora, para a
constituição de motivos, para a articulação dos signos e sintagmas
(portanto para a organização da micro e da macroestrutura do texto),
sem que estas precisassem, para tanto, estar inseridas em seu
contexto filosófico. Homogeneidade e leis de estruturação deveriam
existir na obra criada, sem submissão ao texto de origem” (1976:
105). As ideias filosóficas, segundo as conclusões da estudiosa,
seriam assim tomadas enquanto “matéria prima” para a criação
literária.

Outro estudo que conclui igualmente pela incorporação eclética de
fontes filosóficas na obra rosiana é o de Francis Utéza,
\emph{Metafísica do grande sertão}. Trata"-se de um estudo que parte
igualmente da biblioteca"-espólio do escritor, acrescida de
declarações, confidências, principalmente de cartas. Francis Utéza
identifica no \emph{Grande sertão: veredas} os elementos tanto da
tradição hermético"-alquímica, indicando o princípio “solve/\,coagula”
enquanto procedimento artístico e identificando, entremeados à obra,
símbolos e arquétipos próprios dessa tradição. Sua análise crítica
assinala ainda a presença do taoísmo e zen budismo, dando a ver a
amplitude das tradições que entram em jogo na poética rosiana.

De um modo geral, os estudos críticos que exploram o diálogo que Rosa
propõe com as ideias filosóficas insistem nessa incorporação eclética
de tradições que apareceriam condensadas em sua obra (será também a
posição de Benedito Nunes, como veremos adiante). Todavia, esse
aspecto merece melhor exame, se considerarmos a \emph{função} que
tais citações exercem na composição de cada livro. Ou seja, como os
temas e referências às ideias filosóficas se articulam no texto? Em
\emph{Corpo de baile}, por exemplo, temos epígrafes de Plotino,
enquanto \emph{Tutameia} é emoldurada com epígrafes de
Schopenhauer. Em seu uso padrão, a função da epígrafe seria a de
apresentar, digamos, a “ideia"-síntese” do que o texto irá
desenvolver; se assim é, não deveríamos entender aquelas epígrafes, e
seus autores, então, como diretrizes norteadoras das obras? Em
confidência a Paulo Dantas, Rosa teria dito que “não foi à toa
aquelas epígrafes de Plotino ou de Ruysbroeck, o Admirável, para o
meu \emph{Corpo de baile}. São um \emph{complemento} de minha
obra” (\textsc{Dantas}, 1975: 26, grifo nosso). Entendidas enquanto
“complemento”, as epígrafes não assinalariam, por conseguinte, o
\emph{interlocutor} do diálogo proposto pela obra? Aquele com a
qual ela dialoga, propondo, portanto, uma chave de leitura? É o que
pretendemos examinar em nossa comparação da representação do amor em
algumas narrativas rosianas.

\section{A idealização do amor}

Embora pouco estudada, há em geral, por parte da crítica, um consenso
acerca da presença de certa concepção metafísica na obra de Guimarães
Rosa.\footnote{ Estamos, a exemplo de Utéza (1994), utilizando o
termo “metafísica” em sentido largo: referindo"-se a toda e qualquer
tradição que conceba um outro plano, absoluto e imutável, com o qual
a finitude da vida sensível entraria em relação.}  Longe de
tentar definir a metafísica da obra rosiana, nossa preocupação
centra"-se na função que exerce na obra: aquele universo ficcional
aparece regido por algum plano do inteligível, do Absoluto, de Deus?
Ou a ideia de transcendência surgiria enquanto constructo cultural?
Ou, ainda, simplesmente tema nas preocupações dos personagens? Nossa
hipótese é a de que não haveria propriamente uma \emph{defesa} da
dimensão metafísica do mundo na obra rosiana, mas um debate sobre seu
funcionamento. Questão a que pretendemos nos aproximar confrontando a
idealização amorosa de algumas estórias. Quando comparamos a
configuração do amor nas diversas narrativas, percebe"-se não uma
reiteração, mas perspectivas diversas sobre o mesmo tema. Como um
movimento de tomada de cena, inicialmente realista em
\emph{Sagarana},\footnote{ \emph{Sagarana} foi a primeira obra
publicada por Rosa, de cunho mais realista. O amor é colocado na
dinâmica social, como nos mostra os conflitos: temos a estória do
valentão que vai visitar a noiva de Fulô; Maria Irma e seu ardil que
leva o primo a conquistar a amiga por estar interessada no noivo
dela; em “O Burrinho Pedrês” a vingança anunciada pelo vaqueiros que
perdeu a namorada para outro etc. Em \emph{Sagarana}, a mulher,
com raras exceções, é representada indiretamente: é referida pelas
personagens masculinas, não assume estatuto de personagem nas
estórias, não aparece em cena. A mulher, então, distanciada dos
leitores, mas presente naquele universo masculino, cumpre normalmente
a função de “estopim da intriga”. Não há propriamente uma idealização
amorosa, já que a perspectiva privilegia o universo das relações
sociais, muito embora já esteja dado o fascínio que as mulheres
representam aos homens, são elas que movimentam suas vidas. }
 o amor ganha dimensão idealizada em \emph{Grande sertão: veredas}
e \emph{Corpo de baile} e, em sua derradeira obra,
\emph{Tutameia}, o amor idealizado aparece como construção
linguística, representação do desejo.

\begin{center}
*\quad*\quad*
\end{center}

Em \emph{Grande sertão: veredas}, o narrador protagonista, Riobaldo,
conta suas andanças de jagunço pelo sertão, incluíndo suas
experiências amorosas. A primeira dessas experiências, considerando a
trama, é a dúbia paixão pelo amigo Diadorim, que se relaciona a um
estado de confusão e encantamento Veja"-se a fala de Riobaldo
referindo"-se a Diadorim: “Aquela mandante amizade. Eu não pensava em
adiação nenhuma, de pior propósito. Mas eu gostava dele, dia mais
dia, mais gostava. Diga o senhor: como um feitiço? Isso. Feito
coisa"-feita” (1986: 146--7). Encantamento que perdura por toda
narrativa, só será quebrado com a morte de Diadorim. Momento de
revelação para Riobaldo, pois descobre que o amigo, por quem sentiu e
reprimiu um confuso amor, era uma mulher: “Ela era. Tal que assim se
desencantava, num encanto tão terrível; e levantei mão para me benzer
--- mas com ela tapei foi um soluçar, e enxuguei as lágrimas maiores”
(1986: 585).

Outra espécie de amor será a da prostituta Nhorinhá, que exerce seu
ofício com arte, e que imprime em Riobaldo recordações voluptuosas do
amor sensível: “Se chamava Nhorinhá. Recebeu meu carinho no cetim do
pêlo --- alegria que foi, feito casamento, esponsal. Ah, mangaba boa,
só se colhe já caída no chão, de baixo\ldots{} Nhorinhá”. Amor simples e
natural, o prazer sensível que ela lhe proporcionou, cresce em suas
recordações até confundir"-se com o amor beatífico que Otacília lhe
inspirava.

Finalmente, Otacilia, a moça encontrada na fazenda Santa Catarina,
“forte como a paz”, representa a terceira espécie de amor vivenciado
por Riobaldo: “só vislumbrei graça de carinha e riso e boca, e os
compridos cabelos, num enquadro de janela, por o mal aceso de uma
lamparina. [\ldots{}] Otacília era a mais” (1986: 176--7). A imagem nos
lembra a princesa na torre do castelo e a parca adjetivação completa
a idealização: “graça”, “a mais”. A lembrança de Otacília exerce
sobre Riobaldo um efeito purificador, funcionando como contraponto: é
o sonho da vida fora das andanças de guerrear e pelejar.

Riobaldo conhece, então, três amores ou três paixões qualitativamente
diversas, mas que por vezes interpenetram"-se, dinâmica que Benedito
Nunes correlaciona ao \emph{eros} platônico de \emph{O Banquete}:
“diferentes formas ou estágios de um mesmo impulso erótico, que é
primitivo e caótico em Diadorim, sensual em Nhorinhá e espiritual em
Otacília” (1991: 145)

Enfim, em sua leitura do amor em \emph{Grande sertão: veredas},
Benedito Nunes aponta a atualização da idealização amorosa ou, em
outros termos, identifica a interlocução que Rosa propõe com as
ideias platônicas ao construir na obra a possível experiência dessa
concepção de amor. Apesar da ênfase atribuída a esse diálogo com o
\emph{eros} platônico, Nunes lembra que essa ascese erótica
“ajusta"-se também ao simbolismo da \emph{transubstanciação}
alquímica. O platonismo de Guimarães Rosa é inseparável da tradição
hermético"-alquímica”, conclui (1991: 150).

A idealização amorosa parece guardar vínculos mais estreitos, no caso
de \emph{Corpo de Baile}, com o pensamento de Plotino, autor de
suas epígrafes. Nessa obra, desdobrada em três volumes a partir da
terceira edição, há um verdadeiro jogo de epígrafes: algumas se
referem ao conjunto dos volumes; outras a cada um dos volumes
especificamente. Em sua análise de \emph{Corpo de baile}, Araujo
(1992) atenta também para as possíveis relações urdidas pelas
epígrafes na obra. Uma dessas relações diz respeito à disposição das
várias estórias e à simbologia que encerram, interpretadas por Araújo
a partir da seguinte epígrafe de Plotino:

\begin{quote}
O melhor, sem dúvida, é escutar Platão: é preciso --- diz ele --- que haja
no universo um sólido que seja resistente; é por isso que a Terra
está situada no centro, como uma ponte sobre o abismo; ela oferece um
solo firme a quem sobre ela caminha, os animais que estão em sua
superfície dela tiram necessariamente uma solidez semelhante à sua.
(\emph{Enéadas}, \textsc{ii}, i, 7)
\end{quote}

Segundo a pesquisa de Araújo, Plotino está se referindo ao
\emph{Timeu} de Platão, de acordo com o qual, em volta da Terra,
formando uma circunferência, movimentam"-se os planetas: o Sol, a Lua,
Marte, Vênus, Júpiter, Saturno e Mercúrio. Os movimentos dos
planetas, suas evoluções, são descritas no \emph{Timeu} como uma
“dança córica”, um bailado, um \emph{Corpo de baile.} A estória do
meio, em \textsc{cb}, a que está no centro, é “Recado do Morro”, que
“concentra em sua narrativa a ideia de repouso, solidez e permanência
da terra” (\textsc{Araujo}, 1992: 19), o centro em torno do qual ocorre a
“dança córica” das estórias, que alegorizam aqueles sete planetas: 

Resumindo, a análise de Araújo vai indicando a que planetas
correspoderiam cada uma das estórias: “Campo Geral” seria o Sol; “Uma
estória de amor”, corresponderia a Jupiter; “A estória de Lelio e
Lina”, a Marte; “Recado do Morro”, a Terra e a Mercúrio;
“Dão"-Lalalão”, a Vênus; “Cara"-de"-Bronze”, a Saturno; e, finalmente,
“Buriti”, a Lua.\footnote{ A primeira estória, “Campo Geral”, é
protagonizada pelo menino Miguilim e conta com a presença de seo
Aristeu, “uma das personificações de Apollo” dirá Rosa em carta a
Bizarri (1980: 21). Deus solar, da visão, da luz, do olhar, da
infância e do início da vida. “Buriti”, o último conto do livro,
narra a estória de D. Lalinha, “mulher que espera numa indefinição,
mulher reflexiva, cuja sexualidade só é despertada no decorrer de um
ritual noturno: deusa noturna, lunar, do escuro, dos ruídos e sons
[\ldots{}] Encontramos a Lua e a noite. Encontramos a idade adulta de
Miguilim, Miguel” (Araújo, 1992: 20).  }  Há outro nível de
correspondência estabelecida em \emph{Corpo de baile}, sobreposta
àquela, que seria a dos sentidos do corpo, tomados tanto positiva
(afirmação do corpóreo), quanto negativamente (âmbito do incorpóreo,
do espírito). O Sol e a Lua abrem o campo do visível e do invisível;
em “Uma estória de amor” reitera"-se o tema do audível e do inaudível;
o campo do odoroso e do inodoro em “A Estória de Lélio e Lina”. A
essas inter"-relações soma"-se ainda o movimento, \emph{Corpo de
baile}, conclui Araujo, “é um \emph{kosmos} ordenado, vivo, que se
movimenta segundo leis próprias --- é um cosmo racional
(\emph{Timeu}, 32, \textsc{cd}; 33, \textsc{ab}).” O segundo índice ao final do livro
que propõe uma releitura das narrativas em sentido inverso, do fim
para o começo, indicaria uma ida"-e"-volta, “este passar de um conto
para o outro, este entrecruzar"-se dos planetas/\,deuses, dos elementos
e sentidos corporais --- movimento que os liga, os transforma, os
combina” (1992: 24). Enfim, o trabalho de Araujo deixa entrever a
complexa coreografia de \emph{Corpo de baile}.

De um modo geral, com maior ou menor ênfase, as estórias reiteram o
tema da “travessia”, dos ritos de passagem, das transformações,
percursos que partem invariavelmente do âmbito sensível, associados
sempre à ideia de ascese. Isso se aplica também à idealização
amorosa, aliás, mote central de pelo menos duas estórias: “Uma
estória de amor” e “Lélio e Lina”. Se Otacília parece ocupar no
imaginário de Riobaldo aquele lugar absoluto e distante, o ideal que
se pretende alcançar; em “Lélio e Lina”, o sensível é que parece
engendrar a plenitude.

Trata"-se da estória de Lélio, um jovem vaqueiro, já destro no ofício,
mas inexperiente no amor. O percurso narrativo do herói será
justamente o da aquisição dessa experiência e a onisciência narrativa
deixa entrever os pensamentos obsessivos de Lélio, que gravitam
sempre em torno das relações amorosas. Lélio guarda consigo a imagem
da moça do Paracatu --- a “Mocinha”, a “Sinhá-Linda” --- responsável pela
construção no espírito do herói de um amor idealizado, distante,
impossível. A narrativa, no entanto, é construída de modo a
percebermos que a visão idealizada de Lélio não encontra
correspondência naquela realidade. Os outros vaqueiros não achavam a
moça tão bonita, “só espevitada e malcriadinha, gostando de se
sobressair” (1969: 138). Lélio fica inclusive feliz com esse juízo
sobre ela, pois assim

\begin{quote}
a beleza dela pudesse ficar para ele só, por nada e suspendida, que
mesmo assim o vencia pelos olhos. Porque, desde o momento, nessas
ocasiões, ele ouviu de si e se afirmou que, sôbre bonita, por algum
destino de encanto ela para ele havia de ser sempre linda no mundo,
um confim, uma saudade sem razão (1969: 138). 
\end{quote}

Contrapõe"-se a essa visão etérea de Sinhá-Linda o amor sexual das
“tias”, “mulher no simples, pra precisão da gente”, nas palavras de
Lélio, que manterá também um caso adúltero com Jini, por quem chega a
ficar perdidamente apaixonado, até perceber o “minguo amor, não
sentia que ele mesmo fosse para ela uma pessoa, mas só uma coisa
apreciada no momento, um pé de pau de que ela carecesse” (1969: 198).
Nesse âmbito do amor sensível, Lélio experimenta as duas posições, de
sujeito e de objeto: tanto desfruta das “tias”, quanto se sente
desfrutado pela Jini.

A princípio, parece que estamos diante da representação daquela mesma
ascese erótica que Nunes identifica em \emph{Grande sertão:
veredas}, se pensarmos nos correlatos amor natural, sexual
(Jini/\,Nhorinhá) e amor etéreo, ideal (Sinhá Linda/\,Otacília).
Entretanto, em \emph{Grande sertão: veredas}, ficamos restritos ao
modo como Riobaldo percebe o mundo; em \emph{Corpo de baile},
diversamente, a onisciência narrativa marca constantemente o
desencontro entre a visão de Lélio e a das demais personagens. Há
inclusive um episódio que chega a ser jocoso. Lélio estava voltando
de um daqueles domingos de comida, bebida, jogo e carinhos na casa
das “tias”, era uma longa caminhada de volta, ainda sob sol forte,
quando avista uma mulher colhendo lenha:

\begin{quote}
Vestida de claro, ali perto, de costas para ele, uma moça se curvava,
por pegar alguma coisa no chão. Uma mocinha. [\ldots{}] Era um estado ---
sem surpresa, sem repente --- durou como um rio vai passando. A gente
pode levar um bote de paz, transpassado de tranquilo por um firo de
raio. Lélio não se sentia, achou que estava ouvindo ainda um segredo
[\ldots{}] (1969: 179).
\end{quote}

Depois de dar a ver esse enlevo a que Lélio se entrega, o narrador
adverte: “Mas: era uma velhinha! Uma velha\ldots{} Uma senhora” (1969:
179). Exclamando inicialmente e repetindo mais duas vezes, o narrador
não deixa dúvida, parece que Lélio está alucinando ao,
quixotescamente, enxergar naquela realidade a projeção de seu desejo
do ideal. 

Essa velha, D. Rosalina, senhora experiente, cumpre em boa extensão do
enredo, o papel de confidente e conselheira de Lélio. Ela assinala,
inclusive, uma mudança de orientação ao condenar aquela ideia de
absoluto inalcançável que Sinhá-Linda representava para Lélio. Ao
aconselhá-lo sobre suas experiências amorosas, dirá D. Rosalina:

\begin{quote}
--- “Modo outro, meu Mocinho, eu vejo que isso é um madrastio que você
arranjou para si, nessa Mocinha de fantasma\ldots{}” Lélio não respondeu.
E ela foi dizendo: --- “Do que estou sabendo, por trás de você, pode
ser que essa moça nem seja boa, nem saúde verdadeira de mulher ela
não demonstra ter. Escuta: mulher que não é fêmea nos fogos do corpo,
essa é que não floresce de alma nos olhos, e é seca no coração\ldots{}”
(1969: 194)
\end{quote}

Em sua fala, a velha Rosalina desaprova em Lélio aquele amor descolado
do sensível e localiza na energia sexual primária a fonte geradora do
espiritual. O amor sensível não seria reflexo do absoluto, mas
condição necessária para sua realização: o corpóreo engendraria o
espiritual, já que sem os fogos do corpo não há o florescimento da
alma.

Tanto em \emph{Grande sertão: veredas} quanto em \emph{Corpo de
baile}, guardadas suas diferenças, temos a idealização amorosa
construída a partir de relações dicotômicas (corpo/\,alma;
corpóreo/\,incorpóreo; relativo/\,absoluto etc.) que regem o percurso do
herói. Em \emph{Grande sertão: veredas}, a escalada em direção às
essências; inversamente, em \emph{Corpo de baile}, não há o
afastamento do mundo empírico. Em \emph{Grande sertão: veredas}, a
idealização é apresentada pelo próprio Riobaldo; em \emph{Corpo de
baile}, o narrador onisciente apresenta o contraste de visões,
indicando a idealização como construção subjetiva, já que não é
consensual. No caso de \emph{Tutameia}, última obra publicada em
vida por Rosa, a idealização estará presente, mas será relativizada,
já que é representada como uma construção de linguagem.

Em “João Porém, o criador de perus”, o tímido e pacato herói que
intitula a estória prospera em seus negócios, causando inveja nos
moradores da vila que, certa feita, por brincadeira “inventaram, a
despautação, de espevitar o espírito”: uma certa moça “de
além"-cercanias” apaixonada por ele chamada Lindalice, de olhos azuis,
cabelo liso, vistosa.

\begin{quote}
João Porém ouviu, de sus brusco, firmes vezes; \emph{miúdo meditou}.
Precisava daquilo, para sua saudade sem saber de quê, causa para
ternura intacta. Amara"-a por fé --- diziam, lá eles. Ou o que mais,
porque amar não é verbo; é luz lembrada (1967: 75 --- grifo nosso).
\end{quote}

A mentira inventada pelos moradores desencadeia a ideia arquetípica do
amor, idealização que parece reeditar o mito da reminiscência:
“precisava daquilo, para sua saudade sem saber de quê”. Nesse
sentido, a ideia do amor se sobreporia à ação de amar: “o amar não é
verbo; é luz lembrada”. Note"-se, entretanto, que João “miúdo meditou”
para entender que “precisava daquilo”. Ora, o herói, então, não é
enganado pela mentira chistosa dos outros. Na verdade, se faz de
enganado para instituir seu ideal de amor. Quando instado a ir em
busca da moça:

\begin{quote}
\emph{Se bem pensou, melhor adiou}: aficado, com recopiada
paciência, de entre os perus, como um tutor de órfãos. Sustentava"-se
nisso, sem mecanismos no conformar"-se, feito uma porção de
não relógios. A moça, o amor? A esperança, talvez, sempre cabedora. A
vida é nunca e onde (1967: 75 --- grifo nosso).
\end{quote}

Mais uma vez a ação do herói é \emph{pensada}, ele parece consciente
de sua opção por sustentar"-se na ideia de um amor que seja só
essência etérea, atemporal (“não relógios”). Assim vai vivendo,
trabalhando e prosperando, mas os moradores, pasmos diante da crença
de João na inventada moça, e que queria sempre “antigas novidades
dela” (1967: 75), resolvem, então, por “dó ou cansaço” desfazer a
brincadeira dizendo que a moça havia morrido. Tal desfecho que
imaginam para desfazer o engano acaba servindo, em verdade, para
reafirmá-lo: João Porém “segurava"-se à falecida --- pré-anteperdida”,
intensifica"-se a idealização ao tornar totalmente inalcançável a
amada que, no entanto, ele sentia por perto: “Porém, Lindalice, ele a
persentia”. Fica, assim, fortalecida a idealização amorosa (“E
fechou"-se"-lhe a estrada em círculo”), e João Porém não se demove nem
mesmo diante de outra pretendente “mocinha, de lá, também olhos
azuis, lisos cabelos, bonita e esperta, igual à outra, a urdida e
consumida” (1967: 76). Também inútil essa derradeira tentativa,
desistem finalmente de João Porém, que vive seus dias e morre “imóvel
apaixonado” sem deixar herdeiros. Sentencia então o narrador: “Ele
fôra ali a mente mestra. Mas, com ele não aprendiam, nada. Ainda
repetiam só: --- ``\emph{Porém! Porém\ldots{}} Os perus, também” (1967:
76).

O percurso narrativo do herói inverte a brincadeira inicial: se João
foi a “mente mestra”, ele na verdade usou a brincadeira chistosa, a
mentira, a moça inventada, para instituir seu ideal de amor. É ele,
afinal, que engana os outros. O que inicialmente era uma brincadeira
maldosa, João Porém reverte a seu favor, utilizando"-a como forma de
criar a idealização amorosa e de vivenciá-la sob o testemunho da
comunidade.

De modo análogo, em “Desenredo”, outra estória de
\emph{Tutameia}, o amor idealizado é instituído socialmente através
do discurso. O protagonista da estória, Jó Joaquim, apaixona"-se por
uma mulher casada e consegue entender"-se com ela. O marido, por ser
valente e ciumento, representa o perigo que eles parecem subestimar.
Ocorre o imprevisto: “apanhara o marido a mulher: com outro, um
terceiro”, Jó Joaquim fica arrasado, pois “imaginara"-a jamais a ter o
pé em três estribos”. (1967: 38). A surpresa leva Jó Joaquim ao
recolhimento. Quando o marido morre, o herói e a mulher se
reencontram e se casam. Passa"-se o tempo e, certa feita, é Jó Joaquim
quem flagra a mulher em adultério e a expulsa, mas “suas lágrimas
corriam atrás dela, como formiguinhas brancas”. Trata"-se novamente de
uma personagem que vive a idealização amorosa: “Desejava ele, Jó
Joaquim, a felicidade”; “Ele queria apenas os arquétipos,
platonizava. Ela era um aroma” (1967: 39). Por desejar a felicidade,
transformado num inédito poeta, Jó Joaquim passa a operar o passado:

\begin{quote}
Nunca tivera ela amantes! Não um. Não dois. Disse"-se e dizia isso
Jó Joaquim. Reportava a lenda a embustes, falsas lérias escabrosas.
Cumpria"-lhe descaluniá-la, obrigava"-se por tudo. [\ldots{}]

O ponto está em que o soube, de tal arte: por antipesquisas,
acronologia miúda, conversinhas escudadas, remendados testemunhos. Jó
Joaquim, genial, operava o passado --- plástico e contraditório
rascunho. Criava nova, transformada realidade, mais alta. Mais certa?
(1967: 40).
\end{quote}

Por meio de inversões, Jó Joaquim chega a ser bem"-sucedido em sua
reconstrução da “realidade”, “todos já acreditavam. Jó Joaquim
primeiro que todos” (1967: 40). O passado reconstruído por ele ganha
o estatuto de existência quando a comunidade o aceita e, portanto, o
confirma. Assim como João Porém, Jó Joaquim constrói a possibilidade
arquetípica, pois a realidade construída e aceita como “real” pela
comunidade é na verdade idealizada. Mesmo a mulher adúltera e
reincidente parece ter sido persuadida, pois volta sem culpa:
“soube"-se nua e pura. Veio sem culpa” (1967: 40).

De modo análogo a João Porém, que fingiu"-se enganado para vivenciar a
idealização amorosa; também Jó Joaquim persuade a comunidade e
reconstrói o passado segundo seu desejo. Valeria lembrar que Jó
Joaquim transformara"-se em poeta, aquele que, por operar a linguagem,
é capaz de transformar a “realidade”. De modo geral, poderíamos dizer
que, em ambos os casos, língua e realidade se confundem. João Porém
quer acreditar e, para isso, faz acreditar que acredita; Jó Joaquim,
fazendo acreditar, reconstrói o passado segundo seu desejo. O
ficcional (o passado de Virília; Lindalice, a moça inventada) ganha
estatuto de “real” por meio da linguagem. A “realidade”, nesses
casos, aparece concebida como representação do desejo.

Lembremos que \emph{Tutameia} é encabeçada por epígrafes de
Schopenhauer, e a aproximação entre aquele mundo ficcional regido
pelo desejo e a tese de \emph{O mundo como vontade e representação}
parece evidente. O engano humano ao crer em interpretações
verossímeis do “real” é constantemente reiterado em
\emph{Tutameia}, enfatizando"-se a ideia de que vivemos nesse
universo construído pela linguagem. Uma linguagem que, por sua vez,
não aponta para os objetos do mundo, antes doa significados segundo
as possibilidades da língua e o desejo do enunciador.

\section{Considerações finais}

Nesta rápida e sumária aproximação da obra rosiana, procuramos indicar
suas correlações com a filosofia e em especial com o tema do amor
idealizado, cuja representação varia em função da perspectiva
adotada. Tal diversidade de perspectivas, por sua vez, parece
respaldada por concepções filosóficas específicas, se pensarmos na
hipótese da funcionalidade das epígrafes como chaves de leitura ou
diretrizes norteadoras das obras.

Se assim for, os \emph{topoi} do pensamento filosófico identificados
na obra rosiana deixariam de ser mera “matéria"-prima”, já que
funcionariam como linhas de força na composição textual. O que não
invalida, por outro lado, o ecletismo de tradições promovido na obra,
mesmo porque o princípio da condensação é largamente explorado por
Rosa na composição da alta densidade semântica de suas estórias.

Se o tema muda em função da perspectiva adotada, entramos no campo da
percepção, dos modos de perceber e representar o mundo, da relação do
homem com o mundo, do modo pelo qual ele lhe confere significação. Em
\emph{Grande sertão: veredas}, a idealização amorosa se afirma
enquanto princípio dinamizador da experiência amorosa do herói, mas
não podemos nos esquecer que esse é o ponto de vista de Riobaldo, o
narrador protagonista da narrativa. Em \emph{Corpo de baile}, a
narração em terceira pessoa revela o desencontro entre a visão
idealizada de Lélio e a realidade social da moça rica e arrogante,
que menospreza os vaqueiros. Lélio vê além ou aquém da realidade? Vê
o que os outros não conseguem enxergar ou enxerga o que quer ver? Em
\emph{Tutameia}, é o desejo que movimenta as estórias e que se
realiza na linguagem.

\epigraph{“E o resto já vinha. O senhor verá, pois”}
{\textsc{Rosa}, 1986: 457}

\section{Bibliografia}

\begin{description}\labelsep0ex\parsep0ex
\newcommand{\tit}[1]{\item[\textnormal{\textsc{\MakeTextLowercase{#1}}}]}
\newcommand{\titidem}{\item[\line(1,0){25}]}
\tit{ARAUJO}, Heloisa Vilhena. \emph{A raiz da alma.} São Paulo: Edusp,
1992.

\tit{DANTAS}, Paulo. \emph{Sagarana emotiva.} São Paulo: Duas Cidades,
1975.

\tit{NUNES}, Benedito. “O amor na obra de Guimarães Rosa” in
\textsc{coutinho}, A.
(org.). \emph{Fortuna crítica n. 6.} 2ª ed. Rio de Janeiro:
Civilização Brasileira, 1991.

\tit{ROSA}, Guimarães. \emph{Correspondência com seu tradutor italiano
Edoardo Bizzarri.} 2ª. ed. São Paulo: T. A. Queiros/ Instituto
Cultural Italo"-Brasileiro, 1980.

\titidem. \emph{Sagarana.} 31ª. ed. Rio de Janeiro: Nova
Fronteira, 1984.

\titidem. \emph{Grande sertão: veredas.} 20ª. ed. Rio de
Janeiro: Nova Fronteira, 1986.

\titidem. \textit{Corpo de Baile.} 4ª. ed. Rio de Janeiro: José
Olympio, 1969.

\titidem. \textit{Tutameia.} Rio de Janeiro: José Olympio, 1967.

\tit{SPERBER}, Suzi Frankl. \emph{Caos e cosmos: leituras de Guimarães
Rosa.} São Paulo: Duas Cidades, 1976.

\titidem. \emph{Guimarães Rosa: signo e sentimento.} São Paulo:
Ática, 1982.

\tit{UTÉZA}, Francis. \emph{Metafísica do grande sertão.} São Paulo:
Edusp, 1994.
\end{description}


\capitulo{Neoplatonismo e contemporaneidade --- sobre a presença de
	traços neoplatônicos na posição de Wittgenstein quanto ao estatuto da
	linguagem matemática}%
	{Eduardo Gomes de Siqueira}{ufrrj}

\markboth{Neoplatonismo e contemporaneidade}{Eduardo Gomes de Siqueira}


Inicio ressaltando minha \emph{ousadia} em procurar falar
aqui de duas coisas nas quais não sou especialista, a saber, o
Neoplatonismo e a Matemática --- que, no entanto, interessam
enquanto perspectivas de problematização geral do estatuto da
linguagem em sua relação com o conhecimento do mundo e da
linguagem matemática e seu tipo especial de certeza, em
particular. Outro aspecto de interesse levantado pelo tema são
as conexões possíveis entre o pensamento medieval e o pensamento
contemporâneo em que desenvolvimentos atuais sugerem a retomada
de linhas de pensamento que pareciam já ter sido modernamente
``superadas'', sem mais.

Em que medida podemos apontar uma
``resiliência neoplatônica'' ---
ou algo que pudesse ser assim aproximadamente chamado --- na
compreensão das posições contemporâneas em debate sobre o
estatuto da linguagem matemática? Kurt Gödel, que foi
caracterizado como “um platônico entre positivistas” por Rebecca
Goldstein (2008) --- caracterização ainda por ser devidamente
avaliada ---, teria posto fim ao “programa de Hilbert” de
\emph{formalização completa} da linguagem matemática que
visava a completa eliminação do apelo a \emph{intuições} de
sua axiomatização tradicional. Sem adentrar em aspectos formais
de suas provas, este artigo visa considerar a possibilidade da
presença de traços neoplatônicos na posição ainda bem pouco
compreendida de Wittgenstein em filosofia da matemática, e
especialmente em sua reação ao trabalho de Gödel, seu
conterrâneo austro"-húngaro.

Iniciamos indicando (\textsc{i}) o que entendemos por “traços
neoplatônicos”, (\textsc{ii}) alguns aspectos do impacto que o(s)
teorema(s) da incompletude de Gödel exerceram sobre o programa
formalista da matemática moderna, para assinalar (\textsc{iii}) algumas
marcas da reação de Wittgenstein a Gödel a partir de alguns
aspectos centrais de sua concepção de matemática, procurando
enfim caracterizar a presença de traços platônicos ou
neoplatônicos nas atitudes tanto do grande lógico"-matemático
como na do radical e obscuro filósofo"-da"-linguagem"-matemática.

\section{O que entendemos por <<traços
neoplatônicos>>}

Reconstruindo a posição de Philip Merlan  em \emph{Dal
platonismo al neoplatonismo} (1975) sobre o “rompimento ou
conciliação” entre platonismo e neoplatonismo (\textsc{bezerra},
2006:43--44), ressaltando a importância da tradição indireta (que
indica a presença de um neoplatonismo seis séculos antes de
Plotino), e supondo que o platonismo de Aristóteles era já um
neoplatonismo (\textsc{merlan}, 1975:10); o ponto a assinalar é que se o
neoplatonismo surgiu já na Academia platônica, é em torno do
\emph{significado epistêmico e ontológico} da
\emph{matemática} que a questão dessa continuidade é
formulada.

Nessa tradição, para Platão, entre o sensível e as Formas
estariam os \emph{entes matemáticos intermediários}
(caracterizando o realismo platônico), distintos tanto dos
entes sensíveis, por sua permanência, como dos seres
inteligíveis, por sua multiplicidade. O real se dividiria
portanto em três esferas: o inteligível, o matemático e o
sensível --- e a \emph{alma}, realidade intermediária, estaria
em perfeita consonância (potencial) com os entes matemáticos,
permitindo oferecer a seguinte caracterização, em sete pontos,
do neoplatonismo, ainda segundo Merlan:

\begin{enumerate}
\item as esferas do ser estão subordinadas em graus superiores e
inferiores (estes, os seres perceptíveis no espaço e tempo); 

\item há uma derivação das esferas inferiores das superiores,
derivação ocorrida ela mesma fora do espaço e tempo (fora de
qualquer relação causal ou de implicação lógica); 

\item a esfera superior deriva de um princípio primeiro superior a
todo ser; 

\item o princípio, chamado Uno, é simplicidade absoluta superior a
toda determinação; 

\item a multiplicidade das esferas inferiores implica aumento dos
entes e de suas determinações; 

\item o conhecimento do princípio primeiro não pode ser predicativo
(uma vez que ele não exibe determinações); e 

\item a dificuldade fundamental do neoplatonismo seria a de explicar
a passagem da unidade à multiplicidade.
\end{enumerate}

O problema dos entes matemáticos está na convicção desta
tradição neoplatônica (Xenócrates, Espeusipo, Heráclides por um
lado; Jâmblico, Proclo, Posidônio por outro) de sua
\emph{existência totalmente separada} (aspecto ontológico),
correlata ao seu \emph{modo específico de conhecimento}
(aspecto epistemológico). A alma como intermediário (entre o
divisível e o indivisível) estaria intrinsecamente relacionada
ao caráter intermédio dos entes matemáticos (permitindo a
ligação ou ``participação'' em
ambas esferas). Sua posição intermediária garantiria à alma (um
“número dotado de movimento”) o poder de participar tanto da
identidade (eternidade) como da mudança (alteridade), e assim os
entes matemáticos mediadores teriam de possuir mais que uma
existência ao nível do pensamento: possuiriam uma existência
\emph{rerum natura} (fundada na natureza das coisas mesmas).

O neoplatonismo seguiria assim, segundo Merlan, uma linha de
\emph{continuidade} desde a antiga Academia (raízes puramente
platônicas acrescidas de inovações e contribuições); a
\emph{descontinuidade} estaria marcada pela natureza
transcendente e numinosa do Uno neoplatônico (vinculado ao
êxtase e à assimilação com o divino) que introduz uma
“monopolaridade” contraposta à \emph{bipolaridade} típica do
platonismo (o Uno e a Díada no Neoplatonismo se convertem em
absoluto e derivado): o par de opostos é
``irradiação'' (Proclo) do Uno
supremo --- monopólo típico do neoplatonismo. Quatro pontos podem
ser assinalados no intuito de marcar a diferenciação do
neoplatonismo:

\begin{enumerate}
\item O neoplatonismo é conjugação de elementos
platônico"-acadêmicos (das “doutrinas não escritas”), com
elementos aristotélicos e pitagóricos;

\item A hierarquia platônica é mantida com lugar privilegiado para o
Uno em relação aos graus derivados (a Díada inclusive);

\item O neoplatonismo é sistema monopolar em contraste com o
\emph{bipolarismo} platônico;

\item O neoplatonismo redimensiona a contemplação platônica (o fim
último do processo de conhecimento) como “união plena com Deus”,
união esta marcada pela completa supressão (\emph{aphairesis})
de todo saber e todo dizer.
\end{enumerate}

\section{Do ``sistema
axiomático'' ao ``sistema
formal'' completo e consistente: impactos do(s)
teorema(s) da incompletude de Gödel no programa formalista}

\medskip\paragraph{Os alvos do programa formalista}

\let\subsection\paragraph

O que torna a matemática peculiar estaria em seu modo de
conceber verdades por meio do raciocínio \emph{a priori} de
maneira tal (firme e certa) que nada de empírico poderia
derrubá-la ou sequer ameaçá-la. Modelo de conhecimento
inatacável, posto que infalível, a demonstração geométrica
deveria ser imitada por todas as formas de conhecimento que se
pretendam rigorosas (cf. os modernos discursos sobre o método).

Mas é justamente devido a esta sua inatacabilidade
constitutiva, à sua incrível infalibilidade ($7+5$ é
\emph{sempre} $12$), que podemos suspeitar que falta à
matemática o caráter próprio de \emph{conhecimento}: daquilo
que \emph{nunca pode errar} dificilmente pode fazer sentido
dizer que alguma vez \emph{acerte} (ou em outros termos, o que
não pode ser usado para mentir também não pode ser usado para
dizer a verdade). A matemática talvez não passe de mero jogo
formal (sintático), segundo regras estipuladas, que
``diz nada sobre coisa nenhuma''
fora dele mesmo (semântica nula). A certeza absoluta matemática
talvez signifique apenas que Nada há lá garantindo essa certeza
total. O problema epistemológico poderia ser resolvido com uma
resposta a respeito da \emph{fonte} \emph{da certeza}
matemática. Se o conhecimento empírico, sempre atacável,
falível, sempre sujeito à revisão, tem como fonte as percepções
dos sentidos, qual poderia ser a base do \emph{conhecimento
infalível} (se é que isso faz sentido) matemático? Haveria
percepções extrassensoriais de entes matemáticos?
\emph{Intuições} matemáticas constituiriam seu
\emph{fundamento}? A certeza matemática dependeria de um
acesso intuitivo a um dado essencial ele mesmo ultraempírico?
Esta forma de intuição nos daria acesso ao que existe (lá fora)
como realidade matemática em si e por si? Ou é justamente esse
``lá fora'' que a infalibilidade
matemática não pode fazer presumir que \emph{existe}? Ou
ainda, para dizê-lo de modo mais comedido, seria algo de que não
se pode pretender \emph{falar com sentido}? O fundamento
objetivo da certeza matemática são entes em si e por si
subsistentes (realismo platônico)? Sua certeza incomparável se
baseia em meras regras internas do jogo (operações formais) sem
\emph{nada} de externo que a garanta? Ou ainda, haveria algo
(como o ``Místico'') a
sustentá-la, mas de que \emph{nada se pode falar} (com
sentido)?

O problema, para a epistemologia da matemática, está no
\emph{ponto de partida} de suas \emph{provas} (que uma vez
fixado permitiria demonstrações rigorosas daí em diante
irrepreensíveis apenas com base em regras formais públicas e
evidentes). E o problema com ``alegações de
intuições'' é que elas variam muito de pessoa a
pessoa. Como saber se estamos diante de uma intuição genuína e
não diante de uma crença sugerida, seja pelo desejo, pelo
apetite, pela vontade de poder, pela mera imaginação, por uma
alucinação ou pela fugaz esperança? Tudo o que parece
intuitivamente \emph{óbvio} para alguns pode (e deve) ser
colocado filosoficamente em questão, por outros --- como o tem
praticado a história da filosofia. Aliás, o trabalho
propriamente filosófico pode ser entendido como esforço
reiterado de colocar em questão o que é tido como
``óbvio'', tanto pelos outros
como --- tarefa ainda mais difícil --- por nós próprios. A
matemática, de qualquer modo, enquanto depender de
intuições"-do"-que-é-óbvio em sua base, nunca ficaria livre de
suspeitas epistêmicas.

David Hilbert se notabilizou justamente por constituir o
campo de estudo das teorias matemáticas formalizadas (a
``metamatemática''), visando
garantir sua consistência e completude e assim livrar as
matemáticas do recurso suspeito às intuições, substituindo"-as
por regras formais comuns mais rigorosas, trabalho que se volta
para as bases de toda demonstração: o ``sistema
axiomático'' --- colocando em questão seu
estatuto.

A ideia do sistema axiomático é a de que as múltiplas
verdades (evidentes) de qualquer ramo particular das matemáticas
podem ser organizadas na forma de axiomas, regras de inferência
e teoremas. Axiomas não requerem nenhuma prova adicional. E a
partir deles, munidos apenas com regras de inferência,
obteríamos outras verdades menos óbvias (as dos teoremas) que
decorrem desses dados básicos --- dos quais tudo mais depende.
Teorias matemáticas podem ser assim axiomatizadas:
\emph{escolhem"-se} determinadas proposições básicas depois
desenvolvidas apenas com recurso à lógica.

Uma teoria formalizada qualquer ($T$) é dita \emph{consistente}
se não há contradição entre seus teoremas (isto é, se nenhum de
seus teoremas é negação do outro), caso contrário ela é
inconsistente. Já uma teoria formalizada será \emph{completa}
se para cada uma das sentenças formais que possam ser
construídas em $T$ podemos \emph{decidir} que essa sentença (ou
sua negação) é um teorema de $T$. Um dos objetivos do programa
hilbertiano do início do século passado era provar tanto a
\emph{consistência} das várias teorias matemáticas (a ausência
de contradição entre seus vários teoremas), bem como a
\emph{completude} das mesmas, nos casos em que fosse possível,
libertando a matemática moderna dos paradoxos que no início do
século \textsc{xx} pareciam ameaçar suas pretensões cognitivas. Se
uma teoria formalizada é \emph{inconsistente} (se fere a lei
da não contradição) isso implica que ela é \emph{trivial},
isto é, qualquer uma de suas fórmulas será um teorema (se com
ela qualquer coisa pode ser dita isso equivale a nada dizer). Se
ela é \emph{incompleta} significa que ela é incapaz de
descrever tudo o que há de verdadeiro em seu domínio (ferindo a
lei do terceiro excluso ao admitir que há proposições cuja
verdade é \emph{indecidível} dentro do sistema: não há como
decidir se são verdadeiras ou falsas).

O sistema axiomático (público) impõe um controle aberto e comum
às intuições (privadas) dos cidadãos matemáticos. Nesse sentido
Wittgenstein, ao alinhar"-se a esse tipo de formalismo (com
ressalvas), foi acusado, com outros, de defender uma
``matemática bolchevique'', isto
é, uma visão de matemática da qual toda propriedade privada de
significados matemáticos estaria abolida em função de regras
públicas comunais.

A própria construção do sistema da aritmética é que
explicaria sua certeza particular --- ou o sistema é
\emph{completo} ou não é um sistema. O objetivo de eliminar
\emph{por completo} o recurso às intuições, como condição de
eliminar da matemática tudo o que há nela de potencialmente
obscuro, é o que levou à noção de \emph{sistema formal}: um
sistema axiomático plenamente exteriorizado, desprovido de
qualquer apelo à intuição ou a outros eventos misteriosos
ocorrendo no lado oculto da mente humana.

Bem, para eliminar as intuições é preciso manipular a
linguagem matemática e destituir o sistema axiomático de
qualquer significado, exceto os que podem ser definidos pelas
regras internas do próprio sistema --- o que, se chegasse a ser
realizado, equivaleria a uma refutação definitiva do platonismo
em matemática. Tais regras não podem mais pretender descrever
realidade objetiva alguma fora do sistema. Nesse ideal de
linguagem, os símbolos significativos na base do sistema teriam
de ser formados por sinais completamente sem sentido, ou seja,
meras marcas no papel cujo único significado é definido por suas
relações conforme a regras (por nós) estipuladas. Um sistema
formal não fala \emph{sobre} nada (é fechado sobre si mesmo).
Essa formalização visava transformar a atividade matemática em
um processo completamente determinado por regras bem definidas
que de nada falam senão de suas próprias \emph{operações}.
Seguir regras matemáticas tornar"-se"-ia atividade
\emph{combinatória} segundo \emph{funções recursivas}, como
as programáveis por um computador --- o uso de um algoritmo (ou
seja, sequências de operações que informam o que fazer em cada
passo, com base no resultado do passo anterior) pelo qual regras
são aplicáveis a resultados anteriores de aplicações de regras
sem levar em conta nenhum
``significado'', exceto o da
própria regra operatória.

A aquisição matemática da certeza completa teria por condição
a completa erradicação dos símbolos matemáticos de qualquer
significado extramatemático. A racionalidade \emph{a priori}
não teria sua raiz no acesso a algo externo ou em poderes
mentais misteriosos, mas na autonomia das regras internas que o
sistema formal a si mesmo se dá. Uma matemática completamente
formalizada eliminaria de si qualquer recurso a uma base
inquestionável “na verdadeira natureza das coisas” (in
\emph{rerum natura}). A \emph{construção} de um sistema
formal logicamente consistente mostraria que a matemática é uma
linguagem que \emph{não refere} intrinsecamente a coisa
alguma: mostraria que a matemática, simplesmente, \emph{não é
descritiva}. A visão metamatemática com este alvo, o
\emph{formalismo,} diz que não há, como imaginou Platão, uma
\emph{realidade objetiva} captada pelo sistema. Seriam as
regras estipuladas o que constituiria \emph{todas} as sua
verdades. A matemática ficaria estabelecida como algo que não
depende em nada de qualquer relação com a natureza
(\emph{physis}) --- nem com algo de pitagoramente sobrenatural
(\emph{meta ta physika}) ---, devendo satisfação apenas ao
\emph{nomos} (as convenções, estipulações por nós
estabelecidas). Não haveria qualquer verdade \emph{externa} à
qual as verdades matemáticas teriam de se adequar (ou
corresponder).

\section{impactos gödelianos sobre o programa
formalista} 

Parece haver muitos paralelos entre a posição formalista de
Hilbert e a de Wittgenstein (criticado por seu convencionalismo
radical) em aspectos básicos de suas concepções de matemática ---
assim como certos paralelos importantes de sua posição em
relação ao programa logicista (de redução da matemática à
lógica) de Frege e Russell, o que não visamos examinar aqui em
detalhes ---, de modo que pode ser útil considerar o impacto (dito
“devastador”) dos teoremas de Gödel sobre esses programas, antes
de focalizar enfim a posição (da não posição) de nosso
incompreendido filósofo.

O primeiro teorema da incompletude de Gödel afirma,
condicionalmente, de uma teoria matemática formalizada --- que
contenha a aritmética (elementar) --- que, \emph{se} ela for
consistente, então \emph{não pode} ser completa. Afirma com
isso a incompletude de qualquer sistema formal (capaz de
expressar a aritmética): “Toda axiomática da aritmética é
incompleta”; “Existem proposições aritméticas tais que nem elas,
nem suas negações, são demonstráveis na axiomática adotada”,
implicando que pode haver sempre axiomas aritméticos envolvendo
verdades deriváveis ulteriores.

Isso parece trazer implicações para a (im)possibilidade de
eliminar as intuições da matemática. Se a intuição não pode ser
eliminada, o entendimento do significado dos símbolos básicos
dependeria de apreendermos diretamente a existência real dos
objetos matemáticos (``daquilo''
que os signos matemáticos designam). A ineliminabilidade das
intuições implicaria a ineliminabilidade de uma realidade
matemática externa ao simbolismo matemático,
``realidade'' essa ela mesma
\emph{inefável,} o que levaria ao realismo matemático --- o
platonismo teria razão afinal contra a sofística positivista
protagoriana. Ou haveria aí antes, dependendo do modo como se
encara essa inefabilidade (a transcendência da realidade
matemática), algo de neoplatônico?

O segundo teorema de Gödel, derivado do primeiro, diz também
condicionalmente que a própria fórmula aritmética por ele
utilizada para expressar a consistência de uma teoria
formalizada (que contenha a aritmética) \emph{não é
demonstrável} nessa mesma teoria, caso ela seja consistente. Ou
precisamente: “A consistência de cada axiomática da aritmética
não pode ser demonstrada nessa axiomática” (ou ainda: “a prova
de ausência de contradição de uma axiomática da aritmética não
pode ser realizada apenas com os recursos dessa axiomática”),
implicando que é preciso introduzir um novo elemento para
demonstrar sua consistência, de modo que sua coerência está
sempre em aberto, dependendo desse elemento externo ao sistema,
ou de outro sistema, mais abrangente.

As conclusões de Gödel parecem cheias de sugestões, soam
prenhes de querer dizer algo sobre a realidade matemática, sobre
aquilo que ``tem de existir'',
para que nossa linguagem matemática faça sentido. Mas ainda que
seus leitores e intérpretes tenham querido ver muitas dessas
implicações, Gödel ele mesmo permaneceu sempre extremamente
comedido em relação às consequências de sua demonstração
metamatemática.

\section{Da reação de Wittgenstein a
Gödel}

Neste artigo não nos propusemos uma apreciação formal do
trabalho lógico"-matemático ou em metamatemática, nem mesmo uma
discussão aprofundada da epistemologia da matemática, mas antes
sondar a posição, ainda bem pouco compreendida, buscada por
Wittgenstein em relação aos descaminhos das teorias matemáticas
contemporâneas. Comentadores não hesitam em afirmar que
Wittgenstein simplesmente não foi capaz de compreender aquele
trabalho, tornando sua recusa dos resultados de Gödel uma
posição tida por inconsistente. Newton da Costa, por exemplo,
afirma que “embora Wittgenstein sempre houvesse achado que os
teoremas de incompletude eram apenas jogos sem maior
relevância”, isto era algo “sobre o qual Wittgenstein estava
enganado” (\textsc{oesp}, 2009). Também Dummett afirmava em 1959 que as
passagens de Wittgenstein sobre Gödel “são de pouca qualidade ou
contém definitivamente erros” (\textsc{Dummett}, 1959: 324), entre os
quais o de haver confundido
``verdade'' e
``provabilidade'' no campo
matemático. Vamos tentar reconstruir alguns traços dessa posição
difícil de Wittgenstein em filosofia da matemática a fim de
situarmos sua reação a Gödel e podermos enfim tecer algumas
considerações finais.

\subsection{A passagem de Gödel}

Em Königsberg, no Congresso sobre a Epistemologia das
Ciências Exatas em que Gödel se apresentou (em 07 de outubro de
1930), batiam"-se as posições do logicismo (Frege e Russell), do
intuicionismo (Leo Brouwer), do formalismo (Hilbert) e,
poder"-se"-ia acrescentar, a ``de
Wittgenstein'', que não participou do evento e
parecia partilhar algo de cada uma das anteriores, recusando"-as
completamente, por outro lado (ele foi representado por
Waismann, que à época procurava captar, sem sucesso, o seu
pensamento). O ponto central em discussão no Congresso era como
provar a dedutividade da matemática. No Congresso Gödel indicou
que proposições aritméticas verdadeiras, mas não dedutíveis,
eram possíveis: há proposições \emph{verdadeiras}, mas que não
podem ser \emph{provadas} (no interior do sistema). Com isso
ele refutava todas as grandes metaposições defendidas até então
--- o que custou a ser devidamente assimilado.

Contatou"-se assim que o problema de Hilbert não teria como
ser solucionado: jamais poderia haver uma prova formal
\emph{finitária} da consistência dos axiomas da aritmética
dentro do sistema da aritmética. Significou a devastação do
formalismo teorético de Hilbert. Mas e quanto ao
``formalismo antiteorético'' de Wittgenstein? 

Na matemática pós"-Gödel entende"-se que os aspectos sintáticos
dos sistemas formais não podem captar todas as verdades sobre o
sistema, inclusive a verdade de sua própria consistência. A
consistência do sistema, pressuposta para assegurar seus
fundamentos, \emph{transcende} o alcance do programa que
queria prová-lo (o que nunca foi o caso de Wittgenstein --- pelo
contrário). E o que vale para o sistema mais simples e exato que
possuímos (a aritmética dos números naturais) deve valer em
maior medida para sistemas incapazes de sequer se aproximar de
seu grau de perfeição. É no sentido dessa
``transcendência'' que estamos
procurando apontar a resiliência possível de traços
neoplatônicos.

A estratégia de Gödel para provar isso envolve tanto uma nova
forma de vínculo entre a linguagem objeto e a metalinguagem (a
aritmetização de Gödel), como a absorção do \emph{paradoxo da
referência} em sua demonstração, pelo que Gödel afirma algo
similar ao paradoxo de Epimênides (ou do mentiroso: “Todos os
cretenses são mentirosos”): “essa própria sentença é falsa” ---
que é verdadeira se e somente se é falsa; Gödel apenas
complementaria: “Essa própria sentença não é dedutível dentro do
sistema”. O resultado de Gödel teria mostrado a robustez da
noção matemática de infinito: a estrutura infinita dos números
naturais não pode ser reduzida a sistemas formais finitários; as
\emph{intuições do infinito} do matemático, essa impureza
formal, não poderiam, pois, ser expurgadas da matemática. Em
outras palavras, os números naturais \emph{transcendem} o
sistema formal da aritmética (aquilo de que trata o sistema
formal) --- resta sempre ``algo''
(\emph{Etwas}) que escapa à captura no sistema.

Agora a consistência de um sistema só pode ser provada saindo do
sistema. Em resumo, os aspectos puramente transparentes dos
sistemas formais (sintáticos) \emph{não podem provar} todas as
proposições verdadeiras expressáveis dentro do sistema (diz a
primeira prova) nem podem fornecer uma prova da consistência do
sistema (diz a segunda prova da incompletude). Hilbert teria se
enfurecido ao saber da prova de Gödel, mas a reconheceu e se
conformou (ele era um matemático trabalhando em um paradigma que
se quebrou). Já reação de Wittgenstein à incompletude foi bem
diversa, ele não teve de se adaptar ao resultado de Gödel
(afinal, ele não era um matemático e estava questionando o
paradigma), mas teve de “passar por cima” dele. Sua discordância
era de outra ordem.

\subsection{Wittgenstein e a <<passagem por
cima>> de Gödel}

A incompatibilidade lógica e filosófica entre as visões da
matemática de Gödel e Wittgenstein nos parece ser ampla e
irreconciliável, não sendo nossa proposta aqui sequer tentar
pontuá-la exaustivamente, apenas aspectualmente. Reconhecendo a
incompatibilidade de visões (filosóficas) sobre o assunto,
Wittgenstein teria replicado basicamente que Gödel simplesmente
\emph{não poderia} ter provado o que pensou ter provado. A
contestação se coloca no plano do sentido (de uma ``prova'').

Wittgenstein descartou a prova de Gödel como cheia de
“artimanhas” ou “truques lógicos” (\emph{logische
Kunststücken}) destituídos da importância matemática que lhe
atribuíram. Esse mesmo \emph{tipo de prova} é que ficaria
impossibilitado desde a visão de Wittgenstein da matemática: ele
permaneceu inflexível sobre a impossibilidade de \emph{dizer
algo} ``sobre'' uma linguagem
formal (sobre o que nela se mostra), tal como a aritmetização de
Gödel permitiu que ele pretendesse. O ponto é que “não há
metalinguagem”, argumenta Wittgenstein seguidamente (enquanto
linguagem de ordem superior que explique e funde a linguagem de
ordem inferior: todas as proposições estão no mesmo nível, ainda
que em jogos de linguagem diferentes e legitimamente
comparáveis). Gödel teria apenas articulado um jogo com outro e
não alcançado teoreticamente os fundamentos que se pretendeu que
ele houvesse alcançado.

Wittgenstein também negou que paradoxos pudessem ter as
consequências que Gödel parece ter querido tirar deles (e nesse
ponto Wittgenstein alinha"-se mais a Kierkegaard na ideia de que
paradoxos são antes limites da linguagem contra os quais nos
arremessamos, mas que nunca ultrapassamos para ver o que há
``do outro lado''). “Um cálculo
não pode dar informações sobre os fundamentos da matemática” diz
Wittgenstein (\textsc{pb}, 296), de modo que uma prova como a de Gödel
simplesmente \emph{não é possível} (não faz sentido) --- a não
ser com recurso a truques retóricos.

Essa posição antiteorética de Wittgenstein concebe o trabalho
filosófico antes como uma terapia gramatical (cujo alvo é a
clareza completa e não o \emph{progresso} pela construção de
novas teorias) e permite compreender algo da polêmica frase de
Wittgenstein sobre Gödel: “Minha tarefa não é falar sobre a
prova de Gödel, por exemplo, e sim passar por cima dela” (\textsc{rfm},
\textsc{v}, 16).

Quando Wittgenstein se ocupa de Gödel nas \emph{Observações
sobre os Fundamentos da Matemática}, é sempre para mostrar que
seus teoremas não podem significar o que pretendem: o
``sentido profundo'' que parecem
encarnar serão sempre apenas “nuvens de filosofia” a partir de
“gotinhas de gramática”. É compreensível que muitos matemáticos
e filósofos da lógica e da matemática continuem preferindo
acreditar que o idiossincrático filósofo austríaco simplesmente
não foi capaz de entender os progressos teoréticos para a
ciência da lógica alcançados pelo gênio morávio. Mas talvez
possa valer a pena supor, nem que seja em função do Princípio de
Caridade, que o que ele pensou pode fazer algum sentido, e de
que modo ele tentaria nos convencer de que o que \emph{não faz
sentido} é falar em ``ciência da
lógica'', por exemplo.

Apesar de ter sido o tema a que mais extensamente se dedicou
em sua vida filosófica, a filosofia da matemática de
Wittgenstein continua sendo parte menos conhecida e mais
subestimada de sua obra. Desde pontos deixados no
\emph{Tractatus}, suas anotações sobre filosofia da matemática
se estendem pela Fase Intermediária nas \textsc{pb}
(\emph{Philosophiche Bemerkungen}/ \emph{Observações
Filosóficas}) de 1929--30 e \textsc{pg} (\emph{Philosophical
Grammar}/ \emph{Gramática Filosófica}) de 1931--33, e
especialmente as \textsc{rfm} (\emph{Remarks on the Foundations of the
Mathematics}/ \emph{Observações sobre os Fundamentos da
Matemática}) que englobam anotações desde 1937 até 1944, já na
``fase madura'' das \emph{Investigações
Filosóficas}. De alguns traços reiterados dessas notas em
evolução (desde 1918 até 1944) pode"-se extrair alguns pontos de
caracterização de sua gramática da matemática.

\begin{enumerate}
\item A verdade matemática é não referencial; sua natureza é
puramente sintática; uma ``verdade formal''. Só proposições
contingentes podem ser usadas para fazer afirmações sobre a
realidade, sobre o mundo, sobre como as coisas estão, isto é,
podem ser verdadeiras ou falsas; podemos inventar e expandir a
matemática o quanto quisermos, mas as novas provas não
pré-existem ao caminho que as construiu. A proposição matemática
não é verdadeira por correspondência com alguma realidade. Sua
verdade formal deriva da obediência às regras internas do
sistema dentro do qual a proposição vive sua vida.

\item Tautologias, contradições e proposições matemáticas (equações)
não são verdadeiras nem falsas. ``Verdadeiro'' aqui tem sentido
totalmente diferente daquele pelo qual uma proposição empírica
(contingente) pode ser verdadeira ou falsa; donde proposições
matemáticas \emph{não exprimem conhecimento} \emph{sobre o
mundo}, mas apenas sobre as regras do jogo (as operações
formais); equações matemáticas são pseudo"-proposições.
Verdadeiro em matemática apenas “marca a equivalência de sentido
de duas expressões” (\textsc{tlp}, 6.2323) e têm em comum com a
tautologia estarem no limite da desintegração da combinatória
dos signos, sem qualquer relação de figuração com a realidade;
proposições matemáticas não tem sentido: não podemos usá-las
para afirmar ou negar fatos; equações dizem apenas que duas
expressões são equivalentes em significado e são, portanto,
intersubstituíveis --- do mesmo modo pelo qual pode"-se reconhecer
que proposições lógicas são verdadeiras a partir do símbolo
apenas. Nunca precisamos comparar “o que elas expressam” com os
fatos (\textsc{tlp}, 6.2321; cf. \textsc{rfm}, \textsc{iii}, 4).

\item “A matemática é um método da lógica” (\textsc{tlp}, 6.234). A lógica do
mundo é mostrada tanto nas tautologias da lógica quanto nas
equações da matemática (\textsc{tlp}, 6.22). Conhecemos sua verdade por
operações puramente formais. A relação entre matemática e lógica
aqui não é de redutibilidade (nem de identidade). Realizamos
operações lógicas com proposições e operações aritméticas com
números. Não há razão para fazer da concepção de matemática do
\textsc{tlp} uma versão de logicismo: não há reducionismo, mas antes
``semelhanças de família'' entre lógica e matemática --- o que só
fica esclarecido na fase posterior à do \textsc{tlp}.

\item A matemática é puramente sintática, desprovida de referência e
de semântica. Os sinais e proposições de um cálculo \emph{a
nada referem}: não faz sentido procurar ``as
coisas que são'' o seu significado. A relação
importante não se dá entre significado matemático e objetos
matemáticos (a objetividade), nem com operações mentais
matemáticas (a subjetividade), mas sim com o preocedimento, com
a \emph{decidibilidade algorítmica}.

\item Cálculos matemáticos são compostos de extensões (símbolos,
conjuntos, proposições, axiomas) e intensões (regras de
inferência e de transformação). Falar em
``extensão de objetos
matemáticos'' por comparação com uma extensão de
objetos físicos é apoiar"-se em uma analogia enganosa. Não existe
extensão infinita em matemática (``extensão
infinta concluída'' é autocontraditório).  Não
pode, pois, haver quantificação infinita; toda proposição
matemática é decidível --- por decidibilidade algorítmica (implica
saber como decidir uma proposição por um processo de decisão
conhecido). Não existem infinitas extensões em matemática (e por
isso Wittgenstein rejeita a prova de Cantor dos conjuntos
infinitos de cardinalidade maior ou menor). Nossos cálculos são
sempre apenas extensões finitas de regras intensionais
infinitamente reiteráveis. Supor um infinito real é apenas
confundir intensões e extensões matemáticas.

\item O infinito não é uma quantidade.
``Infinito'' e
``cinco'' não têm a mesma
sintaxe. Uma classe infinita é uma regra recursiva (ou “uma
indução”, nos termos de Wittgenstein) enquanto um símbolo para
uma classe finita é uma lista ou uma extensão (\textsc{pg}, 461).
Extensões matemáticas são sempre necessariamente sequências
finitas de símbolos (o construtivismo finitista
wittgensteiniano). Uma série infinita é a “possibilidade
infinita de séries finitas de números” (\textsc{pr}, 144). Entendido
corretamente o infinito não é uma quantidade, mas antes algo
como uma “possibilidade infinita” (\textsc{pr}, 138). E note"-se que não
poder haver uma quantidade infinita não se deve a limitação
nossa: nem deus seria capaz de totalizar um produto lógico
infinito (a série não pode acabar também para Ele).

\item Em matemática \emph{tudo} é algoritmo e \emph{nada} é
significado (\textsc{pg}, 368). Aqui a distinção sintaxe x semântica não
se aplica: tudo é sintaxe (as regras do jogo). Uma expressão só
é uma proposição matemática em um cálculo dado (\textsc{pg}, 155) e
somente se esse cálculo contém um procedimento de decisão
conhecido (e aplicável) (\textsc{pg}, 379).

\item A matemática é decidível. Nada na matemática (no cálculo) é
indecidível; somente na
``prosa'', na moldura que
acompanha o cálculo (nas pretensas teorias matemáticas) pode
haver (pseudo)proposições indecidíveis. Wittgenstein rejeita a
indecidibilidae matemática com base em que expressões numéricas
que quantificam sobre um domínio infinito não são
algoritmicamente decidíveis (sendo este o critério para decidir
que não são proposições matematicamente
\emph{significativas}).

\item Nessa perspectiva de compreensão das matemáticas pode ficar
mais claro por que Wittgenstein teria querido refutar ou
inviabilizar, em seus próprios termos, a alegada prova de Gödel
da existência de proposições verdadeiras mas não dedutíveis em
um sistema ``S'' qualquer (que
contenha a aritmética, tal como o de \emph{Principia
Matemática}). Ao mostrar que “verdadeiro no cálculo Y” não pode
significar senão “provado no cálculo Y”, a própria ideia de uma
proposição verdadeira, mas não provável no cálculo Y, deve
\emph{perder o sentido}.
\end{enumerate}

Os aspectos aqui elencados não pretendem esgotar os elementos
e muito menos os extensos argumentos apresentados por
Wittgenstein para defender sua concepção (impalatável para a
maioria dos matemáticos) do estatuto e dos fundamentos da
matemática. Quisemos apenas fornecer elementos que permitam
começar a entender que pode haver uma coerência interna na
posição wittgensteiniana (nem isso nem aquilo) ao rejeitar tanto
as posições metamatemáticas em jogo em meados do século passado
quanto a estratégia teorética de Gödel que as teria refutado.
Podemos então voltar para nossa pergunta inicial: pode"-se
indicar algum traço neoplatônico na atitude de Wittgenstein
diante dos progressos da lógica"-matemática do século \textsc{xx}?

\section{CONSIDERAÇÕES FINAIS}

Gödel parecia acreditar, de acordo com Rebecca Goldstein, no
mínimo, que sua prova respaldava a posição platônica sobre a
existência de um domínio supra"-sensível de verdades eternas. Se
Gödel pode ou não por isso ser dito um platônico (entre
positivistas protagóricos), por apontar para a necessidade de
uma realidade matemática que transcende nossos sistemas formais,
isso é algo que deixamos indecidido; mas nos restaria ainda
aventar a possibilidade de indicar certo índice de neoplatonismo
nas atitudes de Wittgenstein ao negar ao seu conterrâneo
justamente o direito de \emph{querer dizer} o que pretendia
ter \emph{dito} (sobre os fundamentos da matemática).

“Não podemos dizer verdades indizíveis, mas elas existem”,
diria a incompletude tractariana. Diz o jovem Wittgenstein:
“Existem, de fato, coisas que não podem ser expressas em
palavras. Elas se fazem manifestas. Elas são o que é místico”
(\textsc{tlp}, 6.522). Gödel e Wittgenstein reconhecem partes esquivas da
realidade, mas de modos irreconciliavelmente diferentes. Para (o
jovem) Wittgenstein o que sistematicamente escapa aos nossos
sistemas é o indizível (que inclui tudo o que é importante --- e
querer falar disso só pode gerar contra"-sensos: deve ser objeto
de místico silêncio). Gödel já parece acreditar que nosso
conhecimento expressável é maior que nossos sistemas: o que não
podemos formalizar, ainda assim podemos \emph{conhecer} --- e
nesse ponto o caminho de ambos, nos parece, se bifurca
definitivamente.

Wittgenstein, que era tão dramaticamente loquaz preferiu
manter o silêncio necessário (sobre o fundamental) e passar por
cima da loquacidade dos intérpretes dos teoremas do taciturno e
comedido Gödel. Os cursos sobre os “Fundamentos da Matemática”
que Wittgenstein continuou oferecendo em Cambridge continuaram
argumentando contra a possibilidade de uma lógica matemática em
geral e contra a possibilidade de uma metamatemática, em
particular --- como corolário da impossibilidade de uma
metalinguagem e de suas pretensões teoréticas fundacionistas,
exemplo clássico da confusão fundamental ainda reinante acerca
da essência da linguagem humana.

Se retomamos agora os sete pontos inicialmente indicados para
caracterizar, de acordo com Merlan, a posição neoplatônica e sua
diferenciação do platonismo, podemos traçar uma comparação com a
posição de Wittgenstein nos seguintes termos:

\begin{enumerate}
\item No Neoplatonismo as esferas do ser estão subordinadas em
graus superiores e inferiores (seres perceptíveis no espaço e
tempo); em Wittgenstein pode"-se dizer que há uma hierarquia
entre o inferior (os fatos dizíveis) e o superior (os valores
indizíveis), mas todas as proposições estão no mesmo nível e
sobre o superior “deve"-se calar”. 

\item No Neoplatonismo há uma derivação das esferas inferiores
das superiores, derivação ocorrida ela mesma fora do espaço e
tempo (fora de qualquer relação causal ou de implicação lógica);
 em Wittgenstein não se coloca o problema da derivação
ontológica ou física dos seres, mas sobre a derivação do sentido
dos seres se pode dizer que todas as formas linguísticas as mais
elaboradas têm sua origem na práxis linguística cotidiana e em
nenhum lugar mais. Nem de tipo lógico nem de tipo causal, as
regras para a geração de sentido são de tipo gramatical.

\item No Neoplatonismo a esfera superior deriva de um princípio
primeiro superior a todo ser; em Wittgenstein, sobre isso, creio
que nada se pode falar (ou só se pode falar ``nada'').

\item No Neoplatonismo o princípio, chamado Uno, é simplicidade
absoluta superior a toda determinação;  em Wittgenstein, se
existe algo similar, devemos chamá-lo de
``Místico'', que pode até ser
dito uno enquanto pseudo"-categoria negativa que engloba tudo o
que há de essencial e de indizível, mas que não podemos dizer
que seja simples (uma vez que simples e complexo são relativos
ao jogo que se joga); porém, com certeza não é objeto de
conhecimento ou expressão teórica, estando acima (ou debaixo) de
toda e qualquer determinação.

\item No Neoplatonismo a multiplicidade das esferas inferiores
implica aumento dos entes e de suas determinações; em
Wittgenstein o abandono do ideal inicial de exatidão lógica faz
reconhecer que nossos termos linguísticos são essencialmente
vagos, que os jogos de linguagem possíveis são inumeráveis; não
é possível eliminar a multiplicidade e é preciso aprender a
conviver com ela; toda generalização e exatidão implicam
simplificação e unilateralidade, vícios filosóficos tradicionais
que o filósofo gramatical deve estar sempre pronto a combater.

\item No Neoplatonismo o conhecimento do princípio primeiro não
pode ser predicativo (uma vez que ele não exibe determinações);
em Wittgenstein, inicialmente, só fatos podiam ser descritos por
proposições e ser, portanto, \emph{conhecidos}: nem mesmo
verdades matemáticas (intermediárias) podem ser propriamente
``conhecidas''. No segundo
momento é preciso investigar a gramática de ``conhecer'' e de
``predicar'' para abordar a questão, mas de modo algum a filosofia
seria encarregada de elaborar por si só um tipo de conhecimento
como esse, uma vez que a filosofia não é uma teoria cognitiva,
mas antes uma atividade esclarecitiva.

\item Se no Neoplatonismo a dificuldade fundamental seria a de
explicar a passagem da unidade à multiplicidade, em Wittgenstein
a dificuldade ainda mais fundamental e diretamente enfrentada
pelo filósofo está em compreender a multiplicidade ela mesma,
tal como se dá.
\end{enumerate}

Parece enfim um contra"-senso pretender classificar
Wittgenstein como um neoplatônico. Mas um traço forte dessa
atitude ao menos parece ressaltar quando o contrastamos com
filósofos e matemáticos como Frege, Russell, Hilbert e Gödel,
por exemplo: a inefabilidade mística do fundamental --- algo sobre
o que seus adversários querem continuar falando (e dizendo algo
com isso) incontinentemente e felizes da vida. Se, por fim, o
neoplatonismo redimensiona a contemplação platônica como “união
plena com Deus”, onde não cabe nenhum
``dizer'' e nenhum
``saber'', de acordo com Merlan,
podemos admitir que essas expressões não são estranhas ao
vocabulário inicial wittgensteiniano (cf. os \emph{Notebooks}
de 1914--16), desde que não sejam vistas como descrições de
eventos sobrenaturais, mas antes como expressão da atitude, algo
estoica, de “aceitar o mundo tal como ele é”, ao contemplarmos o
mundo \emph{sub specie aeternis}. Mas nem por isso estamos em
melhores condições para decidir se nosso modo de vida faz
sentido: segue da constatação de nossa impossibilidade de obter
uma prova de um sistema formal dentro do próprio sistema,
apoiada por diferentes razões tanto por Gödel como por
Wittgenstein, que também \emph{não podemos pretender validar a
nossa racionalidade usando nossa própria racionalidade}. Quer
dizer que uma pessoa, dentro de seu sistema de crenças (o que
inclui crenças sobre suas crenças) não tem como sair desse
sistema para descobrir se ele é racional. E, no entanto, a vida
segue em frente, sem que possamos nada decidir sobre essas
``questões fundamentais''.

Para o Neoplatonismo e a Contemporaneidade deixo aqui, por
fim, duas perguntas:

\begin{enumerate}
\item A incompletude em si (ou seja, a indicação da necessidade de
intuir \emph{algo} de verdadeiro que transcende o sistema e
não se deixa deduzir por ele) é ela mesma uma tese neoplatônica,
ou ao menos indica uma atitude neoplatônica em relação aos
fundamentos do conhecimento humano?

\item Devendo a atitude de Gödel ser caracterizada antes de
platônica, a ``incompletude
mística'' de Wittgenstein (sua atitude radical
contra a pretensão de um meta"-discurso que queira explicar os
fundamentos do próprio discurso) seria então expressão de uma
atitude neoplatônica sobre os fundamentos da matemática, do
conhecimento e sobre a essência da linguagem de que é capaz o
ser humano (\emph{zoon ekon logon})?
\end{enumerate}

\section{Bibliografia}

\begin{description}\labelsep0ex\parsep0ex
\newcommand{\tit}[1]{\item[\textnormal{\textsc{\MakeTextLowercase{#1}}}]}
\newcommand{\titidem}{\item[\line(1,0){25}]}
\tit{BEZERRA}, C.C., \emph{Compreender Plotino e Proclo,} Petrópolis:
Vozes, 2006.

\tit{Da COSTA}, N.A., “A Rigorosa Lógica Kafkiana de Gödel”, in: O
Estado de São Paulo (01/\,03/\,2009).

\tit{GOLDSTEIN}, R., \emph{A Prova e o paradoxo de Kurt Gödel}, São
Paulo: Cia das Letras, 2008.

\tit{NAGEL}, E. e \textsc{newman}, J., \emph{Prova de Gödel}, São Paulo:
Perspectiva, 1998, 2a  edição.

\tit{MERLAN}, \emph{Dal platonismo al neoplatonismo}, traduzione
Enrico Peroli, Milano: Vita e Pensiero, 1975.

\tit{WITTGENSTEIN}, L., \emph{Philosophische Bemerkungen}, Suhrkamp
(1970), trad. inglesa \emph{Philosophical Remarks} (\textsc{pr}) by R.
Heargraves and R. White, Oxford: Blackwell, 1975.

\titidem. \emph{Philosophical Grammar} (\textsc{pg}), trad. A. Kenny,
California: Univ. of California Press, 1974.

\titidem. \emph{Philosophical Investigations},
3\textsuperscript{rd} edition, Oxford: Blackwell, 2001,
translated by G.E.M. Anscombe; trad. bras. J. C. Bruni, São
Paulo: Abril Cultural, 1975.

\titidem. \emph{Remarks on the Foudations of Mathematics}
(\textsc{rfm}), Oxford: Blackwell, 1978, translated by G.E.M. Anscombe.

\titidem. \emph{Tractatus Logico"-Philosophicus} (\textsc{tlp}),
London: Routledge, 1961, translated by D. Pears and B.
McGuiness; trad. bras. de L. H. L. dos Santos, São Paulo: Edusp,
1996. 
\end{description}


